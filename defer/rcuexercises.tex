% defer/rcuexercises.tex
% mainfile: ../perfbook.tex
% SPDX-License-Identifier: CC-BY-SA-3.0

\subsection{RCU Exercises}
\label{sec:defer:RCU Exercises}

이 섹션은 여러분을 이 책의 앞부분에 나온 여러 예제들에 RCU 를 적용하도록 하는
여러 Quick Quizz 들로 이루어져 있습니다.
각 Quick Quiz 에의 답은 어떤 힌트를 제공하며, 그 해결책이 길게 설명되는 뒤의
섹션들로의 연결도 제공합니다.
\co{rcu_read_lock()}, \co{rcu_read_unlock()}, \co{rcu_dereference()},
\co{rcu_assign_pointer()}, 그리고 \co{synchronize_rcu()} 기능들은 이 문제들의
대부분에 충분할 겁니다.

\iffalse

This section is organized as a series of Quick Quizzes that invite you
to apply RCU to a number of examples earlier in this book.
The answer to each Quick Quiz gives some hints, and also contains a
pointer to a later section where the solution is explained at length.
The \co{rcu_read_lock()}, \co{rcu_read_unlock()}, \co{rcu_dereference()},
\co{rcu_assign_pointer()}, and \co{synchronize_rcu()} primitives should
suffice for most of these exercises.

\fi

\EQuickQuiz{
	Listing~\ref{lst:count:Per-Thread Statistical Counters}
	(\path{count_end.c}) 에 보인 통계적 카운터 구현은 \co{read_count()} 의
	합산을 보호하기 위해 전역적 락을 사용했는데, 이는 나쁜 성능과 음의
	확장성으로 이어졌습니다.
	여러분은 \co{read_count()} 에 훌륭한 성능과 좋은 확장성을 제공하기 위해
	RCU 를 어떻게 사용할 수 있겠습니다.
	(\co{read_count()} 의 확장성은 모든 쓰레드의 카운터를 스캔해야 한다는
	필요로 인해 제한되야 할 수 있음을 명심하십시오.)

	\iffalse

	The statistical-counter implementation shown in
	Listing~\ref{lst:count:Per-Thread Statistical Counters}
	(\path{count_end.c})
	used a global lock to guard the summation in \co{read_count()},
	which resulted in poor performance and negative scalability.
	How could you use RCU to provide \co{read_count()} with
	excellent performance and good scalability.
	(Keep in mind that \co{read_count()}'s scalability will
	necessarily be limited by its need to scan all threads'
	counters.)

	\fi

}\EQuickQuizAnswer{
	힌트: 전역 변수 \co{finalcount} 와 배열 \co{counterp[]} 를 하나의 RCU
	로 보호되는 구조체에 위치시키십시오.
	초기화 시점에, 이 구조체는 할당되고 0 과 \co{NULL} 로 설정됩니다.

	\co{inc_count()} 함수는 변경되지 않습니다.

	\co{read_count()} 함수는 \co{final_mutex} 를 잡는 대신
	\co{rcu_read_lock()} 을 사용하며, 현재 구조체로의 참조를 얻기 위해
	\co{rcu_dereference()} 를 사용해야 할겁니다.

	\co{count_register_thread()} 함수는 새로 생성된 쓰레드의 per-thread
	\co{counter} 변수를 참조하기 위해 그 쓰레드에 연관된 배열 내 원소의
	값을 설정할 겁니다.

	\iffalse

	Hint: place the global variable \co{finalcount} and the
	array \co{counterp[]} into a single RCU-protected struct.
	At initialization time, this structure would be allocated
	and set to all zero and \co{NULL}.

	The \co{inc_count()} function would be unchanged.

	The \co{read_count()} function would use \co{rcu_read_lock()}
	instead of acquiring \co{final_mutex}, and would need to
	use \co{rcu_dereference()} to acquire a reference to the
	current structure.

	The \co{count_register_thread()} function would set the
	array element corresponding to the newly created thread
	to reference that thread's per-thread \co{counter} variable.

	\fi

	\co{count_unregister_thread()} 함수는 새 구조체를 할당하고,
	\co{final_mutex} 를 잡고, 기존 구조체의 값을 새 것으로 복사하고, 끝나는
	쓰레드의 \co{counter} 변수를 total 로 더하고, 이 같은 \co{counter}
	변수로의 포인터를 \co{NULL} 값으로 쓰고, 새 구조체로 기존 것을 교체하기
	위해 \co{rcu_assign_pointer()} 를 사용하고, \co{final_mutex} 를 놓고,
	하나의 grace period 를 기다린 후, 마침내 기존 구조체를 메모리 해제할
	겁니다.

	이게 정말 잘 동작할까요?
	왜 일까요 또는 왜 아닐까요?

	더 자세한 내용을 위해선
	page~\pageref{sec:together:RCU and Per-Thread-Variable-Based Statistical Counters}
	의
	Section~\ref{sec:together:RCU and Per-Thread-Variable-Based Statistical Counters}
	을 보시기 바랍니다.

	\iffalse

	The \co{count_unregister_thread()} function would need to
	allocate a new structure, acquire \co{final_mutex},
	copy the old structure to the new one, add the outgoing
	thread's \co{counter} variable to the total, \co{NULL}
	the pointer to this same \co{counter} variable,
	use \co{rcu_assign_pointer()} to install the new structure
	in place of the old one, release \co{final_mutex},
	wait for a grace period, and finally free the old structure.

	Does this really work?
	Why or why not?

	See
	Section~\ref{sec:together:RCU and Per-Thread-Variable-Based Statistical Counters}
	on
	page~\pageref{sec:together:RCU and Per-Thread-Variable-Based Statistical Counters}
	for more details.

	\fi

}\EQuickQuizEnd

\EQuickQuiz{
	Section~\ref{sec:count:Applying Exact Limit Counters}
	은 제거될 수 있는 기기로의 I/O 액세스를 세기 위한 두 쌍의 코드 조각을
	보였습니다.
	이 코드 조각들은 reader-writer 락을 잡아야 하는 이유로 (I/O 를
	시작하는) fastpath 에서의 높은 오버헤드로 고생이 심했습니다.
	여러분은 훌륭한 성능과 확장성을 제공하기 위해 RCU 를 어떻게
	사용하겠습니까?
	(I/O 액세스를 하는 일반적인 경우를 위한 코드 조각의 성능은 기기 제거
	코드 조각의 것보다 훨씬 중요함을 명심하십시오.)

	\iffalse

	Section~\ref{sec:count:Applying Exact Limit Counters}
	showed a fanciful pair of code fragments that dealt with counting
	I/O accesses to removable devices.
	These code fragments suffered from high overhead on the fastpath
	(starting an I/O) due to the need to acquire a reader-writer
	lock.
	How would you use RCU to provide excellent performance and
	scalability?
	(Keep in mind that the performance of the common-case first
	code fragment that does I/O accesses is much more important
	than that of the device-removal code fragment.)

	\fi

}\EQuickQuizAnswer{
	힌트: reader-writer 락의 read-잡기를 RCU read-side 크리티컬 섹션으로
	대체하고, 기기 제거 코드를 그에 맞게 변경하십시오.

	이 문제를 위한 한가지 해결책을 위해
	page~\pageref{sec:together:RCU and Counters for Removable I/O Devices}
	의
	Section~\ref{sec:together:RCU and Counters for Removable I/O Devices}
	을 보십시오.

	\iffalse

	Hint: replace the read-acquisitions of the reader-writer lock
	with RCU read-side critical sections, then adjust the
	device-removal code fragment to suit.

	See
	Section~\ref{sec:together:RCU and Counters for Removable I/O Devices}
	on
	page~\pageref{sec:together:RCU and Counters for Removable I/O Devices}
	for one solution to this problem.

	\fi

}\EQuickQuizEnd
