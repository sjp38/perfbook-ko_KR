% defer/rcu.tex
% mainfile: ../perfbook.tex
% SPDX-License-Identifier: CC-BY-SA-3.0

\section{Read-Copy Update (RCU)}
\label{sec:defer:Read-Copy Update (RCU)}
%
\epigraph{``Free'' is a \emph{very} good price!}{\emph{Tom Peterson}}

앞의 섹션들에서 다루어진 모든 메커니즘은 특정한 행동을 그것이 안전히 이루어질
수 있을 때까지 미루는 여러 전략 중 하나를 사용했습니다.
Section~\ref{sec:defer:Reference Counting}
에서 다룬 레퍼런스 카운터는 읽기 쓰레드를 방해할 수 있는 행동을 뒤로 미루기
위해 명시적 카운터를 사용했고 이는 읽기 쪽 컨텐션과 그로 인한 낮은 확장성의
결과를 가져왔습니다.
Section~\ref{sec:defer:Hazard Pointers}
은 쓰레드별 포인터 리스트의 모습을 한 암시적 카운터를 사용했습니다.
이는 읽기 쪽 컨텐션은 제거했으나, 읽기 쓰레드가 쓰기와 조건적 브랜치는 물론,
읽기쪽 완전한 메모리 배리어 또는 리억타임에 친화적이지 않은 프로세서간 인터럽트
(inter-processor interrupt) 기능을 사용해야 했습니다.\footnote{
	어떤 중요한 특수 경우에는, 이 추가적 일이 Folly 오픈소스 라이브러리
	(\url{https://github.com/facebook/folly}) 에 구현되어 있는
	UnboundedQueue 와 ConcurrentHashMap 데이터 구조에서 보여지듯 연결
	카운팅을 사용하는 것으로 제거될 수 있습니다.}
Section~\ref{sec:defer:Sequence Locks}
에서 선보인 시퀀스 락 또한 읽기 쪽 컨텐션은 제거하나, 포인터 순회를 보호하지
않으며, 해저드 포인터와 같이, 읽기 쪽에서의 완전한 메모리 배리어 또는 업데이트
쪽에서의 프로세서간 인터럽트를 필요로 합니다.
이 방법들의 단점들은 더 나은 것은 불가능한지 질문을 던지게 합니다.

\iffalse

All of the mechanisms discussed in the preceding sections
used one of a number of approaches to defer specific actions
until they may be carried out safely.
The reference counters discussed in
Section~\ref{sec:defer:Reference Counting}
use explicit counters to defer actions that could disturb readers,
which results in read-side contention and thus poor scalability.
The hazard pointers covered by
Section~\ref{sec:defer:Hazard Pointers}
uses implicit counters in the guise of per-thread lists of pointer.
This avoids read-side contention, but requires readers to do stores and
conditional branches, as well as either full memory barriers in read-side
primitives or real-time-unfriendly inter-processor interrupts in
update-side primitives.\footnote{
	In some important special cases, this extra work can be avoided
	by using link counting as exemplified by the UnboundedQueue
	and ConcurrentHashMap data structures implemented in Folly
	open-source library
	(\url{https://github.com/facebook/folly}).}
The sequence lock presented in
Section~\ref{sec:defer:Sequence Locks}
also avoids read-side contention, but does not protect pointer
traversals and, like hazard pointers, requires either full memory barriers
in read-side primitives, or inter-processor interrupts in update-side
primitives.
These schemes' shortcomings raise the question of
whether it is possible to do better.

\fi

이 섹션은 \emph{read-copy update} (RCU) 를 소개하는데, 자주 업데이트 되는 공유
데이터로의 비싼 쓰기 없이 읽기 쓰레드가 소스 코드의 특정 영역과 연관지어지게
하는 API 를 제공합니다.
이 섹션의 나머지 부분은 여러 다른 관점에서 RCU 를 알아봅니다.
\Cref{sec:defer:Introduction to RCU} 는 고전적 RCU 소개를 제공하고,
\cref{sec:defer:RCU Fundamentals} 는 기본적인 RCU 컨셉을 다루며,
\cref{sec:defer:RCU Linux-Kernel API} 는 RCU 의 리눅스 커널 API 를 선보이고,
\cref{sec:defer:RCU Usage} 는 흔한 RCU 사용 예를 소개하고,
\cref{sec:defer:RCU Related Work} 는 RCU 와 연관된 최근의 작업들을 다루고,
\cref{sec:defer:RCU Exercises} 는 일부 RCU 연습문제를 제공하고, 마지막으로
\cref{sec:defer:What About Updates?} 는 업데이트를 다룹니다.

\iffalse

This section introduces \emph{read-copy update} (RCU), which provides
an API that allows readers to be associated with regions in the source code,
rather than with expensive updates to frequently updated shared data.
The remainder of this
section examines RCU from a number of different perspectives.
\Cref{sec:defer:Introduction to RCU} provides the classic
introduction to RCU,
\cref{sec:defer:RCU Fundamentals} covers fundamental RCU
concepts,
\cref{sec:defer:RCU Linux-Kernel API} presents the Linux-kernel
API,
\cref{sec:defer:RCU Usage} introduces some common RCU use cases,
\cref{sec:defer:RCU Related Work} covers recent work related
to RCU,
\cref{sec:defer:RCU Exercises} provides some RCU exercises,
and finally
\cref{sec:defer:What About Updates?}
discusses updates.

\fi

% defer/rcuintro.tex

\subsection{Introduction to RCU}
\label{sec:defer:Introduction to RCU}

앞의 섹션들에서 이야기된 방법들은 어느정도 확장성 있긴 했지만 모두 Pre-BSD
라우팅 테이블을 위한 성능에 있어서 이상적이지 못했습니다.
Pre-BSD 탐색 오버헤드가 싱글 쓰레드 탐색에서와 동일하게끔, 병렬로 수행되는
탐색이 싱글 쓰레드에서의 탐색과 동일한 어셈블리어 인스트럭션 시퀀스를
수행한다면 좋을 겁니다.
이는 좋은 목표가 될 수 있지만, 그러기 위해서는 구현 단계에서 많은 심각한 질문을
이끌어냅니다.
하지만 이걸 시도하면 어떤 일이 벌어질지 알아보고 삽입과 삭제를 구분해서
다뤄봅시다.
\iffalse

The approaches discussed in the preceding sections have provided
some scalability but decidedly non-ideal performance for the
Pre-BSD routing table.
It would be nice if the overhead of Pre-BSD lookups was the same as
that of a single-threaded lookup, so that the parallel lookups would
execute the same sequence of assembly language instructions as would a
single-threaded lookup.
Although this is a nice goal, it does raise some serious implementability
questions.
But let's see what happens if we try, treating insertion and deletion
separately.
\fi

\begin{figure}[tb]
\begin{center}
\resizebox{3in}{!}{\includegraphics{defer/RCUListInsertClassic}}
\end{center}
\caption{Insertion With Concurrent Readers}
\label{fig:defer:Insertion With Concurrent Readers}
\end{figure}

아이템 추가를 위한 고전적인 방법이
Figure~\ref{fig:defer:Insertion With Concurrent Readers} 에 표현되어 있습니다.
첫번째 열은 기본 상태를 보이는데, \co{gptr} 은 \co{NULL} 값을 갖습니다.
두번째 열에서는 하나의 구조체를 메모리 할당하는데, 초기화 되지 않은 부분들은
물음표로 표시되어 있습니다.
세번째 열에서는 이 구조체를 초기화 시킵니다.
다음으로, \co{gptr} 이 이 새로운 원소를 가리키도록 그 값을
할당합니다.\footnote{
	많은 컴퓨터 시스템들에서, 컴파일러와 CPU가 간섭을 끼칠 수 있기 때문에
	단순한 값 할당은 충분하지 못합니다.
	이런 문제에 대해서는 Section~\ref{sec:defer:RCU Fundamentals} 에서
	다루게 될 겁니다.}
최근의 범용 시스템에서 이 값 할당은 어토믹해서 동시에 수행되는, 읽기를 하는
쓰레드들은 \co{NULL} 포인터 또는 새로운 구조체 \co{p} 로의 포인터 둘 중
하나만을 보게 되지, 두 값 중 일부가 합쳐져 있는 값은 보지 못합니다.
따라서, 각각의 읽기를 하는 쓰레드는 기본값인 \co{NULL} 을 보거나
기본값이 아닌 값을 보게 되거나일 것이며 어느 쪽이든 읽기는 일관적인 결과만을
보게 됨이 보장됩니다.
이뿐만이 아니라, 읽기를 수행하는 쓰레드는 다른 비싼 동기화 기능을 사용할 필요가
없어서, 이 방법은 리얼타임 쪽의 사용에 꽤 적합할 겁니다.\footnote{
	다시 말하지만, 많은 컴퓨터 시스템들에서 컴파일러와 DEC Alpha
	시스템에서라면 CPU가 간섭을 행하는 것을 막기 위한 추가 작업이
	필요합니다.
	이에 대해서는 Section~\ref{sec:defer:RCU Fundamentals} 에서
	이야기합니다.}
\iffalse

A classic approach for insertion is shown in
Figure~\ref{fig:defer:Insertion With Concurrent Readers}.
The first row shows the default state, with \co{gptr} equal to \co{NULL}.
In the second row, we have allocated a structure which is uninitialized,
as indicated by the question marks.
In the third row, we have initialized the structure.
Next, we assign \co{gptr} to reference this new element.\footnote{
	On many computer systems, simple assignment is insufficient
	due to interference from both the compiler and the CPU.
	These issues will be covered in
	Section~\ref{sec:defer:RCU Fundamentals}.}
On modern general-purpose systems, this assignment is atomic in the
sense that concurrent readers will see either a \co{NULL} pointer
or a pointer to the new structure \co{p}, but not some mash-up
containing bits from both values.
Each reader is therefore guaranteed to either get the
default value of \co{NULL} or to get the newly installed
non-default values, but either way each reader will see
a consistent result.
Even better, readers need not use any expensive synchronization
primitives, so this approach is quite suitable for real-time use.\footnote{
	Again, on many computer systems, additional work is required
	to prevent interference from the compiler, and, on DEC Alpha
	systems, the CPU as well.
	This will be covered in
	Section~\ref{sec:defer:RCU Fundamentals}.}
\fi

\begin{figure}[tb]
\begin{center}
\resizebox{3in}{!}{\includegraphics{defer/RCUListDeleteClassic}}
\end{center}
\caption{Deletion From Linked List With Concurrent Readers}
\label{fig:defer:Deletion From Linked List With Concurrent Readers}
\end{figure}

하지만 동시에 읽기 쓰레드에 의해 레퍼런스 되고 있는 데이터는 언젠가는 없어져야
할 겁니다.
Figure~\ref{fig:defer:Deletion From Linked List With Concurrent Readers} 와
같이 링크드 리스트에서 원소를 삭제하는, 더 복잡한 예제를 봅시다.
이 리스트는 초기에 원소~\co{A}, \co{B}, 그리고 \co{C} 를 가지고 있으며, 여기서
원소~\co{B} 를 삭제해야 합니다.
먼저 \co{list_del()} 을 사용해 삭제를 진행하는데,\footnote{
	역시 앞서 말했듯 이는 추상화된 예이고,
	Section~\ref{sec:defer:RCU Fundamentals} 에서 이에 관련해서 더 이야기
합니다.}
모든 새로운 읽기 쓰레드들은 원소~\co{B} 를 리스트에서 삭제된
것으로 보게 될 겁니다.
하지만, 이 원소를 여전히 보고 있는 오래된 읽기 쓰레드들도 있을 수
있습니다.
이 오래된 읽기 쓰레드들이 종료되고 나면, 원소~\co{B} 를 안전히 메모리
해제시켜서 그림의 아래쪽에 그려진 상태를 만들 수 있을 겁니다.
\iffalse

But sooner or later, it will be necessary to remove data that is
being referenced by concurrent readers.
Let us move to a more complex example where we are removing an element
from a linked list, as shown in
Figure~\ref{fig:defer:Deletion From Linked List With Concurrent Readers}.
This list initially contains elements~\co{A}, \co{B}, and \co{C},
and we need to remove element~\co{B}.
First, we use \co{list_del()} to carry out the removal,\footnote{
	And yet again, this approximates reality, which will be expanded
	on in Section~\ref{sec:defer:RCU Fundamentals}.}
at which point all new readers will see element~\co{B} as having been
deleted from the list.
However, there might be old readers still referencing this element.
Once all these old readers have finished, we can safely free
element~\co{B}, resulting in the situation shown at the bottom of
the figure.
\fi

하지만, 그 종료 시점을 어떻게 알까요?

레퍼런스 카운팅 방법의 경우는,
Chapter~\ref{chp:Counting} 의
Figure~\ref{fig:count:Atomic Increment Scalability on Nehalem}
가 락킹과 시퀀스 락킹처럼 긴 딜레이를 가져올 수 있음을 보이므로 어렵습니다.

읽기 쓰레드들이 그 존재를 알리기 위한 일을 전혀 안하는, 극단적인 상황을 생각해
봅시다.
이는 읽기 쓰레드들에 최적의 성능을 가능하게 하겠지만 (아무것도 안해도
되니까요), 업데이트 쓰레드는 어떻게 모든 예전 읽기 쓰레드들의 종료를 알지에
대한 질문이 남습니다.
여기에 합리적 답을 하기 위해서는 추가적인 제약이 분명 필요합니다.
\iffalse

But how can we tell when the readers are finished?

It is tempting to consider a reference-counting scheme, but
Figure~\ref{fig:count:Atomic Increment Scalability on Kaby Lake}
in
Chapter~\ref{chp:Counting}
shows that this can also result in long delays, just as can
the locking and sequence-locking approaches that we already rejected.

Let's consider the logical extreme where the readers do absolutely
nothing to announce their presence.
This approach clearly allows optimal performance for readers
(after all, free is a very good price),
but leaves open the question of how the updater can possibly
determine when all the old readers are done.
We clearly need some additional constraints if we are to provide
a reasonable answer to this question.
\fi

일부 운영체제 커널에 적합한 제약은 쓰레드가 CPU 를 빼앗기지 않는 상황
(non-preemptible) 을 고려하는 겁니다.
CPU 를 빼앗길 수 없는 환경에서 쓰레드는 명시적이고 자발적으로 블락킹 되기
전까지는 수행을 계속합니다.
즉, 블락킹 없이 반복되는 무한루프는 CPU 를 무한루프 이외의 목적으로는 사용될 수
없게 하다는 의미입니다.\footnote{
	반면, CPU 를 빼앗길 수 있는 환경에서의 무한루프는 여전히 CPU 시간을
	낭비하고 있긴 하지만, 이 CPU 는 다른 일을 할 수 있을 겁니다.}
CPU 를 뺏길 수 없다는 특성은 또한 쓰레드들이 스핀락을 잡고 있는 동안은 블락킹
되지 않아야 할 것을 필요로 합니다.
이런 금지사항이 없다면, 블락된 쓰레드에 의해 잡혀 있는 스핀락을 획득하려
시도하며 루프를 도는 쓰레드들에 의해 모든 CPU 가 소모되게 될수도 있습니다.
루프를 도는 쓰레드는 락을 잡기 전까지는 자신의 CPU 를 놓지 않을텐데, 락을 잡고
있는 쓰레드는 이 루프를 돌고 있는 쓰레드들이 CPU 를 놓기 전까지는 그 락을 놓을
수가 없습니다.
이는 고전적인 deadlock 상황입니다.
\iffalse
One constraint that fits well with some operating-system kernels is to
consider the case where threads are not subject to preemption.
In such non-preemptible environments, each thread runs until it
explicitly and voluntarily blocks.
This means that an infinite loop without blocking will render a CPU
useless for any other purpose from the start of the infinite loop
onwards.\footnote{
	In contrast, an infinite loop in a preemptible environment
	might be preempted.
	This infinite loop might still waste considerable CPU time,
	but the CPU in question would nevertheless be able to do
	other work.}
Non-preemptibility also requires that threads be prohibited from blocking
while holding spinlocks.
Without this prohibition, all CPUs might be consumed by threads
spinning attempting to acquire a spinlock held by a blocked thread.
The spinning threads will not relinquish their CPUs until they acquire
the lock, but the thread holding the lock cannot possibly release it
until one of the spinning threads relinquishes a CPU.
This is a classic deadlock situation.
\fi

이와 똑같은 제약사항을 링크드 리스트를 횡단하며 읽기를 하는 쓰레드들에도
가해봅시다: 그런 쓰레드들은 리스트 횡단이 완료되기 전까지는 블락되는 것이
허용되지 않습니다.
업데이트 쓰레드가 \co{list_del()} 을 실행 완료한 직후인
Figure~\ref{fig:defer:Deletion From Linked List With Concurrent Readers} 의
두번째 줄로 돌아가서, CPU~0 가 컨텍스트 스위칭을 한다고 생각해 봅시다.
읽기 쓰레드들은 링크드 리스트 횡단 중에 블락되는 것은 허용되지 않으므로, CPU~0
에서 수행되던 모든 시간상 앞의 읽기 쓰레드들은 완료되었음이 보장됩니다.
이 이야기를 다른 CPU 들에도 적용해 보자면, 각 CPU 가 일단 컨텍스트 스위칭이
수행됨을 확인했다면, 모든 시간상 앞의 읽기 쓰레드들은 완료되었고, 더이상
원소~\co{B} 를 레퍼런스 하고 있는 읽기 쓰레드는 더이상 없을 것이 보장된다고 볼
수 있습니다.
그렇다면 업데이트 쓰레드는 안전하게 원소~\co{B} 를 메모리 해제해서
Figure~\ref{fig:defer:Deletion From Linked List With Concurrent Readers} 의
가장 아래의 상태를 만들어낼 수 있습니다.
\iffalse

Let us impose this same constraint on reader threads traversing the
linked list: such threads are not allowed to block until after
completing their traversal.
Returning to the second row of
Figure~\ref{fig:defer:Deletion From Linked List With Concurrent Readers},
where the updater has just completed executing \co{list_del()},
imagine that CPU~0 executes a context switch.
Because readers are not permitted to block while traversing the linked
list, we are guaranteed that all prior readers that might have been running on
CPU~0 will have completed.
Extending this line of reasoning to the other CPUs, once each CPU has
been observed executing a context switch, we are guaranteed that all
prior readers have completed, and that there are no longer any reader
threads referencing element~\co{B}.
The updater can then safely free element~\co{B}, resulting in the
state shown at the bottom of
Figure~\ref{fig:defer:Deletion From Linked List With Concurrent Readers}.
\fi

\begin{figure}[tb]
\centering
\resizebox{3in}{!}{\includegraphics{defer/QSBRGracePeriod}}
\caption{RCU QSBR: Waiting for Pre-Existing Readers}
\label{fig:defer:RCU QSBR: Waiting for Pre-Existing Readers}
\end{figure}

이런 방법은 \emph{quiescent state based
reclamation}(QSBR)~\cite{ThomasEHart2006a} 이라 명명되어 있습니다.
QSBR 방법이
Figure~\ref{fig:defer:RCU QSBR: Waiting for Pre-Existing Readers}
에 그림의 꼭대기부터
바닥까지로 시간의 흐름에 따라 그려져 있습니다.

이런 방법의 제품 수준 구현은 매우 복잡할 수 있지만, 장난감 수준 구현은 상당히
간단합니다:
\iffalse

This approach is termed \emph{quiescent state based reclamation}
(QSBR)~\cite{ThomasEHart2006a}.
A QSBR schematic is shown in
Figure~\ref{fig:defer:RCU QSBR: Waiting for Pre-Existing Readers},
with time advancing from the top of the figure to the bottom.

Although production-quality implementations of this approach can be
quite complex, a toy implementation is exceedingly simple:
\fi

\vspace{5pt}
\begin{minipage}[t]{\columnwidth}
\scriptsize
\begin{verbatim}
  1 for_each_online_cpu(cpu)
  2   run_on(cpu);
\end{verbatim}
\end{minipage}
\vspace{5pt}

이 \co{for_each_online_cpu()} 함수는 모든 CPU 들에 루프를 돌고, \co{run_on()}
함수는 현재 쓰레드가 특정 CPU 에서 수행되게 해서 목적지 CPU 가 컨텍스트
스위칭을 수행하게 만듭니다.
따라서, 일단 한번 \co{for_each_online_cpu()} 가 완료되면, 각 CPU 는 컨텍스트
스위치를 수행한 것이고, 따라서 모든 시간상 앞의 읽기 쓰레드들은 완료되었음이
보장됩니다.
\iffalse

The \co{for_each_online_cpu()} primitive iterates over all CPUs, and
the \co{run_on()} function causes the current thread to execute on the
specified CPU, which forces the destination CPU to execute a context
switch.
Therefore, once the \co{for_each_online_cpu()} has completed, each CPU
has executed a context switch, which in turn guarantees that
all pre-existing reader threads have completed.
\fi

이 방법은 제품 수준의 품질이 \emph{아님} 을 알아 두시기 바랍니다.
일반적이지 않은 경우에서의 정확성 처리와 강력한 최적화 여럿을 필요로 하는 제품
수준 품질의 구현의 성질은 상당한 추가적 복잡도를 의미합니다.
또한, CPU 강탈이 가능한 환경에서의 RCU 구현은 읽기 쓰레드들이 실제로 뭔가를
할것을 필요로 합니다.
하지만, 이 간단한 CPU 강탈이 불가한 상황의 접근법은 개념적으로는 완벽하고, 다음
섹션에서 다루어질 RCU 의 기본을 이해하기 위한 좋은 초기 토대가 될겁니다.
\iffalse

Please note that this approach is \emph{not} production quality.
Correct handling of a number of corner cases and the need for a number
of powerful optimizations mean that production-quality implementations
have significant additional complexity.
In addition, RCU implementations for preemptible environments
require that readers actually do something.
However, this simple non-preemptible approach is conceptually complete,
and forms a good initial basis for understanding the RCU fundamentals
covered in the following section.
\fi

% defer/rcufundamental.tex

\subsection{RCU Fundamentals}
\label{sec:defer:RCU Fundamentals}
\OriginallyPublished{Section}{sec:defer:RCU Fundamentals}{RCU Fundamentals}{Linux Weekly News}{PaulEMcKenney2007WhatIsRCUFundamentally}

Authors: Paul E. McKenney and Jonathan Walpole

Read-copy update (RCU) 는 2002년 10월에 리눅스 커널에 추가된, 하나의 동기화
메커니즘입니다.
RCU 는 읽기 작업들이 업데이트 작업들과 동시에 일어날 수 있도록 함으로써 확장성
개선을 달성합니다.
기존에 일반적으로 사용되어온, 동시에 수행되는 쓰레드들에 대해 그것들이 읽기를
하는지 업데이트를 하는지와 상관없이 상호 배제를 보장하는 락킹 기능들 또는
동시의 읽기 작업은 허용하지만 업데이트가 함께 수행되는 것은 막는 reader-writer
락들과는 대조적으로 RCU 는 하나의 업데이트 쓰레드와 여러 읽기 쓰레드들 사이의
동시성을 지원합니다.
RCU 는 오브젝트들의 여러 버전들을 유지하고 그것들이 이전부터 존재해온 모든
읽기쪽 크리티컬 섹션들이 완료되기 전까지는 메모리에서 해제하지 않음으로써 읽기
작업들이 일관적임을 보장합니다.
RCU 는 오브젝트의 새 버전을 공개하고 읽는데, 그리고 예전 버전들의 정리 작업을
뒤로 미루어 한번에 처리하는데에 효과적이고 확장성 있는 메커니즘을 정의하고
사용합니다.
이런 메커니즘들은 작업을 읽기와 업데이트쪽 수행경로로 분산시키되 읽기 쪽
수행경로가 극단적으로 빠르게 하는데에 해저드 포인터와 유사한 복사와 규칙 완화를
골자로 하는 최적화 기술을 사용하지만, 읽기 쪽의 재시도는 필요 없게 합니다.
일부 경우에는 (CPU 를 뺏기지 않는 커널들), RCU 의 읽기 쪽 기능들은 아예
오버헤드가 없습니다.
\iffalse

Read-copy update (RCU) is a synchronization mechanism that was added to
the Linux kernel in October of 2002.
RCU achieves scalability
improvements by allowing reads to occur concurrently with updates.
In contrast with conventional locking primitives that ensure mutual exclusion
among concurrent threads regardless of whether they be readers or
updaters, or with reader-writer locks that allow concurrent reads but not in
the presence of updates, RCU supports concurrency between a single
updater and multiple readers.
RCU ensures that reads are coherent by
maintaining multiple versions of objects and ensuring that they are not
freed up until all pre-existing read-side critical sections complete.
RCU defines and uses efficient and scalable mechanisms for publishing
and reading new versions of an object, and also for deferring the collection
of old versions.
These mechanisms distribute the work among read and
update paths in such a way as to make read paths extremely fast, using
replication and weakening optimizations in a manner similar to
hazard pointers, but without the need for read-side retries.
In some cases (non-preemptible kernels), RCU's read-side primitives have
zero overhead.
\fi

\QuickQuiz{}
	하지만 Section~\ref{sec:defer:Sequence Locks} 의 시퀀스 락 역시 읽기
	쓰레드들과 업데이트 쓰레드들이 동시에 일을 할 수 있도록 하지 않던가요?
	\iffalse

	But doesn't Section~\ref{sec:defer:Sequence Locks}'s seqlock
	also permit readers and updaters to get work done concurrently?
	\fi
\QuickQuizAnswer{
	맞는 말이고 아닌 말이기도 합니다.
	시퀀스 락의 읽기 쓰레드들은 쓰기 쓰레드들과 동시에 수행될 수 있지만,
	이런 상황이 발생할 때마다, {\tt read\_seqretry()} 기능이 읽기 쓰레드는
	다시 수행을 시도하도록 강제할 겁니다.
	이는 시퀀스 락을 사용하명 업데이트 쓰레드와 동시에 수행되는 읽기
	쓰레드가 하는 일은 모두 취소되고 다시 수행될 것을 의미합니다.
	따라서 시퀀스 락을 사용하는 읽기 작업은 업데이트 작업과 동시에
	\emph{수행} 될 수 있지만, 이런 경우에 실제로는 어떤 일도 만들어내지
	못합니다.

	반면에, RCU 읽기 쓰레드들은 동시에 RCU 업데이트 쓰레드들이 존재하더라도
	의미있는 일을 처리해낼 수 있습니다.
	\iffalse

	Yes and no.
	Although seqlock readers can run concurrently with
	seqlock writers, whenever this happens, the {\tt read\_seqretry()}
	primitive will force the reader to retry.
	This means that any work done by a seqlock reader running concurrently
	with a seqlock updater will be discarded and redone.
	So seqlock readers can \emph{run} concurrently with updaters,
	but they cannot actually get any work done in this case.

	In contrast, RCU readers can perform useful work even in presence
	of concurrent RCU updaters.
	\fi
} \QuickQuizEnd

이는 ``RCU 는 정확히 무엇인가?'' 하는 질문과, 아마도 ``RCU 는 어떻게 동작할 수
\emph{있는가}?'' 하는 질문을 이끌어낼 수 있을 겁니다 (또는, 드물지 않게, RCU 는
동작할 수 없을 것이라는 단정을).
이 문서는 이런 질문들을 기본적 관점에서부터 다룹니다; 뒤의 일부는 RCU 를
사용법과 API 관점에서 살펴봅니다.
마지막 부분은 또한 참고할 문서 목록을 포함합니다.

RCU 는 세개의 기본적 메커니즘으로 만들어지는데, 첫번째는 항목 삽입에 사용되고,
두번째 것은 항목 삭제에, 그리고 세번째 것은 읽기 쓰레드들이 동시의 항목 추가와
삭제에 문제 없이 동작하도록 하는데 사용됩니다.
Section~\ref{sec:defer:Publish-Subscribe Mechanism}
은 항목 추가를 위한 publish-subscribe 메커니즘을 설명하고,
Section~\ref{sec:defer:Wait For Pre-Existing RCU Readers to Complete}
에서는 먼저 시작된 RCU 읽기 쓰레드들을 어떻게 기다려서 항목 삭제가 가능하게
하는지 설명하며,
Section~\ref{sec:defer:Maintain Multiple Versions of Recently Updated Objects}
에서는 최근에 업데이트된 오브젝트들의 여러 버전들을 어떻게 관리해서 동시의 항목
추가와 삭제를 가능하게 하는지 설명합니다.
마지막으로,
Section~\ref{sec:defer:Summary of RCU Fundamentals}
에서는 RCU 기본사항을 요약합니다.
\iffalse

This leads to the question ``what exactly is RCU?'', and perhaps also
to the question ``how can RCU \emph{possibly} work?'' (or, not
infrequently, the assertion that RCU cannot possibly work).
This document addresses these questions from a fundamental viewpoint;
later installments look at them from usage and from API viewpoints.
This last installment also includes a list of references.

RCU is made up of three fundamental mechanisms, the first being
used for insertion, the second being used for deletion, and the third
being used to allow readers to tolerate concurrent insertions and deletions.
Section~\ref{sec:defer:Publish-Subscribe Mechanism}
describes the publish-subscribe mechanism used for insertion,
Section~\ref{sec:defer:Wait For Pre-Existing RCU Readers to Complete}
describes how waiting for pre-existing RCU readers enabled deletion,
and
Section~\ref{sec:defer:Maintain Multiple Versions of Recently Updated Objects}
discusses how maintaining multiple versions of recently updated objects
permits concurrent insertions and deletions.
Finally,
Section~\ref{sec:defer:Summary of RCU Fundamentals}
summarizes RCU fundamentals.
\fi

\subsubsection{Publish-Subscribe Mechanism}
\label{sec:defer:Publish-Subscribe Mechanism}

\begin{figure}[tbp]
{ \scriptsize
\begin{verbatim}
  1 struct foo {
  2   int a;
  3   int b;
  4   int c;
  5 };
  6 struct foo *gp = NULL;
  7
  8 /* . . . */
  9
 10 p = kmalloc(sizeof(*p), GFP_KERNEL);
 11 p->a = 1;
 12 p->b = 2;
 13 p->c = 3;
 14 gp = p;
\end{verbatim}
}
\caption{Data Structure Publication (Unsafe)}
\label{fig:defer:Data Structure Publication (Unsafe)}
\end{figure}

RCU 의 핵심 요소 중 하나는 데이터가 동시에 수정되고 있는데도 불구하고 안전하게
그 데이터를 읽을 수 있는 능력입니다.
동시의 항목 삽입에 이런 능력을 제공하기 위해, RCU 는 공개-구독
(publish-subscribe) 메커니즘이라 생각될 수 있는 방법을 상요합니다.
예를 들어, 초기에 \co{NULL} 인 전역 포인터 \co{gp} 가 새로 할당되고 초기화된
데이터 구조체로의 포인터로 수정되려 한다고 생각해 봅시다.
Figure~\ref{fig:defer:Data Structure Publication (Unsafe)} 에 보이는 코드 조각
(추가로 적절한 락킹을 포함해서) 이 이 목적으로 사용될 수 있을 것입니다.
\iffalse

One key attribute of RCU is the ability to safely scan data, even
though that data is being modified concurrently.
To provide this ability for concurrent insertion,
RCU uses what can be thought of as a publish-subscribe mechanism.
For example, consider an initially \co{NULL} global pointer
\co{gp} that is to be modified to point to a newly allocated
and initialized data structure.
The code fragment shown in
Figure~\ref{fig:defer:Data Structure Publication (Unsafe)}
(with the addition of appropriate locking)
might be used for this purpose.
\fi

안타깝게도, 컴파일러와 CPU 가 마지막 네개의 할당문이 순서대로 수행하도록
강제하는 것이 전혀 없습니다.
\co{gp} 로의 할당이 \co{p} 필드들의 초기화 전에 일어난다면, 동시 수행중인 읽기
작업들은 이 초기화되지 않은 값들을 볼 수 있을 겁니다.
이것들이 순서를 지키도록 하기 위해 메모리 배리어들이 필요합니다만, 메모리
배리어들은 사용하기가 어렵기로 악명높습니다.
따라서 그것들을 공개 의미를 갖는 \co{rcu_assign_pointer()} 기능에 집어넣습니다.
그렇게 되면 마지막 네줄은 다음과 같이 될겁니다:
\iffalse

Unfortunately, there is nothing forcing the compiler and CPU to execute
the last four assignment statements in order.
If the assignment to \co{gp} happens before the initialization
of \co{p} fields, then concurrent readers could see the
uninitialized values.
Memory barriers are required to keep things ordered, but memory barriers
are notoriously difficult to use.
We therefore encapsulate them into a primitive
\co{rcu_assign_pointer()} that has publication semantics.
The last four lines would then be as follows:
\fi

\vspace{5pt}
\begin{minipage}[t]{\columnwidth}
\scriptsize
\begin{verbatim}
  1 p->a = 1;
  2 p->b = 2;
  3 p->c = 3;
  4 rcu_assign_pointer(gp, p);
\end{verbatim}
\end{minipage}
\vspace{5pt}

이 \co{rcu_assign_pointer()} 는 새 구조체를 \emph{공개}하고, 컴파일러와 CPU 가
\co{gp} 로의 할당이 \co{p} 로 참조되는 필드들의 할당 \emph{후에} 수행하도록
강제할 겁니다.

하지만, 업데이트 작업에 순서를 맞추는 것만으로는 충분치 않은데, 읽기 작업도
적절하게 순서가 맞춰져야 하기 때문입니다.
다음과 같은 코드 조각의 예를 생각해 봅시다:
\iffalse

The \co{rcu_assign_pointer()}
would \emph{publish} the new structure, forcing both the compiler
and the CPU to execute the assignment to \co{gp} \emph{after}
the assignments to the fields referenced by \co{p}.

However, it is not sufficient to only enforce ordering at the
updater, as the reader must enforce proper ordering as well.
Consider for example the following code fragment:
\fi

\vspace{5pt}
\begin{minipage}[t]{\columnwidth}
\scriptsize
\begin{verbatim}
  1 p = gp;
  2 if (p != NULL) {
  3   do_something_with(p->a, p->b, p->c);
  4 }
\end{verbatim}
\end{minipage}
\vspace{5pt}

이 코드 조각은 잘못된 순서에 문제가 없을 것처럼 보이지만, 안타깝게도 DEC Alpha
CPU~\cite{PaulMcKenney2005i,PaulMcKenney2005j} 와 값을 예측하는 컴파일러
최적화는, 믿든 안믿든, \co{p->a}, \co{p->b}, 그리고 \co{p->c} 가 \co{p} 의 값
전에 메모리로부터 가져와질 수 있게 할 수 있습니다.
이런 현상은 컴파일러가 \co{p} 의 값을 추측하고 \co{p->a}, \co{p->b}, 그리고
\co{p->c} 값을 가져온 후에 그 추측이 맞았는지 보기 위해 \co{p} 의 실제 값을
가져오는, 컴파일러의 값 추측 최적화의 경우에서 보기가 가장 쉬울 겁니다.
이런 종류의 최적화는 상당히 공격적이고 미친 행위 같지만, 프로파일 기반의
최적화의 문맥에서는 실제로 일어나는 일입니다.
\iffalse

Although this code fragment might well seem immune to misordering,
unfortunately, the
DEC Alpha CPU~\cite{PaulMcKenney2005i,PaulMcKenney2005j}
and value-speculation compiler optimizations can, believe it or not,
cause the values of \co{p->a}, \co{p->b}, and
\co{p->c} to be fetched before the value of \co{p}.
This is perhaps easiest to see in the case of value-speculation
compiler optimizations, where the compiler guesses the value
of \co{p} fetches \co{p->a}, \co{p->b}, and
\co{p->c} then fetches the actual value of \co{p}
in order to check whether its guess was correct.
This sort of optimization is quite aggressive, perhaps insanely so,
but does actually occur in the context of profile-driven optimization.
\fi

분명히, 우리는 컴파일러와 CPU 로부터 이런 종류의 야바위질을 막아야 합니다.
\co{rcu_dereference()} 기능은 이런 목적을 위해 필요한 어떤 메모리 배리어
인스트럭션과 컴파일러 지시어들을 사용합니다:\footnote{
	리눅스 커널에서, \co{rcu_dereference()} 는 volatile 캐스팅으로
	구현되고, DEC Alpha 에서는 메모리 배리어 인스트럭션으로 구현됩니다.
	C11 과 C++11 표준에서는 \co{memory_order_consume} 이
	\co{rcu_dereference()} 지원을 제공하기 위한 의도로 만들어졌습니다만,
	이를 native로 구현한 컴파일러는 아직 없습니다.
	(컴파일러들은 대신 \co{memory_order_consume} 을
	\co{memory_order_acuire} 로 강화시켜서, 약한 순서 규칙의 시스템에서는
	필요없는 메모리 배리어 인스트럭션을 만듭니다.)}
\iffalse

Clearly, we need to prevent this sort of skullduggery on the
part of both the compiler and the CPU.
The \co{rcu_dereference()} primitive uses
whatever memory-barrier instructions and compiler
directives are required for this purpose:\footnote{
	In the Linux kernel, \co{rcu_dereference()} is implemented via
	a volatile cast, and, on DEC Alpha, a memory barrier instruction.
	In the C11 and C++11 standards, \co{memory_order_consume}
	is intended to provide longer-term support for \co{rcu_dereference()},
	but no compilers implement this natively yet.
	(They instead strengthen \co{memory_order_consume} to
	\co{memory_order_acquire}, thus emitting a needless memory-barrier
	instruction on weakly ordered systems.)}
\fi

\vspace{5pt}
\begin{minipage}[t]{\columnwidth}
\scriptsize
\begin{verbatim}
  1 rcu_read_lock();
  2 p = rcu_dereference(gp);
  3 if (p != NULL) {
  4   do_something_with(p->a, p->b, p->c);
  5 }
  6 rcu_read_unlock();
\end{verbatim}
\end{minipage}
\vspace{5pt}

따라서 \co{rcu_dereference()} 함수는 특정 포인터로 주어지는 값에 대한
\emph{구독} 으로, 뒤따르는 dereference 오퍼레이션들은 해당 포인터를 공개한,
연관된 \co{rcu_assign_pointer()} 오퍼레이션 전에 발생한 초기화 작업의 결과들을
보게 될 것이 보장된다고 이해될 수 있습니다.
\co{rcu_read_lock()} 과 \co{rcu_read_unlock()} 함수 호출들은 반드시 필요합니다:
이것들은 RCU read-side 크리티컬 섹션을 정의합니다.
이것들의 목적인
Section~\ref{sec:defer:Wait For Pre-Existing RCU Readers to Complete} 에서
설명됩니다만, 이것들은 결코 스피닝하거나 블락킹 하지 않고, \co{list_add_rcu()}
가 동시에 수행되는 것을 막지도 않습니다.
사실, \co{CONFIG_PREEMPT} 옵션이 켜져있지 않은 커널에서 이것들은 아무 코드도
생성하지 않습니다.
\iffalse

The \co{rcu_dereference()} primitive can thus be thought of
as \emph{subscribing} to a given value of the specified pointer,
guaranteeing that subsequent dereference operations will see any
initialization that occurred before the corresponding
\co{rcu_assign_pointer()} operation that published that pointer.
The \co{rcu_read_lock()} and \co{rcu_read_unlock()}
calls are absolutely required: they define the extent of the
RCU read-side critical section.
Their purpose is explained in
Section~\ref{sec:defer:Wait For Pre-Existing RCU Readers to Complete},
however, they never spin or block, nor do they prevent the
\co{list_add_rcu()} from executing concurrently.
In fact, in non-\co{CONFIG_PREEMPT} kernels, they generate
absolutely no code.
\fi

\begin{figure}[tb]
\begin{center}
\resizebox{3in}{!}{\includegraphics{defer/Linux_list}}
\end{center}
\caption{Linux Circular Linked List}
\label{fig:defer:Linux Circular Linked List}
\end{figure}

\begin{figure}[tb]
\begin{center}
\resizebox{3in}{!}{\includegraphics{defer/Linux_list_abbr}}
\end{center}
\caption{Linux Linked List Abbreviated}
\label{fig:defer:Linux Linked List Abbreviated}
\end{figure}

이론상으로는 \co{rcu_assign_pointer()} 와 \co{rcu_dereference()} 는 상상할 수
있는 RCU 로 보호되는 데이터 구조는 얼마든지 만들 수 있지만, 실제로는
고차원의 방법을 사용하는게 나은 경우가 많이 있습니다.
그런 이유로, \co{rcu_assign_pointer()} 와 \co{rcu_dereference()} 함수들이
리눅스에 있는 리스트 조정 API 의 특별한 RCU 사용 버전에 내장되어 있습니다.
리눅스는 이중 링크드 리스트의 두가지 버전을 가지고 있는데, 순환 형태의 {\tt
struct list\_head} 와 선형의 {\tt struct hlist\_head}/{\tt struct hlist\_node}
쌍입니다.
앞의 버전은
Figure~\ref{fig:defer:Linux Circular Linked List} 에 그려져 있는데, (왼쪽의)
초록 상자는 리스트 헤더를 나타내고 (오른쪽의 세개의) 파란 박스들은 리스트의
원소들을 의미합니다.
이 방법은 다루기가 힘들기 때문에
Figure~\ref{fig:defer:Linux Linked List Abbreviated} 에 보인 것처럼 헤더 없이
(파란) 원소들만을 보이는 형태로 간략화해서 나타낼 수 있습니다.
\iffalse

Although \co{rcu_assign_pointer()} and
\co{rcu_dereference()} can in theory be used to construct any
conceivable RCU-protected data structure, in practice it is often better
to use higher-level constructs.
Therefore, the \co{rcu_assign_pointer()} and
\co{rcu_dereference()}
primitives have been embedded in special RCU variants of Linux's
list-manipulation API.
Linux has two variants of doubly linked list, the circular
{\tt struct list\_head} and the linear
{\tt struct hlist\_head}/{\tt struct hlist\_node} pair.
The former is laid out as shown in
Figure~\ref{fig:defer:Linux Circular Linked List},
where the green (leftmost) boxes represent the list header and the blue
(rightmost three) boxes represent the elements in the list.
This notation is cumbersome, and will therefore be abbreviated as shown in
Figure~\ref{fig:defer:Linux Linked List Abbreviated},
which shows only the non-header (blue) elements.
\fi

\begin{figure}[tbp]
{ \scriptsize
\begin{verbatim}
  1 struct foo {
  2   struct list_head *list;
  3   int a;
  4   int b;
  5   int c;
  6 };
  7 LIST_HEAD(head);
  8
  9 /* . . . */
 10
 11 p = kmalloc(sizeof(*p), GFP_KERNEL);
 12 p->a = 1;
 13 p->b = 2;
 14 p->c = 3;
 15 list_add_rcu(&p->list, &head);
\end{verbatim}
}
\caption{RCU Data Structure Publication}
\label{fig:defer:RCU Data Structure Publication}
\end{figure}

이 링크드 리스트에 포인터 공개 예제의 기법을 적용하는 것은
Figure~\ref{fig:defer:RCU Data Structure Publication} 에 보여진 코드와 같은
형태로 귀결될 겁니다.

Line~15 는 여러개의 \co{list_add_rcu()} 가 동시에 수행되는 것을 막기 위해 어떤
다른 동기화 메커니즘 (가장 흔하게는 어떤 종류의 락) 을 사용해야만 합니다.
하지만, 그런 동기화는 이 \co{list_add()} 의 수행을 RCU 읽기 작업들과 동시에
수행되는 것을 못하게 하지는 않습니다.

RCU 로 보호되는 리스트를 구독하는 행위는 간단합니다:
\iffalse

Adapting the pointer-publish example for the linked list results in
the code shown in
Figure~\ref{fig:defer:RCU Data Structure Publication}.

Line~15 must be protected by some synchronization mechanism (most
commonly some sort of lock) to prevent multiple \co{list_add_rcu()}
instances from executing concurrently.
However, such synchronization does not prevent this \co{list_add()}
instance from executing concurrently with RCU readers.

Subscribing to an RCU-protected list is straightforward:
\fi

\vspace{5pt}
\begin{minipage}[t]{\columnwidth}
\scriptsize
\begin{verbatim}
  1 rcu_read_lock();
  2 list_for_each_entry_rcu(p, head, list) {
  3   do_something_with(p->a, p->b, p->c);
  4 }
  5 rcu_read_unlock();
\end{verbatim}
\end{minipage}
\vspace{5pt}

앞의 \co{list_add_rcu()} 함수는 원소를 공개하고, 특정 리스트의 헤드에 집어넣고,
이에 연관된 \co{list_for_each_entry_rcu()} 호출이 정상적으로 같은 원소를
구독하게 될것을 보장합니다.
\iffalse

The \co{list_add_rcu()} primitive publishes an entry, inserting it at
the head of the specified list, guaranteeing that the corresponding
\co{list_for_each_entry_rcu()} invocation will properly subscribe to
this same entry.
\fi

\QuickQuiz{}
	{\tt list\_add\_rcu()} 와 정확히 똑같은 시간에
	{\tt list\_for\_each\_entry\_rcu()} 이 수행되면 segfault 가 날 수 있을
	것 같은데, 이걸 무엇이 방지해 주나요?
	\iffalse

	What prevents the {\tt list\_for\_each\_entry\_rcu()} from
	getting a segfault if it happens to execute at exactly the same
	time as the {\tt list\_add\_rcu()}?
	\fi
\QuickQuizAnswer{
	리눅스가 돌아가는 모든 시스템에서 포인터로의 로드와 스토어는 모두
	어토믹한데, 즉, 포인터로의 스토어가 같은 포인터로부터의 로드와 같은
	시점에 일어난다면, 이 로드는 초기값이나 저장된 값을 가져오지, 그 두
	값이 섞여진 값을 가져오는 일은 결코 없습니다.
	또한, {\tt list\_for\_each\_entry\_rcu()} 는 항상 리스트를 앞으로만
	움직이지, 결코 뒤로 움직이지는 않습니다.
	따라서, {\tt list\_for\_each\_entry\_rcu()} 는 {\tt list\_add\_rc()} 를
	통해 들어온 원소를 보거나 보지 않을 뿐이지만 어떤 경우든 적합하게 잘
	형성된 리스트를 볼 겁니다.
	\iffalse

	On all systems running Linux, loads from and stores
	to pointers are atomic, that is, if a store to a pointer occurs at
	the same time as a load from that same pointer, the load will return
	either the initial value or the value stored, never some bitwise
	mashup of the two.
	In addition, the {\tt list\_for\_each\_entry\_rcu()} always proceeds
	forward through the list, never looking back.
	Therefore, the {\tt list\_for\_each\_entry\_rcu()} will either see
	the element being added by {\tt list\_add\_rcu()} or it will not,
	but either way, it will see a valid well-formed list.
	\fi
} \QuickQuizEnd

\begin{figure}[tb]
\begin{center}
\resizebox{3in}{!}{\includegraphics{defer/Linux_hlist}}
\end{center}
\caption{Linux Linear Linked List}
\label{fig:defer:Linux Linear Linked List}
\end{figure}

리눅스의 다른 이중 링크드 리스트인 hlist 는 선형 리스트인데, 이는 헤더로의
포인터만이 필요하지
Figure~\ref{fig:defer:Linux Linear Linked List} 에 보여진 순환 형태의
리스트처럼 두개의 포인터가 필요하진 않습니다.
따라서, hlist 의 사용은 해시 버킷 배열들이나 커다란 해시 테이블에서는 메모리
사용량을 반으로 줄일 수 있습니다.
앞에서와 같이, 이 형태는 다루기가 까다로우므로, hlist 들은
Figure~\ref{fig:defer:Linux Linked List Abbreviated} 에 보인 것과 같은 형태로
간략화 될 겁니다.
\iffalse

Linux's other doubly linked list, the hlist,
is a linear list, which means that
it needs only one pointer for the header rather than the two
required for the circular list, as shown in
Figure~\ref{fig:defer:Linux Linear Linked List}.
Thus, use of hlist can halve the memory consumption for the hash-bucket
arrays of large hash tables.
As before, this notation is cumbersome, so hlists will be abbreviated
in the same way lists are, as shown in
Figure~\ref{fig:defer:Linux Linked List Abbreviated}.
\fi

\begin{figure}[tbp]
{ \scriptsize
\begin{verbatim}
  1 struct foo {
  2   struct hlist_node *list;
  3   int a;
  4   int b;
  5   int c;
  6 };
  7 HLIST_HEAD(head);
  8
  9 /* . . . */
 10
 11 p = kmalloc(sizeof(*p), GFP_KERNEL);
 12 p->a = 1;
 13 p->b = 2;
 14 p->c = 3;
 15 hlist_add_head_rcu(&p->list, &head);
\end{verbatim}
}
\caption{RCU {\tt hlist} Publication}
\label{fig:defer:RCU hlist Publication}
\end{figure}

새로운 원소를 RCU 로 보호되는 hlist 에 공개하는 건
as shown in Figure~\ref{fig:defer:RCU hlist Publication} 에 보여진 것처럼,
순환형 리스트에서 했던 것과 상당히 유사합니다.

앞에서와 같이, line~15 는 예를 들면 락과 같은, 어떤 종류의 동기화 메커니즘으로
보호되어야만 합니다.

RCU 로 보호되는 hlist 를 구독하는 행위 역시 순환형 리스트에서와 비슷합니다:
\iffalse

Publishing a new element to an RCU-protected hlist is quite similar
to doing so for the circular list,
as shown in Figure~\ref{fig:defer:RCU hlist Publication}.

As before, line~15 must be protected by some sort of synchronization
mechanism, for example, a lock.

Subscribing to an RCU-protected hlist is also similar to the
circular list:
\fi

\vspace{5pt}
\begin{minipage}[t]{\columnwidth}
\scriptsize
\begin{verbatim}
  1 rcu_read_lock();
  2 hlist_for_each_entry_rcu(p, q, head, list) {
  3   do_something_with(p->a, p->b, p->c);
  4 }
  5 rcu_read_unlock();
\end{verbatim}
\end{minipage}
\vspace{5pt}

\QuickQuiz{}
	{\tt list\_for\_each\_entry\_rcu()} 에서는 한개면 되었던 포인터가 왜
	{\tt hlist\_for\_each\_entry\_rcu()} 에서는 두개나 넘겨줘야 하는거죠?
	\iffalse

	Why do we need to pass two pointers into
	{\tt hlist\_for\_each\_entry\_rcu()}
	when only one is needed for {\tt list\_for\_each\_entry\_rcu()}?
	\fi
\QuickQuizAnswer{
	hlist 에서는 헤드를 만나는 대신에 NULL 체크를 할 필요가 있기
	때문입니다.
	(하나의 포인터만 사용하는 {\tt hlist\_for\_each\_entry\_rcu()} 를
	한번 구현해 보세요.
	만약 좋은 해결책을 찾는다면, 정말 좋은 일일 겁니다!)
	\iffalse

	Because in an hlist it is necessary to check for
	NULL rather than for encountering the head.
	(Try coding up a single-pointer {\tt hlist\_for\_each\_entry\_rcu()}.
	If you come up with a nice solution, it would be a very good thing!)
	\fi
} \QuickQuizEnd

\begin{table*}[tb]
\begin{center}
\scriptsize
\begin{tabular}{l||l|l|l}
Category  & Publish	& Retract	& Subscribe \\
\hline
\hline
Pointers  & \co{rcu_assign_pointer()}
			& \co{rcu_assign_pointer(..., NULL)}~~
					& \co{rcu_dereference()} \\
\hline
Lists     & \parbox{1.5in}{
		\co{list_add_rcu()} \\
		\co{list_add_tail_rcu()} \\
		\co{list_replace_rcu()} }
			& \co{list_del_rcu()}
					& \co{list_for_each_entry_rcu()}~~~ \\
\hline
Hlists    & \parbox{1.5in}{
		\co{hlist_add_after_rcu()} \\
		\co{hlist_add_before_rcu()}  \\
		\co{hlist_add_head_rcu()} \\
		\co{hlist_replace_rcu()} }
			& \co{hlist_del_rcu()}
					& \co{hlist_for_each_entry_rcu()}~~~~~
\end{tabular}
\end{center}
\caption{RCU Publish and Subscribe Primitives}
\label{tab:defer:RCU Publish and Subscribe Primitives}
\end{table*}

RCU 공개와 구독 기능들의 집합이
Table~\ref{tab:defer:RCU Publish and Subscribe Primitives} 에 ``구독취소'' 또는
철회를 위한 추가적인 기능들과 함께 표시 되어 있습니다.

\co{list_replace_rcu()}, \co{list_del_rcu()},
\co{hlist_replace_rcu()}, and \co{hlist_del_rcu()}
API 들이 복잡도를 더함을 알아두세요.
교체되거나 삭제된 데이터 원소를 메모리에서 해제하는데 안전한 시점은 언제일까요?
자세히 들어가서, 모든 읽기 작업들이 특정 데이터 원소로의 레퍼런스들을
해제한 시점을 어떻게 하면 알 수 있을까요?

이 질문들은 다음의 섹션에서 다루어집니다.
\iffalse

The set of RCU publish and subscribe primitives are shown in
Table~\ref{tab:defer:RCU Publish and Subscribe Primitives},
along with additional primitives to ``unpublish'', or retract.

Note that the \co{list_replace_rcu()}, \co{list_del_rcu()},
\co{hlist_replace_rcu()}, and \co{hlist_del_rcu()}
APIs add a complication.
When is it safe to free up the data element that was replaced or
removed?
In particular, how can we possibly know when all the readers
have released their references to that data element?

These questions are addressed in the following section.
\fi

\subsubsection{Wait For Pre-Existing RCU Readers to Complete}
\label{sec:defer:Wait For Pre-Existing RCU Readers to Complete}

가장 기본적인 형태에서, RCU 는 일들이 끝나기를 기다리는 방법입니다.
물론, RCU 외에도 일들이 끝나길 기다리는 훌륭한 방법들이 여럿 있는데, 레퍼런스
카운트, reader-writer lock, 이벤트 등등이 포함됩니다.
RCU 의 커다란 장점은 각각의 (대략) 20,000 개의 서로 다른 일들을 명시적으로 그
모든 것들을 각각 정보를 쫓아가지 않고, 성능 하락, 확장성 제한, 복잡한 데드락
시나리오, 그리고 명시적으로 정보 쫓는 방법에서 필연적인 메모리 누수 문제들을
걱정할 필요 없이 기다릴 수 있다는 겁니다.
\iffalse

In its most basic form, RCU is a way of waiting for things to finish.
Of course, there are a great many other ways of waiting for things to
finish, including reference counts, reader-writer locks, events, and so on.
The great advantage of RCU is that it can wait for each of
(say) 20,000 different things without having to explicitly
track each and every one of them, and without having to worry about
the performance degradation, scalability limitations, complex deadlock
scenarios, and memory-leak hazards that are inherent in schemes
using explicit tracking.
\fi

RCU 의 경우에, 기다려지고 있는 일들은 ``RCU read-side 크리티컬 섹션'' 이라
불립니다.
하나의 RCU read-side 크리티컬 섹션은 \co{rcu_read_lock()} 함수로 시작되고, 그에
연관되는 \co{rcu_read_unlock()} 함수로 종료됩니다.
RCU read-side 크리티컬 섹션들은 중첩될 수 있고, 어떤 코드든 그 코드가
명시적으로 블락하거나 잠들지 않는 한 (SRCU~\cite{PaulEMcKenney2006c} 라고
불리는 특별한 형태의 RCU 가 SRCU read-side 크리티컬 섹션 내에서의 일반적인
잠들기를 가능하게 하긴 하지만), 그 안에 얼마든지 들어갈 수 있습니다.
이런 규칙에 동의한다면, 코드에서 원하는 부분이라면 \emph{어떤 부분이든}
완료되기를 기다리는데에 RCU 를 사용할 수 있습니다.
\iffalse

In RCU's case, the things waited on are called
``RCU read-side critical sections''.
An RCU read-side critical section starts with an
\co{rcu_read_lock()} primitive, and ends with a corresponding
\co{rcu_read_unlock()} primitive.
RCU read-side critical sections can be nested, and may contain pretty
much any code, as long as that code does not explicitly block or sleep
(although a special form of RCU called SRCU~\cite{PaulEMcKenney2006c}
does permit general sleeping in SRCU read-side critical sections).
If you abide by these conventions, you can use RCU to wait for \emph{any}
desired piece of code to complete.
\fi

RCU 는 언제 이런 기다리는 중인 일들이 종료되었는지를 간접적으로 판단해내는
것으로 이 기능을 구현합니다~\cite{PaulEMcKenney2007whatisRCU,
PaulEMcKenney2007PreemptibleRCU}.
\iffalse

RCU accomplishes this feat by indirectly determining when these
other things have finished~\cite{PaulEMcKenney2007whatisRCU,
PaulEMcKenney2007PreemptibleRCU}.
\fi

\begin{figure}[tb]
\begin{center}
\resizebox{3in}{!}{\includegraphics{defer/GracePeriodGood}}
\end{center}
\caption{Readers and RCU Grace Period}
\label{fig:defer:Readers and RCU Grace Period}
\end{figure}

자세히 말하자면,
Figure~\ref{fig:defer:Readers and RCU Grace Period} 에 보여진 것처럼, RCU 는
전부터 존재했던 RCU read-side 크리티컬 섹션들이 그 크리티컬 센션들에서 수행되는
메모리 오퍼레이션 등이 완전히 끝나기를 기다리는 한가지 방법입니다.
하지만, 주어진 grace period 의 시작 후에 시작된 RCU read-side 크리티컬 섹션들은
그 grace period 의 종료 이후까지 수행될 수도 있음을 기억해 두십시오.

다음의 슈도코드는 RCU 를 이용해 읽기 작업들을 기다리는 알고리즘들의 기본적
형태를 보입니다:
\iffalse

In particular, as shown in
Figure~\ref{fig:defer:Readers and RCU Grace Period},
RCU is a way of
waiting for pre-existing RCU read-side critical sections to completely
finish, including memory operations executed by those critical sections.
However, note that RCU read-side critical sections
that begin after the beginning
of a given grace period can and will extend beyond the end of that grace
period.

The following pseudocode shows the basic form of algorithms that use
RCU to wait for readers:
\fi

\begin{enumerate}
\item	링크드 리스트에서 한 원소를 바꿔치기 하거나 하는 식으로 변경을
	만듭니다.
\item	전부터 존재했던 RCU read-side 크리티컬 섹션들이 완전히 종료되길
	기다립니다 (예를 들어, \co{synchronize_rcu()} 기능을 이용해서).
	여기서의 핵심은 뒤이어지는 RCU read-side 크리티컬 섹션들은 이제는
	제거된 원소로의 레퍼런스를 얻을 수 없다는 점입니다.
\item	예를 들어 앞서 교체된 원소를 메모리에서 해제하는 식으로 정리를 합니다.
\iffalse

\item	Make a change, for example, replace an element in a linked list.
\item	Wait for all pre-existing RCU read-side critical sections to
	completely finish (for example, by using the
	\co{synchronize_rcu()} primitive).
	The key observation here is that subsequent RCU read-side critical
	sections have no way to gain a reference to the newly removed
	element.
\item	Clean up, for example, free the element that was replaced above.
\fi
\end{enumerate}

\begin{figure}[tbp]
{ \scriptsize
\begin{verbatim}
  1 struct foo {
  2   struct list_head *list;
  3   int a;
  4   int b;
  5   int c;
  6 };
  7 LIST_HEAD(head);
  8
  9 /* . . . */
 10
 11 p = search(head, key);
 12 if (p == NULL) {
 13   /* Take appropriate action, unlock, & return. */
 14 }
 15 q = kmalloc(sizeof(*p), GFP_KERNEL);
 16 *q = *p;
 17 q->b = 2;
 18 q->c = 3;
 19 list_replace_rcu(&p->list, &q->list);
 20 synchronize_rcu();
 21 kfree(p);
\end{verbatim}
}
\caption{Canonical RCU Replacement Example}
\label{fig:defer:Canonical RCU Replacement Example}
\end{figure}

Figure~\ref{fig:defer:Canonical RCU Replacement Example} 에 보여진 코드 조각은
Section~\ref{sec:defer:Publish-Subscribe Mechanism} 에서 가져와진 것으로, 이
프로세스를 보여주는데,  필드 \co{a} 는 검색을 위한 키로 사용됩니다.

Line~19, 20, 21 은 앞에서 이야기한 세개의 스텝을 보입니다.
Line~16-19 이 RCU (``read-copy update'') 에 그 이름을 줍니다: 동시의
\emph{read} 를 허용하면서, line~16 은 \emph{copy} 를 하고 line~17-19 에서 실제
\emph{update} 를 합니다.
\iffalse

The code fragment shown in
Figure~\ref{fig:defer:Canonical RCU Replacement Example},
adapted from those in Section~\ref{sec:defer:Publish-Subscribe Mechanism},
demonstrates this process, with field \co{a} being the search key.

Lines~19, 20, and 21 implement the three steps called out above.
Lines~16-19 gives RCU (``read-copy update'') its name: while permitting
concurrent \emph{reads}, line~16 \emph{copies} and lines~17-19
do an \emph{update}.
\fi

Section~\ref{sec:defer:Introduction to RCU} 에서 이야기된 것처럼,
\co{synchronize_rcu()} 기능은 매우 간단할 수 있습니다 (``장난감'' RCU 구현을 더
보기 위해선 Section~\ref{sec:defer:``Toy'' RCU Implementations} 을 참고하시기
바랍니다).
하지만, 상품 수준의 구현들은 복잡하고 희귀한 경우들을 처리해야 하고 강력한
최적화를 포함해야 하는데, 둘 다 상당한 복잡도가 생기게 하고 맙니다.
\co{synchronize_rcu()} 의 간단한 개념적 구현이 있음을 알게 된 건 좋지만, 다른
질문들이 남습니다.
예를 들어, RCU 읽기 작업들은 동시에 업데이트 되는 리스트를 돌아다니면서 정확히
뭘 보게 되는 걸까요?
이 질문을 다음 섹션에서 다루도록 합니다.
\iffalse

As discussed in Section~\ref{sec:defer:Introduction to RCU},
the \co{synchronize_rcu()} primitive can be quite simple
(see Section~\ref{sec:defer:``Toy'' RCU Implementations}
for additional ``toy'' RCU implementations).
However, production-quality implementations must deal with
difficult corner cases and also incorporate
powerful optimizations, both of which result in significant complexity.
Although it is good to know that there is a simple conceptual
implementation of \co{synchronize_rcu()}, other questions remain.
For example, what exactly do RCU
readers see when traversing a concurrently updated list?
This question is addressed in the following section.
\fi

\subsubsection{Maintain Multiple Versions of Recently Updated Objects}
\label{sec:defer:Maintain Multiple Versions of Recently Updated Objects}

This section demonstrates how RCU maintains multiple versions of
lists to accommodate synchronization-free readers.
Two examples are presented showing how an element
that might be referenced by a given reader must remain intact
while that reader remains in its RCU read-side critical section.
The first example demonstrates deletion of a list element,
and the second example demonstrates replacement of an element.

\paragraph{Example 1: Maintaining Multiple Versions During Deletion}
\label{sec:defer:Example 1: Maintaining Multiple Versions During Deletion}

We can now revisit the deletion example from
Section~\ref{sec:defer:Introduction to RCU},
but now with the benefit of a firm understanding of the fundamental
concepts underlying RCU.
To begin this new version of the deletion example,
we will modify lines~11-21 in
Figure~\ref{fig:defer:Canonical RCU Replacement Example}
to read as follows:

\vspace{5pt}
\begin{minipage}[t]{\columnwidth}
\scriptsize
\begin{verbatim}
  1 p = search(head, key);
  2 if (p != NULL) {
  3   list_del_rcu(&p->list);
  4   synchronize_rcu();
  5   kfree(p);
  6 }
\end{verbatim}
\end{minipage}
\vspace{5pt}

\begin{figure}[tb]
\begin{center}
\resizebox{3in}{!}{\includegraphics{defer/RCUDeletion}}
\end{center}
\caption{RCU Deletion From Linked List}
\label{fig:defer:RCU Deletion From Linked List}
\end{figure}

This code will update the list as shown in
Figure~\ref{fig:defer:RCU Deletion From Linked List}.
The triples in each element represent the values of fields \co{a},
\co{b}, and \co{c}, respectively.
The red-shaded elements
indicate that RCU readers might be holding references to them,
so in the initial state at the top of the diagram, all elements
are shaded red.
Please note that
we have omitted the backwards pointers and the link from the tail
of the list to the head for clarity.

After the \co{list_del_rcu()} on
line~3 has completed, the \co{5,6,7}~element
has been removed from the list, as shown in the second row of
Figure~\ref{fig:defer:RCU Deletion From Linked List}.
Since readers do not synchronize directly with updaters,
readers might be concurrently scanning this list.
These concurrent readers might or might not see the newly removed element,
depending on timing.
However, readers that were delayed (e.g., due to interrupts, ECC memory
errors, or, in \co{CONFIG_PREEMPT_RT} kernels, preemption)
just after fetching a pointer to the newly removed element might
see the old version of the list for quite some time after the
removal.
Therefore, we now have two versions of the list, one with element
\co{5,6,7} and one without.
The \co{5,6,7}~element in the second row of the figure is now
shaded yellow, indicating
that old readers might still be referencing it, but that new
readers cannot obtain a reference to it.

Please note that readers are not permitted to maintain references to
element~\co{5,6,7} after exiting from their RCU read-side
critical sections.
Therefore,
once the \co{synchronize_rcu()} on
line~4 completes, so that all pre-existing readers are
guaranteed to have completed,
there can be no more readers referencing this
element, as indicated by its green shading on the third row of
Figure~\ref{fig:defer:RCU Deletion From Linked List}.
We are thus back to a single version of the list.

At this point, the \co{5,6,7}~element may safely be
freed, as shown on the final row of
Figure~\ref{fig:defer:RCU Deletion From Linked List}.
At this point, we have completed the deletion of
element~\co{5,6,7}.
The following section covers replacement.

\paragraph{Example 2: Maintaining Multiple Versions During Replacement}
\label{sec:defer:Example 2: Maintaining Multiple Versions During Replacement}

To start the replacement example,
here are the last few lines of the
example shown in
Figure~\ref{fig:defer:Canonical RCU Replacement Example}:

\vspace{5pt}
\begin{minipage}[t]{\columnwidth}
\scriptsize
\begin{verbatim}
  1 q = kmalloc(sizeof(*p), GFP_KERNEL);
  2 *q = *p;
  3 q->b = 2;
  4 q->c = 3;
  5 list_replace_rcu(&p->list, &q->list);
  6 synchronize_rcu();
  7 kfree(p);
\end{verbatim}
\end{minipage}
\vspace{5pt}

\begin{figure}[tbp]
\begin{center}
\resizebox{2.7in}{!}{\includegraphics{defer/RCUReplacement}}
\end{center}
\caption{RCU Replacement in Linked List}
\label{fig:defer:RCU Replacement in Linked List}
\end{figure}

The initial state of the list, including the pointer \co{p},
is the same as for the deletion example, as shown on the
first row of
Figure~\ref{fig:defer:RCU Replacement in Linked List}.

As before,
the triples in each element represent the values of fields \co{a},
\co{b}, and \co{c}, respectively.
The red-shaded elements might be referenced by readers,
and because readers do not synchronize directly with updaters,
readers might run concurrently with this entire replacement process.
Please note that
we again omit the backwards pointers and the link from the tail
of the list to the head for clarity.

The following text describes how to replace the \co{5,6,7} element
with \co{5,2,3} in such a way that any given reader sees one of these
two values.

Line~1 \co{kmalloc()}s a replacement element, as follows,
resulting in the state as shown in the second row of
Figure~\ref{fig:defer:RCU Replacement in Linked List}.
At this point, no reader can hold a reference to the newly allocated
element (as indicated by its green shading), and it is uninitialized
(as indicated by the question marks).

Line~2 copies the old element to the new one, resulting in the
state as shown in the third row of
Figure~\ref{fig:defer:RCU Replacement in Linked List}.
The newly allocated element still cannot be referenced by readers, but
it is now initialized.

Line~3 updates \co{q->b} to the value ``2'', and
line~4 updates \co{q->c} to the value ``3'', as shown on the fourth row of
Figure~\ref{fig:defer:RCU Replacement in Linked List}.

Now, line~5 does the replacement, so that the new element is
finally visible to readers, and hence is shaded red, as shown on
the fifth row of
Figure~\ref{fig:defer:RCU Replacement in Linked List}.
At this point, as shown below, we have two versions of the list.
Pre-existing readers might see the \co{5,6,7} element (which is
therefore now shaded yellow), but
new readers will instead see the \co{5,2,3} element.
But any given reader is guaranteed to see some well-defined list.

After the \co{synchronize_rcu()} on line~6 returns,
a grace period will have elapsed, and so all reads that started before the
\co{list_replace_rcu()} will have completed.
In particular, any readers that might have been holding references
to the \co{5,6,7} element are guaranteed to have exited
their RCU read-side critical sections, and are thus prohibited from
continuing to hold a reference.
Therefore, there can no longer be any readers holding references
to the old element, as indicated its green shading in the sixth row of
Figure~\ref{fig:defer:RCU Replacement in Linked List}.
As far as the readers are concerned, we are back to having a single version
of the list, but with the new element in place of the old.

After the \co{kfree()} on line 7 completes, the list will
appear as shown on the final row of
Figure~\ref{fig:defer:RCU Replacement in Linked List}.

Despite the fact that RCU was named after the replacement case,
the vast majority of RCU usage within the Linux kernel relies on
the simple deletion case shown in
Section~\ref{sec:defer:Maintain Multiple Versions of Recently Updated Objects}.

\paragraph{Discussion}
\label{sec:defer:Discussion}

These examples assumed that a mutex was held across the entire
update operation, which would mean that there could be at most two
versions of the list active at a given time.

\QuickQuiz{}
	How would you modify the deletion example to permit more than two
	versions of the list to be active?
\QuickQuizAnswer{
	One way of accomplishing this is as shown in
	Figure~\ref{fig:defer:Concurrent RCU Deletion}.

\begin{figure}[htbp]
{ \centering
\begin{verbatim}
  1 spin_lock(&mylock);
  2 p = search(head, key);
  3 if (p == NULL)
  4   spin_unlock(&mylock);
  5 else {
  6   list_del_rcu(&p->list);
  7   spin_unlock(&mylock);
  8   synchronize_rcu();
  9   kfree(p);
 10 }
\end{verbatim}
}
\caption{Concurrent RCU Deletion}
\label{fig:defer:Concurrent RCU Deletion}
\end{figure}

	Note that this means that multiple concurrent deletions might be
	waiting in \co{synchronize_rcu()}.
} \QuickQuizEnd

\QuickQuiz{}
	How many RCU versions of a given list can be
	active at any given time?
\QuickQuizAnswer{
	That depends on the synchronization design.
	If a semaphore protecting the update is held across the grace period,
	then there can be at most two versions, the old and the new.

	However, suppose that only the search, the update, and the
	\co{list_replace_rcu()} were protected by a lock, so that
	the \co{synchronize_rcu()} was outside of that lock, similar
	to the code shown in
	Figure~\ref{fig:defer:Concurrent RCU Deletion}.
	Suppose further that a large number of threads undertook an
	RCU replacement at about the same time, and that readers
	are also constantly traversing the data structure.

	Then the following sequence of events could occur, starting from
	the end state of
	Figure~\ref{fig:defer:RCU Replacement in Linked List}:

	\begin{enumerate}
	\item	Thread~A traverses the list, obtaining a reference to
		the 5,2,3 element.
	\item	Thread~B replaces the 5,2,3 element with a new
		5,2,4 element, then waits for its \co{synchronize_rcu()}
		call to return.
	\item	Thread~C traverses the list, obtaining a reference to
		the 5,2,4 element.
	\item	Thread~D replaces the 5,2,4 element with a new
		5,2,5 element, then waits for its \co{synchronize_rcu()}
		call to return.
	\item	Thread~E traverses the list, obtaining a reference to
		the 5,2,5 element.
	\item	Thread~F replaces the 5,2,5 element with a new
		5,2,6 element, then waits for its \co{synchronize_rcu()}
		call to return.
	\item	Thread~G traverses the list, obtaining a reference to
		the 5,2,6 element.
	\item	And the previous two steps repeat quickly, so that all
		of them happen before any of the \co{synchronize_rcu()}
		calls return.
	\end{enumerate}

	Thus, there can be an arbitrary number of versions active,
	limited only by memory and by how many updates could be completed
	within a grace period.
	But please note that data structures that are updated so frequently
	probably are not good candidates for RCU.
	That said, RCU can handle high update rates when necessary.
} \QuickQuizEnd

This sequence of events shows how RCU updates use multiple versions
to safely carry out changes in presence of concurrent readers.
Of course, some algorithms cannot gracefully handle multiple versions.
There are techniques
for adapting such algorithms to RCU~\cite{PaulEdwardMcKenneyPhD},
but these are beyond the scope of this section.

\subsubsection{Summary of RCU Fundamentals}
\label{sec:defer:Summary of RCU Fundamentals}

This section has described the three fundamental components of RCU-based
algorithms:

\begin{enumerate}
\item	a publish-subscribe mechanism for adding new data,

\item	a way of waiting for pre-existing RCU readers to finish, and

\item	a discipline of maintaining multiple versions to permit
	change without harming or unduly delaying concurrent RCU readers.
\end{enumerate}

\QuickQuiz{}
	How can RCU updaters possibly delay RCU readers, given that the
	{\tt rcu\_read\_lock()} and {\tt rcu\_read\_unlock()}
	primitives neither spin nor block?
\QuickQuizAnswer{
	The modifications undertaken by a given RCU updater will cause the
	corresponding CPU to invalidate cache lines containing the data,
	forcing the CPUs running concurrent RCU readers to incur expensive
	cache misses.
	(Can you design an algorithm that changes a data structure
	\emph{without}
	inflicting expensive cache misses on concurrent readers?
	On subsequent readers?)
} \QuickQuizEnd

These three RCU components
allow data to be updated in face of concurrent readers, and
can be combined in different ways to
implement a surprising variety of different types of RCU-based algorithms,
some of which are described in the following section.

% defer/rcuAPI.tex
% mainfile: ../perfbook.tex
% SPDX-License-Identifier: CC-BY-SA-3.0

\subsection{RCU Linux-Kernel API}
\label{sec:defer:RCU Linux-Kernel API}
\OriginallyPublished{Section}{sec:defer:RCU Linux-Kernel API}{RCU Linux-Kernel API}{Linux Weekly News}{PaulEMcKenney2008WhatIsRCUAPI}

이 섹션은 RCU 를 리눅스 커널 API 의 관점에서 봅니다.\footnote{
	Userspace RCU 의 API 는 다른 곳에 문서화 되어
	있습니다~\cite{PaulMcKenney2013LWNURCU}.}
Section~\ref{sec:defer:RCU has a Family of Wait-to-Finish APIs}
은 RCU 의 종료까지 기다리기 API 를 보이고,
Section~\ref{sec:defer:RCU has Publish-Subscribe and Version-Maintenance APIs}
은 RCU 의 발행-구독과 버전 관리 API 들을,
Section~\ref{sec:defer:RCU has List-Processing APIs}
은 RCU 의 리스트 처리 API 를,
Section~\ref{sec:defer:RCU Has Diagnostic APIs}
은 RCU 의 분석 API 를, 그리고
Section~\ref{sec:defer:Where Can RCU's APIs Be Used?}
은 RCU 의 다양한 API 들이 어떤 맥락에서 사용될 수 있는지 보입니다.
마지막으로,
Section~\ref{sec:defer:So, What is RCU Really?}
은 결론적 요약을 제공합니다.

커널 내부에 큰 관심이 없는 독자 여러분은 이 섹션을 건너뛰어
page~\pageref{sec:defer:RCU Usage} 의
\cref{sec:defer:RCU Usage} 으로 넘어가실 수도 있겠습니다.

\iffalse

This section looks at RCU from the viewpoint of its Linux-kernel API\@.\footnote{
	Userspace RCU's API is documented
	elsewhere~\cite{PaulMcKenney2013LWNURCU}.}
Section~\ref{sec:defer:RCU has a Family of Wait-to-Finish APIs}
presents RCU's wait-to-finish APIs,
Section~\ref{sec:defer:RCU has Publish-Subscribe and Version-Maintenance APIs}
presents RCU's publish-subscribe and version-maintenance APIs,
Section~\ref{sec:defer:RCU has List-Processing APIs}
presents RCU's list-processing APIs,
Section~\ref{sec:defer:RCU Has Diagnostic APIs}
presents RCU's diagnostic APIs, and
Section~\ref{sec:defer:Where Can RCU's APIs Be Used?}
describes in which contexts RCU's various APIs may be used.
Finally,
Section~\ref{sec:defer:So, What is RCU Really?}
presents concluding remarks.

Readers who are not excited about kernel internals may wish to skip
ahead to \cref{sec:defer:RCU Usage}
on page~\pageref{sec:defer:RCU Usage}.

\fi

\subsubsection{RCU has a Family of Wait-to-Finish APIs}
\label{sec:defer:RCU has a Family of Wait-to-Finish APIs}

``RCU 란 무엇인가'' 에 대한 가장 간단한 답은 RCU 는 API 라는 것입니다.
예를 들어, 리눅스 커널에서 사용되는 RCU 구현이 RCU, ``잠들 수 있는'' RCU
(SRCU), 태스크 기반 RCU (Tasks RCU) 의 읽기 쓰레드 기다리기 부분과 일반적 API
를 각각 보이고 있는 Table~\ref{tab:defer:RCU Wait-to-Finish APIs} 로, 그리고
이 API 의 발행-구독 부분을 보이는
Table~\ref{tab:defer:RCU Publish-Subscribe and Version Maintenance APIs}
로 요약되어 있습니다.\footnote{
	이 인용은 v4.20 과 이후 버전을 다룹니다.
	그 전 버전의 리눅스 커널 RCU API 에 대한 문서는 다른 곳에서 찾을 수
	있습니다~\cite{PaulEMcKenney2008WhatIsRCUAPI,PaulEMcKenney2014RCUAPI}.}

\iffalse

The most straightforward answer to ``what is RCU'' is that RCU is
an API\@.
For example, the RCU implementation used in the Linux kernel is
summarized by
Table~\ref{tab:defer:RCU Wait-to-Finish APIs},
which shows the wait-for-readers portions of the RCU, ``sleepable'' RCU
(SRCU), Tasks RCU, and generic APIs, respectively,
and by
Table~\ref{tab:defer:RCU Publish-Subscribe and Version Maintenance APIs},
which shows the publish-subscribe portions of the
API~\cite{PaulEMcKenney2019RCUAPI}.\footnote{
	This citation covers v4.20 and later.
	Documetation for earlier versions of the Linux-kernel RCU API may
	be found elsewhere~\cite{PaulEMcKenney2008WhatIsRCUAPI,PaulEMcKenney2014RCUAPI}.}

\fi

\begin{sidewaystable*}[tbp]
\rowcolors{1}{}{lightgray}
\renewcommand*{\arraystretch}{1.3}
\centering
\caption{RCU Wait-to-Finish APIs}
\label{tab:defer:RCU Wait-to-Finish APIs}
\scriptsize\hspace*{-.125in}
\begin{tabularx}{8.5in}{>{\raggedright\arraybackslash}p{0.94in}
    >{\raggedright\arraybackslash}X
    >{\raggedright\arraybackslash}X
    >{\raggedright\arraybackslash}p{1.1in}
    >{\raggedright\arraybackslash}p{1.35in}
    >{\raggedright\arraybackslash}p{1.45in}}
\toprule
&
    {\bf RCU}: Original &
	{\bf SRCU}: Sleeping readers &
	    {\bf Tasks RCU}: Free tracing trampolines &
		{\bf Tasks RCU Rude}: Free idle-task tracing trampolines &
		    {\bf Tasks RCU Trace}: Protect sleepable BPF programs \\
\midrule
{\bf Read-side critical-section markers} &
    \tco{rcu_read_lock()}~! \tco{rcu_read_unlock()}~!
    \tco{rcu_read_lock_bh()} \tco{rcu_read_unlock_bh()}
    \tco{rcu_read_lock_sched()} \tco{rcu_read_unlock_sched()}
    (Plus anything disabing bottom halves, preemption, or interrupts.) &
	\tco{srcu_read_lock()} \tco{srcu_read_unlock()} &
	    Voluntary context switch &
		Voluntary context switch and preempt-enable regions of code &
		    \tco{rcu_read_lock_trace()} \tco{rcu_read_unlock_trace()} \\
{\bf Update-side primitives (synchronous) } &
    { \tco{synchronize_rcu()}
      \tco{synchronize_net()}
      \tco{synchronize_rcu_expedited()} } &
	\tco{synchronize_srcu()} \tco{synchronize_srcu_expedited()} &
	    \tco{synchronize_rcu_tasks()} &
		\tco{synchronize_rcu_tasks_rude()} &
		    \tco{synchronize_rcu_tasks_trace()} \\
{\bf Update-side primitives (asynchronous / callback) } &
    \tco{call_rcu()} ! &
	\tco{call_srcu()} &
	    \tco{call_rcu_tasks()} &
		\tco{call_rcu_tasks_rude()} &
		    \tco{call_rcu_tasks_trace()} \\
{\bf Update-side primitives (wait for callbacks) } &
    \tco{rcu_barrier()} &
	\tco{srcu_barrier()} &
	    \tco{rcu_barrier_tasks()} &
		\tco{rcu_barrier_tasks_rude()} &
		    \tco{rcu_barrier_tasks_trace()} \\
{\bf Update-side primitives (initiate / wait)} &
    \tco{get_state_synchronize_rcu()}
    \tco{cond_synchronize_rcu()} &
	&
	    &
		&
		    \\
{\bf Update-side primitives (free memory) } &
    \tco{kfree_rcu()} &
	&
	    &
		&
		    \\
{\bf Type-safe memory } &
    \tco{SLAB_TYPESAFE_BY_RCU} &
	&
	    &
		&
		    \\
{\bf Read side constraints } &
    No blocking (only preemption) &
	No \tco{synchronize_srcu()} with same \tco{srcu_struct} &
	    No voluntary context switch &
		Neither blocking nor preemption &
			No RCU tasks trace grace period \\
{\bf Read side overhead } &
    CPU-local accesses (\tco{barrier()} on \tco{PREEMPT=n}) &
	Simple instructions, memory barriers &
	    Free &
		CPU-local accesses (free on \tco{PREEMPT=n}) &
		    CPU-local accesses \\
{\bf Asynchronous update-side overhead } &
    sub-microsecond &
	sub-microsecond &
	    sub-microsecond &
		sub-microsecond &
		    sub-microsecond \\
{\bf Grace-period latency } &
    10s of milliseconds &
        Milliseconds &
	    Seconds &
		Milliseconds &
		    10s of milliseconds \\
{\bf Expedited grace-period latency } &
    10s of microseconds &
        Microseconds &
	    N/A &
		N/A &
		    N/A \\
\bottomrule
\end{tabularx}
\end{sidewaystable*}

RCU 를 처음 접하신다면,
각각 리눅스 커널의 RCU API 집합 중 하나의 멤버를 요약하는
Table~\ref{tab:defer:RCU Wait-to-Finish APIs} 의 열 중 하나에만 집중하는 걸
고려해 보시기 바랍니다.
예를 들어, 리눅스 커널에서 RCU 가 어떻게 사용되는지 이해하는게 주요 관심사라면,
``RCU'' 가 가장 빈번하게 사용되므로 시작하기 좋을 겁니다.
다른 한편, RCU 를 그 자체만으로 이해하고자 한다면, ``Tasks RCU'' 가 가장 간단한
API 를 가졌습니다.
나중에 언제든 다른 열로 돌아오셔도 됩니다.

만약 여러분이 RCU 와 친숙하다면, 이 표들은 유용한 레퍼런스로 사용될 수 있을
겁니다.

\iffalse

If you are new to RCU, you might consider focusing on just one
of the columns in
Table~\ref{tab:defer:RCU Wait-to-Finish APIs},
each of which summarizes one member of the Linux kernel's RCU API family.
For example, if you are primarily interested in understanding how RCU
is used in the Linux kernel, ``RCU'' would be the place to start,
as it is used most frequently.
On the other hand, if you want to understand RCU for its own sake,
``Task RCU'' has the simplest API\@.
You can always come back for the other columns later.

If you are already familiar with RCU, these tables can
serve as a useful reference.

\fi

\QuickQuiz{
	Table~\ref{tab:defer:RCU Wait-to-Finish APIs}
	의 일부 셀들은 왜 느낌표를 (``!'') 가지고 있나요?

	\iffalse

	Why do some of the cells in
	Table~\ref{tab:defer:RCU Wait-to-Finish APIs}
	have exclamation marks (``!'')?

	\fi

}\QuickQuizAnswer{
	느낌표를 가지고 있는 API 멤버들 (\co{rcu_read_lock()},
	\co{rcu_read_unlock()}, 그리고 \co{call_rcu()}) 은 Paul E. McKenney 가
	90년대 중반에 인지하고 있던 리눅스 RCU API 멤버의 전부였습니다.
	이 시간 동안, 그는 그가 RCU 에 대해 알아야 할 걸 모두 알고 있다는
	잘못된 느낌을 가지고 있었습니다.

	\iffalse

	The API members with exclamation marks (\co{rcu_read_lock()},
	\co{rcu_read_unlock()}, and \co{call_rcu()}) were the
	only members of the Linux RCU API that Paul E. McKenney was aware
	of back in the mid-90s.
	During this timeframe, he was under the mistaken impression that
	he knew all that there is to know about RCU\@.

	\fi

}\QuickQuizEnd

``RCU'' 열은 세가지 리눅스 커널 RCU
구현의~\cite{PaulEMcKenney2019RCUCVE,McKenney:2019:CRS:3319647.3325836} 집합에
연관되는데, RCU read-side 크리티컬 섹션이 \co{rcu_read_lock()},
\co{rcu_read_lock_bh()}, 또는 \co{rcu_read_lock_sched()} 로 시작하여
\co{rcu_read_unlock()}, \co{rcu_read_unlock_bh()}, 또는
\co{rcu_read_unlock_sched()} 로 각각 종료됩니다.
Bottom halves, 인터럽트, 또는 preemption 을 불능화 시키는 모든 코드 영역 또한
RCU read-side 크리티컬 섹션으로 동작합니다.
연관된 동기 update-side 기능들은 \co{synchronize_rcu()} 와
\co{synchronize_rcu_expedited()}, 그리고 그와 유사한 \co{synchronize_net()}
으로, 현재 수행 중인 모든 종류의 RCU read-side 크리티컬 섹션들이 완료되기를
기다립니다.
이 기다림의 길이가 ``grace period'' 라고 알려져 있으며,
\co{synchronize_rcu_expedited()} 는 증가된 CPU 오버헤드와 IPI 를 댓가로 grace
period 응답시간을 줄이게끔 설계되었습니다.
비동기적 update-side 기능인 \co{call_rcu()} 는 다음 grace period 후에 특정
함수를 특정 인자와 함께 수행합니다.
예를 들어, \co{call_rcu(p,f);} 는 다음 grace period 후에 ``RCU callback''
\co{f(p)} 이 호출되게 할 겁니다.
\co{call_rcu()} 를 사용하는 리눅스 커널 모듈이 언로딩 될 때라던지와 같이 모든
RCU callback 들이 완료되기를 기다려야 하는 상황도
있습니다~\cite{PaulEMcKenney2007rcubarrier}.
\co{rcu_barrier()} 기능이 그 일을 합니다.

\iffalse

The ``RCU'' column corresponds to the consolidation of the
three Linux-kernel RCU
implementations~\cite{PaulEMcKenney2019RCUCVE,McKenney:2019:CRS:3319647.3325836},
in which RCU read-side critical sections start with
\co{rcu_read_lock()}, \co{rcu_read_lock_bh()}, or \co{rcu_read_lock_sched()}
and end with \co{rcu_read_unlock()}, \co{rcu_read_unlock_bh()},
or \co{rcu_read_unlock_sched()}, respectively.
Any region of code that disables bottom halves, interrupts, or preemption
also acts as an RCU read-side critical section.
RCU read-side critical sections may be nested.
The corresponding synchronous update-side primitives,
\co{synchronize_rcu()} and \co{synchronize_rcu_expedited()}, along with
their synonym \co{synchronize_net()}, wait for any type of currently
executing RCU read-side critical sections to complete.
The length of this wait is known as a ``grace period'', and
\co{synchronize_rcu_expedited()} is designed to reduce grace-period
latency at the expense of increased CPU overhead and IPIs.
The asynchronous update-side primitive, \co{call_rcu()},
invokes a specified function with a specified argument after a
subsequent grace period.
For example, \co{call_rcu(p,f);} will result in
the ``RCU callback'' \co{f(p)}
being invoked after a subsequent grace period.
There are situations,
such as when unloading a Linux-kernel module that uses \co{call_rcu()},
when it is necessary to wait for all
outstanding RCU callbacks to complete~\cite{PaulEMcKenney2007rcubarrier}.
The \co{rcu_barrier()} primitive does this job.

\fi

\QuickQuizSeries{%
\QuickQuizB{
	엄청나게 많은 RCU read-side 크리티컬 섹션이 무한정
	\co{synchronize_rcu()} 호출을 기다리게 하는 것은 어떻게 예방하나요?

	\iffalse

	How do you prevent a huge number of RCU read-side critical
	sections from indefinitely blocking a \co{synchronize_rcu()}
	invocation?

	\fi

}\QuickQuizAnswerB{
	RCU read-side 크리티컬 섹션들이 무한정 \co{synchronize_rcu()} 를
	기다리게 하는걸 막기 위해 어떤 일도 할 필요가 없는데,
	\co{synchronize_rcu()} 는 \emph{앞서서부터 존재해온} RCU read-side
	크리티컬 섹션들만을 기다리면 되기 때문입니다.
	따라서 각 RCU read-side 크리티컬 섹션이 한정된 길이인 한, RCU grace
	period 역시 한정적이게 됩니다.

	\iffalse

	There is no need to do anything to prevent RCU read-side
	critical sections from indefinitely blocking a
	\co{synchronize_rcu()} invocation, because the
	\co{synchronize_rcu()} invocation need wait only for
	\emph{pre-existing} RCU read-side critical sections.
	So as long as each RCU read-side critical section is
	of finite duration, RCU grace periods will also remain
	finite.

	\fi

}\QuickQuizEndB
%
\QuickQuizE{
	\co{synchronize_rcu()} API 는 모든 앞서서부터 존재해온 인터럽트
	핸들러가 완료되길 기다립니다, 맞죠?

	\iffalse

	The \co{synchronize_rcu()} API waits for all pre-existing
	interrupt handlers to complete, right?

	\fi

}\QuickQuizAnswerE{
	v4.20 이후의 리눅스 커널에서는
	그렇습니다~\cite{PaulEMcKenney2019RCUCVE,McKenney:2019:CRS:3319647.3325836}.

	하지만 그 전의 커널에서는, 특히 preemption 가능한 RCU 를 사용할 때는
	아닙니다!
	여러분은 그대신 \co{synchronize_irq()} 를 원할 겁니다.
	대안적으로, 여러분은 여러분이 \co{synchronize_rcu()} 로 하여금 기다리게
	하고 싶은 특정 인터럽트 핸들러 내에 \co{rcu_read_lock()} 과
	\co{rcu_read_unlock()} 을 넣을 수 있습니다.
	하지만 그럴 때에도, preemption 가능한 RCU 는 \co{rcu_read_lock()} 앞의,
	또는 \co{rcu_read_lock()} 뒤의 부분은 기다린다는 보장이 없으니
	조심하세요.

	\iffalse

	In v4.20 and later Linux kernels,
	yes~\cite{PaulEMcKenney2019RCUCVE,McKenney:2019:CRS:3319647.3325836}.

	But not in earlier kernels, and especially not when using
	preemptible RCU!
	You instead want \co{synchronize_irq()}.
	Alternatively, you can place calls to \co{rcu_read_lock()}
	and \co{rcu_read_unlock()} in the specific interrupt handlers that
	you want \co{synchronize_rcu()} to wait for.
	But even then, be careful, as preemptible RCU will not be guaranteed
	to wait for that portion of the interrupt handler preceding the
	\co{rcu_read_lock()} or following the \co{rcu_read_unlock()}.

	\fi

}\QuickQuizEndE
}

마지막으로, RCU 는
Section~\ref{sec:defer:RCU Provides Type-Safe Memory} 에서 이야기 되었듯
type-safe 메모리~\cite{Cheriton96a} 를 제공하기 위해 사용될 수 있습니다.
RCU 의 맥락에서, type-safe 메모리는 특정 데이터 원소의 타입이 그 원소에
접근하는 RCU read-side 크리티컬 섹션 내에서는 바뀌지 않음을 보장합니다.
RCU 기반 type-safe 메모리의 사용을 위해선, \co{kmem_cache_create()} 에
\co{SLAB_TYPESAFE_BY_RCU} 를 넘기면 됩니다.

\iffalse

Finally, RCU may be used to provide
type-safe memory~\cite{Cheriton96a}, as described in
Section~\ref{sec:defer:RCU Provides Type-Safe Memory}.
In the context of RCU, type-safe memory guarantees that a given
data element will not change type during any RCU read-side critical section
that accesses it.
To make use of RCU-based type-safe memory, pass
\co{SLAB_TYPESAFE_BY_RCU} to \co{kmem_cache_create()}.

\fi

Table~\ref{tab:defer:RCU Wait-to-Finish APIs}
의 ``SRCU'' 열은 \co{srcu_read_lock()} 과 \co{srcu_read_unlock()} 으로
구분지어지는 SRCU read-side 크리티컬 섹션 내에서 일반적인 잠들기를 허용하는
특수화된 RCU API 를 보입니다~\cite{PaulEMcKenney2006c}
하지만, RCU 와 달리, SRCU 의 \co{srcu_read_lock()} 은 연관된
\co{srcu_read_unlock()} 에 넘겨져야만 하는 값을 리턴합니다.
이 차이점은 SRCU 사용자는 각 SRCU 사용을 위해 \co{srcu_struct} 를 하나 할당해야
하는 사실 때문입니다.
이 개별의 \co{srcu_struct} 구조체는 SRCU read-side 크리티컬 섹션이 연관되지
않은 \co{synchronize_srcu()} 와 \co{synchronize_srcu_expedited()} 수행을
블록되게 하는 것을 방지합니다.
물론, SRCU read-side 크리티컬 섹션 내에서의 \co{synchronize_srcu()} 나
\co{synchronize_srcu_expedited()} 호출은 스스로의 데드락을 초래할 수 있으므로,
방지되어야 합니다.
RCU 에서와 같이, SRCU 의 \co{synchronize_srcu_expedited()} 는
\co{synchronize_srcu()} 에 비해 grace period 응답시간을 줄여 줍니다만, 증가된
CPU 오버헤드를 댓가로 치룹니다.

\iffalse

The ``SRCU'' column in
Table~\ref{tab:defer:RCU Wait-to-Finish APIs}
displays a specialized RCU API that permits general sleeping in SRCU
read-side critical
sections~\cite{PaulEMcKenney2006c}
delimited by \co{srcu_read_lock()} and \co{srcu_read_unlock()}.
However, unlike RCU, SRCU's \co{srcu_read_lock()} returns a value that
must be passed into the corresponding \co{srcu_read_unlock()}.
This difference is due to the fact that the SRCU user allocates an
\co{srcu_struct} for each distinct SRCU usage.
These distinct \co{srcu_struct} structures prevent SRCU read-side critical
sections from blocking unrelated \co{synchronize_srcu()} and
\co{synchronize_srcu_expedited()} invocations.
Of course, use of either \co{synchronize_srcu()} or
\co{synchronize_srcu_expedited()} within an SRCU read-side critical
section can result in self-deadlock, so should be avoided.
As with RCU, SRCU's \co{synchronize_srcu_expedited()} decreases
grace-period latency compared to \co{synchronize_srcu()}, but at
the expense of increased CPU overhead.

\fi

\QuickQuiz{
	어떤 조건에서 \co{synchronize_srcu()} 는 SRCU read-side 크리티컬 섹션
	내에서 안전하게 사용될 수 있나요?

	\iffalse

	Under what conditions can \co{synchronize_srcu()} be safely
	used within an SRCU read-side critical section?

	\fi

}\QuickQuizAnswer{
	원칙적으로, 여러분은 특정 \co{srcu_struct} 를 가지고 다른
	\co{srcu_struct} 를 사용하는 SRCU read-side 크리티컬 섹션 내에서
	\co{synchronize_srcu()} 나 \co{synchronize_srcu_expedited()} 를 사용할
	수 있습니다.
	하지만, 실제로는 이러는 건 거의 분명 나쁜 생각입니다.
	특히,
	Listing~\ref{lst:defer:Multistage SRCU Deadlocks}
	에 보인 코드는 여전히 데드락을 초래할 수 있습니다.

	\iffalse

	In principle, you can use either \co{synchronize_srcu()} or
	\co{synchronize_srcu_expedited()} with a given \co{srcu_struct}
	within an SRCU read-side critical section that uses some other
	\co{srcu_struct}.
	In practice, however, doing this is almost certainly a bad idea.
	In particular, the code shown in
	Listing~\ref{lst:defer:Multistage SRCU Deadlocks}
	could still result in deadlock.

	\fi

\begin{listing}[htbp]
\begin{VerbatimL}
idx = srcu_read_lock(&ssa);
synchronize_srcu(&ssb);
srcu_read_unlock(&ssa, idx);

/* . . . */

idx = srcu_read_lock(&ssb);
synchronize_srcu(&ssa);
srcu_read_unlock(&ssb, idx);
\end{VerbatimL}
\caption{Multistage SRCU Deadlocks}
\label{lst:defer:Multistage SRCU Deadlocks}
\end{listing}
}\QuickQuizEnd

일반적인 RCU 와 비슷하게, 비동기적 \co{call_srcu()} 함수를 통해 스스로 데드락
걸리는 것을 막을 수 있습니다.
하지만, 단일 태스크가 SRCU 콜백들을 매우 빠르게 등록할 수 있으므로
\co{call_srcu()} 를 사용할 때에는 특별한 주의가 필요합니다.
SRCU 가 읽기 쓰레드들이 임의의 기간동안 블록할 수 있다는 점을 두고 볼 때, 이는
임의의 커다란 양의 메모리를 소모할 수 있습니다.
대조적으로, 동기적인 \co{synchronous_srcu()} 인터페이스에서 태스크는 다음 grace
period 를 기다리기 시작하기 전에 현재 grace period 대기를 완료해야만 합니다.

또한 RCU 와 비슷하게, 모든 앞서 호출된 \co{call_srcu()} 콜백이 완료되기를
기다리는 \co{srcu_barrier()} 함수도 있습니다.

달리 말하면, SRCU 는 개발자들이 그 범위를 제한할 수 있게 허용함으로써 자신의
극단적으로 완화된 진행 보장을 보상합니다.

\iffalse

Similar to normal RCU, self-deadlock can be avoided using the
asynchronous \co{call_srcu()} function.
However, special care must be taken when using \co{call_srcu()} because
a single task could register SRCU callbacks very quickly.
Given that SRCU allows readers to block for arbitrary periods of
time, this could consume an arbitrarily large quantity of memory.
In contrast, given the synchronous \co{synchronize_srcu()}
interface, a given task must finish waiting for a given grace period
before it can start waiting for the next one.

Also similar to RCU, there is an \co{srcu_barrier()} function that waits
for all prior \co{call_srcu()} callbacks to be invoked.

In other words, SRCU compensates for its extremely weak
forward-progress guarantees by permitting the developer to restrict
its scope.

\fi

\cref{tab:defer:RCU Wait-to-Finish APIs} 의 ``Tasks RCU'' 열은 리눅스 커널
트레이싱 (tracing) 에서 사용되는 trampoline 들의 메모리 해제를 중재하는데
특수화된 RCU API 를 보입니다.
이 trampoline 들은 트레이싱 되는 코드의 한 지점에서 실제 트레이싱을 하는 코드로
제어를 전환하는 데 사용됩니다.
특정 trampoline 에서 수행되는 모든 코드가 이 trampoline 을 메모리 해제하기 전에
종료되었음을 보장할 것이 당연히 필요합니다.

Tracing 되는 코드에의 변화는 일반적으로 하나의 jump 또는 call 인스트럭션으로
제한되며, 따라서 \co{rcu_read_lock()} 과 \co{rcu_read_unlock()} 을 구현하는데
필요한 코드를 수용할 수 없습니다.
이 trampoline 은 \co{rcu_read_lock()} 과 \co{rcu_read_unlock()} 호출을 담을
수도 없습니다.
이를 이해하기 위해, 특정 trampoline 을 이제 수행하기 시작하려는 CPU 를 생각해
봅시다.
이 CPU 는 아직 \co{rcu_read_lock()} 을 수행하지 않았으므로, 언제든 메모리
해제될 수 있어서 이 CPU 에게 치명적 놀라움을 안겨줄 수 있습니다.
따라서, trampoline 은 traceing 된 코드 또는 trampoline 자신에 의해 수행되는
동기화 기능으로 보호될 수 없습니다.
이는 어떻게 trampoline 이 보호되어야 하는지 질문을 떠오르게 합니다.

\iffalse

The ``Tasks RCU'' column in
\cref{tab:defer:RCU Wait-to-Finish APIs} displays a specialized
RCU API that mediates freeing of the trampolines used in Linux-kernel
tracing.
These trampolines are used to transfer control from a point in the
code being traced to the code doing the actual tracing.
It is of course necessary to ensure that all code executing within
a given trampoline has finished before freeing that trampoline.

Changes to the code being traced are typically limited to a single jump
or call instruction, and thus cannot accommodate the sequence of code
required to implement \co{rcu_read_lock()} and \co{rcu_read_unlock()}.
Nor can the trampoline contain these calls to \co{rcu_read_lock()} and
\co{rcu_read_unlock()}.
To see this, consider a CPU that is just about to start executing a
given trampoline.
Because it has not yet executed the \co{rcu_read_lock()}, that
trampoline could be freed at any time, which would come as a fatal
surprise to this CPU\@.
Therefore, trampolines cannot be protected by synchronization primitives
executed in either the traced code or in the trampoline itself.
Which does raise the question of exactly how the trampoline is to be
protected.

\fi

이 질문에 대답하기 위한 열쇠는 trampoline 코드가 직접적으로도 간접적으로도
자발적 컨텍스트 스위치를 하는 코드를 결코 포함하지 않는다는 것을 파악하는
것입니다.
이 코드는 preemption 당할 수 있지만, 직접적으로든 간접적으로든 \co{schedule()}
을 결코 호출하지 않습니다.
이는 자발적 컨텍스트 스위치와 idle 수행을 유일한 quiescent state 로 간주하는
RCU 변종을 제안합니다.
이 변종이 Tasks RCU 입니다.

Tasks RCU 는 read-side 마킹 기능이 없다는 점에서 일반적이지 않은데, 그 주요
사용 케이스가 어디에도 그런 마킹을 할 수 없는 경우라는 걸 생각하면 좋은
일입니다.
대신, \co{schedule()} 호출이 곧 quiescent state 로 동작합니다.
업데이트는 모든 앞서서부터 존재해온 trampoline 수행이 완료되기를 기다리기 위해
\co{synchronize_rcu_tasks()} 를, 또는 그 비동기 버전인 \co{call_rcu_tasks()} 를
사용할 수 있습니다.
또한 모든 앞서 호출된 \co{call_rcu_tasks()} 에 연관된 콜백들의 완료를 기다리기
위한 \co{rcu_barrier_tasks()} 도 존재합니다.
아직 요구가 없었기 때문에 \co{synchronize_rcu_tasks_expedited()} 함수는
존재하지 않습니다만, 그것의 유용한 변종을 구현하는 건 도전해볼만 할 것입니다.

\iffalse

The key to answering this question is to note that trampoline code
never contains code that either directly or indirectly does a
voluntary context switch.
This code might be preempted, but it will never directly or indirectly
invoke \co{schedule()}.
This suggests a variant of RCU having voluntary context switches and
idle execution as its only quiescent states.
This variant is Tasks RCU\@.

Tasks RCU is unusual in having no read-side marking functions, which
is good given that its main use case has nowhere to put such markings.
Instead, calls to \co{schedule()} serve directly as quiescent states.
Updates can use \co{synchronize_rcu_tasks()} to wait for all pre-existing
trampoline execution to complete, or they can use its asynchronous
counterpart, \co{call_rcu_tasks()}.
There is also an \co{rcu_barrier_tasks()} that waits for completion
of callbacks corresponding to all prior invocations of \co{call_rcu_tasks()}.
There is no \co{synchronize_rcu_tasks_expedited()} because there has
not yet been a request for it, though implementing a useful variant of
it would not be free of challenges.

\fi

\QuickQuiz{
	\co{CONFIG_PREEMPT_NONE=y} 로 빌드된 커널에서라면 preemption 이 불능화
	되었고 trampoline 은 직접적으로든 간접적으로든 \co{schedule()} 을
	호출하지 않으니  \co{synchronize_rcu()} 는 모든 trampoline 을 기다리지
	않나요?

	\iffalse

	In a kernel built with \co{CONFIG_PREEMPT_NONE=y}, won't
	\co{synchronize_rcu()} wait for all trampolines, given
	that preemption is disabled and that trampolines never
	directly or indirectly invoke \co{schedule()}?

	\fi

}\QuickQuizAnswer{
	바로 그렇습니다!

	실제로, nonpreemptible 커널에서, \co{synchronize_rcu_tasks()} 는
	\co{synchronize_rcu()} 의 wrapper 입니다.

	\iffalse

	You are quite right!

	In fact, in nonpreemptible kernels, \co{synchronize_rcu_tasks()}
	is a wrapper around \co{synchronize_rcu()}.

	\fi

}\QuickQuizEnd

``Tasks RCU Rude'' 열은
\cref{sec:defer:Toy Implementation} 에서 선보인 장난감 구현의 보다 효과적인
변종을 제공합니다.
이 변종은 각 CPU 가 컨텍스트 스위치를 수행해서 모든 자발적 컨텍스트 스위치나
모든 preemption 가능한 코드 영역이 quiescent state 로 동작하게 합니다.
Tasks RCU Rude 변종은 리눅스 커널의 workqueue 기능을 동시적 컨텍스트 스위치를
강제하기 위한 장치로 사용하는데, 그 장난감 구현에서는 CPU 하나씩 순차적으로
사용하는 접근법과 대조됩니다.
이 API 는 Tasks RCU 의 것과 대칭되는데, 명시적 read-side 마커의 부재 또한
그렇습니다.

마지막으로, ``Tasks RCU Trace'' 열은 SRCU 의 그것과 비슷한 기능을 갖지만 훨씬
빠른 read-side 마커는 예외입니다.\footnote{
	그리고 따라서 Tasks RCU 계열에서 명시적 read-side 마커를 갖는게
	비일상적입니다!}
하지만, 이 속도는 이 마커들이 메모리 배리어 인스트럭션을 수행하지 않는다는
사실에서 기인하는 것으로, Tasks RCU Trace grace period 는 종종 모든 CPU 에게
IPI 를 보내야 하고 항상 전체 태스크 리스트를 스캔해야 함을 의미합니다.
그러나, 그로 인한 grace period 응답시간은 무척 짧아서 RCU 의 그것과 라이벌이
될만 합니다.

\iffalse

The ``Tasks RCU Rude'' column provides a more effective variant
of the toy implementation presented in
\cref{sec:defer:Toy Implementation}.
This variant causes each CPU to execute a context switch,
so that any voluntary context switch or any preemptible region of
code can serve as a quiescent state.
The Tasks RCU Rude variant uses the Linux-kernel workqueues facility to
force concurrent context switches, in contrast to the serial
CPU-by-CPU approach taken by the toy implementation.
The API mirrors that of Tasks RCU, including the lack of explicit
read-side markers.

Finally, the ``Tasks RCU Trace'' column provides an RCU implementation
with functionality similar to that of SRCU, except with much faster
read-side markers.\footnote{
	And thus is unusual for the Tasks RCU family for having
	explicit read-side markers!}
However, this speed is a consequence of the fact that these markers
do not execute memory-barrier instructions, which means that Tasks RCU
Trace grace periods must often send IPIs to all CPUs and must always
scan the entire task list.
Nevertheless, the resulting grace-period latency is reasonably short,
rivaling that of RCU\@.

\fi

\subsubsection{RCU has Publish-Subscribe and Version-Maintenance APIs}
\label{sec:defer:RCU has Publish-Subscribe and Version-Maintenance APIs}

다행히,
Table~\ref{tab:defer:RCU Publish-Subscribe and Version Maintenance APIs}
에 보인 RCU 발행-구독과 버전 관리 기능은 앞서 이야기한 모든 RCU 변종에
적용됩니다.
이 동일성이 더 많은 코드가 공유되고 API 증식을 줄일 수 있게 합니다.
RCU 발행-구독 API 의 원래 목적은 메모리 배리어를 이 API 내로 묻어버려서 리눅스
커널 프로그래머가 리눅스가 지원하는 20+ CPU 군~\cite{Spraul01} 각각의 메모리
순서 규칙 모델에 대한 전문가가 될 필요 없이 RCU 를 사용할 수 있게 하는
것이었습니다.

\iffalse

Fortunately, the RCU publish-subscribe and version-maintenance
primitives shown in
Table~\ref{tab:defer:RCU Publish-Subscribe and Version Maintenance APIs}
apply to all of the variants of RCU discussed above.
This commonality can allow more code to be shared, and reduces API
proliferation.
The original purpose of the RCU publish-subscribe APIs was to
bury memory barriers into these APIs, so that Linux kernel
programmers could use RCU without needing to become expert on
the memory-ordering models of each of the 20+ CPU families
that Linux supports~\cite{Spraul01}.

\fi

\begin{table*}[tb]
\renewcommand*{\arraystretch}{1.15}
\footnotesize
\centering\OneColumnHSpace{-.4in}
\begin{tabular}{llp{2.2in}}
\toprule
Category &
	Primitives &
		Overhead \\
\midrule
Pointer publish &
	\tco{rcu_assign_pointer()} &
		Memory barrier \\
&
	\tco{rcu_replace_pointer()} &
		Memory barrier (two of them on Alpha) \\
&
	\tco{rcu_pointer_handoff()} &
		Simple instructions \\
&
	\tco{RCU_INIT_POINTER()} &
		Simple instructions \\
&
	\tco{RCU_POINTER_INITIALIZER()} &
		Compile-time constant \\
\midrule
Pointer subscribe (traveral) &
	\tco{rcu_access_pointer()} &
		Simple instructions \\
&
	\tco{rcu_dereference()} &
		Simple instructions (memory barrier on Alpha) \\
&
	\tco{rcu_dereference_check()} &
		Simple instructions (memory barrier on Alpha) \\
&
	\tco{rcu_dereference_protected()} &
		Simple instructions \\
&
	\tco{rcu_dereference_raw()} &
		Simple instructions (memory barrier on Alpha) \\
&
	\tco{rcu_dereference_raw_notrace()} &
		Simple instructions (memory barrier on Alpha) \\
\bottomrule
\end{tabular}
\caption{RCU Publish-Subscribe and Version Maintenance APIs}
\label{tab:defer:RCU Publish-Subscribe and Version Maintenance APIs}
\end{table*}

이 기능들은 포인터에 직접 동작하며, RCU 로 보호되는 배열과 트리 같은 연결된
데이터 구조들을 생성하는데 유용합니다.
링크드 리스트의 특수한 경우는
Section~\ref{sec:defer:RCU has List-Processing APIs}
에 설명된 별도의 API 집합으로 처리됩니다.

첫번째 카테고리는 새 데이터 항목으로의 포인터를 발행합니다.
\co{rcu_assign_pointer()} 기능은 모든 앞의 초기화가 완화된 순서 규칙 기계에서도
포인터 할당 전에 행해졌을 것을 보장합니다.
\co{rcu_replace_pointer()} 기능은 \co{rcu_assign_pointer()} 가 하는 것과 똑같이
이 포인터를 업데이트 합니다만, 기존의 값을 리턴하는데,
\co{rcu_dereference_protected()} 가 하는 것과 동일한데 (아래를 보시기
바랍니다), lockdep 표현 역시 포함합니다.
이 교체는 업데이트 쓰레드가 새 포인터를 발행하면서 기존 포인터에 의해 참조되는
구조체를 해제해야 할 때 유용합니다.

\iffalse

These primitives operate directly on pointers, and are useful for
creating RCU-protected linked data structures, such as RCU-protected
arrays and trees.
The special case of linked lists is handled by a separate set of
APIs described in
Section~\ref{sec:defer:RCU has List-Processing APIs}.

The first category publishes pointers to new data items.
The \co{rcu_assign_pointer()} primitive ensures that any
prior initialization remains ordered before the assignment to the
pointer on weakly ordered machines.
The \co{rcu_replace_pointer()} primitive updates the pointer just like
\co{rcu_assign_pointer()} does, but also returns the previous value,
just like \co{rcu_dereference_protected()} (see below) would, including
the lockdep expression.
This replacement is convenient when the updater must both publish a new
pointer and free the structure referenced by the old pointer.

\fi

\QuickQuizSeries{%
\QuickQuizB{
	일반적으로, \co{rcu_dereference()} 에 넘겨지는 포인터는 항상
	Table~\ref{tab:defer:RCU Publish-Subscribe and Version Maintenance APIs}
	의 포인터 발행 함수 중 하나, 예를 들면 \co{rcu_assign_pointer()} 를
	사용해 업데이트 \emph{되어야만} 합니다.

	이 규칙에서 예외는 무엇이겠습니까?

	\iffalse

	Normally, any pointer subject to \co{rcu_dereference()} \emph{must}
	always be updated using one of the pointer-publish functions in
	Table~\ref{tab:defer:RCU Publish-Subscribe and Version Maintenance APIs},
	for example, \co{rcu_assign_pointer()}.

	What is an exception to this rule?

	\fi

}\QuickQuizAnswerB{
	그런 예외 중 하나는 여러 원소를 갖는 연결된 데이터 구조가 다른 CPU 들에
	의한 액세스가 불가능한 동안 하나의 단위로 초기화 되었고 이어서 이
	데이터 구조에 하나의 전역 포인터를 넣기 위해 \co{rcu_assign_pointer()}
	가 사용되었을 때입니다.
	이 초기화 시점 포인터 할당은 \co{rcu_assign_pointer()} 를 사용할 필요가
	없습니다, 그런 구조체가 전역적으로 보여질 수 있게 된 후의 그런 할당은
	\co{rcu_assign_pointer()} 를 \emph{사용해야만} 하지만말입니다.

	하지만, 이 초기화 코드가 상당히 자주 수행되는 코드 경로에 있지 않다면,
	\co{rcu_assign_pointer()} 가 이론상으론 필요치 않더라도 이 함수를
	사용하는게 현명합니다.
	``사소한'' 변경이 초기화가 사적으로 일어난다는 여러분의 소중한 가정을
	바꿔놓기는 너무 쉽습니다.

	\iffalse

	One such exception is when a multi-element linked
	data structure is initialized as a unit while inaccessible to other
	CPUs, and then a single \co{rcu_assign_pointer()} is used
	to plant a global pointer to this data structure.
	The initialization-time pointer assignments need not use
	\co{rcu_assign_pointer()}, though any such assignments that
	happen after the structure is globally visible \emph{must} use
	\co{rcu_assign_pointer()}.

	However, unless this initialization code is on an impressively hot
	code-path, it is probably wise to use \co{rcu_assign_pointer()}
	anyway, even though it is in theory unnecessary.
	It is all too easy for a ``minor'' change to invalidate your cherished
	assumptions about the initialization happening privately.

	\fi

}\QuickQuizEndB
%
\QuickQuizE{
	이 순회와 업데이트 기능들이 모든 RCU API 집합 멤버들과 사용될 수 있다는
	데에 어떤 단점은 없습니까?

	\iffalse

	Are there any downsides to the fact that these traversal and update
	primitives can be used with any of the RCU API family members?

	\fi

}\QuickQuizAnswerE{
	``sparse'' 와 같은 자동화된 코드 검사기가 (또는 실제 사람이) 특정 RCU
	순회 기능이 어떤 종류의 RCU read-side 크리티컬 섹션에 연관되어 있는지
	알기 어렵습니다.
	예를 들어,
	Listing~\ref{lst:defer:Diverse RCU Read-Side Nesting}
	에 보인 코드를 생각해 봅시다.

	\iffalse

	It can sometimes be difficult for automated
	code checkers such as ``sparse'' (or indeed for human beings) to
	work out which type of RCU read-side critical section a given
	RCU traversal primitive corresponds to.
	For example, consider the code shown in
	Listing~\ref{lst:defer:Diverse RCU Read-Side Nesting}.

	\fi

\begin{listing}[htbp]
\begin{VerbatimL}
rcu_read_lock();
preempt_disable();
p = rcu_dereference(global_pointer);

/* . . . */

preempt_enable();
rcu_read_unlock();
\end{VerbatimL}
\caption{Diverse RCU Read-Side Nesting}
\label{lst:defer:Diverse RCU Read-Side Nesting}
\end{listing}

	여기서의 \co{rcu_dereference()} 기능은 바닐라 RCU 크리티컬 섹션 안에
	있나요, RCU Sched 크리티컬 섹션 안에 있나요?
	이걸 알기 위해 여러분은 뭘 해야 합니까?

	하지만 v4.20 리눅스 커널에서의 RCU 변종들의 통합 이후부터는 더이상
	신경쓸 필요 없습니다!

	\iffalse

	Is the \co{rcu_dereference()} primitive in a vanilla RCU critical
	section or an RCU Sched critical section?
	What would you have to do to figure this out?

	But perhaps after the consolidation of the RCU flavors in
	the v4.20 Linux kernel we no longer need to care!

	\fi

}\QuickQuizEndE
}

\co{rcu_pointer_handoff()} 기능은 간단히 그것의 전체 인자를 리턴합니다만, RCU
read-side 크리티컬 섹션 밖으로 새어나가는 포인터들을 검사하는 도구를 만드는 데
유용합니다.
\co{rcu_pointer_handoff()} 의 사용은 그런 도구에게 이 구조체의 보호가 RCU 에서
예를 들면 락킹이나 레퍼런스 카운팅 같은 어떤 다른 메커니즘으로 넘겨졌음을
알립니다.

\co{RCU_INIT_POINTER()} 매크로는 아직 읽기 쓰레드에게 노출되지 않은 RCU 로
보호되는 포인터를 초기화 하기 위해, 또는 RCU 로 보호되는 포인터를 \co{NULL} 로
만들기 위해 사용될 수 있습니다.
이 제한된 경우들에서, \co{rcu_assign_pointer()} 에 의해 제공되는 메모리 배리어
인스트럭션은 필요치 않습니다.
비슷하게, \co{RCU_POINTER_INITIALIZER()} 는 데이터 구조들 내의 RCU 로 보호되는
포인터들의 쉬운 초기화를 허용하기 위한 \GCC 스타일 구조체 초기화를 제공합니다.

\iffalse

The \co{rcu_pointer_handoff()} primitive simply returns its sole argument,
but is useful to tooling checking for pointers being leaked from
RCU read-side critical sections.
Use of \co{rcu_pointer_handoff()} indicates to such tooling that protection
of the structure in question has been handed off from RCU to some other
mechanism, such as locking or reference counting.

The \co{RCU_INIT_POINTER()} macro can be used to initialize RCU-protected
pointers that have not yet been exposed to readers, or alternatively,
to set RCU-protected pointers to \co{NULL}.
In these restricted cases, the memory-barrier instructions provided by
\co{rcu_assign_pointer()} are not needed.
Similarly, \co{RCU_POINTER_INITIALIZER()} provides a \GCC-style
structure initializer to allow easy initialization of RCU-protected
pointers in structures.

\fi

두번째 카테고리는 데이터 항목으로의 포인터를 구독하거나, 또는 대안적으로,
안전하게 RCU 로 보호되는 포인터들을 순회합니다.
다시 말하지만, 단순히 C 언어 액세스를 사용한 포인터 로딩은 가리켜지는 데이터의
초기화 전 쓰레기 값을 볼 수 있게 할 수 있습니다.
비슷하게, 이 포인터를 어떤 수단을 사용해서든 RCU read-side 크리티컬 섹션
바깥에서 읽는 행위는 가리켜진 객체가 언제든 메모리 해제되게 할 수 있습니다.
하지만, 이 포인터가 그저 테스트 되고 역참조 되지 않을 것이라면, 가리켜진 객체의
메모리 해제는 문제가 아닐 겁니다.
이 경우, \co{rcu_access_pointer()} 가 사용될 수 있습니다.
하지만, 보통 RCU read-side 보호가 필요하며, 따라서 \co{rcu_dereference()}
기능은 이 \co{rcu_dereference()} 호출이 \co{rcu_read_lock()},
\co{srcu_read_lock()}, 또는 어떤 다른 RCU read-side 마커의 보호 아래
이루어졌음을 검증하기 위해 리눅스 커널의 \co{lockdep}
기능~\cite{JonathanCorbet2006lockdep} 을 사용합니다.
대조적으로, \co{rcu_access_pointer()} 기능은 \co{lockdep} 과 연관되지 않으며,
따라서 RCU read-side 크리티컬 섹션 외부에서 사용되어도 \co{lockdep} 의 불평을
일으키지 않을 겁니다.

\iffalse

The second category subscribes to pointers to data items, or,
alternatively, safely traverses RCU-protected pointers.
Again, simply loading these pointers using C-language accesses
could result in seeing pre-initialization garbage in the pointed-to data.
Similarly, loading these pointer by any means outside of an RCU
read-side critical section could result in the pointed-to object being
freed at any time.
However, if the pointer is merely to be tested and not dereferenced,
the freeing of the pointed-to object is not necessarily a problem.
In this case, \co{rcu_access_pointer()} may be used.
Normally, however, RCU read-side protection is required, and so the
\co{rcu_dereference()} primitive uses the Linux kernel's \co{lockdep}
facility~\cite{JonathanCorbet2006lockdep} to verify that this
\co{rcu_dereference()} invocation is under the protection of
\co{rcu_read_lock()}, \co{srcu_read_lock()}, or some other RCU read-side
marker.
In contrast, the \co{rcu_access_pointer()} primitive does not involve
\co{lockdep}, and thus will not provoke \co{lockdep} complaints when
used outside of an RCU read-side critical section.

\fi

보호가 필요치 않은 또다른 상황은 업데이트 쪽 코드가 RCU 로 보호되는 포인터를
업데이트 쪽 락을 잡은 채 액세스 하는 경우입니다.
\co{rcu_dereference_protected()} API 멤버가 이 상황을 위해 제공됩니다.
이것의 첫번째 패러미터는 RCU 로 보호되는 포인터이며, 두번째 패러미터는 이
액세슥 ㅏ안전하기 위해 어떤 락이 잡혀져 있어야만 하는지를 설명하는 lockdep
표현입니다.
읽기 쓰레드와 업데이트 쓰레드 양쪽에서 수행되는 코드는 역시 lockdep 표현을
받지만 이 락을 잡지 않는 읽기 쪽 코드에서도 호출될 수 있는
\co{rcu_dereference_check()} 를 사용할 수 있습니다.
어떤 경우에, 이 lockdep 표현은 매우 복잡할 수 있는데, 예를 들어 세밀한 락킹을
사용할 때에는, 큰 수의 락들 중 어떤 것도 잡혀 있을 수 있고, 그 중 어떤 것이
적용되는지 파악하기 무척 어려울 수도 있습니다.
이러한 (바라건대 드문) 경우, \co{rcu_dereference_raw()} 는 보호를 제공하지만
읽기 쓰레드 또는 특수한 락이 잡힌 상태에서 호출되는지를 검사하지는 않습니다.
\co{rcu_dereference_raw_notrace()} API 멤버는 비슷하게 동작하지만 추적될 수
없으며, 따라서 트레이싱 코드에서 안전히 사용될 수 있습니다.

어떤 연결된 구조든 포인터를 사용해 액세스 될 수 있지만, 더 높은 수준의 구조가
도움될 수 있습니다.
따라서 다음 섹션은 다양한 종류의 RCU 로 보호되며 리눅스 커널에서 사용되는
링크드 리스트들을 알아봅니다.

\iffalse

Another situation where protection is not required is when update-side code
accesses the RCU-protected pointer while holding the update-side lock.
The \co{rcu_dereference_protected()} API member is provided for this
situation.
Its first parameter is the RCU-protected pointer, and the second
parameter takes a lockdep expression describing which locks must be
held in order for the access to be safe.
Code invoked both from readers and updaters can use
\co{rcu_dereference_check()}, which also takes a lockdep expression, but
which may also be invoked from read-side code not holding the locks.
In some cases, the lockdep expressions can be very complex, for example,
when using fine-grained locking, any of a very large number of locks
might be held, and it might be quite difficult to work out which applies.
In these (hopefully rare) cases, \co{rcu_dereference_raw()} provides
protection but does not check for being invoked within a reader or with
any particular lock being held.
The \co{rcu_dereference_raw_notrace()} API member acts similarly, but
cannot be traced, and may therefore be safely used by tracing code.

Although pretty much any linked structure can be accessed by manipulating
pointers, higher-level structures can be quite helpful.
The next section therefore looks at various sorts of RCU-protected
linked lists used by the Linux kernel.

\fi

\subsubsection{RCU has List-Processing APIs}
\label{sec:defer:RCU has List-Processing APIs}

\begin{figure}[tb]
\centering
\resizebox{3in}{!}{\includegraphics{defer/Linux_list}}
\caption{Linux Circular Linked List (\tco{list})}
\label{fig:defer:Linux Circular Linked List (list)}
\end{figure}

\begin{figure}[tb]
\centering
\resizebox{3in}{!}{\includegraphics{defer/Linux_list_abbr}}
\caption{Linux Linked List Abbreviated}
\label{fig:defer:Linux Linked List Abbreviated}
\end{figure}

\co{rcu_assign_pointer()} 와 \co{rcu_dereference()} 가 이론상으로는 모든 상상할
수 있는 RCU 로 보호되는 데이터 구조를 만들 수 있지만, 실제로는 더 높은 단계의
것을 쓰는게 종종 낫습니다.
따라서, \co{rcu_assign_pointer()} 와 \co{rcu_dereference()} 는 리눅스의 리스트
조작 API 의 특수한 RCU 변종에 내포되어 있습니다.
리눅스는 네개의 양방향 링크드 리스트 변종을 갖는데, 순환형 \co{struct
list_head} 와 선형 \co{struct hlist_head}/\co{struct hlist_node}, \co{struct
hlist_nulls_head}/\co{struct hlist_nulls_node}, 그리고 \co{struct
hlist_bl_head}/\co{struct hlist_bl_node} 쌍들입니다.
앞의 것은
Figure~\ref{fig:defer:Linux Circular Linked List (list)} 에 보여져 있는 형태를
띄는데, (가장 왼쪽의) 녹색 상자가 리스트 헤더를 표현하며 (오른쪽 세개의) 파란
상자들은 이 리스트의 원소들을 표현합니다.
이 표기법은 다루기 성가시며 따라서 헤더가 아닌 (파랑) 원소들만 보이는
Figure~\ref{fig:defer:Linux Linked List Abbreviated} 에 보인 것처럼 간략화 할
겁니다.

\iffalse

Although \co{rcu_assign_pointer()} and
\co{rcu_dereference()} can in theory be used to construct any
conceivable RCU-protected data structure, in practice it is often better
to use higher-level constructs.
Therefore, the \co{rcu_assign_pointer()} and
\co{rcu_dereference()}
primitives have been embedded in special RCU variants of Linux's
list-manipulation API\@.
Linux has four variants of doubly linked list, the circular
\co{struct list_head} and the linear
\co{struct hlist_head}/\co{struct hlist_node},
\co{struct hlist_nulls_head}/\co{struct hlist_nulls_node}, and
\co{struct hlist_bl_head}/\co{struct hlist_bl_node}
pairs.
The former is laid out as shown in
Figure~\ref{fig:defer:Linux Circular Linked List (list)},
where the green (leftmost) boxes represent the list header and the blue
(rightmost three) boxes represent the elements in the list.
This notation is cumbersome, and will therefore be abbreviated as shown in
Figure~\ref{fig:defer:Linux Linked List Abbreviated},
which shows only the non-header (blue) elements.

\fi

\begin{figure}[tb]
\centering
\resizebox{3in}{!}{\includegraphics{defer/Linux_hlist}}
\caption{Linux Linear Linked List (\tco{hlist})}
\label{fig:defer:Linux Linear Linked List (hlist)}
\end{figure}

리눅스의 \co{hlist}\footnote{
	``h'' 는 해쉬테이블을 의미하는데, 리눅스의 양방향 포인터 순환형 링크드
	리스트에 비해 메모리 사용량을 절반으로 줄입니다.}
는 선형 리스트로, 이는
Figure~\ref{fig:defer:Linux Linear Linked List (hlist)}
에 보이듯헤더에  순환형 리스트에 필요한 두개의 포인터가 아니라 한개의 포인터만
필요합니다.
따라서, \co{hlist} 의 사용은 커다란 해쉬 테이블의 해쉬 버킷 배열을 위한 메모리
사용량을 절반으로 줄일 수 있습니다.
앞에서와 같이, 이 표기법은 다루기 귀찮으므로, \co{hlist} 구조는
Figure~\ref{fig:defer:Linux Linked List Abbreviated}
에 보인 것과 같은 \co{list_head} 스타일 리스트로 간략화 해 표기하겠습니다.

\iffalse

Linux's \co{hlist}\footnote{
	The ``h'' stands for hashtable, in which it reduces memory
	use by half compared to Linux's double-pointer circular
	linked list.}
is a linear list, which means that
it needs only one pointer for the header rather than the two
required for the circular list, as shown in
Figure~\ref{fig:defer:Linux Linear Linked List (hlist)}.
Thus, use of \co{hlist} can halve the memory consumption for the hash-bucket
arrays of large hash tables.
As before, this notation is cumbersome, so \co{hlist} structures will
be abbreviated in the same way \co{list_head}-style lists are, as shown in
Figure~\ref{fig:defer:Linux Linked List Abbreviated}.

\fi

\co{hlist_nulls} 라 이름지어진 리눅스의 \co{hlist} 의 한 변종은 여러 구별된
\co{NULL} 포인터들을 제공합니다만, 그것 외에는
Figure~\ref{fig:defer:Linux Linear Linked List (hlist)}
에 보인것과 같은 레이아웃을 갖습니다.
이 변종에서, 0 값의 낮은 위치 비트를 갖는 \co{->next} 포인터는 \co{NULL} 포인터
타입을 의미합니다.
이 타입의 리스트는 lockless 읽기 쓰레드들이 어떤 노드가 한 리스트에서 다른
리스트로 옮겨진 것을 파악할 수 있게 하기 위해 사용됩니다.
예를 들어, 해쉬 테이블의 각 버킷은 그것의 \co{NULL} 포인터를 마킹하기 위해 그
인덱스를 사용할 수도 있습니다.
읽기 쓰레드가 자신이 시작한 버킷의 인덱스와 맞지 않는 \co{NULL} 포인터를
마주치면, 이 읽기 쓰레드는 자신이 순회하고 있는 원소가 순회 사이에 다른
버킷으로 옮겨졌음을 알게 됩니다.
이 읽기 쓰레드는 리스트의 끝에 도달했는지 알기 위해 \co{is_a_nulls()} 함수
(\co{hlist_nulls} \co{NULL} 포인터를 받으면 true 를 리턴합니다) 를, \co{NULL}
포인터의 타입을 가져오기 위해 \co{get_nulls_value()} (이 함수는 인자의
\co{NULL} 포인터 식별자를 리턴합니다) 를 사용할 수 있습니다.
\co{get_nulls_value()} 가 예상치 않은 값을 리턴하면, 읽기 쓰레드는 올바른
행동을 취할 수 있는데, 예를 들면 시작부터 순회를 재시작 하는 것입니다.

\iffalse

A variant of Linux's \co{hlist}, named \co{hlist_nulls}, provides multiple
distinct \co{NULL} pointers, but otherwise uses the same layout as shown in
Figure~\ref{fig:defer:Linux Linear Linked List (hlist)}.
In this variant, a \co{->next} pointer having a zero low-order bit is
considered to be a pointer.
However, if the low-order bit is set to one, the upper bits identify
the type of \co{NULL} pointer.
This type of list is used to allow lockless readers to detect when a
node has been moved from one list to another.
For example, each bucket of a hash table might use its index to mark
its \co{NULL} pointer.
Should a reader encounter a \co{NULL} pointer not matching the index of
the bucket it started from, that reader knows that an element it was
traversing was moved to some other bucket during the traversal, taking
that reader with it.
The reader can use the \co{is_a_nulls()} function (which returns true
if passed an \co{hlist_nulls} \co{NULL} pointer) to determine when
it reaches the end of a list, and the \co{get_nulls_value()} function
(which returns its argument's \co{NULL}-pointer identifier) to fetch
the type of \co{NULL} pointer.
When \co{get_nulls_value()} returns an unexpected value, the reader
can take corrective action, for example, restarting its traversal from
the beginning.

\fi

\QuickQuiz{
	하지만 \co{hlist_nulls} 읽기 쓰레드가 다른 버킷으로 갔다가 다시
	돌아오면 어떻게 하죠?

	\iffalse

	But what if an \co{hlist_nulls} reader gets moved to some other
	bucket and then back again?

	\fi

}\QuickQuizAnswer{
	이를 제어하기 위한 한가지 방법은 항상 노드를 목적 버켓의 시작점으로
	옮겨서 읽기 쓰레드가 매치 되는 \co{NULL} 포인터를 갖는 리스트의 끝에
	닿았을 때에는 이 쓰레드가 이 리스트 전체를 탐색했음을 보장하는
	것입니다.

	물론, 해쉬 테이블에 버킷당 많은 원소가 있고 너무 많은 옮기기
	오퍼레이션이 존재한다면 읽기 쓰레드는 영영 리스트의 끝에 도달할 수 없을
	수도 있을 겁니다.
	평범한 경우에 이를 막는 한가지 방법은 해쉬 테이블을 잘 튜닝되어서 짧은
	리스트들만을 갖게 하는 것입니다.
	이 문제를 파악하고 처리하는 한가지 방법은 읽기 쓰레드가 어떤 많은 수의
	노드를 순회한 다음에는 탐색을 종료하고 update-side 락을 잡은 후 탐색을
	다시 하는 것이지만 이는 데드락을 초래할 수 있습니다.
	이 문제를 완전히 막는 또다른 방법은 읽기 쓰레드가 RCU read-side
	크리티컬 섹션 내에서 탐색을 하고, 이어지는 업데이트들 사이의 RCU grace
	period 하나를 기다리는 겁니다.
	중간의 위치는 어떤 적당한 $N$ 값을 가지고 매 $N$ 업데이트마다 RCU grace
	period 를 하나 기다릴 겁니다.

	\iffalse

	One way to handle this is to always move nodes to the beginning
	of the destination bucket, ensuring that when the reader reaches
	the end of the list having a matching \co{NULL} pointer, it will
	have searched the entire list.

	Of course, if there are too many move operations in a hash table
	with many elements per bucket, the reader might never reach the
	end of a list.
	One way of avoiding this in the common case is to keep hash
	tables well-tuned, thus with short lists.
	One way of detecting the problem and handling it is for the
	reader to terminate the search after traversing some large
	number of nodes, acquire the update-side lock, and redo the
	search, but this might introduce deadlocks.
	Another way of avoiding the problem entirely is for readers to
	search within RCU read-side critical sections, and to wait for
	an RCU grace period between successive updates.
	An intermediate position might wait for an RCU grace period
	every $N$ updates, for some suitable value of $N$.

	\fi

}\QuickQuizEnd

\co{hlist_nulls} 에 대한 더 많은 정보는 \path{rculist_nulls.rst} 파일 (오래된
커널에선 \path{rculist_nulls.txt}) 에서 제공되는 도움 될만한 예제 코드들과 함께
리눅스 커널의 소스 트리에서 얻을 수 있습니다.

리눅스의 \co{hlist} 의 또다른 변종은 비트 락킹을 결합하며, \co{hlist_bl} 이라
이름지어졌습니다.
이 변종은
Figure~\ref{fig:defer:Linux Linear Linked List (hlist)}
에 보인 것과 동일한 레이아웃을 사용하지만, 헤드 포인터의 (그림의 ``첫번째'')
낮은 위치 비트를 이 리스트를 잠그기 위해 사용됩니다.
이 방법 또한 메모리 사용량을 줄이는데, 이게 아니면 포인터 자체와 함께 별도의
스핀락이 저장되어야 하기 때문입니다.

\iffalse

More information on \co{hlist_nulls} is available in the Linux-kernel
source tree, with helpful example code provided in the
\path{rculist_nulls.rst} file (\path{rculist_nulls.txt} in older kernels).

Another variant of Linux's \co{hlist} incorporates bit-locking,
and is named \co{hlist_bl}.
This variant uses the same layout as shown in
Figure~\ref{fig:defer:Linux Linear Linked List (hlist)},
but reserves the low-order bit of the head pointer (``first'' in the
figure) to lock the list.
This approach also reduces memory usage, as it allows what would otherwise
be a separate spinlock to be stored with the pointer itself.

\fi

\begin{sidewaystable*}[htbp]
\rowcolors{1}{}{lightgray}
\renewcommand*{\arraystretch}{1.3}
\centering
\caption{RCU-Protected List APIs}
\label{tab:defer:RCU-Protected List APIs}
\footnotesize
\newlength{\cwa}\newlength{\cwb}\newlength{\cwc}\newlength{\cwd}
\IfNimbusAvail{
  \renewcommand{\ttdefault}{NimbusMonoN}
  \setlength{\cwa}{1.9in}\setlength{\cwb}{2.1in}
  \setlength{\cwc}{1.8in}\setlength{\cwd}{1.6in}
}{
  \setlength{\cwa}{1.95in}\setlength{\cwb}{2.15in}
  \setlength{\cwc}{1.9in}\setlength{\cwd}{1.7in}
}
\begin{tabular}{>{\raggedright\arraybackslash}p{\cwa}
    >{\raggedright\arraybackslash}p{\cwb}
    >{\raggedright\arraybackslash}p{\cwc}
    >{\raggedright\arraybackslash}p{\cwd}}
\toprule
\pmb{\tco{list}}: Circular doubly linked list &
    \pmb{\tco{hlist}}: Linear doubly linked list &
	\pmb{\tco{hlist_nulls}}: Linear doubly linked list with marked
	NULL pointer, with up to 31~bits of marking &
	    \pmb{\tco{hlist_bl}}: Linear doubly linked list with bit locking \\
\midrule
\multicolumn{4}{l}{{\bf Structures}} \\
\tco{struct list_head} &
    \tco{struct}{\tt ~}\tco{hlist_head} ~~~~~~~~~~~~~~
    \tco{struct}{\tt ~}\tco{hlist_node} &
	\tco{struct}{\tt ~}\tco{hlist_nulls_head}
	\tco{struct}{\tt ~}\tco{hlist_nulls_node} &
	    \tco{struct}{\tt ~}\tco{hlist_bl_head}
	    \tco{struct}{\tt ~}\tco{hlist_bl_node} \\
\multicolumn{4}{l}{{\bf Initialization}} \\
&
    \tco{INIT_LIST_HEAD_RCU()} &
	&
	    \\
\multicolumn{4}{l}{{\bf Full traversal}} \\
\tco{list_for_each_entry_rcu()}
\tco{list_for_each_entry_lockless()} &
    \tco{hlist_for_each_entry_rcu()}
    \tco{hlist_for_each_entry_rcu_bh()}
    \tco{hlist_for_each_entry_rcu_notrace()} &
	\tco{hlist_nulls_for_each_entry_rcu()}
	\tco{hlist_nulls_for_each_entry_safe()} &
	    \tco{hlist_bl_for_each_entry_rcu()} \\
\multicolumn{4}{l}{{\bf Resume traversal}} \\
\tco{list_for_each_entry_continue_rcu()}
\tco{list_for_each_entry_from_rcu()} &
    \tco{hlist_for_each_entry_continue_rcu()}
    \tco{hlist_for_each_entry_continue_rcu_bh()}
    \tco{hlist_for_each_entry_from_rcu()} &
	&
	    \\
\multicolumn{4}{l}{{\bf Stepwise traversal}} \\
\tco{list_entry_rcu()}
\tco{list_entry_lockless()}
\tco{list_first_or_null_rcu()}
\tco{list_next_rcu()}
\tco{list_next_or_null_rcu()} &
    \multicolumn{1}{p{1.2in}}{\tco{hlist_first_rcu()}
			      \tco{hlist_next_rcu()}
			      \tco{hlist_pprev_rcu()}} &
	\tco{hlist_nulls_first_rcu()}
	\tco{hlist_nulls_next_rcu()} &
	    \tco{hlist_bl_first_rcu()} \\
\multicolumn{4}{l}{{\bf Add}} \\
\multicolumn{1}{p{1.2in}}{\tco{list_add_rcu()}
			  \tco{list_add_tail_rcu()}} &
    \tco{hlist_add_before_rcu()}
    \tco{hlist_add_behind_rcu()}
    \tco{hlist_add_head_rcu()}
    \tco{hlist_add_tail_rcu()} &
	\tco{hlist_nulls_add_head_rcu()} &
	    \tco{hlist_bl_add_head_rcu()}
	    \tco{hlist_bl_set_first_rcu()} \\
\multicolumn{4}{l}{{\bf Delete}} \\
\tco{list_del_rcu()} &
    \multicolumn{1}{p{1.2in}}{\tco{hlist_del_rcu()}
			      \tco{hlist_del_init_rcu()}} &
	\tco{hlist_nulls_del_rcu()}
	\tco{hlist_nulls_del_init_rcu()} &
	    \tco{hlist_bl_del_rcu()}
	    \tco{hlist_bl_del_init_rcu()} \\
\multicolumn{4}{l}{{\bf Replace}} \\
\tco{list_replace_rcu()} &
    \tco{hlist_replace_rcu()} &
	&
	    \\
\multicolumn{4}{l}{{\bf Splice}} \\
\tco{list_splice_init_rcu()} &
    \tco{list_splice_tail_init_rcu()} &
	&
	    \\
\bottomrule
\end{tabular}
\end{sidewaystable*}

이 링크드 리스트 변종을 위한 API 멤버들이
Table~\ref{tab:defer:RCU-Protected List APIs}
에 요약되어 있습니다.
더 많은 정보는 리눅스 커널 소스 트리의 \path{Documentation/RCU} 디렉토리와
Linux Weekly News~\cite{PaulEMcKenney2019RCUAPI} 에서 얻을 수 있습니다.

하지만, 이 섹션의 나머지 부분은 \co{list_replace_rcu()} 의 사용에 대해 확장해
보는데, 이 API 멤버가 RCU 에게 그 이름을 주었기 때문입니다.
이 API 멤버는 여러 필드를 가지는 이 리스트의 중간에 있는 원소가 원자적으로
업데이트 되어서 어떤 읽기 쓰레드든 기존의 값들의 집합 또는 새로운 값들의 집합을
봐야 하지, 그 두 집합의 섞인 값을 보지는 않아야 하는 복잡한 업데이트를 행하기
위해 사용됩니다.
예를 들어, 링크드 리스트의 각 노드는 정수 필드 \co{->a}, \co{->b},
그리고~\co{->c} 를 가질 수도 있으며, 그 값을 5, 6, 그리고~7 에서 5, 2, 그리고~3
으로 각각 업데이트 해야 할 수도 있습니다.

이 원자적 업데이트를 행하는 코드 구현은 단순합니다:

\iffalse

The API members for these linked-list variants are summarized in
Table~\ref{tab:defer:RCU-Protected List APIs}.
More information is available in the \path{Documentation/RCU}
directory of the Linux-kernel source tree and at
Linux Weekly News~\cite{PaulEMcKenney2019RCUAPI}.

However, the remainder of this section expands on the use of
\co{list_replace_rcu()}, given that this API member gave RCU its name.
This API member is used to carry out more complex updates in which an
element in the middle of the list having multiple fields is atomically
updated, so that a given reader sees either the old set of values or
the new set of values, but not a mixture of the two sets.
For example, each node of a linked list might have integer fields
\co{->a}, \co{->b}, and~\co{->c}, and it might be necessary to update
a given node's fields from 5, 6, and~7 to 5, 2, and~3, respectively.

The code implementing this atomic update is straightforward:

\fi

\begin{fcvlabel}[ln:defer:Canonical RCU Replacement Example (2nd)]
\begin{VerbatimN}[samepage=true,commandchars=\\\[\],firstnumber=15]
q = kmalloc(sizeof(*p), GFP_KERNEL);	\lnlbl[kmalloc]
*q = *p;				\lnlbl[copy]
q->b = 2;				\lnlbl[update1]
q->c = 3;				\lnlbl[update2]
list_replace_rcu(&p->list, &q->list);	\lnlbl[replace]
synchronize_rcu();			\lnlbl[sync_rcu]
kfree(p);				\lnlbl[kfree]
\end{VerbatimN}
\end{fcvlabel}

\begin{figure}[tbp]
\centering
\resizebox{2.7in}{!}{\includegraphics{defer/RCUReplacement}}
\caption{RCU Replacement in Linked List}
\label{fig:defer:RCU Replacement in Linked List}
\end{figure}

다음 이야기는 이 코드를 따라가는데, 그 상태 변화를 그리기 위해
Figure~\ref{fig:defer:RCU Replacement in Linked List} 를 사용합니다.
각 원소의 세 값은 필드 \co{->a}, \co{->b}, 그리고~\co{->c} 를 각각 나타냅니다.
빨간 색으로 칠해진 원소들은 읽기 쓰레드에 의해 참조될 수도 있으며, 읽기
쓰레드는 업데이트 쓰레드와 직접적으로 동기화 하지 않기 때문에, 읽기 쓰레드는 이
전체 교체 과정과 동시에 수행될 수도 있습니다.
뒤쪽으로의 포인터와 tail 로부터 head 로의 링크는 간략화를 위해 생략되었음을
알아두시기 바랍니다.

포인터 \co{p} 를 포함해 이 리스트의 처음 상태는 앞의 삭제 예와 동일한데, 이
그림의 첫번째 행에 보여져 있습니다.

다음의 텍스트는 어떤 읽기 쓰레드든 이 두 값들 중 하나만을 보게 하는 방식으로
\co{5,6,7} 원소를 \co{5,2,3} 으로 교체하는지 설명합니다.

\iffalse

The following discussion walks through this code, using
Figure~\ref{fig:defer:RCU Replacement in Linked List} to illustrate
the state changes.
The triples in each element represent the values of fields \co{->a},
\co{->b}, and~\co{->c}, respectively.
The red-shaded elements might be referenced by readers,
and because readers do not synchronize directly with updaters,
readers might run concurrently with this entire replacement process.
Please note that backwards pointers and the link from the tail to the
head are omitted for clarity.

The initial state of the list, including the pointer \co{p},
is the same as for the deletion example, as shown on the
first row of the figure.

The following text describes how to replace the \co{5,6,7} element
with \co{5,2,3} in such a way that any given reader sees one of these
two values.

\fi

\begin{fcvref}[ln:defer:Canonical RCU Replacement Example (2nd)]
라인~\lnref{kmalloc} 은 교체 원소를 할당하여,
Figure~\ref{fig:defer:RCU Replacement in Linked List} 의 두번째 행에 이릅니다.
이 지점에서, 어떤 읽기 쓰레드도 새로 할당된 원소로의 참조를 잡지 못하며 (초록색
색깔로 표시되어 있습니다), 초기화 되어 있지 않습니다 (물음표로 표시되어
있습니다).

라인~\lnref{copy} 는 기존 원소를 새 원소로 복사하여,
Figure~\ref{fig:defer:RCU Replacement in Linked List} 의 세번째 행의 상태에
이릅니다.
새로 할당된 원소는 여전히 읽기 쓰레드에 의해 참조될 수 없지만, 이제 초기화
되었습니다.

Figure~\ref{fig:defer:RCU Replacement in Linked List} 의 네번째 행에 보여져
있듯 라인~\lnref{update1} 은 \co{q->b} 의 값을 ``2'' 로, 그리고
라인~\lnref{update2} 는 \co{q->c} 의 값을 ``3'' 으로 업데이트 합니다.
새로 할당된 구조체는 여전히 읽기 쓰레드에게 액세스 될 수 없음을 생각해 주시기
바랍니다.

\iffalse

\begin{fcvref}[ln:defer:Canonical RCU Replacement Example (2nd)]
Line~\lnref{kmalloc} allocates a replacement element,
resulting in the state as shown in the second row of
Figure~\ref{fig:defer:RCU Replacement in Linked List}.
At this point, no reader can hold a reference to the newly allocated
element (as indicated by its green shading), and it is uninitialized
(as indicated by the question marks).

Line~\lnref{copy} copies the old element to the new one, resulting in the
state as shown in the third row of
Figure~\ref{fig:defer:RCU Replacement in Linked List}.
The newly allocated element still cannot be referenced by readers, but
it is now initialized.

Line~\lnref{update1} updates \co{q->b} to the value ``2'', and
line~\lnref{update2} updates \co{q->c} to the value ``3'',
as shown on the fourth row of
Figure~\ref{fig:defer:RCU Replacement in Linked List}.
Note that the newly allocated structure is still inaccessible to readers.

\fi

이제, 라인~\lnref{replace} 는 새 원소가 마침내 읽기 쓰레드에게 보여질 수 있게끔
교체를 행하며, 따라서
Figure~\ref{fig:defer:RCU Replacement in Linked List} 의 다섯번째 행에 보여져
있듯 빨갛게 칠해집니다.
이 시점에서는 아래에서 보여지듯 우리는 이 리스트의 두 버전을 갖게 됩니다.
기존부터 존재해온 읽기 쓰레드는 \co{5,6,7} 원소를 볼 수도 있지만 (따라서
노란색으로 칠해져 있습니다), 새로운 읽기 쓰레드는 \co{5,2,3} 원소를 볼
것입니다.
하지만 모든 읽기 쓰레드는 이 값들 중 하나만 보지, 두 값들의 혼합된 것을 보지는
못할 것이 보장됩니다.

\iffalse

Now, line~\lnref{replace} does the replacement, so that the new element is
finally visible to readers, and hence is shaded red, as shown on
the fifth row of
Figure~\ref{fig:defer:RCU Replacement in Linked List}.
At this point, as shown below, we have two versions of the list.
Pre-existing readers might see the \co{5,6,7} element (which is
therefore now shaded yellow), but
new readers will instead see the \co{5,2,3} element.
But any given reader is guaranteed to see one set of values or the
other, not a mixture of the two.

\fi

라인~\lnref{sync_rcu} 의 \co{synchronize_rcu()} 가 리턴한 후에는 하나의 grace
period 가 지나갔을 것이며, 따라서 \co{list_replace_rcu()} 전에 시작된 모든
읽기는 완료되었을 것입니다.
특히, \co{5,6,7} 원소로의 참조를 가지고 있던 모든 읽기 쓰레드는 그들의 RCU
read-side 크리티컬 섹션이 종료되었을 것이 보장되며, 따라서 참조를 계속 쥐고
있는 것이 금지됩니다.
따라서, 더이상 기존 값으로의 참조를 쥐고 있는 읽기 쓰레드는 더이상 존재하지
않는데,
Figure~\ref{fig:defer:RCU Replacement in Linked List} 의 여섯번째 행에
초록색으로 칠함으로써 표시되어 있습니다.
읽기 쓰레드들의 관점에서, 우린 이제 새 원소가 기존 원소를 교체한, 리스트의
단일한 버전으로 돌아왔습니다.

라인~\lnref{kfree} 의 \co{kfree()} 가 완료된 후, 이 리스트는
Figure~\ref{fig:defer:RCU Replacement in Linked List} 의 마지막 행에 보여진
것과 같아질 것입니다.
\end{fcvref}

\iffalse

After the \co{synchronize_rcu()} on line~\lnref{sync_rcu} returns,
a grace period will have elapsed, and so all reads that started before the
\co{list_replace_rcu()} will have completed.
In particular, any readers that might have been holding references
to the \co{5,6,7} element are guaranteed to have exited
their RCU read-side critical sections, and are thus prohibited from
continuing to hold a reference.
Therefore, there can no longer be any readers holding references
to the old element, as indicated its green shading in the sixth row of
Figure~\ref{fig:defer:RCU Replacement in Linked List}.
As far as the readers are concerned, we are back to having a single version
of the list, but with the new element in place of the old.

After the \co{kfree()} on line~\lnref{kfree} completes, the list will
appear as shown on the final row of
Figure~\ref{fig:defer:RCU Replacement in Linked List}.
\end{fcvref}

\fi

RCU 가 이 교체의 경우 후에 이름지어지긴 했으나, 리눅스 커널에서의 주요하고 많은
RCU 사용 예는
Section~\ref{sec:defer:Maintain Multiple Versions of Recently Updated Objects}
의
Figure~\ref{fig:defer:Multiple RCU Data-Structure Versions}
에 보여진 것처럼 간단하고 비의존적인 삽입과 삭제에 기반해 있습니다.

다음 섹션은 RCU 를 사용하는 코드의 디버깅을 하는 개발자들을 돕는 API 들을
봅니다.

\iffalse

Despite the fact that RCU was named after the replacement case,
the vast majority of RCU usage within the Linux kernel relies on
the simple independent insertion and deletion, as was shown in
Figure~\ref{fig:defer:Multiple RCU Data-Structure Versions} in
Section~\ref{sec:defer:Maintain Multiple Versions of Recently Updated Objects}.

The next section looks at APIs that assist developers in debugging
their code that makes use of RCU\@.

\fi

\subsubsection{RCU Has Diagnostic APIs}
\label{sec:defer:RCU Has Diagnostic APIs}

\begin{table}[tb]
\renewcommand*{\arraystretch}{1.15}
\footnotesize
\centering
\begin{tabular}{ll}
\toprule
Category &
	Primitives \\
\midrule
Mark RCU pointer &
	\tco{__rcu} \\
\midrule
Debug-object support &
	\tco{init_rcu_head()} \\
&	\tco{destroy_rcu_head()} \\
&	\tco{init_rcu_head_on_stack()} \\
&	\tco{destroy_rcu_head_on_stack()} \\
\midrule
Stall-warning control &
	\tco{rcu_cpu_stall_reset()} \\
\midrule
Callback checking &
	\tco{rcu_head_init()} \\
&	\tco{rcu_head_after_call_rcu()} \\
\midrule
lockdep support &
	\tco{rcu_read_lock_held()} \\
&	\tco{rcu_read_lock_bh_held()} \\
&	\tco{rcu_read_lock_sched_held()} \\
&	\tco{srcu_read_lock_held()} \\
&	\tco{rcu_is_watching()} \\
&	\tco{RCU_LOCKDEP_WARN()} \\
&	\tco{RCU_NONIDLE()} \\
&	\tco{rcu_sleep_check()} \\
\bottomrule
\end{tabular}
\caption{RCU Diagnostic APIs}
\label{tab:defer:RCU Diagnostic APIs}
\end{table}

Table~\ref{tab:defer:RCU Diagnostic APIs}
은 RCU 의 진단 API 들을 보입니다.

\co{__rcu} 는 RCU 로 보호된 포인터를 표시하는데, 예를 들면 \qtco{struct foo
__rcu *p;} 와 같은 형태입니다.
\co{rcu_dereference()} 로 넘겨질 수도 있는 포인터들이 그렇게 표시될 수
있습니다만, \co{rcu_dereference()} 로부터 리턴된 값을 잡고 있는 포인터들은
그러지 않아야 합니다.
변수들에 이 표시를 제공함으로써, 구조체 필드, 함수 패러미터, 그리고 리턴 값들은
리눅스 커널의 \co{sparse} 툴이 RCU 로 보호되는 포인터들이 평범한 C 언어 로드와
스토어를 통해 부적절하게 액세스 되는 상황을 파악할 수 있게 합니다.

\iffalse

Table~\ref{tab:defer:RCU Diagnostic APIs}
shows RCU's diagnostic APIs.

The \co{__rcu} marks an RCU-protected pointer, for example,
\qtco{struct foo __rcu *p;}.
Pointers that might be passed to \co{rcu_dereference()} can be marked,
but pointers holding values returned from \co{rcu_dereference()}
should not be.
Providing these markings on variables, structure fields, function
parameters, and return values allow the Linux kernel's \co{sparse}
tool to detect situtations where RCU-protected pointers are
incorrectly accessed using plain C-language loads and stores.

\fi

객체 디버깅 지원은 리눅스 커널의 메모리 할당자로부터 얻어진 구조체의 부분인
모든 \co{rcu_head} 구조체에 대해 자동적입니다만, 각자의 특수 목적 메모리
할당자를 사용하는 경우에는 할당과 해제 시에 각각 \co{init_rcu_head()} 와
\co{destroy_rcu_head()} 를 사용할 수 있습니다.
함수 호출 스택에서 할당된 \co{rcu_head} 구조체를 사용하는 경우엔 (그런 경우가
있습니다!) 첫번째 사용 전에 \co{init_rcu_head_on_stack()} 을, 마지막 사용 후,
그러나 해당 함수에서 리턴하기 전에 \co{destroy_rcu_head_on_stack()} 을 사용할
수 있습니다.
객체 디버깅 지원은 동일한 \co{rcu_head} 구조체를 \co{call_rcu()} 와 그
친구들에게 빠르게 잇달아 보냄으로써 이중 메모리 해제 종류의 메모리 할당 버그의
\co{call_rcu()} 버전의 버그를 감지할 수 있게 합니다.

Stall 경고 제어는 \co{rcu_cpu_stall_reset()} 을 통해 제공되는데, 호출자가 현재
grace period 의 남은 시간 동안 RCU CPU stall 경고를 멈출 수 있게 합니다.
RCU CPU stall 경고는 어떤 RCU read-side 크리티컬 섹션이 지나치게 긴 시간 동안
수행되는 상황을 집어낼 수 있게 하며, 커널 디버거와 같은 것들이 예를 들면
breakpoint 를 만났거나 하는 경우엔 이 경고를 끌 수 있게 하는게 유용합니다.

\iffalse

Debug-object support is automatic for any \co{rcu_head} structures
that are part of a structure obtained from the Linux kernel's
memory allocators, but those building their own special-purpose
memory allocators can use \co{init_rcu_head()} and \co{destroy_rcu_head()}
at allocation and free time, respectively.
Those using \co{rcu_head} structures allocated on the function-call
stack (it happens!) may use \co{init_rcu_head_on_stack()}
before first use and \co{destroy_rcu_head_on_stack()} after last use,
but before returning from the function.
Debug-object support allows detection of bugs involving passing the
same \co{rcu_head} structure to \co{call_rcu()} and friends in
quick succession, which is the \co{call_rcu()} counterpart to the
infamous double-free class of memory-allocation bugs.

Stall-warning control is provided by \co{rcu_cpu_stall_reset()}, which
allows the caller to suppress RCU CPU stall warnings for the remainder
of the current grace period.
RCU CPU stall warnings help pinpoint situations where an RCU read-side
critical section runs for an excessive length of time, and it is useful
for things like kernel debuggers to be able to suppress them, for example,
when encountering a breakpoint.

\fi

Callback 검사는 \co{rcu_head_init()} 와 \co{rcu_head_after_call_rcu()} 를 통해
제공됩니다.
앞의 것은 \co{rcu_head} 구조체가 \co{call_rcu()} 에 넘겨지기 전에 호출되며,
그러면 \co{rcu_head_after_call_rcu()} 는 해당 callback 이 명시된 함수와 함께
수행되었는지 검사할 겁니다.

Lockdep~\cite{JonathanCorbet2006lockdep} 지원은
\co{rcu_read_lock_held()},
\co{rcu_read_lock_bh_held()},
\co{rcu_read_lock_sched_held()}, 그리고
\co{srcu_read_lock_held()} 를 포함하는데, 이것들 각각은 연관된 종류의 RCU
read-side 크리티컬 섹션 내에서 호출되었다면 \co{true} 를 리턴합니다.

\iffalse

Callback checking is provided by \co{rcu_head_init()} and
\co{rcu_head_after_call_rcu()}.
The former is invoked on an \co{rcu_head} structure before it is passed
to \co{call_rcu()}, and then \co{rcu_head_after_call_rcu()} will
check to see if the callback is has been invoked with the specified
function.

Support for lockdep~\cite{JonathanCorbet2006lockdep} includes
\co{rcu_read_lock_held()},
\co{rcu_read_lock_bh_held()},
\co{rcu_read_lock_sched_held()}, and
\co{srcu_read_lock_held()},
each of which returns \co{true} if invoked within the corresponding
type of RCU read-side critical section.

\fi

\QuickQuiz{
	Tasks RCU 를 위한 \co{rcu_read_lock_tasks_held()} 는 왜 존재하지
	않나요?

	\iffalse

	Why isn't there a \co{rcu_read_lock_tasks_held()} for Tasks RCU?

	\fi

}\QuickQuizAnswer{
	Tasks RCU 는 read-side 마커가 존재하지 않기 때문입니다.
	대신, Tasks RCU read-side 크리티컬 섹션은 자발적 컨텍스트 스위치 사이로
	제한됩니다.

	\iffalse

	Because Tasks RCU does not have read-side markers.
	Instead, Tasks RCU read-side critical sections are
	bounded by voluntary context switches.

	\fi

}\QuickQuizEnd

\co{rcu_read_lock()} 은 idle 루프 내에서 사용될 수 없으며 전력 효율에 대한
고민이 idle 루프를 상당히 화려하게 만든 이유로, \co{rcu_is_watching()} 은
\co{rcu_read_lock()} 의 사용이 허용되는 컨텍스트에서 호출되었을 때 \co{true} 를
리턴합니다.
\co{srcu_read_lock()} 은 idle 에서도, 심지어 오프라인된 CPU 에서도 사용될 수
있으며, 이는 \co{rcu_is_watching()} 이 SRCU 에는 적용되지 않음을 의미함을 다시
주의하시기 바랍니다.

\co{RCU_LOCKDEP_WARN()} 은 lockdep 이 활성화 되어 있으며 그 인자가 \co{true} 로
평가될 때 경고를 냅니다.
예를 들어, \co{RCU_LOCKDEP_WARN(!rcu_read_lock_held())} 는 RCU read-side
크리티컬 섹션 바깥에서 호출되었을 때 경고를 낼 겁니다.

\co{RCU_NONIDLE()} 은 RCU 를 완전한 인자로 전달된 명령문이 수행될 때를 보게
강제할 수 있습니다.
예를 들어, \co{RCU_NONIDLE(WARN_ON(!rcu_is_watching()))} 은 결코 경고를 내지
않을 겁니다.
하지만, 2020--2021 사이의 변화는 RCU 의 범위를 idle 루프 내까지 확장시켰으며,
이는 \co{RCU_NONIDLE()} 의 필요를 크게 줄이거나 심지어 제거할 수 있을 겁니다.

마지막으로, \co{rcu_sleep_check()} 는 RCU, RCU-bh, 또는 RCU-sched read-side
크리티컬 섹션 내에서 호출되었을 때 경고를 냅니다.

\iffalse

Because \co{rcu_read_lock()} cannot be used from the idle loop,
and because energy-efficiency concerns have caused the idle loop
to become quite ornate, \co{rcu_is_watching()} returns \co{true} if
invoked in a context where use of \co{rcu_read_lock()} is legal.
Note again that \co{srcu_read_lock()} may be used from idle and
even offline CPUs, which means that \co{rcu_is_watching()} does not
apply to SRCU\@.

\co{RCU_LOCKDEP_WARN()} emits a warning if lockdep is enabled and if
its argument evaluated to \co{true}.
For example, \co{RCU_LOCKDEP_WARN(!rcu_read_lock_held())} would emit a
warning if invoked outside of an RCU read-side critical section.

\co{RCU_NONIDLE()} may be used to force RCU to watch when executing
the statement that is passed in as the sole argument.
For example, \co{RCU_NONIDLE(WARN_ON(!rcu_is_watching()))}
would never emit a warning.
However, changes in the 2020--2021 timeframe extend RCU's reach deeper
into the idle loop, which should greatly reduce or even eliminate the
need for \co{RCU_NONIDLE()}.

Finally,  \co{rcu_sleep_check()} emits a warning if invoked within
an RCU, RCU-bh, or RCU-sched read-side critical section.

\fi

\subsubsection{Where Can RCU's APIs Be Used?}
\label{sec:defer:Where Can RCU's APIs Be Used?}

\begin{figure}[tb]
\centering
\resizebox{3in}{!}{\includegraphics{defer/RCUenvAPI}}
\caption{RCU API Usage Constraints}
\label{fig:defer:RCU API Usage Constraints}
\end{figure}

Figure~\ref{fig:defer:RCU API Usage Constraints}
shows which APIs may be used in which in-kernel environments.
The RCU read-side primitives may be used in any environment, including NMI,
the RCU mutation and asynchronous grace-period primitives may be used in any
environment other than NMI, and, finally, the RCU synchronous grace-period
primitives may be used only in process context.
The RCU list-traversal primitives include \co{list_for_each_entry_rcu()},
\co{hlist_for_each_entry_rcu()}, etc.
Similarly, the RCU list-mutation primitives include
\co{list_add_rcu()}, \co{hlist_del_rcu()}, etc.

Note that primitives from other families of RCU may be substituted,
for example, \co{srcu_read_lock()} may be used in any context
in which \co{rcu_read_lock()} may be used.

\subsubsection{So, What \emph{is} RCU Really?}
\label{sec:defer:So, What is RCU Really?}

At its core, RCU is nothing more nor less than an API that supports
publication and subscription for insertions, waiting for all RCU readers
to complete, and maintenance of multiple versions.
That said, it is possible to build higher-level constructs
on top of RCU, including the reader-writer-locking, reference-counting,
and existence-guarantee constructs listed in
Section~\ref{sec:defer:RCU Usage}.
Furthermore, I have no doubt that the Linux community will continue to
find interesting new uses for RCU,
just as they do for any of a number of synchronization
primitives throughout the kernel.

Of course, a more-complete view of RCU would also include
all of the things you can do with these APIs.

However, for many people, a complete view of RCU must include sample
RCU implementations.
Appendix~\ref{chp:app:``Toy'' RCU Implementations} therefore presents a series
of ``toy'' RCU implementations of increasing complexity and capability,
though others might prefer the classic
``User-Level Implementations of Read-Copy
Update''~\cite{MathieuDesnoyers2012URCU}.
For everyone else, the next section gives an overview of some RCU use cases.

% defer/rcuusage.tex
% mainfile: ../perfbook.tex
% SPDX-License-Identifier: CC-BY-SA-3.0

\subsection{RCU Usage}
\label{sec:defer:RCU Usage}
\OriginallyPublished{Section}{sec:defer:RCU Usage}{RCU Usage}{Linux Weekly News}{PaulEMcKenney2008WhatIsRCUUsage}

\begin{table}[tb]
\renewcommand*{\arraystretch}{1.2}
\centering
\small
\begin{tabular}{ll}
\toprule
Mechanism RCU Replaces & Section \\
\midrule
Reader-writer locking &
	Section~\ref{sec:defer:RCU is a Reader-Writer Lock Replacement} \\
Restricted reference-counting &
	Section~\ref{sec:defer:RCU is a Restricted Reference-Counting Mechanism} \\
Bulk reference-counting &
	Section~\ref{sec:defer:RCU is a Bulk Reference-Counting Mechanism} \\
Garbage collector &
	Section~\ref{sec:defer:RCU is a Poor Man's Garbage Collector} \\
Multi-version concurrency control &
	Section~\ref{sec:defer:RCU is an MVCC} \\
Existence Guarantees &
	Section~\ref{sec:defer:RCU Provides Existence Guarantees} \\
Type-Safe Memory &
	Section~\ref{sec:defer:RCU Provides Type-Safe Memory} \\
Wait for things to finish &
	Section~\ref{sec:defer:RCU is a Way of Waiting for Things to Finish} \\
\bottomrule
\end{tabular}
\caption{RCU Usage}
\label{tab:defer:RCU Usage}
\end{table}

이 섹션은 ``RCU 는 무엇인가?'' 라는 질문을 RCU 가 무엇을 제공할 수 있는지
사용의 관점에서 답해봅니다.
RCU 는 존재하는 메커니즘을 교체하는데 가장 자주 사용되므로,
Table~\ref{tab:defer:RCU Usage}
에 보인 것처럼 우린 그런 메커니즘과의 관계에 주로 집중해서 RCU 를 알아 봅니다.
이 표의 섹션들에 이어서
Section~\ref{sec:defer:RCU Usage Summary} 은 요약을 제공합니다.

\iffalse

This section answers the question ``What is RCU?'' from the viewpoint
of the uses to which RCU can be put.
Because RCU is most frequently used to replace some existing mechanism,
we look at it primarily in terms of its relationship to such mechanisms,
as listed in Table~\ref{tab:defer:RCU Usage}.
Following the sections listed in this table,
Section~\ref{sec:defer:RCU Usage Summary} provides a summary.

\fi

\subsubsection{RCU for Pre-BSD Routing}
\label{sec:defer:RCU for Pre-BSD Routing}

Listing~\ref{lst:defer:RCU Pre-BSD Routing Table Lookup}
과~\ref{lst:defer:RCU Pre-BSD Routing Table Add/Delete}
는 RCU 로 보호되는 Pre-BSD 라우팅 테이블의 코드를 보입니다
(\path{route_rcu.c}).
앞의 것은 데이터 구조와 \co{route_lookup()} 을, 뒤의 것은 \co{route_add()} 와
\co{route_del()} 을 보입니다.

\iffalse

Listings~\ref{lst:defer:RCU Pre-BSD Routing Table Lookup}
and~\ref{lst:defer:RCU Pre-BSD Routing Table Add/Delete}
show code for an RCU-protected Pre-BSD routing table
(\path{route_rcu.c}).
The former shows data structures and \co{route_lookup()},
and the latter shows \co{route_add()} and \co{route_del()}.

\fi

\begin{listing}[tbp]
\input{CodeSamples/defer/route_rcu@lookup.fcv}
\caption{RCU Pre-BSD Routing Table Lookup}
\label{lst:defer:RCU Pre-BSD Routing Table Lookup}
\end{listing}

\begin{listing}[tbp]
\input{CodeSamples/defer/route_rcu@add_del.fcv}
\caption{RCU Pre-BSD Routing Table Add/Delete}
\label{lst:defer:RCU Pre-BSD Routing Table Add/Delete}
\end{listing}

\begin{fcvref}[ln:defer:route_rcu:lookup]
Listing~\ref{lst:defer:RCU Pre-BSD Routing Table Lookup} 에서, 라인~\lnref{rh}
는 RCU 교체에 사용되는 \co{->rh} 필드를 더하고 라인~\lnref{re_freed} 는
use-after-free 검사 필드인 \co{->re_freed} 를 더하며, 라인~\lnref{lock},
\lnref{unlock1}, 그리고~\lnref{unlock2} 는 RCU read-side 보호를,
라인~\lnref{chk_freed} 와~\lnref{abort} 는 use-after-free 검사를 더합니다.
\end{fcvref}
\begin{fcvref}[ln:defer:route_rcu:add_del]
Listing~\ref{lst:defer:RCU Pre-BSD Routing Table Add/Delete} 에서,
라인~\lnref{add:lock}, \lnref{add:unlock}, \lnref{del:lock},
라인~\lnref{del:unlock1}, 그리고~\lnref{del:unlock2} 는 update-side 락킹을
더하고 라인~\lnref{add:add_rcu} 와~\lnref{del:del_rcu} 는 RCU update-side
보호를 더하고, 라인~\lnref{del:call_rcu}  는 \co{route_cb()} 가 하나의 grace
period 가 지난 후 호출되게 하며 \clnrefrange{cb:b}{cb:e} 는 \co{route_cb()} 를
정의합니다.
이는 동작하는 동시적 구현을 위한 최소한의 추가된 코드입니다.
\end{fcvref}

\iffalse

\begin{fcvref}[ln:defer:route_rcu:lookup]
In Listing~\ref{lst:defer:RCU Pre-BSD Routing Table Lookup},
line~\lnref{rh} adds the \co{->rh} field used by RCU reclamation,
line~\lnref{re_freed} adds the \co{->re_freed} use-after-free-check field,
lines~\lnref{lock}, \lnref{unlock1}, and~\lnref{unlock2}
add RCU read-side protection,
and lines~\lnref{chk_freed} and~\lnref{abort} add the use-after-free check.
\end{fcvref}
\begin{fcvref}[ln:defer:route_rcu:add_del]
In Listing~\ref{lst:defer:RCU Pre-BSD Routing Table Add/Delete},
lines~\lnref{add:lock}, \lnref{add:unlock}, \lnref{del:lock},
\lnref{del:unlock1}, and~\lnref{del:unlock2} add update-side locking,
lines~\lnref{add:add_rcu} and~\lnref{del:del_rcu} add RCU update-side protection,
line~\lnref{del:call_rcu} causes \co{route_cb()} to be invoked after
a grace period elapses,
and \clnrefrange{cb:b}{cb:e} define \co{route_cb()}.
This is minimal added code for a working concurrent implementation.
\end{fcvref}

\fi

\begin{figure}[tb]
\centering
\resizebox{2.5in}{!}{\includegraphics{CodeSamples/defer/perf-rcu}}
\caption{Pre-BSD Routing Table Protected by RCU}
\label{fig:defer:Pre-BSD Routing Table Protected by RCU}
\end{figure}

Figure~\ref{fig:defer:Pre-BSD Routing Table Protected by RCU}
는 읽기 전용 워크로드에서의 성능을 보입니다.
RCU 는 상당히 잘 확장되며 거의 이상적인 성능을 제공합니다.
하지만, 이 데이터는 \co{rcu_read_lock()} 과 \co{rcu_read_unlock()} 에 작은 양의
코드를 생성하는 userspace
RCU~\cite{MathieuDesnoyers2009URCU,PaulMcKenney2013LWNURCU} 의 \co{RCU_SIGNAL}
버전을 사용해 만들어졌습니다.
\co{rcu_read_lock()} 과 \co{rcu_read_unlock()} 에 어떤 코드도 생성하지 않는
QSBR 버전의 RCU 를 사용하면 어떻게 될까요?
(RCU QSBR 에 대한 논의를 위해선
Section~\ref{sec:defer:Introduction to RCU},
그리고 특히
Figure~\ref{fig:defer:QSBR: Waiting for Pre-Existing Readers} 를 보시기
바랍니다.)

이에 대한 답이 RCU QSBR 의 성능과 확장성은 실제로 이상적인 동기화 없는
워크로드의 것을 넘어섬을 보이는
Figure~\ref{fig:defer:Pre-BSD Routing Table Protected by RCU QSBR}
에 보여져 있습니다.

\iffalse

Figure~\ref{fig:defer:Pre-BSD Routing Table Protected by RCU}
shows the performance on the read-only workload.
RCU scales quite well, and offers nearly ideal performance.
However, this data was generated using the \co{RCU_SIGNAL}
flavor of userspace
RCU~\cite{MathieuDesnoyers2009URCU,PaulMcKenney2013LWNURCU},
for which \co{rcu_read_lock()} and \co{rcu_read_unlock()}
generate a small amount of code.
What happens for the QSBR flavor of RCU, which generates no code at all
for \co{rcu_read_lock()} and \co{rcu_read_unlock()}?
(See Section~\ref{sec:defer:Introduction to RCU},
and especially
Figure~\ref{fig:defer:QSBR: Waiting for Pre-Existing Readers},
for a discussion of RCU QSBR\@.)

The answer to this is shown in
Figure~\ref{fig:defer:Pre-BSD Routing Table Protected by RCU QSBR},
which shows that RCU QSBR's performance and scalability actually exceeds
that of the ideal synchronization-free workload.

\fi

\QuickQuizSeries{%
\QuickQuizB{
	잠깐요, 뭐라고요???
	어떻게 RCU QSBR 이 이상적인 것보다 나을 수 있죠?
	어떤 쓰레기 같은 이상에 대한 정의가 그것이 모든 가능한 결과 중 최선이
	아니게 할 수 있죠???

	\iffalse

	Wait, what???
	How can RCU QSBR possibly be better than ideal?
	Just what rubbish definition of ideal would fail to be the best
	of all possible results???

	\fi

}\QuickQuizAnswerB{
	훌륭한 질문이고, 이에 대한 답은 현대의 CPU 와 컴파일러는 굉장히
	복잡하다는 것입니다.
	하지만 거기까지 가기 전에, RCU QSBR 의 성능 이득은 코어당 하나의
	하드웨어 쓰레드가 제공되는 세계에서만 나타남을 알아둘 가치가 있습니다.
	시스템에 부하가 완전히 걸리고 나면, RCU QSBR 의 성능은 이상적인 것의
	수준으로 떨어집니다.

	RCU 버전의 \co{route_lookup()} 탐색 반복문은 실제 순차적 버전의 것보다
	x86 명령을 하나더 가지는데, 이는 \co{cmp}, \co{je}, \co{mov}, \co{cmp},
	\co{lea}, and \co{jne} 순서에서의 \co{lea} 입니다.
	이 추가된 명령은 RCU 버전의 \co{route_entry} 구조체의 시작지점에 있는
	\co{rcu_head} 구조체 때문으로, 이 때문에 순차적 버전과 달리 RCU 버전의
	\co{->re_next.next} 포인터는 0이 아닌 오프셋을 갖습니다.
	1980년대로 돌아가 보면, 이 추가적인 \co{lea} 명령은 RCU 버전이 느려지는
	결과를 낼 수도 있었습니다만, 우린 지금 21세기에 있으며, 1980년대는 한참
	지났습니다.

	\iffalse

	This is an excellent question, and the answer is that modern
	CPUs and compilers are extremely complex.
	But before getting into that, it is well worth noting that
	RCU QSBR's performance advantage appears only in the
	one-hardware-thread-per-core regime.
	Once the system is fully loaded, RCU QSBR's performance drops
	back to ideal.

	The RCU variant of the \co{route_lookup()} search loop actually
	has one more x86 instruction than does the sequential version,
	namely the \co{lea} in the sequence
	\co{cmp}, \co{je}, \co{mov}, \co{cmp}, \co{lea}, and \co{jne}.
	This extra instruction is due to the \co{rcu_head} structure
	at the beginning of the RCU variant's \co{route_entry} structure,
	so that, unlike the sequential variant, the RCU variant's
	\co{->re_next.next} pointer has a non-zero offset.
	Back in the 1980s, this additional \co{lea} instruction might
	have reliably resulted in the RCU variant being slower, but we
	are now in the 21\textsuperscript{st} century, and the 1980s
	are long gone.

	\fi

	하지만
	\cref{sec:cpu:Pipelined CPUs}
	를 주의 깊게 읽은 여러분들은 이미 이 모든 걸 알고 있을 겁니다!

	이런 반직관적인 결과는 물론 현대 마이크로프로세서에서의 모든 성능
	결과는 약간 회의적으로 보아져야 함을 의미합니다.
	이론상, 이상적인 것보다 나은 성능을 얻는다는 것은 정말 말이 안됩니다만,
	현대의 마이크로프로세서에서는 정말로 일어날 수 있는 일입니다.
	그런 결과는 축복받은 초선형적 성능향상 (그런 예 중 하나를 위해
	Section~\ref{sec:SMPdesign:Beyond Partitioning} 을 보세요) 비슷한 걸로
	생각 될 수 있는데, 즉, 흥미롭지만 또한 실용적 중요도는 제한된다는
	것입니다.
	그러나, RCU 의 강점 중 하나는 그 읽기 쪽 오버헤드가 무척 낮아서 이와
	같은 작은 효과가 실제 성능 측정에도 보여질 수 있다는 것입니다.

	\iffalse

	But those of you who read
	\cref{sec:cpu:Pipelined CPUs}
	carefully already knew all of this!

	These counter-intuitive results of course means that any
	performance result on modern microprocessors must be subject to
	some skepticism.
	In theory, it really does not make sense to obtain performance
	results that are better than ideal, but it really can happen
	on modern microprocessors.
	Such results can be thought of as similar to the celebrated
	super-linear speedups (see
	Section~\ref{sec:SMPdesign:Beyond Partitioning}
	for one such example), that is, of interest but also of limited
	practical importance.
	Nevertheless, one of the strengths of RCU is that its read-side
	overhead is so low that tiny effects such as this one are visible
	in real performance measurements.

	\fi

\begin{figure}[tb]
\centering
\resizebox{2.5in}{!}{\includegraphics{CodeSamples/defer/perf-rcu-qsbr-qq}}
\caption{Pre-BSD Routing Table Protected by RCU QSBR With Non-Initial \tco{rcu_head}}
\label{fig:defer:Pre-BSD Routing Table Protected by RCU QSBR With Non-Initial rcu-head}
\end{figure}

	이는 \co{rcu_head} 구조체가 \co{->re_next.next} 포인터가 0 오프셋을
	갖게끔 옮겨져서 순차적 버전과 동일하게 되면 어떻게 되는지 질문을 하게
	만듭니다.
	그에 대한 답은
	Figure~\ref{fig:defer:Pre-BSD Routing Table Protected by RCU QSBR With Non-Initial rcu-head}
	에 보여져 있듯, RCU QSBR 의 성능이 여전히 이상적인 것에 근접하지만
	이상적인 것을 넘어서지는 않게 한다는 것입니다.

	\iffalse

	This raises the question as to what would happen if the
	\co{rcu_head} structure were to be moved so that RCU's
	\co{->re_next.next} pointer also had zero offset, just the
	same as the sequential variant.
	And the answer, as can be seen in
	Figure~\ref{fig:defer:Pre-BSD Routing Table Protected by RCU QSBR With Non-Initial rcu-head},
	is that this causes RCU QSBR's performance to decrease to where
	it is still very nearly ideal, but no longer super-ideal.

	\fi

}\QuickQuizEndB
%
\QuickQuizE{
	RCU QSBR 의 읽기 성능이 그렇게 좋은데, 왜 다른 userspace 버전을
	신경쓰죠?

	\iffalse

	Given RCU QSBR's read-side performance, why bother with any
	other flavor of userspace RCU?

	\fi

}\QuickQuizAnswerE{
	RCU QSBR 은 허용될 수 없을 수도 있는 어플리케이션 전체의 제약을
	요구하기 때문인데, 예를 들면 어플리케이션 내의 모든 각각의 쓰레드가
	정기적으로 quiescent state 를 지나야 한다는 것입니다.
	다른 것들 중에서도, 이는 RCU QSBR 이 다른 종류의 userspace
	RCU~\cite{PaulMcKenney2013LWNURCU} 에 의해 더 잘 도움받을 수도 있는
	라이브러리 작성자에게는 도움이 될 수 있다는 것을 의미합니다.

	\iffalse

	Because RCU QSBR places constraints on the overall application
	that might not be tolerable,
	for example, requiring that each and every thread in the
	application regularly pass through a quiescent state.
	Among other things, this means that RCU QSBR is not helpful
	to library writers, who might be better served by other
	flavors of userspace RCU~\cite{PaulMcKenney2013LWNURCU}.

	\fi

}\QuickQuizEndE
}

\begin{figure}[tb]
\centering
\resizebox{2.5in}{!}{\includegraphics{CodeSamples/defer/perf-rcu-qsbr}}
\caption{Pre-BSD Routing Table Protected by RCU QSBR}
\label{fig:defer:Pre-BSD Routing Table Protected by RCU QSBR}
\end{figure}

\subsubsection{RCU is a Reader-Writer Lock Replacement}
\label{sec:defer:RCU is a Reader-Writer Lock Replacement}

리눅스 커널에서의 RCU 의 가장 흔한 사용처는 아마도 읽기가 집중적인 상황에서의
reader-writer 락킹 대체일 겁니다.
그러나, 이런 RCU 사용은 처음에 제게 즉각적으로 적절하게 느껴지지 않았고, 실제로
저는 1990년대 초에 범용 RCU 구현을 만들기 전에 가벼운 reader-writer 락을
구현하기로 했습니다~\cite{WilsonCHsieh92a}.\footnote{
	2.4 리눅스 커널의 \co{brlock} 과, 더 최신의 리눅스 커널의 \co{lglock}
	과 비슷합니다.}
이 가벼운 reader-writer 락으로 제가 하고자 했던 모든 것은 결국 RCU 를 이용해
구현되었습니다.
사실, 그건 이 가벼운 reader-writer 락이 처음 사용되기 3년이나 전의
일이었습니다.
정말 바보가 된 것 같았죠!

RCU 와 reader-writer 락킹 사이의 핵심 유사성은 둘 다 병렬로 구생될 수 있는
read-side 크리티컬 섹션을 갖는다는 것입니다.
실제로, 어떤 경우에는 RCU API 멤버들을 연관된 reader-writer 락 API 멤버들로
대체하는게 가능합니다.
하지만 무엇보다, 왜 그런걸 신경쓰죠?

\iffalse

Perhaps the most common use of RCU within the Linux kernel is as
a replacement for reader-writer locking in read-intensive situations.
Nevertheless, this use of RCU was not immediately apparent to me
at the outset, in fact, I chose to implement a lightweight reader-writer
lock~\cite{WilsonCHsieh92a}\footnote{
	Similar to \co{brlock} in the 2.4 Linux kernel and to
	\co{lglock} in more recent Linux kernels.}
before implementing a general-purpose RCU implementation
back in the early 1990s.
Each and every one of the uses I envisioned for the lightweight reader-writer
lock was instead implemented using RCU\@.
In fact, it was more than
three years before the lightweight reader-writer lock saw its first use.
Boy, did I feel foolish!

The key similarity between RCU and reader-writer locking is that
both have read-side critical sections that can execute in parallel.
In fact, in some cases, it is possible to mechanically substitute RCU API
members for the corresponding reader-writer lock API members.
But first, why bother?

\fi

RCU 의 장점은 성능, 데드락 내성, 그리고 리얼타임 응답시간을 포함합니다.
물론, RCU 의 제한점들도 존재하는데 읽기 쓰레드와 업데이트 쓰레드가 동시에
수행된다는 사실, 낮은 우선순위 RCU 읽기 쓰레드가 높은 우선순위 쓰레드를 grace
period 하나가 지나갈 때까지 블록시킨다는 것, 그리고 grace period 응답시간이 수
밀리세컨드까지 길어질 수 있다는 점이 포함됩니다.
이런 장점과 한계점들이 다음 문단들에서 논의됩니다.

\iffalse

Advantages of RCU include performance,
deadlock immunity, and realtime latency.
There are, of course, limitations to RCU, including the fact that
readers and updaters run concurrently, that low-priority RCU readers
can block high-priority threads waiting for a grace period to elapse,
and that grace-period latencies can extend for many milliseconds.
These advantages and limitations are discussed in the following paragraphs.

\fi

\paragraph{Performance}

\begin{figure}[tb]
\centering
\resizebox{2.5in}{!}{\includegraphics{defer/rwlockRCUperf}}
\caption{Performance Advantage of RCU Over Reader-Writer Locking}
\label{fig:defer:Performance Advantage of RCU Over Reader-Writer Locking}
\end{figure}

리눅스 커널 RCU 의 reader-writer 락킹 대비 읽기 성능 장점이
448 개 2.10\,GHz Intel x86 CPU 시스템에서 측정되어 만들어진
Figure~\ref{fig:defer:Performance Advantage of RCU Over Reader-Writer Locking}
에 보여져 있습니다.

\iffalse

The read-side performance advantages of Linux-kernel RCU over
reader-writer locking are shown in
Figure~\ref{fig:defer:Performance Advantage of RCU Over Reader-Writer Locking},
which was generated on a 448-CPU 2.10\,GHz Intel x86 system.

\fi

\QuickQuizSeries{%
\QuickQuizB{
	뭐요?
	2.10\,GHz 에서의 클락 시간은 약 500\,picosecond 인데 RCU 가
	300-picosecond 도 안되는 오버헤드를 갖는다는 말을 나더러 믿으라구요?

	\iffalse

	WTF?
	How the heck do you expect me to believe that RCU can have less
	than a 300-picosecond overhead when the clock period at 2.10\,GHz
	is almost 500\,picoseconds?

	\fi

}\QuickQuizAnswerB{
	우선, 이 측정을 위해 사용된 반복문이 다음과 같음을 고려합시다:

	\iffalse

	First, consider that the inner loop used to
	take this measurement is as follows:

	\fi

\begin{VerbatimN}
	for (i = nloops; i >= 0; i--) {
		rcu_read_lock();
		rcu_read_unlock();
	}
\end{VerbatimN}

	다음으로, 실질적인 \co{rcu_read_lock()} 과 \co{rcu_read_unlock()} 의
	정의를 생각해 봅시다:

	\iffalse

	Next, consider the effective definitions of \co{rcu_read_lock()}
	and \co{rcu_read_unlock()}:

	\fi

\begin{VerbatimN}
#define rcu_read_lock()   barrier()
#define rcu_read_unlock() barrier()
\end{VerbatimN}

	이 정의들은 메모리 참조에 연관되는 컴파일러의 코드 이동 최적화를
	제약하지만, 그것들 자체 내에는 어떤 명령도 넣지 않습니다.
	하지만, 이 반복문 변수가 레지스터에 관리된다면, \co{i} 로의 액세스는
	메모리 참조로 취급되지 않습니다.
	더욱이, 컴파일러는 loop unrolling 을 알 수 있어, 결과적인 코드가 단순히
	\co{i} 값을 1보다 큰 어떤 값으로 증가시키는 것으로 여러 횟수의 반복문
	수행을 ``해내는'' 코드를 만들 수 있습니다.

	따라서 267 picoseconds 의 ``측정'' 은 시간 측정을 \co{rcu_read_lock()}
	과 \co{rcu_read_unlock()} 호출을 포함하는 이 내부 반복문을 통과하는
	반복 횟수로 나눈 것에 이 loop-unrolling 으로 \co{i} 조정 코드를 나눈
	것을 더한 고정된 오버헤드입니다.
	그리고 그래서, 이 측정은 실제로 에러를 포함하고 있으며, 실제로
	오버헤드를 수십 수백배로 부풀려서 이야기 하고 있습니다.
	어쨌건, 만들어진 기계 명령어의 관점에서, \co{rcu_read_lock()} 과
	\co{rcu_read_unlock()} 의 오버헤드는 정확히 제로입니다.

	267 picosecond 의 측정된 시간이 과장된 수치임이 드러나는 건 매일 있는
	일은 분명 아닙니다!

	\iffalse

	These definitions constrain compiler code-movement optimizations
	involving memory references, but emit no instructions in and
	of themselves.
	However, if the loop variable is maintained in a register,
	the accesses to \co{i} will not count as memory references.
	Furthermore, the compiler can do loop unrolling,
	allowing the resulting code to ``execute'' multiple passes
	through the loop body simply by incrementing \co{i} by
	some value larger than the value 1.

	So the ``measurement'' of 267 picoseconds is simply the fixed
	overhead of the timing measurements divided by the number of
	passes through the inner loop containing the calls
	to \co{rcu_read_lock()} and \co{rcu_read_unlock()}, plus
	the code to manipulate \co{i} divided by the loop-unrolling
	factor.
	And therefore, this measurement really is in error, in fact,
	it exaggerates the overhead by an arbitrary number of orders
	of magnitude.
	After all, in terms of machine instructions emitted, the actual
	overheads of \co{rcu_read_lock()} and of \co{rcu_read_unlock()}
	are each precisely zero.

	It certainly is not just every day that a timing measurement
	of 267 picoseconds turns out to be an overestimate!

	\fi
}\QuickQuizEndB

\QuickQuizM{
	이 책의 초기 버전에서는 RCU 읽기 오버헤드가 1 picosecond 미만이라 하지
	않았나요?
	무슨 일이 있었던 거죠???

	\iffalse

	Didn't an earlier release of this book show RCU read-side
	overhead way down in the sub-picosecond range?
	What happened???

	\fi

}\QuickQuizAnswerM{
	훌륭한 기억력이군요!!!
	어떤 초기 버전에서의 오버헤드는 실제로 약 100 femtosecond 이었습니다.

	그사이 무슨 일이 있었느냐면, RCU 사용처가 리눅스 커널 내에 더욱 널리
	퍼졌는데, page fault 처리 코드도 여기 포함됩니다.
	그때로 돌아가면, \co{rcu_read_lock()} 과 \co{rcu_read_unlock()} 은
	\co{CONFIG_PREEMPT=n} 커널에서 완전히 아무 일도 하지 않았습니다.
	불행히도, 그 상황은 컴파일러가 page fault 되는 메모리 액세스를 RCU
	read-side 크리티컬 섹션 내로 재배치 할 수 있게 했습니다.
	물론, page fault 는 블록될 수 있으며, 따라서 이 크리티컬 섹션을
	망가지게 합니다.

	\iffalse

	Excellent memory!!!
	The overhead in some early releases was in fact roughly
	100~femtoseconds.

	What happened was that RCU usage spread more broadly through the
	Linux kernel, including into code that takes page faults.
	Back at that time, \co{rcu_read_lock()} and \co{rcu_read_unlock()}
	were complete no-ops in \co{CONFIG_PREEMPT=n} kernels.
	Unfortunately, that situation allowed the compiler to reorder
	page-faulting memory accesses into RCU read-side critical
	sections.
	Of course, page faults can block, which destroys those critical
	section.

	\fi

	이건 이론적 문제만이 아니었습니다:
	그로 인한 문제가 실제로 2019년에 이야기 되었습니다.
	\ppl{Herber}{Xu} 는 이 문제를 분석했고 \ppl{Linus}{Torvalds} 는
	\co{rcu_read_lock()} 과 \co{rcu_read_unlock()} 이 무조건적으로
	\co{barrier()} 호출을 포함하게 하는 커밋을 대기열에
	올렸습니다~\cite{LinusTorvalds2019:RCUreader.barrier}.
	그리고 \co{barrier()} 가 코드를 추가하진 않지만, 컴파일러 최적화를
	제한합니다.
	따라서 널리 퍼져있는 RCU 사용처의 비용은 \co{rcu_read_lock()} 과
	\co{rcu_read_unlock()} 오버헤드보다 약간 더 높아집니다.

	물론, 더 오래전의 결과는
	Figure~\ref{fig:defer:Performance Advantage of RCU Over Reader-Writer Locking}
	에서 보인 것과 다른 시스템에서 얻어진 겁니다.
	그래서 뭐가 더 영향을 끼쳤을까요, Linus 의 커밋 또는 시스템 변경?
	이 질문은 독자 여러분의 연습문제로 남겨둡니다.

	\iffalse

	Nor was this a theoretical problem:
	A failure actually manifested in 2019.
	\ppl{Herbert}{Xu} tracked down this failure down and
	\ppl{Linus}{Torvalds}
	therefore queued a commit to upgrade \co{rcu_read_lock()} and
	\co{rcu_read_unlock()} to unconditionally include a call to
	\co{barrier()}~\cite{LinusTorvalds2019:RCUreader.barrier}.
	And although \co{barrier()} emits no code, it does constrain
	compiler optimizations.
	And so the price of widespread RCU usage is slightly higher
	\co{rcu_read_lock()} and \co{rcu_read_unlock()} overhead.

	Of course, it is also the case that the older results were obtained
	on a different system than were those shown in
	Figure~\ref{fig:defer:Performance Advantage of RCU Over Reader-Writer Locking}.
	So which change had the most effect, Linus's commit or the change in
	the system?
	This question is left as an exercise to the reader.

	\fi

}\QuickQuizEndM

\QuickQuizE{
	Figure~\ref{fig:defer:Performance Advantage of RCU Over Reader-Writer Locking}
	의 \co{rcu} 에는 왜그리 큰 편차가 존재하는 거죠?

	\iffalse

	Why is there such large variation for the \co{rcu} trace in
	Figure~\ref{fig:defer:Performance Advantage of RCU Over Reader-Writer Locking}?

	\fi

}\QuickQuizAnswerE{
	이는 log-log 그림이며, 따라서 크게 보이는 \co{rcu} 편차는 실제로는 수백
	picosecond 에 불과함을 명심하세요.
	그리고 그건 무엇이든 그것을 일으킬 수 있는 무척 짧은 시간입니다.
	하지만, 편차가 CPU 수가 작든 크든 감소된다는 걸로 보아, 한가지 가능한
	가설은 이 편차는 하나의 CPU 에서 다른 CPU 로의 migration 때문이라는
	것입니다.

	그래요, 이 측정은 인터럽트가 불능화 된 채로 취해졌습니다만, 게스트 OS
	내에서 취해졌으므로, 하이퍼바이저 단계에서의 preemption 은 가능합니다.
	이 게스트 OS 들을 리얼타임 우선순위로 수행함으로써 이런 편차를 줄이려는
	노력은 (Joel Fernandes 에 의해 제안되었습니다) 독자 여러분의 연습문제로
	남겨두겠습니다.

	\iffalse

	Keep in mind that this is a log-log plot, so those large-seeming
	\co{rcu} variances in reality span only a few hundred picoseconds.
	And that is such a short time that anything could cause it.
	However, given that the variance decreases with both small and
	large numbers of CPUs, one hypothesis is that the variation is
	due to migrations from one CPU to another.

	Yes, these measurements were taken with interrupts disabled, but
	they were also taken within a guest OS, so that preemption was
	still possible at the hypervisor level.
	Attempting to reduce these variations by running the guest OSes
	at real-time priority (as suggested by Joel Fernandes) is left
	as an exercise for the reader.

	\fi

}\QuickQuizEndE
}                 % End of \QuickQuizSeries

Reader-writer 락킹은 단일 CPU 에서 RCU 보다 수십배 느리며, 192~CPU 에서는
\emph{수만} 배보다 더 느리다는 것을 보시기 바랍니다.
대조적으로, RCU 는 상당히 잘 확장합니다.
두 경우 모드, 에러 바들은 30회의 측정 결과 전체 범위를 보이며, 선은 그 중간값을
보입니다.

\iffalse

Note that reader-writer locking is more than an order of magnitude slower
than RCU on a single CPU, and is more than \emph{four} orders of magnitude
slower on 192~CPUs.
In contrast, RCU scales quite well.
In both cases, the error bars cover the full range of the measurements
from 30~runs, with the line being the median.

\fi

\begin{figure}[tb]
\centering
\resizebox{2.5in}{!}{\includegraphics{defer/rwlockRCUperfPREEMPT}}
\caption{Performance Advantage of Preemptible RCU Over Reader-Writer Locking}
\label{fig:defer:Performance Advantage of Preemptible RCU Over Reader-Writer Locking}
\end{figure}

더 정확한 그림은 \co{CONFIG_PREEMPT} 커널에서 얻어질 수 있겠지만, 448rodml
2.10\,GHz x86 CPU 시스템에서 측정된
Figure~\ref{fig:defer:Performance Advantage of Preemptible RCU Over Reader-Writer Locking}
에 보이듯 RCU 는 여전히 단일 CPU 에서 약 일곱배, 192~CPU 에서 수천배
reader-writer 락보다 빠릅니다.
큰 수의 CPU 에서 reader-writer 락킹의 높은 편차값을 보세요.
에러바는 데이터의 전체 범위를 그립니다.

\iffalse

A more moderate view may be obtained from a \co{CONFIG_PREEMPT} kernel,
though RCU still beats reader-writer locking by between a factor of seven
on a single CPU and by three orders of magnitude on 192~CPUs, as shown in
Figure~\ref{fig:defer:Performance Advantage of Preemptible RCU Over Reader-Writer Locking},
which was generated on the same 448-CPU 2.10\,GHz x86 system.
Note the high variability of reader-writer locking at larger numbers of CPUs.
The error bars span the full range of data.

\fi

\QuickQuiz{
	시스템은 448 개의 하드웨어 쓰레드를 갖는데, 왜 192~CPU 까지만
	측정하나요?

	\iffalse

	Given that the system had no fewer than 448~hardware threads,
	why only 192~CPUs?

	\fi

}\QuickQuizAnswer{
	이 데이터를 생성하는데 사용된 스크립트 (\path{rcusclae.sh}) 는 모아지는
	지점들의 각 집합을 위해 게스트 운영체제를 시작하며, 이 특정
	시스템에서는 \co{qemu} 와 KVM 이 모두 특정 게스트 OS 에 설정될 수 있는
	CPU 갯수에 제한을 두기 때문입니다.
	그래요, 조금 더 많은 CPU 를 사용하는 것도 가능하겠습니다만 256 은
	불가능하다는 점을 생각할 때 이진수 관점에서 192 는 괜찮은 어림수입니다.

	\iffalse

	Because the script (\path{rcuscale.sh}) that generates this data
	spawn a guest operating system for each set of points gathered,
	and on this particular system, both \co{qemu} and KVM limit the
	number of CPUs that may be configured into a given guest OS\@.
	Yes, it would have been possible to run a few more CPUs, but
	192 is a nice round number from a binary perspective, given
	that 256 is infeasible.

	\fi
}\QuickQuizEnd

\begin{figure}[tb]
\centering
\resizebox{2.5in}{!}{\includegraphics{defer/rwlockRCUperfwt}}
\caption{Comparison of RCU to Reader-Writer Locking as Function of Critical-Section Duration, 192 CPUs}
\label{fig:defer:Comparison of RCU to Reader-Writer Locking as Function of Critical-Section Duration}
\end{figure}

Of course, the low performance of reader-writer locking in
\cref{fig:defer:Performance Advantage of RCU Over Reader-Writer Locking,%
fig:defer:Performance Advantage of Preemptible RCU Over Reader-Writer Locking}
is exaggerated by the unrealistic zero-length critical sections.
The performance advantages of RCU decrease as the overhead of the critical
sections increase.
This decrease can be seen in
Figure~\ref{fig:defer:Comparison of RCU to Reader-Writer Locking as Function of Critical-Section Duration},
which was run on the same system as the previous plots.
Here, the y-axis represents the sum of the overhead of the read-side
primitives and that of the critical section and the x-axis represents
the critical-section overhead in nanoseconds.
But please note the logscale y~axis, which means that the small
separations between the traces still represent significant differences.
This figure shows non-preemptible RCU, but given that preemptible RCU's
read-side overhead is only about three nanoseconds, its plot would be
nearly identical to
Figure~\ref{fig:defer:Comparison of RCU to Reader-Writer Locking as Function of Critical-Section Duration}.

\QuickQuiz{
	Why the larger error ranges for the submicrosecond durations in
	Figure~\ref{fig:defer:Comparison of RCU to Reader-Writer Locking as Function of Critical-Section Duration}?
}\QuickQuizAnswer{
	Because smaller disturbances result in greater relative errors
	for smaller measurements.
	Also, the Linux kernel's \co{ndelay()} nanosecond-scale primitive
	is (as of 2020) less accurate than is the \co{udelay()} primitive
	used for the data for durations of a microsecond or more.
	It is instructive to compare to the zero-length case shown in
	Figure~\ref{fig:defer:Performance Advantage of RCU Over Reader-Writer Locking}.
}\QuickQuizEnd

There are three traces for reader-writer locking, with the upper trace
being for 100~CPUs, the next for 10~CPUs, and the lowest for 1~CPU\@.
So the greater the number of CPUs and the shorter the critical sections,
the greater is RCU's performance advantage.
These performance advantages are underscored by the fact that 100-CPU
systems are no longer uncommon and that a number of system calls (and
thus any RCU read-side critical sections that they contain) complete
within a microsecond.

In addition, as is discussed in the next paragraph,
RCU read-side primitives are almost entirely deadlock-immune.


\paragraph{Deadlock Immunity}

Although RCU offers significant performance advantages for
read-mostly workloads, one of the primary reasons for creating
RCU in the first place was in fact its immunity to read-side
deadlocks.
This immunity stems from the fact that
RCU read-side primitives do not block, spin, or even
do backwards branches, so that their execution time is deterministic.
It is therefore impossible for them to participate in a deadlock
cycle.

\QuickQuiz{
	Is there an exception to this deadlock immunity, and if so,
	what sequence of events could lead to deadlock?
}\QuickQuizAnswer{
	One way to cause a deadlock cycle involving
	RCU read-side primitives is via the following (illegal) sequence
	of statements:

\begin{VerbatimU}
rcu_read_lock();
synchronize_rcu();
rcu_read_unlock();
\end{VerbatimU}

	The \co{synchronize_rcu()} cannot return until all
	pre-existing RCU read-side critical sections complete, but
	is enclosed in an RCU read-side critical section that cannot
	complete until the \co{synchronize_rcu()} returns.
	The result is a classic self-deadlock---you get the same
	effect when attempting to write-acquire a reader-writer lock
	while read-holding it.

	Note that this self-deadlock scenario does not apply to
	RCU QSBR, because the context switch performed by the
	\co{synchronize_rcu()} would act as a quiescent state
	for this CPU, allowing a grace period to complete.
	However, this is if anything even worse, because data used
	by the RCU read-side critical section might be freed as a
	result of the grace period completing.

	In short, do not invoke synchronous RCU update-side primitives
	from within an RCU read-side critical section.
}\QuickQuizEnd

An interesting consequence of RCU's read-side deadlock immunity is
that it is possible to unconditionally upgrade an RCU
reader to an RCU updater.
Attempting to do such an upgrade with reader-writer locking results
in deadlock.
A sample code fragment that does an RCU read-to-update upgrade follows:

\begin{VerbatimN}[samepage=true]
rcu_read_lock();
list_for_each_entry_rcu(p, &head, list_field) {
	do_something_with(p);
	if (need_update(p)) {
		spin_lock(my_lock);
		do_update(p);
		spin_unlock(&my_lock);
	}
}
rcu_read_unlock();
\end{VerbatimN}

Note that \co{do_update()} is executed under
the protection of the lock \emph{and} under RCU read-side protection.

Another interesting consequence of RCU's deadlock immunity is its
immunity to a large class of priority inversion problems.
For example, low-priority RCU readers cannot prevent a high-priority
RCU updater from acquiring the update-side lock.
Similarly, a low-priority RCU updater cannot prevent high-priority
RCU readers from entering an RCU read-side critical section.

\QuickQuiz{
	Immunity to both deadlock and priority inversion???
	Sounds too good to be true.
	Why should I believe that this is even possible?
}\QuickQuizAnswer{
	It really does work.
	After all, if it didn't work, the Linux kernel would not run.
}\QuickQuizEnd

\paragraph{Realtime Latency}

Because RCU read-side primitives neither spin nor block, they offer
excellent realtime latencies.
In addition, as noted earlier, this means that they are
immune to priority inversion
involving the RCU read-side primitives and locks.

However, RCU is susceptible to more subtle priority-inversion scenarios,
for example, a high-priority process blocked waiting for an RCU
grace period to elapse can be blocked by low-priority RCU readers
in \rt\ kernels.
This can be solved by using RCU priority
boosting~\cite{PaulEMcKenney2007BoostRCU,DinakarGuniguntala2008IBMSysJ}.

\paragraph{RCU Readers and Updaters Run Concurrently}

Because RCU readers never spin nor block, and because updaters are not
subject to any sort of rollback or abort semantics, RCU readers and
updaters must necessarily run concurrently.
This means that RCU readers might access stale data, and might even
see inconsistencies, either of which can render conversion from reader-writer
locking to RCU non-trivial.

\begin{figure}[tb]
\centering
\resizebox{3in}{!}{\includegraphics{defer/rwlockRCUupdate}}
\caption{Response Time of RCU vs. Reader-Writer Locking}
\label{fig:defer:Response Time of RCU vs. Reader-Writer Locking}
\end{figure}

However, in a surprisingly large number of situations, inconsistencies and
stale data are not problems.
The classic example is the networking routing table.
Because routing updates can take considerable time to reach a given
system (seconds or even minutes), the system will have been sending
packets the wrong way for quite some time when the update arrives.
It is usually not a problem to continue sending updates the wrong
way for a few additional milliseconds.
Furthermore, because RCU updaters can make changes without waiting for
RCU readers to finish,
the RCU readers might well see the change more quickly than would
batch-fair
reader-writer-locking readers, as shown in
Figure~\ref{fig:defer:Response Time of RCU vs. Reader-Writer Locking}.

Once the update is received, the rwlock writer cannot proceed until the
last reader completes, and subsequent readers cannot proceed until the
writer completes.
However, these subsequent readers are guaranteed to see the new value,
as indicated by the green shading of the rightmost boxes.
In contrast, RCU readers and updaters do not block each other, which permits
the RCU readers to see the updated values sooner.
Of course, because their execution overlaps that of the RCU updater,
\emph{all} of the RCU readers might well see updated values, including
the three readers that started before the update.
Nevertheless only the green-shaded rightmost RCU readers
are \emph{guaranteed} to see the updated values.

Reader-writer locking and RCU simply provide different guarantees.
With reader-writer locking, any reader that begins after the writer begins
is guaranteed to see new values, and any reader that attempts to
begin while the writer is spinning might or might not see new values,
depending on the reader/writer preference of the rwlock implementation in
question.
In contrast, with RCU, any reader that begins after the updater completes
is guaranteed to see new values, and any reader that completes after the
updater begins might or might not see new values, depending on timing.

The key point here is that, although reader-writer locking does
indeed guarantee consistency within the confines of the computer system,
there are situations where this consistency comes at the price of
increased \emph{inconsistency} with the outside world.
In other words, reader-writer locking obtains internal consistency at the
price of silently stale data with respect to the outside world.

Nevertheless, there are situations where inconsistency and stale
data within the confines of the system cannot be tolerated.
Fortunately,
there are a number of approaches that avoid inconsistency and stale
data~\cite{PaulEdwardMcKenneyPhD,Arcangeli03}, and some
methods based on reference counting are discussed in
Section~\ref{sec:defer:Reference Counting}.

\paragraph{Low-Priority RCU Readers Can Block High-Priority Reclaimers}

In Realtime RCU~\cite{DinakarGuniguntala2008IBMSysJ} or
SRCU~\cite{PaulEMcKenney2006c},
a preempted reader will prevent a grace period from completing, even if
a high-priority task is blocked waiting for that grace period to complete.
Realtime RCU can avoid this problem by substituting \co{call_rcu()}
for \co{synchronize_rcu()} or by using RCU priority
boosting~\cite{PaulEMcKenney2007BoostRCU,DinakarGuniguntala2008IBMSysJ},
which is still in experimental status as of early 2008.
It might become necessary to augment SRCU and QRCU with priority boosting,
but not before a clear real-world need is demonstrated.

\paragraph{RCU Grace Periods Extend for Many Milliseconds}

With the exception of userspace
RCU~\cite{MathieuDesnoyers2009URCU,PaulMcKenney2013LWNURCU},
expedited grace periods, and several of the ``toy''
RCU implementations described in
Appendix~\ref{chp:app:``Toy'' RCU Implementations},
RCU grace periods extend milliseconds.
Although there are a number of techniques to render such long delays
harmless, including use of the asynchronous interfaces where available
(\co{call_rcu()} and \co{call_rcu_bh()}), this situation
is a major reason for the rule of thumb that RCU be used in read-mostly
situations.

\paragraph{Code: Reader-Writer Locking vs. RCU Code}

In the best case, the conversion from reader-writer locking to RCU
is quite simple, as shown in
Listings~\ref{lst:defer:Converting Reader-Writer Locking to RCU: Data},
\ref{lst:defer:Converting Reader-Writer Locking to RCU: Search},
and
\ref{lst:defer:Converting Reader-Writer Locking to RCU: Deletion},
all taken from
Wikipedia~\cite{WikipediaRCU}.

\begin{listing*}[htbp]
{ \scriptsize
\begin{verbbox}
 1 struct el {                           1 struct el {
 2   struct list_head lp;                2   struct list_head lp;
 3   long key;                           3   long key;
 4   spinlock_t mutex;                   4   spinlock_t mutex;
 5   int data;                           5   int data;
 6   /* Other data fields */             6   /* Other data fields */
 7 };                                    7 };
 8 DEFINE_RWLOCK(listmutex);             8 DEFINE_SPINLOCK(listmutex);
 9 LIST_HEAD(head);                      9 LIST_HEAD(head);
\end{verbbox}
}
\hspace*{0.9in}\OneColumnHSpace{-0.5in}
\theverbbox
\caption{Converting Reader-Writer Locking to RCU: Data}
\label{lst:defer:Converting Reader-Writer Locking to RCU: Data}
\end{listing*}

\begin{listing*}[htbp]
{ \scriptsize
\begin{verbbox}
 1 int search(long key, int *result)     1 int search(long key, int *result)
 2 {                                     2 {
 3   struct el *p;                       3   struct el *p;
 4                                       4
 5   read_lock(&listmutex);              5   rcu_read_lock();
 6   list_for_each_entry(p, &head, lp) { 6   list_for_each_entry_rcu(p, &head, lp) {
 7     if (p->key == key) {              7     if (p->key == key) {
 8       *result = p->data;              8       *result = p->data;
 9       read_unlock(&listmutex);        9       rcu_read_unlock();
10       return 1;                      10       return 1;
11     }                                11     }
12   }                                  12   }
13   read_unlock(&listmutex);           13   rcu_read_unlock();
14   return 0;                          14   return 0;
15 }                                    15 }
\end{verbbox}
}
\hspace*{0.9in}\OneColumnHSpace{-0.5in}
\theverbbox
\caption{Converting Reader-Writer Locking to RCU: Search}
\label{lst:defer:Converting Reader-Writer Locking to RCU: Search}
\end{listing*}

\begin{listing*}[htbp]
{ \scriptsize
\begin{verbbox}
 1 int delete(long key)                  1 int delete(long key)
 2 {                                     2 {
 3   struct el *p;                       3   struct el *p;
 4                                       4
 5   write_lock(&listmutex);             5   spin_lock(&listmutex);
 6   list_for_each_entry(p, &head, lp) { 6   list_for_each_entry(p, &head, lp) {
 7     if (p->key == key) {              7     if (p->key == key) {
 8       list_del(&p->lp);               8       list_del_rcu(&p->lp);
 9       write_unlock(&listmutex);       9       spin_unlock(&listmutex);
                                        10       synchronize_rcu();
10       kfree(p);                      11       kfree(p);
11       return 1;                      12       return 1;
12     }                                13     }
13   }                                  14   }
14   write_unlock(&listmutex);          15   spin_unlock(&listmutex);
15   return 0;                          16   return 0;
16 }                                    17 }
\end{verbbox}
}
\hspace*{0.9in}\OneColumnHSpace{-0.5in}
\theverbbox
\caption{Converting Reader-Writer Locking to RCU: Deletion}
\label{lst:defer:Converting Reader-Writer Locking to RCU: Deletion}
\end{listing*}

However, the transformation is not always this straightforward.
This is because neither the \co{spin_lock()} nor the
\co{synchronize_rcu()} in
\cref{lst:defer:Converting Reader-Writer Locking to RCU: Deletion}
exclude the readers in
\cref{lst:defer:Converting Reader-Writer Locking to RCU: Search}.
First, the \co{spin_lock()} does not interact in any way with
\co{rcu_read_lock()} and \co{rcu_read_unlock()}, thus not excluding them.
Second, although both \co{write_lock()} and \co{synchronize_rcu()}
wait for pre-existing readers, only \co{write_lock()} prevents
subsequent readers from commencing.\footnote{
	Kudos to whoever pointed this out to Paul.}
Thus, \co{synchronize_rcu()} cannot exclude readers.
It is therefore surprising that a great many situations
using reader-writer locking can be easily converted to RCU\@.

More-elaborate cases of replacing reader-writer locking with RCU
may be found
elsewhere~\cite{NeilBrown2015PathnameLookup,NeilBrown2015RCUwalk}.

\paragraph{Semantics: Reader-Writer Locking vs. RCU Semantics}

Reader-writer locking semantics can be roughly and informally summarized
by the following three temporal constraints:

\begin{enumerate}
\item	Write-side acquisitions wait for any read-holders to release
	the lock.
\item	Writer-side acquisitions wait for any write-holder to release
	the lock.
\item	Read-side acquisitions wait for any write-holder to release
	the lock.
\end{enumerate}

RCU dispenses entirely with constraint~\#3 and weakens the other two
as follows:

\begin{enumerate}
\item	Writers wait for any pre-existing read-holders before progressing
	to the destructive phase of their update (usually the freeing of
	memory).
\item	Writers synchronize with each other as needed.
\end{enumerate}

It is of course this weakening that permits RCU implementations to attain
excellent performance and scalability.
RCU use cases compensate for this weakening in a surprisingly large number
of ways, but most commonly by imposing spatial constraints:

\begin{enumerate}
\item	New data is placed in newly allocated memory.
\item	Old data is freed, but only after:
	\begin{enumerate}
	\item	That data has been unlinked so as to be inaccessible
		to later readers, and
	\item	A subsequent RCU grace period has elapsed.
	\end{enumerate}
\end{enumerate}

In short, RCU attains its read-side performance and scalability by
constructing semantics based on combined temporal and spatial constraints.

\subsubsection{RCU is a Restricted Reference-Counting Mechanism}
\label{sec:defer:RCU is a Restricted Reference-Counting Mechanism}

Because grace periods are not allowed to complete while
there is an RCU read-side critical section in progress,
the RCU read-side primitives may be used as a restricted
reference-counting mechanism.
For example, consider the following code fragment:

\begin{VerbatimN}
rcu_read_lock();  /* acquire reference. */
p = rcu_dereference(head);
/* do something with p. */
rcu_read_unlock();  /* release reference. */
\end{VerbatimN}

The \co{rcu_read_lock()} primitive can be thought of as
acquiring a reference to \co{p}, because a grace period
starting after the \co{rcu_dereference()} assigns to \co{p}
cannot possibly end until after we reach the matching
\co{rcu_read_unlock()}.
This reference-counting scheme is restricted in that
we are not allowed to block in RCU read-side critical sections,
nor are we permitted to hand off an RCU read-side critical section
from one task to another.

Regardless of these restrictions,
the following code can safely delete \co{p}:

\begin{VerbatimN}
spin_lock(&mylock);
p = head;
rcu_assign_pointer(head, NULL);
spin_unlock(&mylock);
/* Wait for all references to be released. */
synchronize_rcu();
kfree(p);
\end{VerbatimN}

The assignment to \co{head} prevents any future references
to \co{p} from being acquired, and the \co{synchronize_rcu()}
waits for any previously acquired references to be released.

\QuickQuiz{
	But wait!
	This is exactly the same code that might be used when thinking
	of RCU as a replacement for reader-writer locking!
	What gives?
}\QuickQuizAnswer{
	This is an effect of the Law of Toy Examples:
	beyond a certain point, the code fragments look the same.
	The only difference is in how we think about the code.
	However, this difference can be extremely important.
	For but one example of the importance, consider that if we think
	of RCU as a restricted reference counting scheme, we would never
	be fooled into thinking that the updates would exclude the RCU
	read-side critical sections.

	It nevertheless is often useful to think of RCU as a replacement
	for reader-writer locking, for example, when you are replacing
	reader-writer locking with RCU\@.
}\QuickQuizEnd

Of course, RCU can also be combined with traditional reference counting,
as discussed in
Section~\ref{sec:together:Refurbish Reference Counting}.

\begin{figure}[tb]
\centering
\resizebox{2.5in}{!}{\includegraphics{defer/refcntRCUperf}}
\caption{Performance of RCU vs. Reference Counting}
\label{fig:defer:Performance of RCU vs. Reference Counting}
\end{figure}

\begin{figure}[tb]
\centering
\resizebox{2.5in}{!}{\includegraphics{defer/refRCUperfPREEMPT}}
\caption{Performance of Preemptible RCU vs. Reference Counting}
\label{fig:defer:Performance of Preemptible RCU vs. Reference Counting}
\end{figure}

But why bother?
Again, part of the answer is performance, as shown in
Figures~\ref{fig:defer:Performance of RCU vs. Reference Counting}
and~\ref{fig:defer:Performance of Preemptible RCU vs. Reference Counting},
again showing data taken on a 448-CPU 2.1\,GHz Intel x86 system
for non-preemptible and preemptible Linux-kernel RCU, respectively.
Non-preemptible RCU's advantage over reference counting ranges from
more than an order of magnitude at one CPU up to about four orders of
magnitude at 192~CPUs.
Preemptible RCU's advantage ranges from about a factor of three at
one CPU up to about three orders of magnitude at 192~CPUs.

\begin{figure}[tb]
\centering
\resizebox{2.5in}{!}{\includegraphics{defer/refRCUperfwt}}
\caption{Response Time of RCU vs. Reference Counting, 192 CPUs}
\label{fig:defer:Response Time of RCU vs. Reference Counting}
\end{figure}

However, as with reader-writer locking, the performance advantages of
RCU are most pronounced for short-duration critical sections and for
large numbers of CPUs, as shown in
Figure~\ref{fig:defer:Response Time of RCU vs. Reference Counting}
for the same system.
In addition, as with reader-writer locking, many system calls (and thus
any RCU read-side critical sections that they contain) complete in
a few microseconds.

However, the restrictions that go with RCU can be quite onerous.
For example, in many cases, the prohibition against sleeping while in an RCU
read-side critical section would defeat the entire purpose.
The next section looks at ways of addressing this problem, while also
reducing the complexity of traditional reference counting, at least in
some cases.\footnote{
	Other cases might be better served by the hazard pointers
	mechanism described in \cref{sec:defer:Hazard Pointers}.}

\subsubsection{RCU is a Bulk Reference-Counting Mechanism}
\label{sec:defer:RCU is a Bulk Reference-Counting Mechanism}

As noted in the preceding section,
traditional reference counters are usually associated with a specific
data structure, or perhaps a specific group of data structures.
However, maintaining a single global reference counter for a large
variety of data structures typically results in bouncing
the cache line containing the reference count.
Such cache-line bouncing can severely degrade performance.

In contrast, RCU's lightweight read-side primitives permit
extremely frequent read-side usage with negligible performance
degradation, permitting RCU to be used as a ``bulk reference-counting''
mechanism with little or no performance penalty.
Situations where a reference must be held by a single task across a
section of code that blocks may be accommodated with
Sleepable RCU (SRCU)~\cite{PaulEMcKenney2006c}.
This fails to cover the not-uncommon situation where a reference is ``passed''
from one task to another, for example, when a reference is acquired
when starting an I/O and released in the corresponding completion
interrupt handler.
(In principle, this could be handled by the SRCU implementation,
but in practice, it is not yet clear whether this is a good tradeoff.)

Of course, SRCU brings restrictions of its own, namely that the
return value from \co{srcu_read_lock()} be passed into the
corresponding \co{srcu_read_unlock()}, and that no SRCU primitives
be invoked from hardware interrupt handlers or from non-maskable interrupt
(NMI) handlers.
The jury is still out as to how much of a problem is presented by
these restrictions, and as to how they can best be handled.

\subsubsection{RCU is a Poor Man's Garbage Collector}
\label{sec:defer:RCU is a Poor Man's Garbage Collector}

A not-uncommon exclamation made by people first learning about
RCU is ``RCU is sort of like a garbage collector!''
This exclamation has a large grain of truth, but it can also be
misleading.

Perhaps the best way to think of the relationship between RCU
and automatic garbage collectors (GCs) is that RCU resembles
a GC in that the \emph{timing} of collection is automatically
determined, but that RCU differs from a GC in that: (1)~the programmer
must manually indicate when a given data structure is eligible
to be collected, and (2)~the programmer must manually mark the
RCU read-side critical sections where references might be held.

Despite these differences, the resemblance does go quite deep.
In fact, the first RCU-like mechanism I am aware of used a
reference-count-based garbage collector to handle the grace
periods~\cite{Kung80}, and the connection between RCU and
garbage collection has been noted more
recently~\cite{HarshalSheth2016goRCU}.
Nevertheless, a better way of thinking of RCU is described in the
following section.

\subsubsection{RCU is an MVCC}
\label{sec:defer:RCU is an MVCC}

RCU can also be thought of as a simplified multi-version concurrency
control (MVCC) mechanism with weak consistency criteria.
The multi-version aspects were touched upon in
\cref{sec:defer:Maintain Multiple Versions of Recently Updated Objects}.
However, in its native form, RCU provides version consistency only
within a given RCU-protected data element.

However, a number of techniques can be used to restore version consistency
at a higher level, for example, using sequence locking
(see \cref{sec:together:Correlated Data Elements})
or imposing additional levels of indirection
(see \cref{sec:together:Correlated Fields}).

\subsubsection{RCU Provides Existence Guarantees}
\label{sec:defer:RCU Provides Existence Guarantees}

Gamsa et al.~\cite{Gamsa99}
discuss existence guarantees and describe how a mechanism
resembling RCU can be used to provide these existence guarantees
(see Section~5 on page 7 of the PDF), and
Section~\ref{sec:locking:Lock-Based Existence Guarantees}
discusses how to guarantee existence via locking, along with the
ensuing disadvantages of doing so.
The effect is that if any RCU-protected data element is accessed
within an RCU read-side critical section, that data element is
guaranteed to remain in existence for the duration of that RCU
read-side critical section.

\begin{listing}[tbp]
\begin{fcvlabel}[ln:defer:Existence Guarantees Enable Per-Element Locking]
\begin{VerbatimL}[commandchars=\\\@\$]
int delete(int key)
{
	struct element *p;
	int b;

	b = hashfunction(key);			\lnlbl@hash$
	rcu_read_lock();			\lnlbl@rdlock$
	p = rcu_dereference(hashtable[b]);
	if (p == NULL || p->key != key) {	\lnlbl@chkkey$
		rcu_read_unlock();		\lnlbl@rdunlock1$
		return 0;			\lnlbl@ret_0:a$
	}
	spin_lock(&p->lock);			\lnlbl@acq$
	if (hashtable[b] == p && p->key == key) {\lnlbl@chkkey2$
		rcu_read_unlock();		\lnlbl@rdunlock2$
		rcu_assign_pointer(hashtable[b], NULL);\lnlbl@remove$
		spin_unlock(&p->lock);		\lnlbl@rel1$
		synchronize_rcu();		\lnlbl@sync_rcu$
		kfree(p);			\lnlbl@kfree$
		return 1;			\lnlbl@ret_1$
	}
	spin_unlock(&p->lock);			\lnlbl@rel2$
	rcu_read_unlock();			\lnlbl@rdunlock3$
	return 0;				\lnlbl@ret_0:b$
}
\end{VerbatimL}
\end{fcvlabel}
\caption{Existence Guarantees Enable Per-Element Locking}
\label{lst:defer:Existence Guarantees Enable Per-Element Locking}
\end{listing}

\begin{fcvref}[ln:defer:Existence Guarantees Enable Per-Element Locking]
Listing~\ref{lst:defer:Existence Guarantees Enable Per-Element Locking}
demonstrates how RCU-based existence guarantees can enable
per-element locking via a function that deletes an element from
a hash table.
Line~\lnref{hash} computes a hash function, and line~\lnref{rdlock} enters an RCU
read-side critical section.
If line~\lnref{chkkey} finds that the corresponding bucket of the hash table is
empty or that the element present is not the one we wish to delete,
then line~\lnref{rdunlock1} exits the RCU read-side critical section and
line~\lnref{ret_0:a}
indicates failure.
\end{fcvref}

\QuickQuiz{
	What if the element we need to delete is not the first element
	of the list on
        line~\ref{ln:defer:Existence Guarantees Enable Per-Element Locking:chkkey} of
	Listing~\ref{lst:defer:Existence Guarantees Enable Per-Element Locking}?
}\QuickQuizAnswer{
	As with
	Listing~\ref{lst:locking:Per-Element Locking Without Existence Guarantees},
	this is a very simple hash table with no chaining, so the only
	element in a given bucket is the first element.
	The reader is again invited to adapt this example to a hash table with
	full chaining.
	Less energetic reader might wish to refer to
	\cref{chp:Data Structures}.
}\QuickQuizEnd

\begin{fcvref}[ln:defer:Existence Guarantees Enable Per-Element Locking]
Otherwise, line~\lnref{acq} acquires the update-side spinlock, and
line~\lnref{chkkey2} then checks that the element is still the one that we want.
If so, line~\lnref{rdunlock2} leaves the RCU read-side critical section,
line~\lnref{remove} removes it from the table, line~\lnref{rel1} releases
the lock, line~\lnref{sync_rcu} waits for all pre-existing RCU read-side critical
sections to complete, line~\lnref{kfree} frees the newly removed element,
and line~\lnref{ret_1} indicates success.
If the element is no longer the one we want, line~\lnref{rel2} releases
the lock, line~\lnref{rdunlock3} leaves the RCU read-side critical section,
and line~\lnref{ret_0:b} indicates failure to delete the specified key.
\end{fcvref}

\QuickQuizSeries{%
\QuickQuizB{
	\begin{fcvref}[ln:defer:Existence Guarantees Enable Per-Element Locking]
	Why is it OK to exit the RCU read-side critical section on
	line~\lnref{rdunlock2} of
	Listing~\ref{lst:defer:Existence Guarantees Enable Per-Element Locking}
	before releasing the lock on line~\lnref{rel1}?
	\end{fcvref}
}\QuickQuizAnswerB{
	\begin{fcvref}[ln:defer:Existence Guarantees Enable Per-Element Locking]
	First, please note that the second check on line~\lnref{chkkey2} is
	necessary because some other
	CPU might have removed this element while we were waiting
	to acquire the lock.
	However, the fact that we were in an RCU read-side critical section
	while acquiring the lock guarantees that this element could not
	possibly have been re-allocated and re-inserted into this
	hash table.
	Furthermore, once we acquire the lock, the lock itself guarantees
	the element's existence, so we no longer need to be in an
	RCU read-side critical section.

	The question as to whether it is necessary to re-check the
	element's key is left as an exercise to the reader.
	% A re-check is necessary if the key can mutate or if it is
	% necessary to reject deleted entries (in cases where deletion
	% is recorded by mutating the key.
	\end{fcvref}
}\QuickQuizEndB
%
\QuickQuizE{
	\begin{fcvref}[ln:defer:Existence Guarantees Enable Per-Element Locking]
	Why not exit the RCU read-side critical section on
	line~\lnref{rdunlock3} of
	Listing~\ref{lst:defer:Existence Guarantees Enable Per-Element Locking}
	before releasing the lock on line~\lnref{rel2}?
	\end{fcvref}
}\QuickQuizAnswerE{
	Suppose we reverse the order of these two lines.
	Then this code is vulnerable to the following sequence of
	events:
	\begin{enumerate}
	\begin{fcvref}[ln:defer:Existence Guarantees Enable Per-Element Locking]
	\item	CPU~0 invokes \co{delete()}, and finds the element
		to be deleted, executing through line~\lnref{rdunlock2}.
		It has not yet actually deleted the element, but
		is about to do so.
	\item	CPU~1 concurrently invokes \co{delete()}, attempting
		to delete this same element.
		However, CPU~0 still holds the lock, so CPU~1 waits
		for it at line~\lnref{acq}.
	\item	CPU~0 executes lines~\lnref{remove} and~\lnref{rel1},
		and blocks at line~\lnref{sync_rcu} waiting for CPU~1
		to exit its RCU read-side critical section.
	\item	CPU~1 now acquires the lock, but the test on line~\lnref{chkkey2}
		fails because CPU~0 has already removed the element.
		CPU~1 now executes line~\lnref{rel2}
                (which we switched with line~\lnref{rdunlock3}
		for the purposes of this Quick Quiz)
		and exits its RCU read-side critical section.
	\item	CPU~0 can now return from \co{synchronize_rcu()},
		and thus executes line~\lnref{kfree}, sending the element to
		the freelist.
	\item	CPU~1 now attempts to release a lock for an element
		that has been freed, and, worse yet, possibly
		reallocated as some other type of data structure.
		This is a fatal memory-corruption error.
	\end{fcvref}
	\end{enumerate}
}\QuickQuizEndE
}

Alert readers will recognize this as only a slight variation on
the original ``RCU is a way of waiting for things to finish'' theme,
which is addressed in
Section~\ref{sec:defer:RCU is a Way of Waiting for Things to Finish}.
They might also note the deadlock-immunity advantages over the lock-based
existence guarantees discussed in
Section~\ref{sec:locking:Lock-Based Existence Guarantees}.

\subsubsection{RCU Provides Type-Safe Memory}
\label{sec:defer:RCU Provides Type-Safe Memory}

A number of lockless algorithms do not require that a given data
element keep the same identity through a given RCU read-side critical
section referencing it---but only if that data element retains the
same type.
In other words, these lockless algorithms can tolerate a given data
element being freed and reallocated as the same type of structure
while they are referencing it, but must prohibit a change in type.
This guarantee, called ``type-safe memory'' in
academic literature~\cite{Cheriton96a},
is weaker than the existence guarantees in the
previous section, and is therefore quite a bit harder to work with.
Type-safe memory algorithms in the Linux kernel make use of slab caches,
specially marking these caches with \co{SLAB_TYPESAFE_BY_RCU}
so that RCU is used when returning a freed-up
slab to system memory.
This use of RCU guarantees that any in-use element of
such a slab will remain in that slab, thus retaining its type,
for the duration of any pre-existing RCU read-side critical sections.

\QuickQuiz{
	But what if there is an arbitrarily long series of RCU
	read-side critical sections in multiple threads, so that at
	any point in time there is at least one thread in the system
	executing in an RCU read-side critical section?
	Wouldn't that prevent any data from a \co{SLAB_TYPESAFE_BY_RCU}
	slab ever being returned to the system, possibly resulting
	in OOM events?
}\QuickQuizAnswer{
	There could certainly be an arbitrarily long period of time
	during which at least one thread is always in an RCU read-side
	critical section.
	However, the key words in the description in
	Section~\ref{sec:defer:RCU Provides Type-Safe Memory}
	are ``in-use'' and ``pre-existing''.
	Keep in mind that a given RCU read-side critical section is
	conceptually only permitted to gain references to data elements
	that were in use at the beginning of that critical section.
	Furthermore, remember that a slab cannot be returned to the
	system until all of its data elements have been freed, in fact,
	the RCU grace period cannot start until after they have all been
	freed.

	Therefore, the slab cache need only wait for those RCU read-side
	critical sections that started before the freeing of the last element
	of the slab.
	This in turn means that any RCU grace period that begins after
	the freeing of the last element will do---the slab may be returned
	to the system after that grace period ends.
}\QuickQuizEnd

It is important to note that \co{SLAB_TYPESAFE_BY_RCU} will
\emph{in no way}
prevent \co{kmem_cache_alloc()} from immediately reallocating
memory that was just now freed via \co{kmem_cache_free()}!
In fact, the \co{SLAB_TYPESAFE_BY_RCU}-protected data structure
just returned by \co{rcu_dereference} might be freed and reallocated
an arbitrarily large number of times, even when under the protection
of \co{rcu_read_lock()}.
Instead, \co{SLAB_TYPESAFE_BY_RCU} operates by preventing
\co{kmem_cache_free()}
from returning a completely freed-up slab of data structures
to the system until after an RCU grace period elapses.
In short, although a given RCU read-side critical section might see a
given \co{SLAB_TYPESAFE_BY_RCU} data element being freed and reallocated
arbitrarily often, the element's type is guaranteed not to change until
that critical section has completed.

These algorithms therefore typically use a validation step that checks
to make sure that the newly referenced data structure really is the one
that was requested~\cite[Section~2.5]{LaninShasha1986TSM}.
These validation checks require that portions of the data structure
remain untouched by the free-reallocate process.
Such validation checks are usually very hard to get right, and can
hide subtle and difficult bugs.

Therefore, although type-safety-based lockless algorithms can be extremely
helpful in a very few difficult situations, you should instead use existence
guarantees where possible.
Simpler is after all almost always better!

\subsubsection{RCU is a Way of Waiting for Things to Finish}
\label{sec:defer:RCU is a Way of Waiting for Things to Finish}

As noted in Section~\ref{sec:defer:RCU Fundamentals}
an important component
of RCU is a way of waiting for RCU readers to finish.
One of
RCU's great strength is that it allows you to wait for each of
thousands of different things to finish without having to explicitly
track each and every one of them, and without having to worry about
the performance degradation, scalability limitations, complex deadlock
scenarios, and memory-leak hazards that are inherent in schemes that
use explicit tracking.

In this section, we will show how \co{synchronize_sched()}'s
read-side counterparts (which include anything that disables preemption,
along with hardware operations and
primitives that disable interrupts) permit you to implement interactions with
non-maskable interrupt
(NMI) handlers that would be quite difficult if using locking.
This approach has been called ``Pure RCU''~\cite{PaulEdwardMcKenneyPhD},
and it is used in a number of places in the Linux kernel.

The basic form of such ``Pure RCU'' designs is as follows:

\begin{enumerate}
\item	Make a change, for example, to the way that the OS reacts to an NMI\@.
\item	Wait for all pre-existing read-side critical sections to
	completely finish (for example, by using the
	\co{synchronize_sched()} primitive).
	The key observation here is that subsequent RCU read-side critical
	sections are guaranteed to see whatever change was made.
\item	Clean up, for example, return status indicating that the
	change was successfully made.
\end{enumerate}

The remainder of this section presents example code adapted from
the Linux kernel.
In this example, the \co{timer_stop} function uses
\co{synchronize_sched()} to ensure that all in-flight NMI
notifications have completed before freeing the associated resources.
A simplified version of this code is shown
Listing~\ref{lst:defer:Using RCU to Wait for NMIs to Finish}.

\begin{listing}[tbp]
\begin{fcvlabel}[ln:defer:Using RCU to Wait for NMIs to Finish]
\begin{VerbatimL}[commandchars=\\\@\$]
struct profile_buffer {				\lnlbl@struct:b$
	long size;
	atomic_t entry[0];
};						\lnlbl@struct:e$
static struct profile_buffer *buf = NULL;	\lnlbl@struct:buf$

void nmi_profile(unsigned long pcvalue)		\lnlbl@nmi_profile:b$
{
	struct profile_buffer *p = rcu_dereference(buf);\lnlbl@nmi_profile:rcu_deref$

	if (p == NULL)				\lnlbl@nmi_profile:if_NULL$
		return;				\lnlbl@nmi_profile:ret:a$
	if (pcvalue >= p->size)			\lnlbl@nmi_profile:if_oor$
		return;				\lnlbl@nmi_profile:ret:b$
	atomic_inc(&p->entry[pcvalue]);		\lnlbl@nmi_profile:inc$
}						\lnlbl@nmi_profile:e$

void nmi_stop(void)				\lnlbl@nmi_stop:b$
{
	struct profile_buffer *p = buf;		\lnlbl@nmi_stop:fetch$

	if (p == NULL)				\lnlbl@nmi_stop:if_NULL$
		return;				\lnlbl@nmi_stop:ret$
	rcu_assign_pointer(buf, NULL);		\lnlbl@nmi_stop:NULL$
	synchronize_sched();			\lnlbl@nmi_stop:sync_sched$
	kfree(p);				\lnlbl@nmi_stop:kfree$
}						\lnlbl@nmi_stop:e$
\end{VerbatimL}
\end{fcvlabel}
\caption{Using RCU to Wait for NMIs to Finish}
\label{lst:defer:Using RCU to Wait for NMIs to Finish}
\end{listing}

\begin{fcvref}[ln:defer:Using RCU to Wait for NMIs to Finish:struct]
\Clnrefrange{b}{e} define a \co{profile_buffer} structure, containing a
size and an indefinite array of entries.
Line~\lnref{buf} defines a pointer to a profile buffer, which is
presumably initialized elsewhere to point to a dynamically allocated
region of memory.
\end{fcvref}

\begin{fcvref}[ln:defer:Using RCU to Wait for NMIs to Finish:nmi_profile]
\Clnrefrange{b}{e} define the \co{nmi_profile()} function,
which is called from within an NMI handler.
As such, it cannot be preempted, nor can it be interrupted by a normal
interrupts handler, however, it is still subject to delays due to cache misses,
ECC errors, and cycle stealing by other hardware threads within the same
core.
Line~\lnref{rcu_deref} gets a local pointer to the profile buffer using the
\co{rcu_dereference()} primitive to ensure memory ordering on
DEC Alpha, and
lines~\lnref{if_NULL} and~\lnref{ret:a} exit from this function if there is no
profile buffer currently allocated, while lines~\lnref{if_oor} and~\lnref{ret:b}
exit from this function if the \co{pcvalue} argument
is out of range.
Otherwise, line~\lnref{inc} increments the profile-buffer entry indexed
by the \co{pcvalue} argument.
Note that storing the size with the buffer guarantees that the
range check matches the buffer, even if a large buffer is suddenly
replaced by a smaller one.
\end{fcvref}

\begin{fcvref}[ln:defer:Using RCU to Wait for NMIs to Finish:nmi_stop]
\Clnrefrange{b}{e} define the \co{nmi_stop()} function,
where the caller is responsible for mutual exclusion (for example,
holding the correct lock).
Line~\lnref{fetch} fetches a pointer to the profile buffer, and
lines~\lnref{if_NULL} and~\lnref{ret} exit the function if there is no buffer.
Otherwise, line~\lnref{NULL} \co{NULL}s out the profile-buffer pointer
(using the \co{rcu_assign_pointer()} primitive to maintain
memory ordering on weakly ordered machines),
and line~\lnref{sync_sched} waits for an RCU Sched grace period to elapse,
in particular, waiting for all non-preemptible regions of code,
including NMI handlers, to complete.
Once execution continues at line~\lnref{kfree}, we are guaranteed that
any instance of \co{nmi_profile()} that obtained a
pointer to the old buffer has returned.
It is therefore safe to free the buffer, in this case using the
\co{kfree()} primitive.
\end{fcvref}

\QuickQuiz{
	Suppose that the \co{nmi_profile()} function was preemptible.
	What would need to change to make this example work correctly?
}\QuickQuizAnswer{
	One approach would be to use
	\co{rcu_read_lock()} and \co{rcu_read_unlock()}
	in \co{nmi_profile()}, and to replace the
	\co{synchronize_sched()} with \co{synchronize_rcu()},
	perhaps as shown in
	Listing~\ref{lst:defer:Using RCU to Wait for Mythical Preemptible NMIs to Finish}.
%
\begin{listing}[tb]
\begin{VerbatimL}
struct profile_buffer {
	long size;
	atomic_t entry[0];
};
static struct profile_buffer *buf = NULL;

void nmi_profile(unsigned long pcvalue)
{
	struct profile_buffer *p;

	rcu_read_lock();
	p = rcu_dereference(buf);
	if (p == NULL) {
		rcu_read_unlock();
		return;
	}
	if (pcvalue >= p->size) {
		rcu_read_unlock();
		return;
	}
	atomic_inc(&p->entry[pcvalue]);
	rcu_read_unlock();
}

void nmi_stop(void)
{
	struct profile_buffer *p = buf;

	if (p == NULL)
		return;
	rcu_assign_pointer(buf, NULL);
	synchronize_rcu();
	kfree(p);
}
\end{VerbatimL}
\caption{Using RCU to Wait for Mythical Preemptible NMIs to Finish}
\label{lst:defer:Using RCU to Wait for Mythical Preemptible NMIs to Finish}
\end{listing}
%
}\QuickQuizEnd

In short, RCU makes it easy to dynamically switch among profile
buffers (you just \emph{try} doing this efficiently with atomic
operations, or at all with locking!).
However, RCU is normally used at a higher level of abstraction, as
was shown in the previous sections.

\subsubsection{RCU Usage Summary}
\label{sec:defer:RCU Usage Summary}

At its core, RCU is nothing more nor less than an API that provides:

\begin{enumerate}
\item	a publish-subscribe mechanism for adding new data,
\item	a way of waiting for pre-existing RCU readers to finish, and
\item	a discipline of maintaining multiple versions to permit change
	without harming or unduly delaying concurrent RCU readers.
\end{enumerate}

That said, it is possible to build higher-level constructs
on top of RCU, including the reader-writer-locking, reference-counting,
and existence-guarantee constructs listed in the earlier sections.
Furthermore, I have no doubt that the Linux community will continue to
find interesting new uses for RCU,
as well as for any of a number of other synchronization primitives.

\begin{figure}[tbh]
\centering
\resizebox{3in}{!}{\includegraphics{defer/RCUApplicability}}
\caption{RCU Areas of Applicability}
\label{fig:defer:RCU Areas of Applicability}
\end{figure}

In the meantime,
Figure~\ref{fig:defer:RCU Areas of Applicability}
shows some rough rules of thumb on where RCU is most helpful.

As shown in the blue box at the top of the figure, RCU works best if
you have read-mostly data where stale and inconsistent
data is permissible (but see below for more information on stale and
inconsistent data).
The canonical example of this case in the Linux kernel is routing tables.
Because it may have taken many seconds or even minutes for the
routing updates to propagate across the Internet, the system
has been sending packets the wrong way for quite some time.
Having some small probability of continuing to send some of them the wrong
way for a few more milliseconds is almost never a problem.

If you have a read-mostly workload where consistent data is required,
RCU works well, as shown by the green ``read-mostly, need consistent data''
box.
One example of this case is the Linux kernel's mapping from user-level
System-V semaphore IDs to the corresponding in-kernel data structures.
Semaphores tend to be used far more frequently than they are created
and destroyed, so this mapping is read-mostly.
However, it would be erroneous to perform a semaphore operation on
a semaphore that has already been deleted.
This need for consistency is handled by using the lock in the
in-kernel semaphore data structure, along with a ``deleted''
flag that is set when deleting a semaphore.
If a user ID maps to an in-kernel data structure with the
``deleted'' flag set, the data structure is ignored, so that
the user ID is flagged as invalid.

Although this requires that the readers acquire a lock for the
data structure representing the semaphore itself,
it allows them to dispense with locking for the
mapping data structure.
The readers therefore locklessly
traverse the tree used to map from ID to data structure,
which in turn greatly improves performance, scalability, and
real-time response.

As indicated by the yellow ``read-write'' box, RCU can also be useful
for read-write
workloads where consistent data is required, although usually in
conjunction with a number of other synchronization primitives.
For example, the directory-entry cache in recent Linux kernels uses RCU in
conjunction with sequence locks, per-CPU locks, and per-data-structure
locks to allow lockless traversal of pathnames in the common case.
Although RCU can be very beneficial in this read-write case, such
use is often more complex than that of the read-mostly cases.

Finally, as indicated by the red box at the bottom of the figure,
update-mostly workloads requiring
consistent data are rarely good places to use RCU, though there are some
exceptions~\cite{MathieuDesnoyers2012URCU}.
In addition, as noted in
Section~\ref{sec:defer:RCU Provides Type-Safe Memory},
within the Linux kernel, the \co{SLAB_TYPESAFE_BY_RCU}
slab-allocator flag provides type-safe memory to RCU readers, which can
greatly simplify non-blocking synchronization and other lockless
algorithms.

In short, RCU is an API that includes a publish-subscribe mechanism for
adding new data, a way of waiting for pre-existing RCU readers to finish,
and a discipline of maintaining multiple versions to allow updates to
avoid harming or unduly delaying concurrent RCU readers.
This RCU API is best suited for read-mostly situations, especially if
stale and inconsistent data can be tolerated by the application.

% defer/rcurelated.tex

\subsection{RCU Related Work}
\label{sec:defer:RCU Related Work}
\OriginallyPublished{Section}{sec:defer:RCU Related Work}{RCU Related Work}{Linux Weekly News}{PaulEMcKenney2014ReadMostly}
\OriginallyPublished{Section}{sec:defer:RCU Related Work}{RCU Related Work}{Linux Weekly News}{PaulEMcKenney2015ReadMostly}

RCU 비슷한 것들에 대한 알려진 첫번째 언급은 FORTRAN 을 위한 J. Weizenbaum 의
SLIP 리스트 처리 설비~\cite{Weizenbaum:1963:SLP:367593.367617} 에 대한 Donal
Knuth 의 버그 레포트~\cite[page 413 of Fundamental Algorithms]{Knuth73}
였습니다.
Knuth 는 SLIP 이 grace-period 보장과 같은 것을 갖고 있지 않다고 이야기했습니다.

RCU 비슷한 것들에 대한 버그 레포트가 아닌 첫번째 언급은 Kung 과 Lehman 의
획기적인 논문~\cite{Kung80} 이었습니다.
이 기법의 학계에서의 추가적인
사용~\cite{Manber82,Manber84,BarbaraLiskov1988ArgusCACM,Pugh90,Andrews91textbook,Pu95a,Cowan96a,Rastogi:1997:LPV:645923.671017,Gamsa99}
이 있었습니다만, 이 영역에서의 대부분의 작업은 실제 일을 하는 사람들에 의해
진행되었습니다~\cite{RichardRashid87a,Hennessy89,Jacobson93,AjuJohn95,Slingwine95,Slingwine97,Slingwine98,McKenney98}.
2000 년에 이르러, 주도권은 오픈 소스 프로젝트들로, 그 중에서도 리눅스 커널
커뮤니티로
넘겨졌습니다~\cite{RustyRussell2000a,RustyRussell2000b,McKenney01b,McKenney01a,McKenney02a,Arcangeli03}.\footnote{
	200 개가 넘는 인용처 목록은 이 책의 {\LaTeX} 소스의 \co{bib/RCU.bib}
	에서 찾을 수 있을 겁니다.}
\iffalse

The known first mention of anything resembling RCU took the form of a bug
report from
Donald Knuth~\cite[page 413 of Fundamental Algorithms]{Knuth73}
against J. Weizenbaum's SLIP list-processing facility for
FORTRAN~\cite{Weizenbaum:1963:SLP:367593.367617}.
Knuth was justified in reporting the bug, as SLIP had no notion of
any sort of grace-period guarantee.

The first known non-bug-report mention of anything resembing RCU appeared
in Kung's and Lehman's landmark paper~\cite{Kung80}.
There was some additional use of this technique in
academia~\cite{Manber82,Manber84,BarbaraLiskov1988ArgusCACM,Pugh90,Andrews91textbook,Pu95a,Cowan96a,Rastogi:1997:LPV:645923.671017,Gamsa99},
but much of the work in this area was carried out by
practitioners~\cite{RichardRashid87a,Hennessy89,Jacobson93,AjuJohn95,Slingwine95,Slingwine97,Slingwine98,McKenney98}.
By the year 2000, the initiative had passed to open-source projects,
most notably the Linux kernel
community~\cite{RustyRussell2000a,RustyRussell2000b,McKenney01b,McKenney01a,McKenney02a,Arcangeli03}.\footnote{
	A list of citations with well over 200 entries may be found in
	\co{bib/RCU.bib} in the {\LaTeX} source for this book.}
\fi

하지만, 2010년대 중반에 이르러, 커뮤니티와 기관들을 통한 RCU 에 대한 연구와
개발이 급증했습니다.
Section~\ref{sec:defer:RCU Uses} 는 RCU 의 사용들에 대해 설명하고,
Section~\ref{sec:defer:RCU Implementations} 는 RCU 구현들을 (그리고 구현을
반들고 사용한 일들도) 설명하고, 마지막으로,
Section~\ref{sec:defer:RCU Validation} 에서는 RCU 의 검증과 그 사용을
이야기합니다.
\iffalse

However, in the mid 2010s, there was a welcome upsurge in RCU research
and development across a number of communities and institutions.
Section~\ref{sec:defer:RCU Uses} describes uses of RCU,
Section~\ref{sec:defer:RCU Implementations} describes RCU implementations
(as well as work that both creates and uses an implementation),
and finally,
Section~\ref{sec:defer:RCU Validation} describes verification and validation
of RCU and its uses.
\fi

\subsubsection{RCU Uses}
\label{sec:defer:RCU Uses}

Phil Howard and Jon Walpole of Portland State University (PSU) have
applied RCU to red-black
trees~\cite{PhilHowardPhD,PhilHoward2011RCUTMRBTree} combined with updates
synchronized using software transactional memory.
Josh Triplett and Jon Walpole (again of PSU) applied RCU to resizable
hash tables~\cite{JoshTriplettPhD,Triplett:2011:RPHash,JonCorbet2014RCUhash1,JonCorbet2014RCUhash2}.
Other RCU-protected resizable hash tables have been created by
Herbert Xu~\cite{HerbertXu2010RCUResizeHash} and by
Mathieu Desnoyers~\cite{PaulMcKenney2013LWNURCUhash}.

Austin Clements, Frans Kaashoek, and Nickolai Zeldovich
of MIT created an RCU-optimized balanced binary tree
(Bonsai)~\cite{AustinClements2012RCULinux:mmapsem}, and applied this
tree to the Linux kernel's VM subsystem in order to reduce read-side
contention on the Linux kernel's \co{mmap_sem}.
This work resulted in order-of-magnitude speedups and scalability up to
at least 80 CPUs for a microbenchmark featuring large numbers of minor
page faults.
This is similar to a patch developed earlier by
Peter Zijlstra~\cite{PeterZijlstra2014SpeculativePageFault}, and both
were limited by the fact that, at the time, filesystem data structures
were not safe for RCU readers.
Clements et al. avoided this limitation by optimizing the page-fault
path for anonymous pages only.
More recently, filesystem data structures have been made safe for RCU
readers~\cite{JonathanCorbet2010dcacheRCU,JonathanCorbet2011dcacheRCUbug},
so perhaps this work can be implemented for all page types, not just
anonymous pages---Peter Zijlstra has, in fact, recently prototyped
exactly this.

Yandong Mao and Robert Morris of MIT and Eddie Kohler of
Harvard University created another RCU-protected tree named
Masstree~\cite{Mao:2012:CCF:2168836.2168855} that combines ideas from B+
trees and tries.
Although this tree is about 2.5x slower than an RCU-protected hash table,
it supports operations on key ranges, unlike hash tables.
In addition, Masstree supports efficient storage of objects with long
shared key prefixes and, furthermore, provides persistence via logging
to mass storage.

The paper notes that Masstree's performance rivals that of memcached, even
given that Masstree is persistently storing updates and memcached is not.
The paper also compares Masstree's performance to the persistent
datastores MongoDB, VoltDB, and Redis, reporting significant performance
advantages for Masstree, in some cases exceeding two orders of magnitude.
Another paper~\cite{Tu:2013:STM:2517349.2522713}, by Stephen Tu,
Wenting Zheng, Barbara Liskov, and Samuel Madden of MIT and Kohler,
applies Masstree to an in-memory database named Silo, achieving 700K
transactions per second (42M transactions per minute) on a well-known
transaction-processing benchmark.
Interestingly enough, Silo guarantees linearizability without incurring
the overhead of grace periods while holding locks.

\subsubsection{RCU Implementations}
\label{sec:defer:RCU Implementations}

Mathieu Desnoyers created a user-space RCU for use in
tracing~\cite{MathieuDesnoyersPhD,MathieuDesnoyers2012URCU},
which has seen use in a number of projects~\cite{MikeDay2013RCUqemu}.

Maya Arbel and Hagit Attiya of Technion took a more rigorous
approach~\cite{MayaArbel2014RCUtree} to an RCU-protected search tree that,
like Masstree, allows concurrent updates.
This paper includes a proof of correctness, including proof that all
operations on this tree are linearizable.
Unfortunately, this implementation achieves linearizability by incurring
the full latency of grace-period waits while holding locks, which degrades
scalability of update-only workloads.
One way around this problem is to abandon
linearizability~\cite{AndreasHaas2012FIFOisnt,PaulEMcKennneyAtomicTreeN4037}),
however, Arbel and Attiya instead created an RCU variant that reduces
low-end grace-period latency.
Of course, nothing comes for free, and this RCU variant appears to hit
a scalability limit at about 32 CPUs.
Although there is much to be said for dropping linearizability, thus
gaining both performance and scalability, it is very good to see academics
experimenting with alternative RCU implementations.
Researchers at Charles University in Prague have also been
working on RCU implementations, including dissertations by
Andrej Podzimek~\cite{AndrejPodzimek2010masters} and Adam
Hraska~\cite{AdamHraska2013RCUHelenOS}.

RCU-like mechanisms are also finding their way into Java.
Sivaramakrishnan et al.~\cite{Sivaramakrishnan:2012:ERB:2258996.2259005}
use an RCU-like mechanism to eliminate the read barriers that are
otherwise required when interacting with Java's garbage collector,
resulting in significant performance improvements.

Ran Liu, Heng Zhang, and Haibo Chen of Shanghai Jiao Tong University
created a specialized variant of RCU that they used for an optimized
``passive reader-writer lock''~\cite{RanLiu2014PassiveRWLock}, similar to
those created by Gautham Shenoy~\cite{GauthamShenoy2006RCUrwlock} and
Srivatsa Bhat~\cite{SrivatsaSBhat2014RCUrwlock}.
The Liu et al.~paper is interesting from a number of
perspectives~\cite{PaulEMcKenney2014ReadMostly}.

Geoff Romer and Andrew Hunter proposed a cell-based API for RCU
protection of singleton data structures for inclusion in the
C++ standard~\cite{GeoffRomer2017C++DeferredReclamation}.

\subsubsection{RCU Validation}
\label{sec:defer:RCU Validation}

In early 2017, it is commonly recognized that almost any bug is a potential
security exploit, so validation and verification are first-class concerns.

Researchers at Stony Brook University have produced an RCU-aware data-race
detector~\cite{AbhinavDuggal2010Masters,JustinSeyster2012PhD,Seyster:2011:RFA:2075416.2075425}.
Alexey Gotsman of IMDEA, Noam Rinetzky of Tel Aviv University,
and Hongseok Yang of the University of Oxford have published a
paper~\cite{AlexeyGotsman2012VerifyGraceExtended} expressing the formal
semantics of RCU in terms of separation logic, and have continued with
other aspects of concurrency.

With some luck, all of this validation work will eventually result in
more and better tools for validating concurrent code.

