% updates.tex
% SPDX-License-Identifier: CC-BY-SA-3.0

\section{What About Updates?}
\label{sec:defer:What About Updates?}

이 챕터에서 이야기된 미뤄서 처리하기 테크닉들은 읽기가 대부분인 환경에는 거의
곧바로 적용할 수 있는데, 이는 곧 ``그렇지만 업데이트는 어떻게 하지?'' 라는
질문을 갖게 합니다.
읽기 쓰레드의 성능과 확장성을 증가시키는 것은 잘 되었지만, 쓰기 쓰레드에도
훌륭한 성능과 확장성을 원하는건 자연스러운 일입니다.
\iffalse

The deferred-processing techniques called out in this chapter are most
directly applicable to read-mostly situations, which begs the question
``But what about updates?''
After all, increasing the performance and scalability of readers is all
well and good, but it is only natural to also want great performance and
scalability for writers.
\fi

이미 쓰기 쓰레드에게 높은 성능과 확장성을 가져다주는 상황을 이미 봤는데,
Chapter~\ref{chp:Counting} 에서 이야기된 카운팅 알고리즘들입니다.
이 알고리즘들은 부분적으로 분할된 데이터 구조를 사용해서 쓰기 쓰레드가
지역적으로 일어날 수 있게 하면서도 더 비싼 읽기들은 전체 데이터 구조를 횡단하며
더하기를 해야만 하게 했습니다.
Silas Boyd-Wickhize 는 이런 방향을 OpLog 를 만들도록 일반화 시켰는데, 그는 이를
리눅스 커널의 경로 탐색, VM 역 매핑, 그리고 \co{stat()} 시스템콜에
적용시켰습니다~\cite{SilasBoydWickizerPhD}.
\iffalse

We have already seen one situation featuring high performance and
scalability for writers, namely the counting algorithms surveyed in
Chapter~\ref{chp:Counting}.
These algorithms featured partially partitioned data structures so
that updates can operate locally, while the more-expensive reads
must sum across the entire data structure.
Silas Boyd-Wickhizer has generalized this notion to produce
OpLog, which he has applied to
Linux-kernel pathname lookup, VM reverse mappings, and the \co{stat()} system
call~\cite{SilasBoydWickizerPhD}.
\fi

``Disruptor'' 라고 불리는 또다른 방법은 입력 데이터의 커다란 스트림을 처리하는
어플리케이션을 위해 설계되었습니다.
이 방법은 single-producer-single-consumer FIFO 큐를 위한 것으로, 동기화의
필요성을 최소화 시킵니다~\cite{AdrianSutton2013LCA:Disruptor}.
자바 어플리케이션에서 Disruptor 는 가비지 콜렉터의 사용을 최소화 시키기도
합니다.

그리고 물론, 가능하다면, 완전히 분할되었거나 ``파편화된 (sharded)'' 시스템들은
Chapter~\ref{cha:Partitioning and Synchronization Design} 에서 이야기한 것처럼
굉장한 성능과 확장성을 제공합니다.

다음 챕터는 여러 종류의 데이터 구조의 맥락에서 업데이트들을 살펴보겠습니다.
\iffalse

Another approach, called ``Disruptor'', is designed for applications
that process high-volume streams of input data.
The approach is to rely on single-producer-single-consumer FIFO queues,
minimizing the need for synchronization~\cite{AdrianSutton2013LCA:Disruptor}.
For Java applications, Disruptor also has the virtue of minimizing use
of the garbage collector.

And of course, where feasible, fully partitioned or ``sharded'' systems
provide excellent performance and scalability, as noted in
Chapter~\ref{cha:Partitioning and Synchronization Design}.

The next chapter will look at updates in the context of several types
of data structures.
\fi
