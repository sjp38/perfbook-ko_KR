% appendix/questions/after.tex
% mainfile: ../../perfbook.tex
% SPDX-License-Identifier: CC-BY-SA-3.0

\section{What Does ``After'' Mean?}
\label{sec:app:questions:What Does ``After'' Mean?}

``나중'' 은 직관적이지만 놀랍게도 어려운 개념입니다.
중요하고 비직관적인 문제 하나는 코드는 언제든 얼만큼이든 지연될 수 있다는
겁니다.
전역 상태를 타임스탬프 \qco{t} 와 정수형 필드 \qco{a}, \qco{b}, 그리고 \qco{c}
로 통신하는 생성 쓰레드와 소비 쓰레드를 생각해 봅시다.
생성 쓰레드는
\cref{lst:app:questions:After Producer Function} 에 보인 것처럼
현재 시간을 (1970 년 이래로의 시간을 초 단위로) 기록하고 \qco{a}, \qco{b},
그리고 \qco{c} 를 업데이트 하는 반복문을 수행합니다.
소비 쓰레드는
\cref{lst:app:questions:After Consumer Function} 에 보인 것처럼 역시 현재
시간을 기록하지만 생성 쓰레드의 타임스탬프를 필드 \qco{a}, \qco{b}, 그리고
\qco{c} 와 함께 복사하는 반복문을 수행합니다.
수행의 마지막에서, 이 소비 쓰레드는 예를 들면 시간이 뒤로 흐른 것으로 보이는 것
같은 이례적인 기록들의 리스트를 출력합니다.

\iffalse

``After'' is an intuitive, but surprisingly difficult concept.
An important non-intuitive issue is that code can be delayed at
any point for any amount of time.
Consider a producing and a consuming thread that communicate using
a global struct with a timestamp \qco{t} and integer fields \qco{a}, \qco{b},
and \qco{c}.
The producer loops recording the current time
(in seconds since 1970 in decimal),
then updating the values of \qco{a}, \qco{b}, and \qco{c},
as shown in \cref{lst:app:questions:After Producer Function}.
The consumer code loops, also recording the current time, but also
copying the producer's timestamp along with the fields \qco{a},
\qco{b}, and \qco{c}, as shown in
\cref{lst:app:questions:After Consumer Function}.
At the end of the run, the consumer outputs a list of anomalous recordings,
e.g., where time has appeared to go backwards.

\fi

\begin{listing}[htbp]
\input{CodeSamples/api-pthreads/QAfter/time@producer.fcv}
\caption{``After'' Producer Function}
\label{lst:app:questions:After Producer Function}
\end{listing}

\begin{listing}[htbp]
\input{CodeSamples/api-pthreads/QAfter/time@consumer.fcv}
\caption{``After'' Consumer Function}
\label{lst:app:questions:After Consumer Function}
\end{listing}

\QuickQuiz{
	이 예들에서 어떤 SMP 코딩 오류를 당신은 찾았나요?
	전체 코드를 위해선 \path{time.c} 를 보세요.

	\iffalse

	What SMP coding errors can you see in these examples?
	See \path{time.c} for full code.

	\fi

}\QuickQuizAnswer{
	\begin{enumerate}
	\item	반복문에서의 \co{barrier()} 나 \co{volatile} 사용 누락.
	\item	업데이트 쪽에서의 메모리 배리어 누락
	\item	생성 쓰레드와 소비 쓰레드 사이의 동기화 부재.

	\iffalse

	\item	Missing \co{barrier()} or \co{volatile} on tight loops.
	\item	Missing memory barriers on update side.
	\item	Lack of synchronization between producer and consumer.

	\fi

	\end{enumerate}
}\QuickQuizEnd

직관적으로는 생성 쓰레드가 타임스탬프나 값을 기록하는 데에는 시간이 많이 들지
않을 것이므로 생성 쓰레드와 소비 쓰레드 사이의 타임스탬프 차이는 상당히 작을
것이라 예상할 수 있겠습니다.
듀얼코어 1\,GHz x86 에서의 샘플 결과가
\cref{tab:app:questions:After Program Sample Output} 에 보여져 있습니다.
여기서, ``seq'' 행은 이 반복문을 통과한 횟수이고, ``time'' 행은 초 단위에서의
이상현상 횟수이며, ``delta'' 행은 생성 쓰레드의 타임스탬프가 소비 쓰레드의
그것을 뒤따르는 초 수이며 (음수는 소비 쓰레드가 타임스탬프를 생성 쓰레드가
취득하기 전에 취득했음을 의미합니다), ``a'', ``b'', 그리고 ``c'' 로 라벨링 된
행들은 이 값들이 소비 쓰레드가 얻어온 앞의 스냅샷 대비 얼마나 증가했는지를
보입니다.

\iffalse

One might intuitively expect that the difference between the producer
and consumer timestamps would be quite small, as it should not take
much time for the producer to record the timestamps or the values.
An excerpt of some sample output on a dual-core 1\,GHz x86 is shown in
\cref{tab:app:questions:After Program Sample Output}.
Here, the ``seq'' column is the number of times through the loop,
the ``time'' column is the time of the anomaly in seconds, the ``delta''
column is the number of seconds the consumer's timestamp follows that
of the producer (where a negative value indicates that the consumer
has collected its timestamp before the producer did), and the
columns labelled ``a'', ``b'', and ``c'' show the amount that these
variables increased since the prior snapshot collected by the consumer.

\fi

\begin{table}[htbp]
\rowcolors{1}{}{lightgray}
\renewcommand*{\arraystretch}{1.2}
\sisetup{group-digits=false}
\centering
\scriptsize
\begin{tabular}{rS[table-format=7.6]rS[table-format=3.0]S[table-format=3.0]S[table-format=3.0]}
\toprule
seq    & \multicolumn{1}{c}{time (seconds)} & delta~  &  a &  b &  c \\
\midrule
17563: & 1152396.251585 & ($-16.928$) & 27 & 27 & 27 \\
18004: & 1152396.252581 & ($-12.875$) & 24 & 24 & 24 \\
18163: & 1152396.252955 & ($-19.073$) & 18 & 18 & 18 \\
18765: & 1152396.254449 & ($-148.773$) & 216 & 216 & 216 \\
19863: & 1152396.256960 & ($-6.914$) & 18 & 18 & 18 \\
21644: & 1152396.260959 & ($-5.960$) & 18 & 18 & 18 \\
23408: & 1152396.264957 & ($-20.027$) & 15 & 15 & 15 \\
\bottomrule
\end{tabular}
\caption{``After'' Program Sample Output}
\label{tab:app:questions:After Program Sample Output}
\end{table}

왜 시간이 거꾸로 흐르는 걸까요?
괄호 안의 수는 마이크로세컨드 단위로, 큰 수 하나는 10 마이크로세컨드를
넘어서며, 어떤건 100 마이크로세컨드까지 넘깁니다!
이 CPU 는 잠재적으로 그 동안 100,000 개의 인스트럭션을 수행할 수 있음을
알아두시기 바랍니다.

\iffalse

Why is time going backwards?
The number in parentheses is the difference in microseconds, with
a large number exceeding 10 microseconds, and one exceeding even
100 microseconds!
Please note that this CPU can potentially execute more than 100,000
instructions in that time.

\fi

\begin{fcvref}[ln:api-pthreads:QAfter:time]
한가지 가능한 경우의 수가 다음 사건의 연속 시에 가능합니다:
\begin{enumerate}
\item	소비 쓰레드가 타임스탬프를 얻습니다
	(\Cref{lst:app:questions:After Consumer Function},
	\clnref{consumer:tod}).
\item	소비 쓰레드가 preemption 당합니다.
\item	임의의 시간이 흐릅니다.
\item	생성 쓰레드가 타임스탬프를 얻습니다
	(\Cref{lst:app:questions:After Producer Function},
	\clnref{producer:tod}).
\item	소비 쓰레드가 수행을 재개하고, 이 생성 쓰레드의 타임스탬프를 가져옵니다
	(\Cref{lst:app:questions:After Consumer Function},
	\clnref{consumer:prodtod}).
\end{enumerate}

\iffalse

One possible reason is given by the following sequence of events:
\begin{enumerate}
\item	Consumer obtains timestamp
	(\Cref{lst:app:questions:After Consumer Function},
	\clnref{consumer:tod}).
\item	Consumer is preempted.
\item	An arbitrary amount of time passes.
\item	Producer obtains timestamp
	(\Cref{lst:app:questions:After Producer Function},
	\clnref{producer:tod}).
\item	Consumer starts running again, and picks up the producer's
	timestamp
	(\Cref{lst:app:questions:After Consumer Function},
	\clnref{consumer:prodtod}).
\end{enumerate}

\fi

이 시나리오에서, 생성 쓰레드의 타임스탬프는 소비 쓰레드가 타임스탬프를 가져온
시간보다 임의의 양만큼 뒤일 겁니다.

여러분의 SMP 코드에서 ``after'' 의 의미로 괴롭힘 당하는 걸 어떻게 막을까요?

단순히 SMP 기능들을 설계된 대로 사용하면 됩니다.

\iffalse

In this scenario, the producer's timestamp might be an arbitrary
amount of time after the consumer's timestamp.

How do you avoid agonizing over the meaning of ``after'' in your
SMP code?

Simply use SMP primitives as designed.

\fi

이 예에서, 가장 쉬운 방법은 락킹을 사용하는 것으로, 예를 들어
\cref{lst:app:questions:After Producer Function} 의
\clnref{producer:tod} 전에 생성 쓰레드에서 락을 잡고
\cref{lst:app:questions:After Consumer Function} 의
\clnref{consumer:tod} 전에 소비 쓰레드에서 락을 잡는 겁니다.
이 락은 또한 
\cref{lst:app:questions:After Producer Function} 의
\clnref{producer:upd:c} 뒤, 그리고
\cref{lst:app:questions:After Consumer Function} 의
\clnref{consumer:upd:c} 뒤에서 해제되어야 합니다.
이 락은
\cref{lst:app:questions:After Producer Function} 의
\clnrefrange{producer:tod}{producer:upd:c} 와
\cref{lst:app:questions:After Consumer Function} 의
\clnrefrange{consumer:tod}{consumer:upd:c} 가 서로를 {\em 배제} 하게 하는데,
달리 말하면 각자에 대해 어토믹하게 수행됩니다.
이 점이
\cref{fig:app:questions:Effect of Locking on Snapshot Collection} 에 표현되어
있습니다:
이 락킹은 코드 상자 모두가 시간상 겹쳐지는 것을 막아서 소비 쓰레드의
타임스탬프는 앞의 생성 쓰레드의 타임스탬프 뒤에 얻어져야만 하게 합니다.
이 그림의 각 상자의 코드 조각은 ``크리티컬 섹션'' 이라 명명됩니다; 한번에 그런
크리티컬 섹션 중 오직 하나만이 수행됩니다.
\end{fcvref}

\iffalse

In this example, the easiest fix is to use locking, for example,
acquire a lock in the producer before \clnref{producer:tod} in
\cref{lst:app:questions:After Producer Function} and in
the consumer before \clnref{consumer:tod} in
\cref{lst:app:questions:After Consumer Function}.
This lock must also be released after \clnref{producer:upd:c} in
\cref{lst:app:questions:After Producer Function} and
after \clnref{consumer:upd:c} in
\cref{lst:app:questions:After Consumer Function}.
These locks cause the code segments in
\clnrefrange{producer:tod}{producer:upd:c} of
\cref{lst:app:questions:After Producer Function} and in
\clnrefrange{consumer:tod}{consumer:upd:c} of
\cref{lst:app:questions:After Consumer Function} to {\em exclude}
each other, in other words, to run atomically with respect to each other.
This is represented in
\cref{fig:app:questions:Effect of Locking on Snapshot Collection}:
the locking prevents any of the boxes of code from overlapping in time, so
that the consumer's timestamp must be collected after the prior
producer's timestamp.
The segments of code in each box in this figure are termed
``critical sections''; only one such critical section may be executing
at a given time.
\end{fcvref}

\fi

\begin{figure}[htb]
\centering
\includegraphics{appendix/questions/after-snapshot}
\caption{Effect of Locking on Snapshot Collection}
\label{fig:app:questions:Effect of Locking on Snapshot Collection}
\end{figure}

이 락킹의 추가는
\cref{fig:app:questions:Locked After Program Sample Output} 에 보인 결과를
초래합니다.
여기엔 시간이 뒤로 흐르는 경우는 사라졌고, 그대신 소비 쓰레드에 의한 뒤따르는
읽기 사이에 1,000 회의 추가가 만들어져 있습니다.

\iffalse

This addition of locking results in output as shown in
\cref{fig:app:questions:Locked After Program Sample Output}.
Here there are no instances of time going backwards, instead,
there are only cases with more than 1,000 counts difference between
consecutive reads by the consumer.

\fi

\begin{table}[htbp]
\renewcommand*{\arraystretch}{1.2}
\sisetup{group-digits=false}
\centering
\scriptsize
\begin{tabular}{rS[table-format=7.6]rS[table-format=4.0]S[table-format=4.0]S[table-format=4.0]}
\toprule
seq    & \multicolumn{1}{c}{time (seconds)} & delta~  &  a &  b &  c \\
\midrule
58597:  & 1156521.556296 & ($3.815$) & 1485 & 1485 & 1485 \\
403927: & 1156523.446636 & ($2.146$) & 2583 & 2583 & 2583 \\
\bottomrule
\end{tabular}
\caption{Locked ``After'' Program Sample Output}
\label{fig:app:questions:Locked After Program Sample Output}
\end{table}

\QuickQuiz{
	연속된 소비 쓰레드의 읽기 사이에 어떻게 그렇게 큰 차이가 발생하죠?
	전체 코드를 위해선 \path{timelocked.c} 를 보시기 바랍니다.

	\iffalse

	How could there be such a large gap between successive
	consumer reads?
	See \path{timelocked.c} for full code.

	\fi

}\QuickQuizAnswer{
	\begin{enumerate}
	\item	이 소비 쓰레드가 오랫동안 preemption 당했을 수도 있습니다.
	\item	오랫동안 수행되는 인터럽트가 이 소비 쓰레드를 지연시켰을 수도
		있습니다.
	\item	캐쉬미스가 이 소비 쓰레드를 지연시켰을 수도 있습니다.
	\item	생성 쓰레드는 소비 쓰레드보다 빠른 CPU 에서 수행되었을 수도
		있습니다 (예를 들어, CPU 들 중 하나는 열 처리나 전력 소비
		제한을 위해 처리 속도를 낮췄을 수도 있습니다).

	\iffalse

	\item	The consumer might be preempted for long time periods.
	\item	A long-running interrupt might delay the consumer.
	\item	Cache misses might delay the consumer.
	\item	The producer might also be running on a faster CPU than is the
		consumer (for example, one of the CPUs might have had to
		decrease its
		clock frequency due to heat-dissipation or power-consumption
		constraints).

	\fi

	\end{enumerate}
}\QuickQuizEnd

요약하자면, 여러분이 배타적 락을 획득했다면, 여러분은 여러분이 그 락을 쥔
상태에서 행한 모든 일이 그 락을 전에 쥔 쓰레드가 행한 모든 일 뒤에 일어나는
것을 {\em 압니다}, 최소한 transactional lock elision
(Section~\ref{sec:future:Semantic Differences} 을 참고하세요) 을
주고받는다면요.
어떤 CPU 가 메모리 배리어를 수행하거나 하지 않았는지 걱정할 필요도, CPU 나
컴파일러의 코드 재배치에 걱정할 필요도 없습니다---삶은 단순합니다.
물론, 이 락킹이 이 두 조각의 코드가 동시에 수행되는 것을 막는다는 것은 이
프로그램의 멀티프로세서에서의 성능 증가를 제한하는, ``안전하지만 느린'' 상황을
초래할 수 있음을 의미합니다.
\Cref{cha:Partitioning and Synchronization Design} 는 많은 경우에서 성능과
확장성을 증가시키는 방법들을 설명합니다.

그러나, 대부분의 경우, 스스로가 코드 조각들이 전이나 후에 수행되는지에 대해
걱정하고 있음을 발견한다면, 여러분은 표준 기능의 사용으로 상황을 낫게 하기 위한
힌트를 여기서 얻어야 합니다.
이 기능들이 그 걱정을 여러분 대신 하게 하세요.

\iffalse

In summary, if you acquire an exclusive lock, you {\em know} that
anything you do while holding that lock will appear to happen after
anything done by any prior holder of that lock, at least give or
take transactional lock elision
(see Section~\ref{sec:future:Semantic Differences}).
No need to worry about which CPU did or did not execute a memory
barrier, no need to worry about the CPU or compiler reordering
operations---life is simple.
Of course, the fact that this locking prevents these two pieces of
code from running concurrently might limit the program's ability
to gain increased performance on multiprocessors, possibly resulting
in a ``safe but slow'' situation.
\Cref{cha:Partitioning and Synchronization Design} describes ways of
gaining performance and scalability in many situations.

However, in most cases, if you find yourself worrying about what happens
before or after a given piece of code, you should take this as a hint to
make better use of the standard primitives.
Let these primitives do the worrying for you.

\fi
