% formal/formal.tex
% SPDX-License-Identifier: CC-BY-SA-3.0

\QuickQuizChapter{chp:Formal Verification}{Formal Verification}
%
\Epigraph{Beware of bugs in the above code; I have only proved it correct,
	  not tried it.}{\emph{Donald Knuth}}

\OriginallyPublished{Chapter}{chp:Formal Verification}{Formal Verification}{Linux Weekly News}{PaulEMcKenney2007QRCUspin,PaulEMcKenney2008dynticksRCU,PaulEMcKenney2011ppcmem}

병렬 알고리즘들은 작성하기 어렵고, 디버깅 하기는 그보다도 더 어렵습니다.
테스트는 필수이긴 하지만 race condition 은 극단적으로 낮은 발현 확률을 가질 수
있기 때문에, 그것만으로는 충분하지 않습니다.
올바름의 증명은 가치있을 수 있습니다만, 그것도 결국은 원래의 알고리즘이
그렇듯이 사람에 의한 에러에 취약합니다.
또한, 올바름의 증명은 가정 자체에 있는 에러, 요구사항의 한계, 아래에 있는
소프트웨어나 하드웨어 기능들에 대한 잘못된 이해, 또는 증명을 할 생각을 하지
못한 에러들은 찾지 못할 것입니다.
이는 정형적 방법들은 테스트를 대체할 수는 없음을 의미합니다만, 정형적 방법들은
검증 도구상자의 또하나의 가치있는 한가지 이상도 이하도 아닙니다.
\iffalse

Parallel algorithms can be hard to write, and even harder to debug.
Testing, though essential, is insufficient, as fatal race conditions
can have extremely low probabilities of occurrence.
Proofs of correctness can be valuable, but in the end are just as
prone to human error as is the original algorithm.
In addition, a proof of correctness cannot be expected to find errors
in your assumptions, shortcomings in the requirements,
misunderstandings of the underlying software or hardware primitives,
or errors that you did not think to construct a proof for.
This means that formal methods can never replace testing, however,
formal methods are nevertheless a valuable addition to your validation toolbox.
\fi

어떻게든 모든 race condition 들을 찾아낼 수 있는 도구를 갖는다면 매우 도움이
될겁니다.
그런 도구들이 여럿 존재하는데, 예를 들어서
Section~\ref{sec:formal:General-Purpose State-Space Search}
는 범용 상태 공간 탐색 도구들인 Promela 와 Spin 을 소개하고,
Section~\ref{sec:formal:Special-Purpose State-Space Search}
는 비슷하게 특수 목적의 ppcmem 와 cppmem 도구들을 소개하며,
Section~\ref{sec:formal:Axiomatic Approaches}
는 자명한 방법의 예를 보고,
Section~\ref{sec:formal:SAT Solvers}
에서는 간단히 SAT solver 들을 살펴보며, 마지막으로
Section~\ref{sec:formal:Stateless Model Checkers}
에서는 stateless model checker 에 대한 간단한 개괄을 제공하고,
Section~\ref{sec:formal:Summary}
에서는 병렬 알고리즘을 검증하기 위해 정형 검증을 사용하는 것에 대해 요약합니다.
\iffalse

It would be very helpful to have a tool that could somehow locate
all race conditions.
A number of such tools exist, for example,
Section~\ref{sec:formal:State-Space Search} provides an
introduction to the general-purpose state-space search tools Promela and Spin,
Section~\ref{sec:formal:Special-Purpose State-Space Search}
similarly introduces the special-purpose ppcmem and cppmem tools,
Section~\ref{sec:formal:Axiomatic Approaches}
looks at an example axiomatic approach,
Section~\ref{sec:formal:SAT Solvers}
briefly overviews SAT solvers,
Section~\ref{sec:formal:Stateless Model Checkers}
briefly overviews stateless model checkers,
and finally
Section~\ref{sec:formal:Summary}
sums up use of formal-verification tools for verifying parallel algorithms.
\fi

% formal/spinhint.html

\section{State-Space Search}
\label{sec:formal:State-Space Search}
%
\epigraph{Follow every byway / Every path you know.}
	 {\emph{``Climb Every Mountain'', Rodgers \& Hammerstein}}

이 섹션은 많은 종류의 멀티 쓰레드 기반 코드의 전체 상태 공간을 탐색하는데에
사용될 수 있는 범용의 도구인 Promela 와 spin 을 알아봅니다.
이것들은 또한 데이터 통신 프로토콜을 증명하는 데에도 유용합니다.
Section~\ref{sec:formal:Promela and Spin}
은 두개의 웜업을 위한 어토믹 하지 않은 버전과 어토믹한 버전의 값 증가를
증명하는 예제를 포함해서 Promela 와 spin 에 대해 소개합니다.
Section~\ref{sec:formal:How to Use Promela}
은 커맨드 라인에서의 사용 예와 Promela 와 C 의 문법 비교를 포함해서 Promela 를
설명합니다.
Section~\ref{sec:formal:Promela Example: Locking}
에서는 락킹을 증명하는데에 Promela 가 어떻게 사용될 수 있는지 보이고,
Section~\ref{sec:formal:Promela Example: QRCU}
는 일반적이지 않은 ``QRCU'' 라는 이름의 RCU 구현을 증명하는데에 Promela 를
사용해 보며, 마지막으로
Section~\ref{sec:formal:Promela Parable: dynticks and Preemptible RCU}
에서는 RCU 의 dyntick 구현에 Promela 를 적용해 봅니다.
\iffalse

This section features the general-purpose Promela and Spin tools,
which may be used to carry out a full
state-space search of many types of multi-threaded code.
They are also quite useful for verifying data communication protocols.
Section~\ref{sec:formal:Promela and Spin}
introduces Promela and Spin, including a couple of warm-up exercises
verifying both non-atomic and atomic increment.
Section~\ref{sec:formal:How to Use Promela}
describes use of Promela, including example command lines and a
comparison of Promela syntax to that of C.
Section~\ref{sec:formal:Promela Example: Locking}
shows how Promela may be used to verify locking,
\ref{sec:formal:Promela Example: QRCU}
uses Promela to verify an unusual implementation of RCU named ``QRCU'',
and finally
Section~\ref{sec:formal:Promela Parable: dynticks and Preemptible RCU}
applies Promela to RCU's dyntick-idle implementation.
\fi

\subsection{Promela and Spin}
\label{sec:formal:Promela and Spin}

Promela 는 증명 프로토콜들을 위해 설계된 언어입니다만, 작은 병렬 알고리즘들을
검증하는 데에도 사용될 수 있습니다.
당신은 당신의 알고리즘과 정확성 제약을 C 같은 언어인 Promela 로 다시 코딩하고,
그다음에 Spin 을 사용해 그걸 C 프로그램으로 변환하고 나면 그걸 컴파일하고
실행해 볼 수 있습니다.
그 결과 나오는 프로그램은 당신의 알고리즘의 전체 상태 공간 탐색을 포함하고 있게
되어서, 당신이 당신의 Promela 프로그램에 넣어둔 단정문들에 대한 반례들을
찾아주거나 검증하게 됩니다.
\iffalse

Promela is a language designed to help verify protocols, but which
can also be used to verify small parallel algorithms.
You recode your algorithm and correctness constraints in the C-like
language Promela, and then use Spin to translate it into a C program
that you can compile and run.
The resulting program conducts a full state-space search of your
algorithm, either verifying or finding counter-examples for
assertions that you can include in your Promela program.
\fi

이 전체 상태 공간은 매우 강력할 수 있습니다만, 양날의 검이 될 수 있기도 합니다.
당신의 알고리즘이 너무 복잡하거나 당신의 Promela 구현이 주의깊게 만들어 지지
않았다면, 메모리에 들어갈 수 있는 것보다 더 많은 상태들이 존재할 수도 있습니다.
더 나아가서, 충분한 메모리를 가졌다 하더라도, 상태 공간 탐색은 예상된 전체
시간보다 더 긴 시간동안 수행될 수 있습니다.
그러므로, 이 도구는 조그맣지만 복잡한 병렬 알고리즘들을 위해 사용하시기
바랍니다.
(전체 리눅스 커널은 말할 것도 없고) 보통 크기의 알고리즘들에 이걸 낙관적으로
적용하는 것도 나쁜 결과로 끝나게 될겁니다.

Promela 와 Spin 은 \url{http://spinroot.com/spin/whatispin.html} 에서 다운로드
받을 수 있습니다.
\iffalse

This full-state search can be extremely powerful, but can also be a two-edged
sword.
If your algorithm is too complex or your Promela implementation is
careless, there might be more states than fit in memory.
Furthermore, even given sufficient memory, the state-space search might
well run for longer than the expected lifetime of the universe.
Therefore, use this tool for compact but complex parallel algorithms.
Attempts to naively apply it to even moderate-scale algorithms (let alone
the full Linux kernel) will end badly.

Promela and Spin may be downloaded from
\url{http://spinroot.com/spin/whatispin.html}.
\fi

앞의 사이트는 또한 Gerard Holzmann 의 Promela 와 Spin 에 대한 훌륭한
책~\cite{Holzmann03a} 으로의 링크를 제공하며,
\url{http://www.spinroot.com/spin/Man/index.html} 에서 시작하는 검색 가능한
온라인 레퍼런스들도 제공합니다.

이 문서의 뒷부분은 병렬 알고리즘들을 디버깅 하는데에 Promela 를 어떻게
사용하는지를 간단한 예제로 시작해서 더 복잡한 경우들로 나아가면서 설명합니다.
\iffalse

The above site also gives links to Gerard Holzmann's excellent
book~\cite{Holzmann03a} on Promela and Spin,
as well as searchable online references starting at:
\url{http://www.spinroot.com/spin/Man/index.html}.

The remainder of this section describes how to use Promela to debug
parallel algorithms, starting with simple examples and progressing to
more complex uses.
\fi

\subsubsection{Promela Warm-Up: Non-Atomic Increment}
\label{sec:formal:Promela Warm-Up: Non-Atomic Increment}
\NoIndentAfterThis

\begin{listing}[tbp]
\input{CodeSamples/formal/promela/increment@whole.fcv}
\caption{Promela Code for Non-Atomic Increment}
\label{lst:formal:Promela Code for Non-Atomic Increment}
\end{listing}

Listing~\ref{lst:analysis:Promela Code for Non-Atomic Increment}
는 어토믹하지 않은 값 증가로 인해 발생하는, 교범적인 race condition 을 보이고
있습니다.
Line~1 은 수행할 프로세스들의 수를 정의하고 (우린 상태 공간에의 영향을 보기
위해 이 값을 바꿔볼 겁니다), line~3 은 카운터를 정의하고, line~4 는 line~29-39
에 있는 단정문을 구현하는데 사용될 겁니다.

Line~6-13 은 카운터를 어토믹하지 않게 증가시키는 프로세스를 정의합니다.
인자 \co{me} 는 프로세스의 번호로, 코드의 뒤에 있는 초기화 블록에서 설정됩니다.
간단한 Promela 구문들은 모두 어토믹한 것으로 가정되기 때문에, 우리는 이 값
증가를 line~10-11 의 두개의 문장으로 쪼개야만 합니다.
Line~12 에서의 값 할당은 프로세스의 완료를 표시합니다.
Spin 시스템은 모든 가능한 상태들의 시퀀스들을 포함해 상태 공간을 모두 탐색하기
때문에, 전통적인 테스트에서라면 사용되었을 수 있는 루프는 필요치 않습니다.
\iffalse

\begin{lineref}[ln:formal:promela:increment:whole]
Listing~\ref{lst:formal:Promela Code for Non-Atomic Increment}
demonstrates the textbook race condition
resulting from non-atomic increment.
Line~\lnref{nprocs} defines the number of processes to run (we will vary this
to see the effect on state space), line~\lnref{count} defines the counter,
and line~\lnref{prog} is used to implement the assertion that appears on
lines~\lnref{assert:b}-\lnref{assert:e}.

Lines~\lnref{proc:b}-\lnref{proc:e} define a process that increments
the counter non-atomically.
The argument \co{me} is the process number, set by the initialization
block later in the code.
Because simple Promela statements are each assumed atomic, we must
break the increment into the two statements on
lines~\lnref{incr:b}-\lnref{incr:e}.
The assignment on line~\lnref{setprog} marks the process's completion.
Because the Spin system will fully search the state space, including
all possible sequences of states, there is no need for the loop
that would be used for conventional testing.
\fi

Line~15-40 은 초기화 블락으로, 제일 처음에 수행됩니다.
Line~19-28 은 정말로 초기화를 하고, line~29-39 는 단정을 수행합니다.
이 두 부분은 불필요하게 상태 공간을 증가시키는 걸 막기 위해 모두 어토믹
블락으로 되어 있습니다: 이것들은 테스트 하려는 알고리즘의 부분이 아니기 때문에,
그것들을 어토믹으로 만듦으로써 검증 범위를 줄이는 것입니다.

Line~21-27 의 do-od 구조는 Promela 루프를 구현하는데, case 라벨에 expression
들을 허용하는 \co{switch} 문을 담고 있는 C {\tt for (;;)} 루프로 생각될 수
있습니다.
({\tt ::} 접두사를 갖는) 조건 블락들은 비결정론적으로 스캔됩니다만, 이 경우에는
한번에 하나의 조건만이 참이 될 것입니다.
Line~22-25 에 있는 do-od 의 첫번째 블락은 i-번째 카운터 증가 프로세스의
progress 셀을 초기화하고, i-번째 카운터 증가 프로세스를 수행시키고, 변수 \co{i}
를 증가시킵니다.
line~26 에 있는 do-od 의 두번째 블락은 이 프로세스들이 모두 시작되면 루프를
빠져나옵니다.
\iffalse

Lines~\lnref{init:b}-\lnref{init:e} are the initialization block,
which is executed first.
Lines~\lnref{doinit:b}-\lnref{doinit:e} actually do the initialization,
while lines~\lnref{assert:b}-\lnref{assert:e}
perform the assertion.
Both are atomic blocks in order to avoid unnecessarily increasing
the state space: because they are not part of the algorithm proper,
we lose no verification coverage by making them atomic.

The \co{do-od} construct on lines~\lnref{dood1:b}-\lnref{dood1:e}
implements a Promela loop,
which can be thought of as a C \co{for (;;)} loop containing a
\co{switch} statement that allows expressions in case labels.
The condition blocks (prefixed by \co{::})
are scanned non-deterministically,
though in this case only one of the conditions can possibly hold at a given
time.
The first block of the \co{do-od} from
lines~\lnref{block1:b}-\lnref{block1:e}
initializes the i-th
incrementer's progress cell, runs the i-th incrementer's process, and
then increments the variable \co{i}.
The second block of the \co{do-od} on
line~\lnref{block2} exits the loop once
these processes have been started.
\fi

Line~29-39 의 어토믹 블락 또한 프로그레스 카운터를 더하는, 비슷한 do-od 루프를
담고 있습니다.
Line~38 의 {\tt assert()} 문은 모든 프로세스가 완료되었는지, 그렇다면 모든
카운트가 정확히 기록되었는지 검증합니다.

독자 여러분은 이 프로그램들을 다음과 같이 빌드하고 실행해 볼 수 있습니다:
\iffalse

The atomic block on lines~\lnref{assert:b}-\lnref{assert:e} also contains
a similar \co{do-od}
loop that sums up the progress counters.
The \co{assert()} statement on line~\lnref{assert} verifies that
if all processes
have been completed, then all counts have been correctly recorded.
\end{lineref}

You can build and run this program as follows:
\fi

\begin{VerbatimU}
spin -a increment.spin      # Translate the model to C
cc -DSAFETY -o pan pan.c    # Compile the model
./pan                       # Run the model
\end{VerbatimU}

\begin{listing}[tbp]
\VerbatimInput[numbers=none,fontsize=\scriptsize]{CodeSamples/formal/promela/increment.spin.lst}
\vspace*{-9pt}
\caption{Non-Atomic Increment Spin Output}
\label{lst:formal:Non-Atomic Increment Spin Output}
\end{listing}

이 수행의 결과로 나올 수 있는 출력이
Listing~\ref{lst:analysis:Non-Atomic Increment spin Output}
에 보여져 있습니다.
첫번째 줄은 우리의 단정이 깨졌음을 이야기 합니다 (어토믹하지 않은 값 증가로
인해 예상되었던대로입니다!).
두번째 줄은 어떻게 이 단정이 깨졌는지에 대한 설명을 \co{trail} 파일에 썼음을
이야기 합니다.
``Warning''  줄은 우리의 모델에 있어서 모든 것이 좋지 않았음을 반복합니다.
두번째 문단은 진행된 상태 탐색의 타입을 설명하는데, 이 경우에는 단정 위반과
무효한 종료 상태들이었습니다.
세번째 문단은 상태 크기에 대한 통계를 보여줍니다: 이 작은 모델은 45개의
상태만을 가졌습니다.
마지막 줄은 메모리 사용량을 보입니다.

\co{trail} 파일 안의 정보는 다음의 커맨드를 통해 사람이 읽을 수 있는 형태로
만들어질 수 있습니다:
\iffalse

This will produce output as shown in
Listing~\ref{lst:formal:Non-Atomic Increment Spin Output}.
The first line tells us that our assertion was violated (as expected
given the non-atomic increment!).
The second line tells us that a \co{trail} file was written describing how the
assertion was violated.
The ``Warning'' line reiterates that all was not well with our model.
The second paragraph describes the type of state-search being carried out,
in this case for assertion violations and invalid end states.
The third paragraph gives state-size statistics: this small model had only
45 states.
The final line shows memory usage.

The \co{trail} file may be rendered human-readable as follows:
\fi

\begin{VerbatimU}
spin -t -p increment.spin
\end{VerbatimU}

\begin{listing*}[htbp]
\VerbatimInput[numbers=none,fontsize=\scriptsize]{CodeSamples/formal/promela/increment.spin.trail.lst}
\vspace*{-9pt}
\caption{Non-Atomic Increment Error Trail}
\label{lst:formal:Non-Atomic Increment Error Trail}
\end{listing*}

이는
Listing~\ref{lst:analysis:Non-Atomic Increment Error Trail}
에 보여진 것과 같은 결과를 보일 겁니다.
보여지듯이, 초기화 블락의 첫번째 부분이 두개의 카운터 증가 프로세스들을
생성했고, 두 프로세스는 모두 카운터 값을 가져간 후에 값을 증가하고 다시
저장시켰으며, 그중 하나의 카운트를 잃었습니다.
그리고 나서, 전체 상태가 표시된 후에 단정문이 판정되었습니다.
\iffalse

This gives the output shown in
Listing~\ref{lst:formal:Non-Atomic Increment Error Trail}.
As can be seen, the first portion of the init block created both
incrementer processes, both of which first fetched the counter,
then both incremented and stored it, losing a count.
The assertion then triggered, after which the global state is displayed.
\fi

\subsubsection{Promela Warm-Up: Atomic Increment}
\label{sec:formal:Promela Warm-Up: Atomic Increment}
\NoIndentAfterThis

\begin{listing}[htbp]
\input{CodeSamples/formal/promela/atomicincrement@incrementer.fcv}
\caption{Promela Code for Atomic Increment}
\label{lst:formal:Promela Code for Atomic Increment}
\end{listing}

\begin{listing}[htbp]
\VerbatimInput[numbers=none,fontsize=\scriptsize]{CodeSamples/formal/promela/atomicincrement.spin.lst}
\vspace*{-9pt}
\caption{Atomic Increment Spin Output}
\label{lst:formal:Atomic Increment Spin Output}
\end{listing}

이 예제의 값 증가 프로세스의 코드를
Listing~\ref{lst:analysis:Promela Code for Atomic Increment} 에 보인 것처럼
고치기는 쉽습니다.
Promela statement 들은 어토믹하기 때문에, 단순히 두개의 statement 들을 {\tt
counter = counter + 1} 로 바꿀수도 있겠습니다.
어떤 경우든, 이 수정된 모델을 돌려보면
Listing~\ref{lst:analysis:Atomic Increment spin Output} 에 보여진 것처럼 에러
없는 상태 공간을 보여줍니다.
\iffalse

It is easy to fix this example by placing the body of the incrementer
processes in an atomic blocks as shown in
Listing~\ref{lst:formal:Promela Code for Atomic Increment}.
One could also have simply replaced the pair of statements with
\co{counter = counter + 1}, because Promela statements are
atomic.
Either way, running this modified model gives us an error-free traversal
of the state space, as shown in
Listing~\ref{lst:formal:Atomic Increment Spin Output}.
\fi

\begin{table}
\rowcolors{1}{}{lightgray}
\small
\renewcommand*{\arraystretch}{1.2}
\centering
\begin{tabular}{S[table-format = 1.0]S[table-format = 7.0]S[table-format = 3.1]}
	\toprule
	\multicolumn{1}{l}{\# incrementers} &
		\multicolumn{1}{r}{\# states} &
			\multicolumn{1}{r}{total memory usage (MB)} \\
	\midrule
	1 &		        11 &        128.7 \\
	2 &		        52 &        128.7 \\
	3 &		       372 &        128.7 \\
	4 &		     3 496 &        128.9 \\
	5 &		    40 221 &        131.7 \\
	6 &		   545 720 &        174.0 \\
	7 &		 8 521 446 &        881.9 \\
	\bottomrule
\end{tabular}
\caption{Memory Usage of Increment Model}
\label{tab:advsync:Memory Usage of Increment Model}
\end{table}

Table~\ref{tab:advsync:Memory Usage of Increment Model}
은 ({\tt NUMPROCS} 로 재정의된) 모델링된 카운터 증가 프로세스의 갯수에 대해
상태의 갯수와 소비된 메모리의 양을 보입니다:

따라서 불필요하게 커다란 모델들을 수행해 보는 것은 약간 권장되지 않습니다, 비록
652MB 는 현대의 데스크탑과 랩탑 기계의 한계 내에 있긴 하지만요.

이 예제를 두고서, Promela 모델을 분석하는데 사용되는 커맨드들을 깊게 알아보고
더 자세한 예제들을 봅시다.
\iffalse

Table~\ref{tab:advsync:Memory Usage of Increment Model}
shows the number of states and memory consumed
as a function of number of incrementers modeled
(by redefining \co{NUMPROCS}):

Running unnecessarily large models is thus subtly discouraged, although
882\,MB is well within the limits of modern desktop and laptop machines.

With this example under our belt, let's take a closer look at the
commands used to analyze Promela models and then look at more
elaborate examples.
\fi

\subsection{How to Use Promela}
\label{sec:formal:How to Use Promela}

소스 파일 \path{qrcu.spin} 을 가지고서, 다음과 같은 커맨드들을 사용할 수
있습니다:
\iffalse

Given a source file \path{qrcu.spin}, one can use the following commands:
\fi

\begin{description}[style=nextline]
\item	[\tco{spin -a qrcu.spin}]
	상태 머신을 완전히 탐색하는 \path{pan.c} 파일을 만들어 냅니다.
\item	[\tco{cc -DSAFETY -o pan pan.c}]
	생성된 상태 머신 탐색 코드를 컴파일 합니다.  \co{-DSAFETY} 는
	단정문만을 (그리고 \co{never} statement 들을) 가지고 있다면 적절한
	최적화를 만들어냅니다.  만약 여러분이 liveness, fairness, 또는
	forward-progress 체크를 가지고 있다면, \co{-DSAFETY} 없이 컴파일을
	해야할 수도 있습니다.  여러분이 그것을 사용할 수 있으면서 \co{-DSAFETY}
	를 사용하지 않는다면, 프로그램은 당신에게 그걸 알릴 겁니다.

	\co{-DSAFETY} 로 만들어지는 최적화는 일의 속도를 엄청나게 높여줄
	것이므로, 사용할 수 있다면 사용해야 합니다.
	\co{-DSAFETY} 를 사용할 수 없는 환경 가운데 예를 들어보자면 \co{-DNP}
	를 통해 (``non-progress cycles'' 라고도 알려져 있는) livelock 을 체크할
	때입니다.
\iffalse

\item	[\tco{spin -a qrcu.spin}]
	Create a file \path{pan.c} that fully searches the state machine.
\item	[\tco{cc -DSAFETY [-DCOLLAPSE] [-DMA=N] -o pan pan.c}]
	Compile the generated state-machine search.  The \co{-DSAFETY}
	generates optimizations that are appropriate if you have only
	assertions (and perhaps \co{never} statements).  If you have
	liveness, fairness, or forward-progress checks, you may need
	to compile without \co{-DSAFETY}.  If you leave off \co{-DSAFETY}
	when you could have used it, the program will let you know.

	The optimizations produced by \co{-DSAFETY} greatly speed things
	up, so you should use it when you can.
	An example situation where you cannot use \co{-DSAFETY} is
	when checking for livelocks (AKA ``non-progress cycles'')
	via \co{-DNP}.
\fi

	The optional \co{-DCOLLAPSE} generates code for a state vector
	compression mode.

	Another optional flag \co{-DMA=N} generates code for a slow
	but aggressive state-space memory compression mode.
\item	[\tco{./pan [-mN] [-wN]}]
	This actually searches the state space.  The number of states
	can reach into the tens of millions with very small state
	machines, so you will need a machine with large memory.
	For example, \path{qrcu.spin} with 3~updaters and 2~readers required
	10.5\,GB of memory even with the \co{-DCOLLAPSE} flag.

	If you see a message from \co{./pan} saying:
	``error: max search depth too small'', you need to increase
	the maximum depth by a \co{-mN} option for a complete search.
	The default is \co{-m10000}.

	The \co{-wN} option specifies the hashtable size.
	The default for full state-space search is \co{-w24}.\footnote{
		As of Spin Version 6.4.6 and 6.4.8. In the online manual of
		Spin dated 10 July 2011, the default for exhaustive search
		mode is said to be \co{-w19}, which does not meet
		the actual behavior.}

	If you aren't sure whether your machine has enough memory,
	run \co{top} in one window and \co{./pan} in another.  Keep the
	focus on the \co{./pan} window so that you can quickly kill
	execution if need be.  As soon as CPU time drops much below
	100\,\%, kill \co{./pan}.  If you have removed focus from the
	window running \co{./pan}, you may wait a long time for the
	windowing system to grab enough memory to do anything for
	you.

	출력을 캡쳐해 두는 것을 잊지 마세요, 특히나 당신이 원격의 기계에서
	작업하고 있다면요.

	만약 당신의 모델이 forward-progress 체크를 포함하고 있다면, \co{./pan}
	에 커맨드 라인 인자로 \co{-f} 를 줌으로써 ``weak fairness'' 를 활성화
	시켜야 할 수 있을 겁니다.
	당신의 forward-progress 체크가 \co{accept} 라벨과 관련되어 있다면,
	\co{-a} 인자 또한 필요할 겁니다.
\iffalse

	Another option to avoid memory exhaustion is the
	\co{-DMEMLIM=N} compiler flag. \co{-DMEMLIM=2000}
	would set the maximum of 2\,GB.

	Don't forget to capture the output, especially
	if you are working on a remote machine.

	If your model includes forward-progress checks, you will likely
	need to enable ``weak fairness'' via the \co{-f} command-line
	argument to \co{./pan}.
	If your forward-progress checks involve \co{accept} labels,
	you will also need the \co{-a} argument.
	% forward reference to model: formal.2009.02.19a in
	% /home/linux/git/userspace-rcu/formal-model.
\fi
\item	[\tco{spin -t -p qrcu.spin}]
	에러를 만나게 된 수행 시에 나온 결과 파일인 \co{trail} 파일을 받아서 그
	에러를 마주하게 된 과정의 시퀀스를 출력합니다.
	\co{-g} 플래그는 또한 변경된 전역 변수들의 값들 또한 포함시킬 것이고,
	\co{-l} 플래그는 변경된 지역 변수들의 값들도 포함시킬 겁니다.
\iffalse

\item	[\tco{spin -t -p qrcu.spin}]
	Given \co{trail} file output by a run that encountered an
	error, output the sequence of steps leading to that error.
	The \co{-g} flag will also include the values of changed
	global variables, and the  \co{-l} flag will also include
	the values of changed local variables.
\fi
\end{description}

\subsubsection{Promela Peculiarities}
\label{sec:formal:Promela Peculiarities}

모든 컴퓨터 언어들이 유사한 점들을 가지고 있음에도 불구하고, C, C++, 또는 Java
를 가지고 코딩하던 사람들에게 Pormela 는 조금 놀라운 것들을 제공할 겁니다.
\iffalse

Although all computer languages have underlying similarities,
Promela will provide some surprises to people used to coding in C,
C++, or Java.
\fi

\begin{enumerate}
\item	C 에서, ``\co{;}'' 는 statement 의 종료를 알립니다.
	Promela 에서는 statement 들을 분리합니다.
	다행히도, Spin 의 더 최신 버전은 ``여분의'' 세미콜론들에 훨씬 더
	너그러워졌습니다.
\item	Promela 에서 루프를 만드는 데 사용되는 \co{do} 문은 조건을 갖습니다.
	이 \co{do} 문은 if-then-else 로 구성된 루프 문과 상당히 닮아 있습니다.
\item	C 의 \co{switch} 문에서, 맞는 케이스가 존재하지 않는다면, 전체
	statement 가 건너뛰어 집니다.
	Promela 의 같은 기능에서, 잘못 사용된 \co{if} 문에서 맞아 떨어지는
	조건이 없다면, 알아들을 수 있는 관련된 에러 메세지 없이 에러를 내게
	됩니다.
	따라서, 에러 출력이 문제 없는 코드 라인을 가리킨다면, \co{if} 나
	\co{do} 문에서 해당되지 않는 조건을 남겨둔 건 아닌지 확인해 보시기
	바랍니다.
\iffalse

\item	In C, ``\co{;}'' terminates statements.  In Promela it separates them.
	Fortunately, more recent versions of Spin have become
	much more forgiving of ``extra'' semicolons.
\item	Promela's looping construct, the \co{do} statement, takes
	conditions.
	This \co{do} statement closely resembles a looping if-then-else
	statement.
\item	In C's \co{switch} statement, if there is no matching case, the whole
	statement is skipped.  In Promela's equivalent, confusingly called
	\co{if}, if there is no matching guard expression, you get an error
	without a recognizable corresponding error message.
	So, if the error output indicates an innocent line of code,
	check to see if you left out a condition from an \co{if} or \co{do}
	statement.
\fi
\item	C 에서 스트레스 테스트를 만들 때, 어떤 사람은 의심되는 오퍼레이션들을
	서로에 대해 반복적으로 경주시키곤 할겁니다.
	Promela 에서, 어떤 사람은 그 대신에 하나의 경주만을 만들텐데, Promela
	는 그 한번의 경주로부터 가능한 모든 결과를 탐색할 것이기 때문입니다.
	어떤 경우에는 Promela 에서 루프를 돌 필요가 있는데, 예를 들어 여러
	오퍼레이션들이 겹치지만, 그렇게 하는게 당신의 상태 공간을 굉장히
	증가시키는 경우가 그런 경우입니다.
\item	C 에서, 하기 가장 쉬운 일은 루프의 진행 정도를 추적하고 종료하기 위해
	루프 카운터를 사용하는 것입니다.
	Promela 에서, 루프 카운터는 역병처럼 방지되어야만 하는데, 그것들은 상태
	공간을 폭발적으로 증가시키기 때문입니다.
	다른 한편, Promela 에서 무한 루프는 변수들 가운데 단조적으로
	증가하거나 감소하는 것들이 없다면 문제가 없습니다---Promela 는 루프에서
	얼마나 수행이 돌아가면 정말로 영향을 끼칠 것인지를 알아챌 것이고,
	자동적으로 그 지점 뒤의 실행을 없애버릴 겁니다.
\iffalse

\item	When creating stress tests in C, one usually races suspect operations
	against each other repeatedly.	In Promela, one instead sets up
	a single race, because Promela will search out all the possible
	outcomes from that single race.	Sometimes you do need to loop
	in Promela, for example, if multiple operations overlap, but
	doing so greatly increases the size of your state space.
\item	In C, the easiest thing to do is to maintain a loop counter to track
	progress and terminate the loop.  In Promela, loop counters
	must be avoided like the plague because they cause the state
	space to explode.  On the other hand, there is no penalty for
	infinite loops in Promela as long as none of the variables
	monotonically increase or decrease---Promela will figure out
	how many passes through the loop really matter, and automatically
	prune execution beyond that point.
\fi
\item	C 로 짜여진 고문 테스트 코드에서는 태스크별 제어 변수를 두는 것이 많은
	경우에 현명합니다.
	그것은 읽기에 편하고, 테스트 코드를 디버깅 하는데에 매우 도움이 됩니다.
	Promela 에서 태스크별 제어 변수는 다른 대안이 없을 때에만 사용되어야
	합니다.
	이를 자세히 보기 위해, 다섯개의 태스크 검증을 해야 하는데 작업 완료를
	나타내는 하나의 비트를 태스크 각각이 갖는다고 생각해 봅시다.
	이는 32개의 상태들을 만들어냅니다.
	반면에, 하나의 간단한 카운터만을 사용한다면 6개의 상태만을 가질
	것이어서, 다섯배가 넘는 상태 갯수의 감소를 이루어냅니다.
	이 다섯배는 문제처럼 보이지 않을 수도 있는데, 검증 프로그램이 1억
	5천만개의 상태들을 10GB 가 넘는 메모리를 소모해 가면서 처리하느라
	고생하고 있지 않을 때에는 그럴 겁니다!
\item	C 고문 테스트 코드와 Promela 둘 다에서 가장 어려운 일들 중 하나는 좋은
	단정문들을 만들어내는 것입니다.
	Promela 는 또한 \co{never} 가 모든 코드 라인들 사이에 복사되어 있는
	단정문과 같은 것들에 대해서 주의를 내도록 하는 것도 가능하게 합니다.
\item	분할하고 지배하기는 Promela 에서 상태 공간을 제어하기에 굉장히 도움이
	됩니다.
	커다란 모델을 두개의 대략적으로 절반씩을 갖는 것들로 분할하는 것은
	각각의 절반이 상태 공간의 루트 값만큼의 양을 갖는 결과를 만들 겁니다.
	예를 들어, 백만개의 상태가 결합된 모델은 두개의 천개 상태 모델들로
	나뉘어질 수 있을 겁니다.
	Promela 가 두개의 더 작은 모델을 더 적은 메모리를 가지고 더 빨리 처리할
	뿐만 아니라, 두개의 작은 알고리즘이 사람이 이해하기에 더 쉽습니다.
\iffalse

\item	In C torture-test code, it is often wise to keep per-task control
	variables.  They are cheap to read, and greatly aid in debugging the
	test code.  In Promela, per-task control variables should be used
	only when there is no other alternative.  To see this, consider
	a 5-task verification with one bit each to indicate completion.
	This gives 32 states.  In contrast, a simple counter would have
	only six states, more than a five-fold reduction.  That factor
	of five might not seem like a problem, at least not until you
	are struggling with a verification program possessing more than
	150 million states consuming more than 10\,GB of memory!
\item	One of the most challenging things both in C torture-test code and
	in Promela is formulating good assertions.  Promela also allows
	\co{never} claims that act sort of like an assertion replicated
	between every line of code.
\item	Dividing and conquering is extremely helpful in Promela in keeping
	the state space under control.  Splitting a large model into two
	roughly equal halves will result in the state space of each
	half being roughly the square root of the whole.
	For example, a million-state combined model might reduce to a
	pair of thousand-state models.
	Not only will Promela handle the two smaller models much more
	quickly with much less memory, but the two smaller algorithms
	are easier for people to understand.
\fi
\end{enumerate}


\subsubsection{Promela Coding Tricks}
\label{sec:formal:Promela Coding Tricks}

Promela 는 프로토콜을 분석하기 위해 설계되었으므로, 병렬 프로그램에 사용하는건
약간 오용에 가깝습니다.
다음의 트릭들은 Promela 를 안전하게 오용하는데 도움을 줄 겁니다:
\iffalse

Promela was designed to analyze protocols, so using it on parallel programs
is a bit abusive.
The following tricks can help you to abuse Promela safely:
\fi

\begin{enumerate}
\item	메모리 재배치.
	전역 변수 x 와 y 를 지역 변수 r1 과 r2 에 복사하는 두개의 statement 가
	있는데, 이것들은 그 순서가 중요한데 (ex: 락으로 보호되지 않음), 메모리
	배리어를 사용하지 않았다고 생각해 봅시다.
	이는 Promela 에서 다음과 같이 모델링될 수 있습니다:
\iffalse

\item	Memory reordering.  Suppose you have a pair of statements
	copying globals x and y to locals r1 and r2, where ordering
	matters (e.g., unprotected by locks), but where you have
	no memory barriers.  This can be modeled in Promela as follows:
\fi

\begin{VerbatimN}[samepage=true]
if
:: 1 -> r1 = x;
        r2 = y
:: 1 -> r2 = y;
        r1 = x
fi
\end{VerbatimN}

	\co{if} 문의 두갈래 경우들은 비결정적으로 선택될 것인데, 둘 다 선택되는
	것이 가능하기 때문입니다.
	전체 상태 공간이 탐색될 것이므로, \emph{둘 다의} 선택들이 결국은 모든
	경우들에 대해 만들어질 것입니다.

	물론, 이 트릭은 너무 과하게 사용된다면 상태 공간을 폭증시켜 버릴 수
	있습니다.
	또한, 이 트릭은 가능한 재배치들을 예측할 것을 필요로 합니다.
	\iffalse

	The two branches of the \co{if} statement will be selected
	nondeterministically, since they both are available.
	Because the full state space is searched, \emph{both} choices
	will eventually be made in all cases.

	Of course, this trick will cause your state space to explode
	if used too heavily.
	In addition, it requires you to anticipate possible reorderings.
	\fi

\item	State reduction.  If you have complex assertions, evaluate
	them under \co{atomic}.  After all, they are not part of the
	algorithm.  One example of a complex assertion (to be discussed
	in more detail later) is as shown in
	Listing~\ref{lst:formal:Complex Promela Assertion}.

	There is no reason to evaluate this assertion
	non-atomically, since it is not actually part of the algorithm.
	Because each statement contributes to state, we can reduce
	the number of useless states by enclosing it in an \co{atomic}
	block as shown in
	Listing~\ref{lst:formal:Atomic Block for Complex Promela Assertion}.

\item	Promela does not provide functions.
	You must instead use C preprocessor macros.
	However, you must use them carefully in order to avoid
	combinatorial explosion.
\end{enumerate}

\begin{listing}[tbp]
\begin{VerbatimL}
i = 0;
sum = 0;
do
:: i < N_QRCU_READERS ->
	sum = sum + (readerstart[i] == 1 &&
	             readerprogress[i] == 1);
	i++
:: i >= N_QRCU_READERS ->
	assert(sum == 0);
	break
od
\end{VerbatimL}
\caption{Complex Promela Assertion}
\label{lst:formal:Complex Promela Assertion}
\end{listing}

\begin{listing}[tbp]
\begin{VerbatimL}
atomic {
	i = 0;
	sum = 0;
	do
	:: i < N_QRCU_READERS ->
		sum = sum + (readerstart[i] == 1 &&
		             readerprogress[i] == 1);
		i++
	:: i >= N_QRCU_READERS ->
		assert(sum == 0);
		break
	od
}
\end{VerbatimL}
\caption{Atomic Block for Complex Promela Assertion}
\label{lst:formal:Atomic Block for Complex Promela Assertion}
\end{listing}

Now we are ready for more complex examples.

\subsection{Promela Example: Locking}
\label{sec:formal:Promela Example: Locking}
\NoIndentAfterThis

\begin{listing}[tbp]
\input{CodeSamples/formal/promela/lock@whole.fcv}
\caption{Promela Code for Spinlock}
\label{lst:formal:Promela Code for Spinlock}
\end{listing}

락은 전반적으로 유용하기 때문에,
Listing~\ref{lst:analysis:Promela Code for Spinlock} 에서 보여진 것처럼 여러
Promela 모델들에 include 될 수 있는 \path{lock.h} 에서 \co{spin_lock()} 과
\co{spin_unlock()}  매크로를 제공합니다.
\co{spin_lock()} 매크로는 line~3 의 단 하나의 조건 ``1'' 덕분에 line~2-11 의
무한한 do-od 루프를 가지고 있습니다.
이 루프의 몸통은 하나의 if-fi 문을 담고 있는 하나의 어토믹 블락입니다.
이 if-fi 문은 루프를 돌기보다는 한번의 패스만을 취한다는 점을 제외하고는 do-od
문과 비슷합니다.
락이 line~5 에서 잡혀있지 않다면 line~6 에서 이를 획득하고 line~7 에서 감싸고
있는 do-od 루프를 깨고 나갑니다 (그리고 어토믹 블락에서도 나갑니다).
한편으로는, 만약 락이 line~8 에서 이미 잡혀 있었다면, 아무 일도 하지 않고
(\co{skip}), if-fi 문과 어토믹 블락을 빠져나오고 그 바깥 루프의 다음 반복을
진행해서 락이 획득 가능해질 때까지 반복하게 됩니다.
\iffalse

\begin{lineref}[ln:formal:promela:lock:whole]
Since locks are generally useful, \co{spin_lock()} and
\co{spin_unlock()}
macros are provided in \path{lock.h}, which may be included from
multiple Promela models, as shown in
Listing~\ref{lst:formal:Promela Code for Spinlock}.
The \co{spin_lock()} macro contains an infinite \co{do-od} loop
spanning lines~\lnref{dood:b}-\lnref{dood:e},
courtesy of the single guard expression of ``1'' on line~\lnref{one}.
The body of this loop is a single atomic block that contains
an \co{if-fi} statement.
The \co{if-fi} construct is similar to the \co{do-od} construct, except
that it takes a single pass rather than looping.
If the lock is not held on line~\lnref{notheld}, then
line~\lnref{acq} acquires it and
line~\lnref{break} breaks out of the enclosing \co{do-od} loop (and also exits
the atomic block).
On the other hand, if the lock is already held on line~\lnref{held},
we do nothing (\co{skip}), and fall out of the \co{if-fi} and the
atomic block so as to take another pass through the outer
loop, repeating until the lock is available.
\fi

\co{spin_unlock()} 매크로는 단순히 이 락을 더이상 잡혀 있지 않았다고
표시합니다.

Promela 는 완전한 순서 규칙을 가정하기 때문에 메모리 배리어들은 필요치 않다는
점을 알아두세요.
어떤 Promela 상태에서도, 모든 프로세스는 현재 상태와 우리가 현재의 상태에
도달하게 되는 과정에서의 상태 변화 순서에 동의하게 됩니다.
이는 (1990년대 MIPS 와 PA-RISC 같은) 일부 컴퓨터 시스템에서 사용되는
``sequentially consistent'' 메모리 모델과 비슷합니다.
앞서 언급되었듯이, 그리고 뒤의 예제에서 알아보게 되듯이, 약한 메모리 순서
규칙은 명시적으로 코딩되어야만 합니다.
\iffalse
\end{lineref}

The \co{spin_unlock()} macro simply marks the lock as no
longer held.

Note that memory barriers are not needed because Promela assumes
full ordering.
In any given Promela state, all processes agree on both the current
state and the order of state changes that caused us to arrive at
the current state.
This is analogous to the ``sequentially consistent'' memory model
used by a few computer systems (such as 1990s MIPS and PA-RISC).
As noted earlier, and as will be seen in a later example,
weak memory ordering must be explicitly coded.
\fi

\begin{listing}[tb]
\input{CodeSamples/formal/promela/lock@spin.fcv}
\caption{Promela Code to Test Spinlocks}
\label{lst:formal:Promela Code to Test Spinlocks}
\end{listing}

이 매크로들은
Listing~\ref{lst:analysis:Promela Code to Test Spinlocks}
에 보여진 Promela 코드에 의해 테스트 되었습니다.
이 코드는 line~3 에서의 \co{N_LOCKERS} 로 정의된 락킹 프로세스의 숫자에 의해 값
증가를 테스트하는데 사용되었던 코드와 비슷합니다.
뮤텍스 자체는 line~5 에 정의되었고, 락 소유자를 추적하기 위한 배열이 line~6 에
정의되어 있으며, line~7 은 하나의 프로세스만이 락을 잡고 있음을 증명하기 위한
단정문 코드에 사용됩니다.

락을 잡는 프로세스는 line~9-18 에 있는데, line~13 에서 락을 획득하고 line~14
에서 락을 잡았음을 공표하고 line~15 에서 락을 잡지 않고 있다고 이야기한 후,
line~16 에서 락을 놓습니다.
\iffalse

\begin{lineref}[ln:formal:promela:lock:spin]
These macros are tested by the Promela code shown in
Listing~\ref{lst:formal:Promela Code to Test Spinlocks}.
This code is similar to that used to test the increments,
with the number of locking processes defined by the \co{N_LOCKERS}
macro definition on line~\lnref{nlockers}.
The mutex itself is defined on line~\lnref{mutex},
an array to track the lock owner
on line~\lnref{array}, and line~\lnref{sum} is used by assertion
code to verify that only one process holds the lock.
\end{lineref}

\begin{lineref}[ln:formal:promela:lock:spin:locker]
The locker process is on lines~\lnref{b}-\lnref{e}, and simply loops forever
acquiring the lock on line~\lnref{lock}, claiming it on line~\lnref{claim},
unclaiming it on line~\lnref{unclaim}, and releasing it on line~\lnref{unlock}.
\end{lineref}
\fi

Line~20-44 의 init 블락은 현재 락 잡는 프로세스의 havelock 배열 원소를 line~26
에서 초기화 하고, 현재 락 잡는 프로세스를 line~27 에서 시작시키며, line~28 에서
다음 락 잡는 프로세스로 넘어갑니다.
일단 모든 락 잡는 프로세스들이 시작되면, do-od 루프의 수행은 단정문을 체크하는
line~29 로 넘어갑니다.
Line~30 과~31 은 제어 변수들을 초기화 하고, line~32-40 은 어토믹하게 havelock
배열 원소들의 합을 구하고, line~41 에서 단정문을 수행하고, line~42 에서 루프를
빠져나옵니다.

우리는
Listing~\ref{lst:analysis:Promela Code for Spinlock}
와~\ref{lst:analysis:Promela Code to Test Spinlocks} 의 두개의 코드 조각들을
\path{lock.h} 와 \path{lock.spin} 에 각각 집어넣는 것으로 이 모델을 수행할 수
있게 되며, 다음의 커맨드들을 사용해 돌릴 수 있습니다:
\iffalse

\begin{lineref}[ln:formal:promela:lock:spin:init]
The init block on lines~\lnref{b}-\lnref{e} initializes the current locker's
havelock array entry on line~\lnref{array}, starts the current locker on
line~\lnref{start}, and advances to the next locker on line~\lnref{next}.
Once all locker processes are spawned, the \co{do-od} loop
moves to line~\lnref{chkassert}, which checks the assertion.
Lines~\lnref{sum} and~\lnref{j} initialize the control variables,
lines~\lnref{atm:b}-\lnref{atm:e} atomically sum the havelock array entries,
line~\lnref{assert} is the assertion, and line~\lnref{break} exits the loop.
\end{lineref}

We can run this model by placing the two code fragments of
Listings~\ref{lst:formal:Promela Code for Spinlock}
and~\ref{lst:formal:Promela Code to Test Spinlocks} into
files named \path{lock.h} and \path{lock.spin}, respectively, and then running
the following commands:
\fi

\begin{VerbatimU}
spin -a lock.spin
cc -DSAFETY -o pan pan.c
./pan
\end{VerbatimU}

\begin{listing}[htbp]
\VerbatimInput[numbers=none,fontsize=\scriptsize]{CodeSamples/formal/promela/lock.spin.lst}
\vspace*{-9pt}
\caption{Output for Spinlock Test}
\label{lst:formal:Output for Spinlock Test}
\end{listing}

출력되는 결과는
Listing~\ref{lst:analysis:Output for Spinlock Test} 에 보여진 것과 비슷할
것입니다.
예상되었듯이, 이 수행은 단정문 실패가 없습니다 (``errors: 0'').
\iffalse

The output will look something like that shown in
Listing~\ref{lst:formal:Output for Spinlock Test}.
As expected, this run has no assertion failures (``errors: 0'').
\fi

\QuickQuiz{}
	왜 locker 에 미치지 못한 statement 가 있는 거죠?
	이건 \emph{전체} 상태-공간 탐색이 아니었나요?
	\iffalse

	Why is there an unreached statement in
	locker?  After all, isn't this a \emph{full} state-space
	search?
	\fi
\QuickQuizAnswer{
	locker 프로세스는 무한 루프이므로, 이 프로세스의 종료까지 제어가 닿지를
	않습니다.
	하지만, 단조적으로 증가되는 변수가 존재하지 않기 때문에, Promela 는 이
	무한 루프를 작은 수의 상태만 가지고도 모델링 할수가 있습니다.
	\iffalse

	The locker process is an infinite loop, so control
	never reaches the end of this process.
	However, since there are no monotonically increasing variables,
	Promela is able to model this infinite loop with a small
	number of states.
	\fi
} \QuickQuizEnd

\QuickQuiz{}
	이 예제에 있어서 Promela 코딩 스타일 문제들은 뭐가 있나요?
	\iffalse

	What are some Promela code-style issues with this example?
	\fi
\QuickQuizAnswer{
	몇가지가 있습니다:
	\iffalse

	There are several:
	\fi
	\begin{enumerate}
	\item	{\co sum} 의 선언은 init 블록 안으로 옮겨져야 하는데, 그 외의
		곳에서는 어디서도 사용되지 않기 때문입니다.
	\item	단정문 코드는 초기화 루프 바깥으로 옮겨져야 합니다.
		그렇게 되면 초기화 루프는 하나의 어토믹 블락 안에 위치할 수
		있어서, 상태 공간을 훨씬 줄일 수 있습니다 (얼마나 줄일 수
		있을까요?).
	\item	단정문 코드를 감싸고 있는 어토믹 블록은 {\tt sum} 과 {\tt j} 의
		초기화를, 그리고 단정문도 포함하도록 확장되어야 합니다.
		이것 역시 상태 공간을 줄일 것입니다 (이번에도, 얼마나 줄일 수
		있을까요?).
	\iffalse

	\item	The declaration of \co{sum} should be moved to within
		the init block, since it is not used anywhere else.
	\item	The assertion code should be moved outside of the
		initialization loop.  The initialization loop can
		then be placed in an atomic block, greatly reducing
		the state space (by how much?).
	\item	The atomic block covering the assertion code should
		be extended to include the initialization of \co{sum}
		and \co{j}, and also to cover the assertion.
		This also reduces the state space (again, by how
		much?).
	\fi
	\end{enumerate}
} \QuickQuizEnd


\subsection{Promela Example: QRCU}
\label{sec:formal:Promela Example: QRCU}

이 마지막 예제는 \co{synchronize_qrcu()} 의 빠른 수행경로를 더 빠르게 하기 위해
수정된 Oleg Nesterov 의 QRCU~\cite{OlegNesterov2006QRCU,OlegNesterov2006aQRCU}
를 위한 실제 세계에서의 Promela 사용을  보입니다.

하지만 먼저, QRCU 란 무엇일까요?
\iffalse

This final example demonstrates a real-world use of Promela on Oleg
Nesterov's
QRCU~\cite{OlegNesterov2006QRCU,OlegNesterov2006aQRCU},
but modified to speed up the \co{synchronize_qrcu()}
fastpath.

But first, what is QRCU?
\fi

QRCU 는 극단적으로 낮은 grace period 대기시간이라는 장점을 더 높은 읽기
오버헤드 (전역 변수에의 어토믹한 값 증가와 감소) 와 맞바꾸는,
SRCU~\cite{PaulEMcKenney2006c} 의 한 변종입니다.
읽기 쓰레드가 없다면, grace period 는 1 마이크로세컨드도 되지 않는 시간에
파악되는데, 이는 대부분의 다른 RCU 구현들의 수 밀리세컨드 grace period
대기시간과 비교됩니다.
\iffalse

QRCU is a variant of SRCU~\cite{PaulEMcKenney2006c}
that trades somewhat higher read overhead
(atomic increment and decrement on a global variable) for extremely
low grace-period latencies.
If there are no readers, the grace period will be detected in less
than a microsecond, compared to the multi-millisecond grace-period
latencies of most other RCU implementations.
\fi

\begin{enumerate}
\item	QRCU 도메인을 정의하는 \co{qrcu_struct} 가 존재합니다.
	SRCU 처럼 (그리고 다른 RCU 변종들과는 달리) QRCU 의 동작은 글로벌하지
	않고, 그대신에 특정한 \co{qrcu_struct} 에 집중됩니다.
\item	QRCU read-side 크리티컬 섹션들을 구분짓는 \co{qrcu_read_lock()} 과
	\co{qrcu_read_unlock()} 이 있습니다.
	연관되는 \co{qrcu_struct} 는 이 함수들에 넘겨져야 하고,
	\co{rcu_read_lock()} 으로부터의 리턴값은 \co{rcu_read_unlock()} 으로
	넘겨져야만 합니다.

	예를 들면 다음과 같습니다:
\iffalse

\item	There is a \co{qrcu_struct} that defines a QRCU domain.
	Like SRCU (and unlike other variants of RCU) QRCU's action
	is not global, but instead focused on the specified
	\co{qrcu_struct}.
\item	There are \co{qrcu_read_lock()} and \co{qrcu_read_unlock()}
	primitives that delimit QRCU read-side critical sections.
	The corresponding \co{qrcu_struct} must be passed into
	these primitives, and the return value from \co{qrcu_read_lock()}
	must be passed to \co{qrcu_read_unlock()}.

	For example:
\fi

\begin{VerbatimU}
idx = qrcu_read_lock(&my_qrcu_struct);
/* read-side critical section. */
qrcu_read_unlock(&my_qrcu_struct, idx);
\end{VerbatimU}

\item	이전부터 존재해온 QRCU read-side 크리티컬 섹션들이 모두 완료될 때까지
	기다리는 \co{synchronize_qrcu()} 기능이 있습니다만, SRCU 의
	\co{synchronize_srcu()} 처럼, QRCU 의 \co{synchronize_qrcu()} 는 같은
	\co{qrcu_struct} 를 사용하는 read-side 크리티컬 섹션들만을 기다리면
	됩니다.

	앞의 예에 이어 예를 들면, \co{synchronize_qruc(&your_qrcu_struct)} 는
	이전의 QRCU read-side 크리티컬 섹션을 기다릴 필요가 \emph{없습니다}.
	반면에, \co{synchronize_qrcu(&my_qrcu_struct)} 는 기다려야 \emph{할 수}
	있는데, 같은 \co{qrcu_struct} 를 공유하기 때문입니다.
\iffalse

\item	There is a \co{synchronize_qrcu()} primitive that blocks until
	all pre-existing QRCU read-side critical sections complete,
	but, like SRCU's \co{synchronize_srcu()}, QRCU's
	\co{synchronize_qrcu()} need wait only for those read-side
	critical sections that are using the same \co{qrcu_struct}.

	For example, \co{synchronize_qrcu(&your_qrcu_struct)}
	would \emph{not} need to wait on the earlier QRCU read-side
	critical section.
	In contrast, \co{synchronize_qrcu(&my_qrcu_struct)}
	\emph{would} need to wait, since it shares the same
	\co{qrcu_struct}.
\fi
\end{enumerate}

QRCU 를 위한 리눅스 커널 패치도
만들어졌습니다만~\cite{PaulMcKenney2007QRCUpatch}, 리눅스 커널에 머지될 것
같지는 않습니다.
\iffalse

A Linux-kernel patch for QRCU has been
produced~\cite{PaulMcKenney2007QRCUpatch},
but is unlikely to ever be included in the Linux kernel.
\fi

\begin{listing}[htbp]
\input{CodeSamples/formal/promela/qrcu@gvar.fcv}
\caption{QRCU Global Variables}
\label{lst:formal:QRCU Global Variables}
\end{listing}

QRCU 를 위한 Promela 코드로 돌아와서, 전역 변수들은
Listing~\ref{lst:analysis:QRCU Global Variables} 에 보인 것과 같습니다.
이 예제는 락킹을 사용하므로, \path{lock.h} 를 include 하고 있습니다.
읽기 쓰레드의 갯수와 쓰기 쓰레드의 갯수는 두개의 \co{#define} 문을 통해
변경될 수 있어서, 두개의 조합 증폭 가능한 방법을 제공합니다.
\co{idx} 변수는 \co{ctr} 배열의 두 원소들 중 무엇이 읽기 쓰레드에 의해 사용될
것인지 결정하며, \co{readerprogress} 변수는 단정문이 언제 모든 읽기 쓰레드가
종료되었는지를 판단할 수 있게 합니다 (QRCU 업데이트는 모든 앞서 존재한 읽기
쓰레드가 그들의 QRCU read-side 크리티컬 섹션을 완료하기 전까지는 완료될 수
없기 때문입니다).
\co{readerprogress} 배열의 원소는 다음과 같이 값을 가져서 연관된 읽기
쓰레드의 상태를 알립니다:
\iffalse

Returning to the Promela code for QRCU, the global variables are as shown in
Listing~\ref{lst:formal:QRCU Global Variables}.
This example uses locking, hence including \path{lock.h}.
Both the number of readers and writers can be varied using the
two \co{#define} statements, giving us not one but two ways to create
combinatorial explosion.
The \co{idx} variable controls which of the two elements of the \co{ctr}
array will be used by readers, and the \co{readerprogress} variable
allows an assertion to determine when all the readers are finished
(since a QRCU update cannot be permitted to complete until all
pre-existing readers have completed their QRCU read-side critical
sections).
The \co{readerprogress} array elements have values as follows,
indicating the state of the corresponding reader:
\fi

\begin{enumerate}[label={\arabic*}:,start=0,itemsep=0pt]
\item	시작되지 않았음.
\item	QRCU read-side 크리티컬 섹션 안에 있음.
\item	QRCU read-side 크리티컬 섹션을 끝냈음.
\iffalse

\item	not yet started.
\item	within QRCU read-side critical section.
\item	finished with QRCU read-side critical section.
\fi
\end{enumerate}

마지막으로, \co{mutex} 변수는 업데이트 쓰레드들의 느린 수행경로들을 직렬화
하는데 사용됩니다.
\iffalse

Finally, the \co{mutex} variable is used to serialize updaters' slowpaths.
\fi

\begin{listing}[htbp]
\input{CodeSamples/formal/promela/qrcu@reader.fcv}
\caption{QRCU Reader Process}
\label{lst:formal:QRCU Reader Process}
\end{listing}

QRCU 읽기 쓰레드는
Listing~\ref{lst:analysis:QRCU Reader Process} 에 보여진 \co{qrcu_reader()}
프로세스에 의해 모델링 됩니다.
do-od 루프가 line~5-16 에 있는데, 하나의 ``1'' 조건을 line~6 에 가지고 있어서
이를 무한 루프로 만들어 줍니다.
Line~7 은 글로벌 인덱스의 현재 값을 가져오고, line~8-15 는 그 값이 0이
아니었다면 이를 원자적으로 증가 (\co{atomic_inc_not_zero()}) 시킵니다 (그리고
루프를 나옵니다).
Line~17 은 RCU read-side 크리티컬 섹션으로의 진입을 표시하고, line~18 은 이
크리티컬 섹션으로부터 나오는 것을 표시하는데, 두 라인 모두 우리가 뒤에서
마주하게 될 {\tt assert()} 문을 위해서입니다.
Line~19 는 우리가 증가시켰던 같은 카운터를 어토믹하게 감소시키는데, 이렇게
함으로써 RCU read-side 크리티컬 섹션을 빠져나오게 됩니다.
\iffalse

\begin{lineref}[ln:formal:promela:qrcu:reader]
QRCU readers are modeled by the \co{qrcu_reader()} process shown in
Listing~\ref{lst:formal:QRCU Reader Process}.
A \co{do-od} loop spans lines~\lnref{do}-\lnref{od},
with a single guard of ``1''
on line~\lnref{one} that makes it an infinite loop.
Line~\lnref{curidx} captures the current value of the global index,
and lines~\lnref{atm:b}-\lnref{atm:e}
atomically increment it (and break from the infinite loop)
if its value was non-zero (\co{atomic_inc_not_zero()}).
Line~\lnref{cs:entry} marks entry into the RCU read-side critical section, and
line~\lnref{cs:exit} marks exit from this critical section,
both lines for the benefit of
the \co{assert()} statement that we shall encounter later.
Line~\lnref{atm:dec} atomically decrements the same counter that we incremented,
thereby exiting the RCU read-side critical section.
\end{lineref}
\fi

\begin{listing}[htbp]
\input{CodeSamples/formal/promela/qrcu@sum_unordered.fcv}
\caption{QRCU Unordered Summation}
\label{lst:formal:QRCU Unordered Summation}
\end{listing}

Listing~\ref{lst:analysis:QRCU Unordered Summation}
에 보인 C 전처리기 매크로는 약한 메모리 순서 규칙을 에뮬레이션 하기 위해 두개의
카운터의 합을 구합니다.
Line~2-13 은 카운터 중 하나를 가져오고, line~14 는 나머지 하나를 가져와서
그것들을 더합니다.
어토믹 블락은 하나의 do-od 문으로 구성되어 있습니다.
이 do-od 문 (line~3-12) 은 line~4 와~8 에 있는 두개의 무조건적인 브랜치들을
가지고 있다는 점에서 일반적이지 않은데, 이는 Promela 가 비결정적으로 두개의
브랜치 중 하나를 선택하게 합니다 (하지만 다시 말하지만, 전체 상태 공간 탐색은
Promela 가 결국은 모든 가능한 상황의 선택지를 만들게 할 겁니다).
첫번째 브랜치는 0번째 카운터를 가져오고 \co{i} 를 1 로 설정하고 (line~14 가
첫번째 카운터를 가져오도록), 두번째 브랜치는 그 반대 일을 해서, 첫번째 카운터를
가져오고 \co{i} 를 0으로 설정합니다 (line~14 에서 두번째 카운터를 가져오도록).
\iffalse

\begin{lineref}[ln:formal:promela:qrcu:sum_unordered]
The C-preprocessor macro shown in
Listing~\ref{lst:formal:QRCU Unordered Summation}
sums the pair of counters so as to emulate weak memory ordering.
Lines~\lnref{fetch:b}-\lnref{fetch:e} fetch one of the counters,
and line~\lnref{sum_other} fetches the other
of the pair and sums them.
The atomic block consists of a single \co{do-od} statement.
This \co{do-od} statement (spanning lines~\lnref{do}-\lnref{od}) is unusual in that
it contains two unconditional
branches with guards on lines~\lnref{g1} and~\lnref{g2}, which causes Promela to
non-deterministically choose one of the two (but again, the full
state-space search causes Promela to eventually make all possible
choices in each applicable situation).
The first branch fetches the zero-th counter and sets \co{i} to 1 (so
that line~\lnref{sum_other} will fetch the first counter), while the second
branch does the opposite, fetching the first counter and setting \co{i}
to 0 (so that line~\lnref{sum_other} will fetch the second counter).
\end{lineref}
\fi

\QuickQuiz{}
	이 \co{do-od} 문을 좀 더 간단하게 코딩하는 방법은 없을까요?
	\iffalse

	Is there a more straightforward way to code the \co{do-od} statement?
	\fi
\QuickQuizAnswer{
	있습니다.
	이걸 \co{if-fi} 로 바꾸고 \co{break} 문을 없애버리세요.
	\iffalse

	Yes.
	Replace it with \co{if-fi} and remove the two \co{break} statements.
	\fi
} \QuickQuizEnd

\begin{listing}[htbp]
\input{CodeSamples/formal/promela/qrcu@updater.fcv}
\caption{QRCU Updater Process}
\label{lst:formal:QRCU Updater Process}
\end{listing}

\co{sum_unordered} 매크로와 함께, 우리는 이제
Listing~\ref{lst:analysis:QRCU Updater Process} 에 보여진 update-side
프로세스로 넘어갈 수가 있게 되었습니다.
이 update-side 프로세스는 line~7-57 의 관련된 do-od 루프를 애매하게 반복하게
됩니다.
이 루프를 통하는 각각의 패스는 먼저 line~12-21 에서 글로벌한 {\tt
readerprogress} 배열을 로컬의 {\tt readerstart} 어레이로 스냅샷 뜹니다.
이 스냅샷은 line~53 에서의 단정문에 사용될 겁니다.
Line~23 은 \co{sum_unordered} 를 호출하고, 빠른 수행경로가 잠재적으로 사용
가능하다면 line~24-27 에서 \co{sum_unordered} 를 다시 호출합니다.
\iffalse

\begin{lineref}[ln:formal:promela:qrcu:updater]
With the \co{sum_unordered} macro in place, we can now proceed
to the update-side process shown in
Listing~\ref{lst:formal:QRCU Updater Process}.
The update-side process repeats indefinitely, with the corresponding
\co{do-od} loop ranging over lines~\lnref{do}-\lnref{od}.
Each pass through the loop first snapshots the global \co{readerprogress}
array into the local \co{readerstart} array on
lines~\lnref{atm1:b}-\lnref{atm1:e}.
This snapshot will be used for the assertion on line~\lnref{assert}.
Line~\lnref{sum_unord} invokes \co{sum_unordered}, and then
lines~\lnref{reinvoke:b}-\lnref{reinvoke:e}
re-invoke \co{sum_unordered} if the fastpath is potentially
usable.
\fi

Line~28-40 은 필요하다면 느린 수행 경로의 코드를 수행하는데, line~30 과~38 에서
update-side 락을 각각 획득하고 해제하며, line~31-33 에서 인덱스를 뒤집고,
line~34-37 에서 모든 이전부터 존재한 읽기 쓰레드들이 완료되길 기다립니다.

Line~44-56 에서는 이제 {\tt readerprogress} 배열의 현재 값들과 {\tt
readerstart} 배열 안의 앞서 수집된 값들과 비교하며, 이 업데이트 전에 시작된
읽기 쓰레드가 여전히 진행중이라면 단정문이 실패하도록 합니다.
\iffalse

Lines~\lnref{slow:b}-\lnref{slow:e} execute the slowpath code if need be, with
lines~\lnref{acq} and~\lnref{rel} acquiring and releasing the update-side lock,
lines~\lnref{flip_idx:b}-\lnref{flip_idx:e} flipping the index, and
lines~\lnref{wait:b}-\lnref{wait:e} waiting for
all pre-existing readers to complete.

Lines~\lnref{atm2:b}-\lnref{atm2:e} then compare the current values
in the \co{readerprogress}
array to those collected in the \co{readerstart} array,
forcing an assertion failure should any readers that started before
this update still be in progress.
\end{lineref}
\fi

\QuickQuiz{}
	왜 line~12-21 과 line~44-56 에 어토믹 블락이 있나요, 그 어토믹 블락들
	안의 오퍼레이션들은 현존하는 제품화된 마이크로프로세서 중 어떤 것도
	어토믹한 구현을 제공하지 않는데도 말이죠?
	\iffalse

	\begin{lineref}[ln:formal:promela:qrcu:updater]
	Why are there atomic blocks at lines~\lnref{atm1:b}-\lnref{atm1:e}
	and lines~\lnref{atm2:b}-\lnref{atm2:e}, when the operations
	within those atomic
	blocks have no atomic implementation on any current
	production microprocessor?
	\end{lineref}
	\fi
\QuickQuizAnswer{
	그 오퍼레이션들은 단정문을 위한 것들일 뿐이기 때문입니다.
	그것들은 알고리즘 자체를 위한 것이 아닙니다.
	따라서 그것들을 어토믹으로 표시해도 문제가 되지 않고, 그것들을
	어토믹으로 표시하는 것은 Promela 모델을 통해 탐색되어야 하는 상태
	공간을 굉장히 줄여주게 됩니다.
	\iffalse

	Because those operations are for the benefit of the
	assertion only.  They are not part of the algorithm itself.
	There is therefore no harm in marking them atomic, and
	so marking them greatly reduces the state space that must
	be searched by the Promela model.
	\fi
} \QuickQuizEnd

\QuickQuiz{}
	Line~24-27 에서 카운터들을 다시 합해보는 것은 \emph{정말로} 필요한
	건가요?
	\iffalse

	\begin{lineref}[ln:formal:promela:qrcu:updater]
	Is the re-summing of the counters on
        lines~\lnref{reinvoke:b}-\lnref{reinvoke:e}
	\emph{really} necessary?
        \end{lineref}
	\fi
\QuickQuizAnswer{
	그렇습니다. 이걸 확실히 보기 위해서, 이 라인들을 지우고 모델을
	돌려보세요.

	대안적으로는, 다음과 같은 단계들을 생각해 보세요:
	\iffalse

	Yes.  To see this, delete these lines and run the model.

	Alternatively, consider the following sequence of steps:
	\fi

	\begin{enumerate}
	\item	한 프로세스가 RCU read-side 크리티컬 섹션 안에 있어서, {\tt
		ctr[0]} 의 값은 0이고 {\tt ctr[1]} 은 2 입니다.
	\item	한 업데이트 쓰레드가 수행을 시작하고, 카운터들의 합이 2임을
		보게 되고 따라서 빠른 수행경로는 실행될 수 없습니다.
		따라서 락을 잡습니다.
	\item	두번째 업데이트 쓰레드가 수행을 시작하고, {\tt ctr[0]} 을 값,
		0을 가져옵니다.
	\item	첫번째 업데이트 쓰레드가 {\tt ctr[0]} 에 1을 더하고, 인덱스를
		바꾸고 (이제 0이 됩니다), {\tt ctr[1]} 에서 1을 뺍니다 (이제
		1이 됩니다).
	\item	두번째 업데이트 쓰레드가 {\tt ctr[1]} 의 값을 가져오는데,
		지금은 1입니다.
	\item	두번째 업데이트 쓰레드는 이제 원래의 읽기 쓰레드가 아직
		완료되지 않았음에도 빠른 수행 경로를 진행해도 된다고 잘못된
		결론을 내리게 됩니다.
	\iffalse

	\item	One process is within its RCU read-side critical
		section, so that the value of \co{ctr[0]} is zero and
		the value of \co{ctr[1]} is two.
	\item	An updater starts executing, and sees that the sum of
		the counters is two so that the fastpath cannot be
		executed.  It therefore acquires the lock.
	\item	A second updater starts executing, and fetches the value
		of \co{ctr[0]}, which is zero.
	\item	The first updater adds one to \co{ctr[0]}, flips
		the index (which now becomes zero), then subtracts
		one from \co{ctr[1]} (which now becomes one).
	\item	The second updater fetches the value of \co{ctr[1]},
		which is now one.
	\item	The second updater now incorrectly concludes that it
		is safe to proceed on the fastpath, despite the fact
		that the original reader has not yet completed.
	\fi
	\end{enumerate}
} \QuickQuizEnd

\begin{listing}[htbp]
\input{CodeSamples/formal/promela/qrcu@init.fcv}
\caption{QRCU Initialization Process}
\label{lst:formal:QRCU Initialization Process}
\end{listing}

이제 남은건
Listing~\ref{lst:analysis:QRCU Initialization Process} 의 초기화 블락 뿐입니다.
이 블락은 단순히 카운터 쌍을 line~5-6 에서 초기화 하고, line~7-14 에서 읽기
프로세스들을 시작시킨 후, line~15-21 에서 업데이트 프로세스들을 시작시킵니다.
이는 상태 공간의 크기를 줄이기 위해 모두 하나의 어토믹 블락 안에서 수행됩니다.
\iffalse

\begin{lineref}[ln:formal:promela:qrcu:init]
All that remains is the initialization block shown in
Listing~\ref{lst:formal:QRCU Initialization Process}.
This block simply initializes the counter pair on
lines~\lnref{i_ctr:b}-\lnref{i_ctr:e},
spawns the reader processes on
lines~\lnref{spn_r:b}-\lnref{spn_r:e}, and spawns the updater
processes on lines~\lnref{spn_u:b}-\lnref{spn_u:e}.
This is all done within an atomic block to reduce state space.
\end{lineref}
\fi

\subsubsection{Running the QRCU Example}
\label{sec:formal:Running the QRCU Example}

이 QRCU 예제를 수행하기 위해서는, 앞 섹션에서의 코드들을 \path{qrcu.spin}
이라는 이름의 하나의 파일로 합치고, \co{spin_lock()} 과 \co{spin_unlock()} 의
정의를 \path{lock.h} 라는 이름의 파일로 옮겨야 합니다.
그리고 나서 다음의 커맨드를 사용해서 QRCU 모델을 빌드하고 수행시킬 수 있습니다:
\iffalse

To run the QRCU example, combine the code fragments in the previous
section into a single file named \path{qrcu.spin}, and place the definitions
for \co{spin_lock()} and \co{spin_unlock()} into a file named
\path{lock.h}.
Then use the following commands to build and run the QRCU model:
\fi

\begin{VerbatimU}
spin -a qrcu.spin
cc -DSAFETY [-DCOLLAPSE] -o pan pan.c
./pan [-mN]
\end{VerbatimU}

\begin{table}
\centering
\begin{threeparttable}
\rowcolors{1}{}{lightgray}
\renewcommand*{\arraystretch}{1.2}
\footnotesize
\begin{tabular}{S[table-format = 1.0]S[table-format = 1.0]S[table-format = 9.0]
		S[table-format = 6.0]S[table-format = 5.1]}
	\toprule
	\multicolumn{1}{r}{updaters} &
	    \multicolumn{1}{r}{readers} &
		\multicolumn{1}{r}{\# states} &
		    \multicolumn{1}{r}{depth} &
			\multicolumn{1}{r}{memory (MB)\tnote{a}} \\
	\midrule
	1 & 1 &         376 &      95 &    128.7 \\
	1 & 2 &       6 177 &     218 &    128.9 \\
	1 & 3 &      99 728 &     385 &    132.6 \\
	2 & 1 &      29 399 &     859 &    129.8 \\
	2 & 2 &   1 071 181 &   2 352 &    169.6 \\
	2 & 3 &  33 866 736 &  12 857 &  1 540.8 \\
	3 & 1 &   2 749 453 &  53 809 &    236.6 \\
	3 & 2 & 186 202 860 & 328 014 & 10 483.7 \\
	\bottomrule
\end{tabular}
\begin{tablenotes}
	\item [a] Obtained with the compiler flag \co{-DCOLLAPSE}
		specified.
\end{tablenotes}
\end{threeparttable}
\caption{Memory Usage of QRCU Model}
\label{tab:advsync:Memory Usage of QRCU Model}
\end{table}

수행 결과 나오는 출력은 이 모델이
Table~\ref{tab:advsync:Memory Usage of QRCU Model} 에 나온 모든 케이스들을
통과함을 보여줍니다.
이제, 이 케이스를 세개의 읽기 쓰레드와 세개의 업데이트 쓰레드를 사용해 돌려보는
것도 좋을겁니다만, 간단한 추정이 이는 최선의 경우에도 0.5 테라바이트의 메모리를
필요로 할 것이란 점을 이야기 합니다.
그럼, 뭘 해야 할까요?
\iffalse

The output shows that this model passes all of the cases shown in
Table~\ref{tab:advsync:Memory Usage of QRCU Model}.
It would be nice to run three readers and three
updaters, however, simple extrapolation indicates that this will
require about half a terabyte of memory.
What to do?
\fi

\co{./pan} 이 메모리가 동날 때 조언을 준다는 게 드러났는데, 예를 들어, 세개의
읽기 쓰레드와 세개의 업데이트 쓰레드를 수행하려 하면:
\iffalse

It turns out that \co{./pan} gives advice when it runs out of memory,
for example, when attempting to run three readers and three updaters:
\fi

\begin{VerbatimU}
hint: to reduce memory, recompile with
  -DCOLLAPSE # good, fast compression, or
  -DMA=96   # better/slower compression, or
  -DHC # hash-compaction, approximation
  -DBITSTATE # supertrace, approximation
\end{VerbatimU}

제안된 컴파일러 플래그 \co{-DMA=N} 을 시도해 봅시다, 이 플래그는 상당히 증가된
탐색 오버헤드를 비용으로 해서 상태 공간을 적극적으로 압축시킵니다.
필요한 커맨드는 아래와 같습니다:
\iffalse

Let's try the suggested compiler flag \co{-DMA=N},
which generates code for aggressive compression of the
state space at the cost of greatly increased search overhead.
The required commands are as follows:
\fi

\begin{VerbatimU}
spin -a qrcu.spin
cc -DSAFETY -DMA=96 -O2 -o pan pan.c
./pan -m20000000
\end{VerbatimU}

여기서, 20,000,000 의 depth limit 은 간단한 외삽으로 나온 예상 depth 보다
몇십배 가량 큽니다.
비록 이게 눈에 띄는 메모리 사용량을 증가시키지만, 너무 타이트한 depth limit 로
인해 발생하는 불완전한 탐색으로 인한 긴 수행의 낭비를 막습니다.
이 수행은 \Power{9} 서버에서 3~일보다 조금 더 걸렸습니다.
그 결과가
Listing~\ref{lst:formal:spinhint:3 Readers 3 Updaters QRCU Spin Output with -DMA=96}
에 보여져 있습니다.
이 Spin 수행은 고작 6.5\,GB 의 총 메모리 사용량과 함께 성공적으로 마무리
되었는데, 이는 약 0.5 테라바이트인 \co{-DCOLLAPSE} 사용의 경우에 비해 수백배
가까이 낮은 크기입니다.
\iffalse

Here, the depth limit of 20,000,000 is an order of magnitude
larger than the expected depth deduced from simple extrapolation.
Although this increases up-front memory usage, it avoids wasting
a long run due to incomplete search resulting from a too-tight
depth limit.
This run took a little more than 3~days on a \Power{9} server.
The result is shown in
Listing~\ref{lst:formal:spinhint:3 Readers 3 Updaters QRCU Spin Output with -DMA=96}.
This Spin run completed successfully with a total memory
usage of only 6.5\,GB, which is almost two orders of magnitude
lower than the \co{-DCOLLAPSE} usage of about half a terabyte.
\fi

\begin{listing}
\VerbatimInput[numbers=none,fontsize=\scriptsize]{CodeSamples/formal/promela/qrcu.spin.33ma.lst}
\vspace*{-9pt}
\caption{3 Readers 3 Updaters QRCU Spin Output with \co{-DMA=96}}
\label{lst:formal:spinhint:3 Readers 3 Updaters QRCU Spin Output with -DMA=96}
\end{listing}

\QuickQuiz{}
	압축률 0.48\,\% 는 상태들로 사용되는 메모리를 200-대-1 로 감소시켰군요!
	이 상태공간 탐색은 \emph{정말로} 소모적입니까?
	\iffalse

	A compression rate of 0.48\,\% corresponds to a 200-to-1 decrease
	in memory occupied by the states!
	Is the state-space search \emph{really} exhaustive???
	\fi
\QuickQuizAnswer{
	Spin 의 문서에 따르면, 네, 그렇습니다.
	\iffalse

	According to Spin's documentation, yes, it is.
	\fi

\begin{listing}
\VerbatimInput[numbers=none,fontsize=\scriptsize]{CodeSamples/formal/promela/qrcu.spin.col-ma.diff.lst}
\vspace*{-9pt}
\caption{Spin Output Diff of \co{-DCOLLAPSE} and \co{-DMA=88}}
\label{lst:formal:promela:Spin Output Diff of -DCOLLAPSE and -DMA=88}
\end{listing}

	간접적 증거로, \co{-DCOLLAPSE} 와 \co{-DMA=88} 의 결과를 비교해 봅시다
	(두 읽기 쓰레드와 세 업데이트 쓰레드).
	이 수행들에서의 출력의 차이점은
	Listing~\ref{lst:formal:promela:Spin Output Diff of -DCOLLAPSE and -DMA=88}
	에 보여져 있습니다.
	볼 수 있듯이, 상태의 수는 (stored 와 matched) 둘 다 같습니다.
	\iffalse

	As an indirect evidence, let's compare the results of
	runs with \co{-DCOLLAPSE} and with \co{-DMA=88}
	(two readers and three updaters).
	The diff of outputs from those runs is shown in
	Listing~\ref{lst:formal:promela:Spin Output Diff of -DCOLLAPSE and -DMA=88}.
	As you can see, they agree on the numbers of states
	(stored and matched).
	\fi
} \QuickQuizEnd

\begin{table*}[tbp]
\rowcolors{6}{}{lightgray}
\renewcommand*{\arraystretch}{1.2}
\footnotesize
\centering
\OneColumnHSpace{-0.7in}%
\begin{tabular}{S[table-format = 1.0]S[table-format = 1.0]S[table-format = 9.0]
		S[table-format = 9.0]S[table-format = 2.0]S[table-format = 5.2]
		S[table-format = 4.2]S[table-format = 2.0]S[table-format = 4.2]
		S[table-format = 6.2]}
	\toprule
	\multicolumn{4}{r}{} & \multicolumn{3}{c}{\tco{-DCOLLAPSE}} &
					\multicolumn{3}{c}{\tco{-DMA=N}} \\
	\cmidrule(l){5-7} \cmidrule(l){8-10}
	\multicolumn{1}{r}{updaters} &
	    \multicolumn{1}{r}{readers} &
		\multicolumn{1}{r}{\# states} &
		    \multicolumn{1}{r}{depth reached} &
			\multicolumn{1}{r}{\tco{-wN}} &
			    \multicolumn{1}{r}{memory (MB)} &
				\multicolumn{1}{r}{runtime (s)} &
				    \multicolumn{1}{r}{\tco{N}} &
					\multicolumn{1}{r}{memory (MB)} &
					    \multicolumn{1}{r}{runtime (s)} \\
	\cmidrule{1-4} \cmidrule(l){5-7} \cmidrule(l){8-10}
	1 & 1 &           376 &         95 & 12 &     0.10 & 0.00 &
		40 &    0.29 &      0.00 \\
	1 & 2 &         6 177 &        218 & 12 &     0.39 & 0.01 &
		47 &    0.59 &      0.02 \\
	1 & 3 &        99 728 &        385 & 16 &     4.60 & 0.14 &
		54 &    3.04 &      0.45 \\
        2 & 1 &        29 399 &        859 & 16 &     2.30 & 0.03 &
		55 &    0.70 &      0.13 \\
        2 & 2 &     1 071 181 &      2 352 & 20 &    49.24 & 1.45 &
		62 &    7.77 &      5.76 \\
        2 & 3 &    33 866 736 &     12 857 & 24 & 1 540.70 & 62.5 &
		69 &  111.66 &    326    \\
        3 & 1 &     2 749 453 &     53 809 & 21 &   125.25 & 4.01 &
		70 &   11.41 &     19.5  \\
        3 & 2 &   186 202 860 &    328 014 & 28 & 10 482.51 & 390 &
		77 &  222.26 &   2560    \\
	3 & 3 & 9 664 707 100 &  2 055 621 &    &          &      &
		84 & 5557.02 & 266000    \\
	\bottomrule
\end{tabular}
\caption{QRCU Spin Result Summary}
\label{tab:formal:promela:QRCU Spin Result Summary}
\end{table*}

참고로, Table~\ref{tab:formal:promela:QRCU Spin Result Summary} 은
\co{-DCOLLAPSE} 와 \co{-DMA=N} 컴파일러 플래그와 함께 돌린 Spin 결과를 요약해
보입니다.
메모리 사용량은 최소의 충분한 탐색 깊이와 테이블에 보인 \co{-DMA=N} 패러미터와
함께 얻어졌습니다.
\co{-DCOLLAPSE} 수행을 위한 해시테이블 크기는 \co{./pan} 의 \co{-wN} 옵션에
의해 너무 많은 메모리를 작은 상태 공간을 해싱하는데 사용하는 것을 막기 위해
조절되었습니다.
따라서 이 메모리 사용량은
해시테이블 크기가 기본 \co{-w24} 에서 시작한
Table~\ref{tab:advsync:Memory Usage of QRCU Model} 에 보인 것보다 작습니다.
이 수행시간은 \Power{9} 서버에서의 것으로, 이 서버는 \co{-DMA=N} 은
\co{-DCOLLAPSE} 에 비해 수십배 높은 CPU 오버헤드를 갖게 됩니다만, 반면에 메모리
오버헤드를 수십배 가량 줄이게 됩니다.
\iffalse

For reference, Table~\ref{tab:formal:promela:QRCU Spin Result Summary}
summarizes the Spin results with \co{-DCOLLAPSE} and \co{-DMA=N}
compiler flags.
The memory usage is obtained with minimal sufficient
search depths and \co{-DMA=N} parameters shown in the table.
Hashtable sizes for \co{-DCOLLAPSE} runs are tweaked by
the \co{-wN} option of \co{./pan} to avoid using too much
memory hashing small state spaces.
Hence the memory usage is smaller than what is shown in
Table~\ref{tab:advsync:Memory Usage of QRCU Model}, where the
hashtable size starts from the default of \co{-w24}.
The runtime is from a \Power{9} server, which shows that \co{-DMA=N}
suffers up to about an order of magnitude higher CPU overhead
than does \co{-DCOLLAPSE}, but on the other hand reduces memory overhead
by well over an order of magnitude.
\fi

적절합니다.
하지만 약간의 업데이트 쓰레드 또는 읽기 쓰레드를 추가하면 \co{-DMA=N} 이
있더라도 메모리를 소모시킬 겁니다.\footnote{
	대안적으로, CPU 소모량이 소모적이 될수도 있습니다.}
그래서 어떡할까요?
여기 몇가지 가능한 방법이 있습니다:
\iffalse

So far so good.
But adding a few more updaters or readers would exhaust memory, even
with \co{-DMA=N}.\footnote{
	Alternatively, the CPU consumption would become excessive.}
So what to do?
Here are some possible approaches:
\fi

\begin{enumerate}
\item	더 적은 수의 읽기 쓰레드와 업데이트 쓰레드가 일반적인 경우를 증명하는데
	충분한지 보세요.
\item	일일이 정확성의 증명을 구하세요.
\item	더 적합한 도구를 사용하세요.
\item	분할하고 정복하세요.
\iffalse

\item	See whether a smaller number of readers and updaters suffice
	to prove the general case.
\item	Manually construct a proof of correctness.
\item	Use a more capable tool.
\item	Divide and conquer.
\fi
\end{enumerate}

다음의 섹션은 이런 방법들 각각을 알아봅니다.
\iffalse

The following sections discuss each of these approaches.
\fi

\subsubsection{How Many Readers and Updaters Are Really Needed?}
\label{sec:formal:How Many Readers and Updaters Are Really Needed?}

한가지 접근법은 \co{qrcu_updater()} 를 위한 Promela 코드를 주의깊게 들여다보고
유일한 전역적 상태 변경은 락을 잡은 아래 일어남을 알아차리는 것입니다.
따라서, 한번에 하나의 업데이트 쓰레드만이 읽기 쓰레드들이나 다른 업데이트
쓰레드들에게 보일 수 있는 상태 변경을 가하고 있을 수 있습니다.
이는, Promela 가 전체 상태 공간 검색을 한다는 사실로 인해, 어떤 시퀀스의 상태
변경들도 하나의 업데이트 쓰레드에 의해 순차적으로 이뤄질 것을 의미합니다.
따라서, 최대 두개의 업데이트 쓰레드가 필요합니다: 하나는 상태를 바꾸기 위해,
나머지 하나는 헷갈려 하기 위해.

읽기 쓰레드들과 함께 있는 상황은 좀 덜 분명한데, 각각의 읽기 쓰레드는 하나의
read-side 크리티컬 섹션만을 만들고 종료되기 때문입니다.
빠른 수행경로는 카운터들에서 최대 0 과 1 만을 읽을 수 있다는 사실로 인해 유용한
읽기 쓰레드의 수가 제한되어 있다고 반론을 제기할 수도 있겠습니다.
실제로, 이는 수사에 있어서 알찬 방법으로, 다음 섹션에서 이야기하는 전체 정확성
증명을 이끌게 됩니다.
\iffalse

One approach is to look carefully at the Promela code for
\co{qrcu_updater()} and notice that the only global state
change is happening under the lock.
Therefore, only one updater at a time can possibly be modifying
state visible to either readers or other updaters.
This means that any sequences of state changes can be carried
out serially by a single updater due to the fact that Promela does a full
state-space search.
Therefore, at most two updaters are required: one to change state
and a second to become confused.

The situation with the readers is less clear-cut, as each reader
does only a single read-side critical section then terminates.
It is possible to argue that the useful number of readers is limited,
due to the fact that the fastpath must see at most a zero and a one
in the counters.
This is a fruitful avenue of investigation, in fact, it leads to
the full proof of correctness described in the next section.
\fi

\subsubsection{Alternative Approach: Proof of Correctness}
\label{sec:formal:Alternative Approach: Proof of Correctness}

비형식적인 증명~\cite{PaulMcKenney2007QRCUpatch} 은 다음과 같습니다:
\iffalse

An informal proof~\cite{PaulMcKenney2007QRCUpatch}
follows:
\fi

\begin{enumerate}
\item	\co{synchronize_qrcu()} 가 너무 일찍 종료되려면, 정의에 의해
	\co{synchronize_qrcu()} 의 전체 수행 사이에 최소 하나의 읽기 쓰레드가
	존재해야 합니다.
\item	이 읽기 쓰레드에 연관된 카운터는 이 시간 간격 동안 최소 1이 되었을
	겁니다.
\item	\co{synchronize_qruc()} 코드는 최소 하나의 카운터는 최소 1이 되도록
	강제합니다.
\item	따라서, 어떤 시점에서든, 카운터들 가운데 하나는 최소 2가 되거나, 두개의
	카운터들이 최소 1을 가질 겁니다.
\item	하지만, \co{synchronize_qrcu()} 빠른 수행 경로 코드는 한번에 하나의
	카운터의 값밖에 읽지 못합니다.
	따라서 빠른 수행 경로 코드가 첫번째 카운터를 값이 0일 동안 읽어오지만
	카운터 뒤집기에서는 경주 상황이 벌어져서 두번째 카운터가 1로 보이는
	경우가 있을 수 있습니다.
\iffalse

\item	For \co{synchronize_qrcu()} to exit too early, then
	by definition there must have been at least one reader
	present during \co{synchronize_qrcu()}'s full
	execution.
\item	The counter corresponding to this reader will have been
	at least 1 during this time interval.
\item	The \co{synchronize_qrcu()} code forces at least one
	of the counters to be at least 1 at all times.
\item	Therefore, at any given point in time, either one of the
	counters will be at least 2, or both of the counters will
	be at least one.
\item	However, the \co{synchronize_qrcu()} fastpath code
	can read only one of the counters at a given time.
	It is therefore possible for the fastpath code to fetch
	the first counter while zero, but to race with a counter
	flip so that the second counter is seen as one.
\fi
\item	그런 경주 조건 동안에 존재하는 최대 하나의 읽기 쓰레드가 있을 수
	있는데, 그렇지 않다면 그 합은 2 이상이 될 것이기 때문으로, 이는
	업데이트 쓰레드가 느린 수행경로를 취하도록 만들 것입니다.
\item	하지만 그 경주 상황이 빠른 수행경로의 첫번째 카운터 읽기에서
	발생한다면, 그리고 그 두번째 읽기에서 다시 일어난다면, 두개의 카운터
	뒤집기가 있었어야만 합니다.
\item	업데이트 쓰레드는 카운터를 한번만 뒤집으므로, 그리고 업데이트 쪽 락은
	두개의 업데이트 쓰레드들이 동시적으로 카운터를 뒤집는 것을 방지하므로,
	빠른 수행경로 코드가 뒤집기와 두번 경주상황을 만들 수 있는건 첫번째
	업데이트 쓰레드가 완료했을 때입니다.
\item	하지만 첫번째 업데이트 쓰레드는 모든 앞서 존재한 읽기 쓰레드들이
	완료되기 전까지는 완료될 수 없습니다.
\item	따라서, 카운터를 두번 뒤집으며 빠른 수행경로 경주상황이 완료된다면,
	모든 앞서 존재한 읽기 쓰레드들은 완료되었어야만 하고, 따라서 빠른
	수행경로를 취해도 안전합니다.
\iffalse

\item	There can be at most one reader persisting through such
	a race condition, as otherwise the sum would be two or
	greater, which would cause the updater to take the slowpath.
\item	But if the race occurs on the fastpath's first read of the
	counters, and then again on its second read, there have
	to have been two counter flips.
\item	Because a given updater flips the counter only once, and
	because the update-side lock prevents a pair of updaters
	from concurrently flipping the counters, the only way that
	the fastpath code can race with a flip twice is if the
	first updater completes.
\item	But the first updater will not complete until after all
	pre-existing readers have completed.
\item	Therefore, if the fastpath races with a counter flip
	twice in succession, all pre-existing readers must have
	completed, so that it is safe to take the fastpath.
\fi
\end{enumerate}

물론 모든 병렬 알고리즘에 이런 간단한 증명이 통하지는 않습니다.
그런 경우에 있어서는 더 적합한 도구들을 모을 필요가 있을 겁니다.
\iffalse

Of course, not all parallel algorithms have such simple proofs.
In such cases, it may be necessary to enlist more capable tools.
\fi

\subsubsection{Alternative Approach: More Capable Tools}
\label{sec:formal:Alternative Approach: More Capable Tools}

Promela 와 Spin 이 상당히 유용하긴 하지만, 훨씬 더 적절한 도구들도 사용할 수
있는데, 특히 하드웨어를 검증할 때 그렇습니다.
이는 낮은 단계의 병렬 알고리즘들이 종종 그렇듯, 당신의 알고리즘이 하드웨어
설계용 VHDL 언어로 변환될 수 있다면, 이 도구들을 코드에 적용해 보는 것도
가능합니다 (예를 들어, 최초의 realtime RCU 알고리즘을 위해 이 방법이
사용되었습니다).
하지만, 그런 도구들은 상당히 비싼 비용을 필요로 할 수 있습니다.

상품화된 멀티프로세싱의 발전이 결국에는 강력하고 멋진 상태 공간 최소화 기능을
가진 프리 소프트웨어 모델 검증기를 만들어낼 수 있겠지만, 현재 여기에 커다란
도움이 되지는 않습니다.
\iffalse

Although Promela and Spin are quite useful,
much more capable tools are available, particularly for verifying
hardware.
This means that if it is possible to translate your algorithm
to the hardware-design VHDL language, as it often will be for
low-level parallel algorithms, then it is possible to apply these
tools to your code (for example, this was done for the first
realtime RCU algorithm).
However, such tools can be quite expensive.

Although the advent of commodity multiprocessing
might eventually result in powerful free-software model-checkers
featuring fancy state-space-reduction capabilities,
this does not help much in the here and now.
\fi

별개로, Spin 은 고정된 양의 메모리를 필요로 하는 대강의 탐색을 지원합니다만,
저는 병렬 알고리즘을 검증할 때 대략적 방법을 신뢰할 수는 없었습니다.

또다른 방법은 분할하고 정복하기가 될겁니다.
\iffalse

As an aside, there are Spin features that support approximate searches
that require fixed amounts of memory, however, I have never been able
to bring myself to trust approximations when verifying parallel
algorithms.

Another approach might be to divide and conquer.
\fi

\subsubsection{Alternative Approach: Divide and Conquer}
\label{sec:formal:Alternative Approach: Divide and Conquer}

커다란 병렬 알고리즘을 개별적으로 증명될 수 있는, 더 작은 조각들로 조각내는
것이 가능한 경우가 종종 있습니다.
예를 들어, 100억개의 상태를 갖는 모델은 두개의 10만개의 상태를 갖는 모델들로
쪼개질 수 있습니다.
이 방법은 Promela 와 같은 도구들이 알고리즘을 검증하는 것을 더 쉽게 해줄 뿐
아니라, 알고리즘을 더 이해하기 쉽게 만들어 줄 수 있습니다.
\iffalse

It is often possible to break down a larger parallel algorithm into
smaller pieces, which can then be proven separately.
For example, a 10-billion-state model might be broken into a pair
of 100,000-state models.
Taking this approach not only makes it easier for tools such as
Promela to verify your algorithms, it can also make your algorithms
easier to understand.
\fi

\subsubsection{Is QRCU Really Correct?}
\label{sec:formal:Is QRCU Really Correct?}

QRCU 는 정말로 올바르게 동작할까요?
우리는 Promela 기반의 기계적 증명과 손으로 하는 증명을 했고 둘 모두 그렇다고
이야기했습니다.
하지만, Alglave 등~\cite{JadeAlglave2013-cav} 의 최근 논문은 다르게
이야기합니다 (해당 논문 page~12  아래쪽의 Section~5.1 을 보세요).
뭐가 맞을까요?

둘 다 맞는 것으로 드러났습니다!
QRCU 가 formal-verification 벤치마크에 추가되었을 때, 그 안의 메모리 배리어들이
제거되어 있었고, 따라서 버그가 존재하는 버전의 QRCU 였습니다.
그러니 여기서의 진짜 새로운 소식은 여러개의 formal-verification 도구들이 이
버그가 존재하는 QRCU 를 올바르지 않게도 올바른 것으로 증명했다는 겁니다.
그리고 이게 formal-verification 도구들은 그들 스스로가 bug 가 추가된 버전의
코드를 검증해 보는 것으로 테스트 되어야 하는 이유입니다.
만약 특정 도구가 주입된 버그를 찾지 못한다면, 그 도구는 믿을 수 없는게
분명합니다.
\iffalse

Is QRCU really correct?
We have a Promela-based mechanical proof and a by-hand proof that both
say that it is.
However, a recent paper by Alglave et al.~\cite{JadeAlglave2013-cav}
says otherwise (see Section~5.1 of the paper at the bottom of page~12).
Which is it?

It turns out that both are correct!
When QRCU was added to a suite of formal-verification benchmarks,
its memory barriers were omitted, thus resulting in a buggy version
of QRCU.
So the real news here is that a number of formal-verification tools
incorrectly proved this buggy QRCU correct.
And this is why formal-verification tools themselves should be tested
using bug-injected versions of the code being verified.
If a given tool cannot find the injected bugs, then that tool is
clearly untrustworthy.
\fi

\QuickQuiz{}
	하지만 다른 formal-verification 도구들은 종종 특정 부류의 버그들을 찾기
	위해 설계됩니다.
	예를 들어, 극히 일부의 formal-verification 도구들은 해당 스펙 상의
	에러만을 찾아낼 겁니다.
	그러니, 이 ``분명히 믿을 수 없습니다'' 라는 말은 약간 너무한 거
	아닌가요?
	\iffalse

	But different formal-verification tools are often designed to
	locate particular classes of bugs.
	For example, very few formal-verification tools will find
	an error in the specification.
	So isn't this ``clearly untrustworthy'' judgment a bit harsh?
	\fi
\QuickQuizAnswer{
	많은 formal-verification 도구들이 어떤 방향으로 특수화 되어 있는 것은
	분명한 사실입니다.
	예를 들어, Promela 는 현실적인 메모리 모델들을 다루지 않고 (비록
	Promela 에 그것들이 프로그램 될 수는
	있지만요~\cite{Desnoyers:2013:MSM:2506164.2506174}),
	CBMD~\cite{EdmundClarke2004CBMC} 는 확률적인 hang 과 데드락을 파악하지
	못하며, Nidhugg~\cite{CarlLeonardsson2014Nidhugg} 는 데이터의
	비결정성에 연관된 버그는 찾지 못합니다.
	하지만 이는 이 도구들이 찾기 위해 설계된 버그 외의 것을 찾을 거라고
	믿을 수는 없음을 의미합니다.
	\iffalse

	It is certainly true that many formal-verification tools are
	specialized in some way.
	For example, Promela does not handle realistic memory models
	(though they can be programmed into
	Promela~\cite{Desnoyers:2013:MSM:2506164.2506174}),
	CBMC~\cite{EdmundClarke2004CBMC} does not detect probabilistic
	hangs and deadlocks, and
	Nidhugg~\cite{CarlLeonardsson2014Nidhugg} does not detect
	bugs involving data nondeterminism.
	But this means that that these tools cannot be trusted to find
	bugs that they are not designed to locate.
	\fi

	또한, 그렇기에 formal-verification 도구들을 만드는 사람들은 분명하게
	어떤 부류의 버그들을 그들의 도구들이 발견할 수 없고 파악할 수 없는지에
	대해 ``진실을 말해야'' 합니다.
	그러지 않는다면, 실제 사용자가 어떤 버그를 파악하지 못하는 도구를 처음
	발견하면, 그 사용자는 그 도구에 상당히 거칠고 극단적으로 공식적인
	비난을 가할 겁니다.
	맞습니다, 맞아요, 여러분이 최고의 노력을 다했다고 말할 것이 존재하겠죠,
	하지만 적절한 면책 조항 없이 그걸 너무 앞에 놓으면 여러분의 도구가
	그로부터 회복될지 안될지도 모를 부정적인 반응의 더미를 쉽게 이끌어낼
	수도 있습니다.

	경고했어요!
	\iffalse

	And therefore people creating formal-verification tools should
	``tell the truth on the label'', clearly calling out what
	classes of bugs their tools can and cannot detect.
	Otherwise, the first time a practitioner finds a tool
	failing to detect a bug, that practitioner is likely to
	make extremely harsh and extremely public denunciations
	of that tool.
	Yes, yes, there is something to be said for putting your
	best foot forward, but putting it too far forward without
	appropriate disclaimers can easily trigger a land mine of
	negative reaction that your tool might or might not be able
	to recover from.

	You have been warned!
	\fi
} \QuickQuizEnd

따라서, 여러분이 QRCU 를 사용하려 한다면, 주의하시기 바랍니다.
그것의 올바름에 대한 증명은 그 자체가 옳을수도, 아닐수도 있습니다.
이는 Donald Knuth 가 오래전부터 지적한대로 formal verification 이 완전히
테스트를 교체할 수는 없을 거라 이야기 되는 이유입니다.
\iffalse

Therefore, if you do intend to use QRCU, please take care.
Its proofs of correctness might or might not themselves be correct.
Which is one reason why formal verification is unlikely to
completely replace testing, as Donald Knuth pointed out so long ago.
\fi

\QuickQuiz{}
	여기에 설명된 QRCU 알고리즘의 정확성에 대한 두개의 독립적인 증명을
	가지고 있고, 올바르지 않음에 대한 증명이 다른 알고리즘은 어떤지를
	다루고 있는데, 왜 여전히 의문의 여지가 남는 거죠?
	\iffalse

	Given that we have two independent proofs of correctness for
	the QRCU algorithm described herein, and given that the
	proof of incorrectness covers what is known to be a different
	algorithm, why is there any room for doubt?
	\fi
\QuickQuizAnswer{
	항상 의문의 여지가 존재합니다.
	이 경우에는, 정확성에 대한 두개의 증명이 실제 세계 메모리 모델들을
	신경쓰지 않고 있으므로, 이 두개의 증명들은 잘못된 메모리 순서규칙
	가정에 기반하고 있을 수 있다는 점을 명심하시기 바랍니다.
	더욱이, 두 증명 모두 같은 사람에 의해 만들어졌으므로, 동일한 에러를
	포함하고 있을 가능성이 농후합니다.
	다시 말하지만, 항상 의심의 여지가 존재합니다.
	\iffalse

	There is always room for doubt.
	In this case, it is important to keep in mind that the two proofs
	of correctness preceded the formalization of real-world memory
	models, raising the possibility that these two proofs are based
	on incorrect memory-ordering assumptions.
	Furthermore, since both proofs were constructed by the same person,
	it is quite possible that they contain a common error.
	Again, there is always room for doubt.
	\fi
} \QuickQuizEnd

% formal/dyntickrcu.tex

\subsection{Promela Parable: dynticks and Preemptible RCU}
\label{sec:formal:Promela Parable: dynticks and Preemptible RCU}

2005년 8월부터 시작된 -rt 패치셋~\cite{IngoMolnar05a} 의 RCU 구현과 비슷한, RCU
의 preemption 가능한 변종이 2008년 초에 메인라인 리눅스에 realtime 워크로드에
대한 지원과 함께 받아들여졌습니다.
Preemption 가능한 RCU 는 기존의 RCU 구현들은 RCU read-side 크리티컬 섹션
내에서의 preemption 을 불가능하게 했기 때문에 지나친 real-time 대기시간을
초래했기 때문에 real-time 워크로드를 위해 필요했습니다.
\iffalse

In early 2008, a preemptible variant of RCU was accepted into
mainline Linux in support of real-time workloads,
a variant similar to the RCU implementations in
the -rt patchset~\cite{IngoMolnar05a}
since August 2005.
Preemptible RCU is needed for real-time workloads because older
RCU implementations disable preemption across RCU read-side
critical sections, resulting in excessive real-time latencies.
\fi

하지만, 기존의 -rt 구현의 한가지 단점은 각각의 grace period 가 모든 각각의 CPU
위에서, 설령 그 CPU 가 저전력의 ``dynticks-idle'' 상태에 있더라도, 작업을 해야
하고, 그로 인해 RCU read-side 크리티컬 섹션들을 수행하기가 불가능하다는
점이었습니다.
Dynticks-idle 상태의 아이디어는 idle CPU 들은 에너지를 아끼기 위해 물리적으로
성능을 낮춰야 한다는 것입니다.
짤게 말해서, preemption 가능한 RCU 는 최신 리눅스 커널에서의 가치있는 에너지
절약 기능을 무효화 시킬 수 있다는 점입니다.
Josh Triplett 과 Paul McKenney 가 CPU 들이 RCU graceperiod 사이에도 저전력
상태를 유지할 수 있는 (따라서 리눅스 커널의 에너지 절약 기능을 유지할 수 있는)
방법에 대해서 토론을 가졌습니다만, Steve Rostedt 가 새로운 dyntick 구현을 -rt
패치셋의 preemtpion 가능한 RCU 와 결합시키기 전까지는 해결책이 나오지
않았습니다.
\iffalse

However, one disadvantage of the older -rt implementation
was that each grace period
requires work to be done on each CPU, even if that CPU is in a low-power
``dynticks-idle'' state,
and thus incapable of executing RCU read-side critical sections.
The idea behind the dynticks-idle state is that idle CPUs
should be physically powered down in order to conserve energy.
In short, preemptible RCU can disable a valuable energy-conservation
feature of recent Linux kernels.
Although Josh Triplett and Paul McKenney
had discussed some approaches for allowing
CPUs to remain in low-power state throughout an RCU grace period
(thus preserving the Linux kernel's ability to conserve energy), matters
did not come to a head until Steve Rostedt integrated a new dyntick
implementation with preemptible RCU in the -rt patchset.
\fi

이 조합은 Steve 의 시스템들 중 하나를 부팅 과정에서 멈춰있게 만들었고, 따라서
10월에, Paul 은 preemption 가능한 RCU 의 grace-period 처리 부분에 dynticks 에
친화적인 수정사항을 코딩 했습니다.
Steve 는 \co{irq_enter()} 와 \co{irq_exit()} 인터럽트 진입/퇴장 함수들로부터
호출되는 \co{rcu_irq_enter()} 와 \co{rcu_irq_exit()} 인터페이스들을
코딩했습니다.
이 \co{rcu_riq_enter()} 와 \co{rcu_irq_exit()} 함수들은 dynticks-idle CPU 들이
RCU read-side 크리티컬 섹션들을 포함하고 있는 인터럽트 핸들러에 의해 순간적으로
성능을 올리게 되는 상황을 안정적으로 처리하기 위해 필요했습니다.
이런 변경 사항들과 함께, Steve 의 시스템은 안정적으로 부팅되었습니다만, Paul 은
첫번째 시도만에 코드가 올바르게 작성되지는 않았을 것이라는 가정 하에 지속적으로
코드를 검사하기를 계속했습니다.
\iffalse

This combination caused one of Steve's systems to hang on boot, so in
October, Paul coded up a dynticks-friendly modification to preemptible RCU's
grace-period processing.
Steve coded up \co{rcu_irq_enter()} and \co{rcu_irq_exit()}
interfaces called from the
\co{irq_enter()} and \co{irq_exit()} interrupt
entry/exit functions.
These \co{rcu_irq_enter()} and \co{rcu_irq_exit()}
functions are needed to allow RCU to reliably handle situations where
a dynticks-idle CPUs is momentarily powered up for an interrupt
handler containing RCU read-side critical sections.
With these changes in place, Steve's system booted reliably,
but Paul continued inspecting the code periodically on the assumption
that we could not possibly have gotten the code right on the first try.
\fi

Paul 은 2007년 10월부터 2008년 2월까지 반복적으로 코드를 리뷰했고 거의 매번
최소 하나의 버그는 찾아냈습니다.
한 경우에는, Paul 은 심지어 그 버그가 실체가 없는 것이란 것을 깨닫기 전에도
수정사항을 코딩하고 테스트 하기도 했고, 사실 모든 경우들에 있어서, ``버그'' 는
실체가 없는 것으로 드러났습니다.

2월 말 즈음에, Paul 은 이 게임에 지쳐버렸습니다.
따라서 그는
Section~\ref{chp:formal:Formal Verification} 에서 이야기한 것처럼
Promela 와 spin~\cite{Holzmann03a} 의 도움을 받기로 했습니다.
다음의 내용은 일곱개의 갈수록 현실적인 Promela 모델들을 보이는데, 그 중
마지막의 것은 상태 공간으로 약 40GB 의 메인 메모리를 소비합니다.

더 중요한건, Promela 와 Spin 은 제게 매우 미묘한 버그를 찾아줬습니다!
\iffalse

Paul reviewed the code repeatedly from October 2007 to February 2008,
and almost always found at least one bug.
In one case, Paul even coded and tested a fix before realizing that the
bug was illusory, and in fact in all cases, the ``bug'' turned out to be
illusory.

Near the end of February, Paul grew tired of this game.
He therefore decided to enlist the aid of
Promela and spin~\cite{Holzmann03a}, as described in
Section~\ref{chp:formal:Formal Verification}.
The following presents a series of seven increasingly realistic
Promela models, the last of which passes, consuming about
40GB of main memory for the state space.

More important, Promela and Spin did find a very subtle bug for me!
\fi

\QuickQuiz{}
	와우, 거 참 대단하네요!
	이제, 저는 40GB 의 메인 메모리를 가진 기계가 없다면 뭘 해야 하는거죠???
	\iffalse

	Yeah, that's just great!
	Now, just what am I supposed to do if I don't happen to have a
	machine with 40GB of main memory???
	\fi
\QuickQuizAnswer{
	진정하세요, 이 질문에 대한 많은 정당한 답들이 있습니다:
	\iffalse

	Relax, there are a number of lawful answers to
	this question:
	\fi
	\begin{enumerate}
	\item	해당 모델을 더 최적화 해서, 메모리 소비량을 줄이세요.
	\item	리눅스 커널의 코드에 있는 주석부터 시작해서 연필과 종이 증명
		방법을 실행하세요.
	\item	비록 코드의 정확성을 증명할 수는 없지만 숨겨진 버그들을 찾아줄
		수 있는, 신중히 생각한 고문 테스트들을 고안하세요.
	\item	작은 기계들의 클러스터들을 사용해서 모델 체크를 하는 도구들을
		만들고 사용하려는 움직임이 일부 있습니다.
		하지만, Paul 은 사용할 수 있는 일부 커다란 기계들로 인해서 그런
		도구들을 직접 사용해 본 적은 없음을 알아두시기 바랍니다.
	\item	당신의 문제를 적용하기에 맞는 크기의 메모리 크기를 가진, 구매할
		만한 각겨대의 시스템이 나오길 기다리세요.
	\item	짧은 시간 동안 커다란 시스템을 대여하기 위해 클라우드 컴퓨팅
		서비스들 가운데 하나를 고려해 보세요.
	\iffalse

	\item	Further optimize the model, reducing its memory consumption.
	\item	Work out a pencil-and-paper proof, perhaps starting with the
		comments in the code in the Linux kernel.
	\item	Devise careful torture tests, which, though they cannot prove
		the code correct, can find hidden bugs.
	\item	There is some movement towards tools that do model
		checking on clusters of smaller machines.
		However, please note that we have not actually used such
		tools myself, courtesy of some large machines that Paul has
		occasional access to.
	\item	Wait for memory sizes of affordable systems to expand
		to fit your problem.
	\item	Use one of a number of cloud-computing services to rent
		a large system for a short time period.
	\fi
	\end{enumerate}
} \QuickQuizEnd

여전히 더 나은 방향은 더 작은 상태 공간을 갖는, 더 간단하고 더 빠른 알고리즘을
사용하는 것일 겁니다.
그보다도 더 나은 방법은 그 정확성이 평상시의 관찰자에 의해서도 분명할 수
있을만큼 간단한 알고리즘들을 사용하는 것이겠죠!
\iffalse

Still better would be to come up with a simpler and faster algorithm
that has a smaller state space.
Even better would be an algorithm so simple that its correctness was
obvious to the casual observer!
\fi

Section~\ref{sec:formal:Introduction to Preemptible RCU and dynticks}
은 preemption 가능한 RCU 의 dynticks 인터페이스에 대해 전체적으로 살펴보고,
Section~\ref{sec:formal:Validating Preemptible RCU and dynticks} 와
Section~\ref{sec:formal:Lessons (Re)Learned} 은 이로써 (다시) 배운 교훈들을
나열합니다.
\iffalse

Section~\ref{sec:formal:Introduction to Preemptible RCU and dynticks}
gives an overview of preemptible RCU's dynticks interface,
Section~\ref{sec:formal:Validating Preemptible RCU and dynticks},
and
Section~\ref{sec:formal:Lessons (Re)Learned} lists
lessons (re)learned during this effort.
\fi

\subsubsection{Introduction to Preemptible RCU and dynticks}
\label{sec:formal:Introduction to Preemptible RCU and dynticks}

Per-CPU \co{dynticks_progress_counter} 변수가 dyntick 과 preemption 가능한 RCU
사이의 인터페이스의 중심 역할을 합니다.
이 변수는 연관된 CPU 가 dynticks-idle 모드일 때에는 짝수를 가지게 되고, 그렇지
않을 때에는 홀수를 갖게 됩니다.
CPU 는 다음의 세가지 이유로 dynticks-idle 모드를 빠져나갑니다:
\iffalse

The per-CPU \co{dynticks_progress_counter} variable is
central to the interface between dynticks and preemptible RCU.
This variable has an even value whenever the corresponding CPU
is in dynticks-idle mode, and an odd value otherwise.
A CPU exits dynticks-idle mode for the following three reasons:
\fi

\begin{enumerate}
\item	태스크를 수행하기 시작하기 위해서,
\item	중첩되어 있을 수 있는 인터럽트 핸들러들 가운데 가장 바깥쪽의 것에
	들어가기 위해서, 그리고
\item	NMI 핸들러에 들어갈 때.
\iffalse

\item	to start running a task,
\item	when entering the outermost of a possibly nested set of interrupt
	handlers, and
\item	when entering an NMI handler.
\fi
\end{enumerate}

Preemption 가능한 RCU 의 grace-period 장치는 언제 dynticks-idle CPU 가 안전하게
무시될 수 있을지를 판단하기 위해서 \co{dynticks_progress_counter} 변수의 값을
샘플링 합니다.

다음의 세개의 섹션들은 태스크 인터페이스, 인터럽트/NMI 인터페이스, 그리고
grace-period 장치에 의한 \co{dynticks_progress_counter} 변수의 사용에 대해서
알아봅니다.
\iffalse

Preemptible RCU's grace-period machinery samples the value of
the \co{dynticks_progress_counter} variable in order to
determine when a dynticks-idle CPU may safely be ignored.

The following three sections give an overview of the task
interface, the interrupt/NMI interface, and the use of
the \co{dynticks_progress_counter} variable by the
grace-period machinery.
\fi

\subsubsection{Task Interface}
\label{sec:formal:Task Interface}

특정 CPU 가 더이상 수행할 태스크가 존재하지 않아 dynticks-idle 모드로 들어갈
때에, 해당 CPU 는 \co{rcu_enter_nohz()} 를 호출합니다:
\iffalse

When a given CPU enters dynticks-idle mode because it has no more
tasks to run, it invokes \co{rcu_enter_nohz()}:
\fi

{ \scriptsize
\begin{verbatim}
 1  static inline void rcu_enter_nohz(void)
 2  {
 3    mb();
 4    __get_cpu_var(dynticks_progress_counter)++;
 5    WARN_ON(__get_cpu_var(dynticks_progress_counter) &
 6            0x1);
 7  }
\end{verbatim}
}

이 함수는 단순히 \co{dynticks_progress_counter} 의 값을 증가시키고 그 결과가
짝수인지를 체크합니다만, 그전에 먼저, \co{dynticks_progress_counter} 의 새로운
값을 보게될 다른 CPU 가 앞의 RCU read-side 크리티컬 섹션의 완료 역시 보게 될
것을 보장하기 위해 메모리 배리어를 실행합니다.

비슷하게, dynticks-idle 모드에 있는 CPU 가 새로운 수행가능한 태스크를
실행하기를 시작할 준비를 할 때에는, \co{rcu_exit_nohz} 를 수행합니다:
\iffalse

This function simply increments \co{dynticks_progress_counter} and
checks that the result is even, but first executing a memory barrier
to ensure that any other CPU that sees the new value of
\co{dynticks_progress_counter} will also see the completion
of any prior RCU read-side critical sections.

Similarly, when a CPU that is in dynticks-idle mode prepares to
start executing a newly runnable task, it invokes
\co{rcu_exit_nohz}:
\fi

{ \scriptsize
\begin{verbatim}
  1 static inline void rcu_exit_nohz(void)
  2 {
  3   __get_cpu_var(dynticks_progress_counter)++;
  4   mb();
  5   WARN_ON(!(__get_cpu_var(dynticks_progress_counter) &
  6             0x1));
  7 }
\end{verbatim}
}

이 함수는 한번더 \co{dynticks_progress_counter} 의 값을 증가시킵니다만, 어떤
다른 CPU 가 뒤의 RCU read-side 크리티컬 섹션의 결과를 보게 된다면 그 CPU 는
\co{dynticks_progress_counter} 의 증가된 값 엯 비로 수 있을 것을 보장하기 위해
값 증가에 이어 메모리 배리어를 실행시킵니다.
마지막으로, \co{rcu_exit_nohz()} 는 값 증가의 결과가 홀수임을 확인합니다.
\iffalse

This function again increments \co{dynticks_progress_counter},
but follows it with a memory barrier to ensure that if any other CPU
sees the result of any subsequent RCU read-side critical section,
then that other CPU will also see the incremented value of
\co{dynticks_progress_counter}.
Finally, \co{rcu_exit_nohz()} checks that the result of the
increment is an odd value.

The \co{rcu_enter_nohz()} and \co{rcu_exit_nohz}
functions handle the case where a CPU enters and exits dynticks-idle
mode due to task execution, but does not handle interrupts, which are
covered in the following section.
\fi

\subsubsection{Interrupt Interface}
\label{sec:formal:Interrupt Interface}

\co{rcu_irq_enter()} 와 \co{rcu_irq_exit()} 함수들은 인터럽트/NMI 진입과 종료를
각각 처리합니다.
물론, 중첩된 인터럽트들 또한 적절하게 처리되어야만 합니다.
중첩 인터럽트의 가능성은 인터럽트나 NMI 핸들러 로의 진입 시에
(\co{rcu_irq_enter()} 에서) 값이 증가되고 종료 시에(\co{rcu_irq_exit()} 에서)
값이 감소되는 두번째 per-CPU 변수, \co{rcu_update_flag} 에 의해 처리됩니다.
추가적으로, 앞서서부터 존재해온 \co{in_interrupt()} 기능은 인터럽트/NMI 가 가장
바깥의 것인지 중첩된 것인지 구별해 내는데에 사용됩니다.

인터럽트 진입은 아래 보여진 \co{rcu_irq_enter()} 에서 처리됩니다:
\iffalse

The \co{rcu_irq_enter()} and \co{rcu_irq_exit()}
functions handle interrupt/NMI entry and exit, respectively.
Of course, nested interrupts must also be properly accounted for.
The possibility of nested interrupts is handled by a second per-CPU
variable, \co{rcu_update_flag}, which is incremented upon
entry to an interrupt or NMI handler (in \co{rcu_irq_enter()})
and is decremented upon exit (in \co{rcu_irq_exit()}).
In addition, the pre-existing \co{in_interrupt()} primitive is
used to distinguish between an outermost or a nested interrupt/NMI.

Interrupt entry is handled by the \co{rcu_irq_enter()}
shown below:
\fi

{ \scriptsize
\begin{verbatim}
  1 void rcu_irq_enter(void)
  2 {
  3   int cpu = smp_processor_id();
  4
  5   if (per_cpu(rcu_update_flag, cpu))
  6     per_cpu(rcu_update_flag, cpu)++;
  7   if (!in_interrupt() &&
  8       (per_cpu(dynticks_progress_counter,
  9                cpu) & 0x1) == 0) {
 10     per_cpu(dynticks_progress_counter, cpu)++;
 11     smp_mb();
 12     per_cpu(rcu_update_flag, cpu)++;
 13   }
 14 }
\end{verbatim}
}

Line~3 는 현재 CPU 의 숫자를 가져오고, line~5 와 6 는 \co{rcu_update_flag} 중첩
카운터가 0이 아니라면 그 값을 증가시킵니다.
Line~7-9 는 자신이 인터럽트의 가장 바깥 단계인지를 체크하고, 만약 그렇다면
\co{dynticks_progress_counter} 가 증가되어야 합니다.
그렇다면, line~10 에서 \co{dynticks_progress_counter} 의 값을 증가시키고,
line~11 에서 메모리 배리어를 실행한 후, line~12 에서 \co{rcu_update_flag} 의
값을 증가시킵니다.
\co{rcu_exit_nohz()} 에서와 마찬가지로, 이 메모리 배리어는 이 인터럽트 핸들러
안에서의 RCU read-side 크리티컬 섹션의 영향을 본 CPU 는
\co{dynticks_progress_counter} 의 값의 증가 결과 역시 볼 수 있을 것을
보장합니다.
\iffalse

Line~3 fetches the current CPU's number, while lines~5 and~6
increment the \co{rcu_update_flag} nesting counter if it
is already non-zero.
Lines~7-9 check to see whether we are the outermost level of
interrupt, and, if so, \co{dynticks_progress_counter}
needs to be incremented.
If so, line~10 increments \co{dynticks_progress_counter},
line~11 executes a memory barrier, and line~12 increments
\co{rcu_update_flag}.
As with \co{rcu_exit_nohz()}, the memory barrier ensures that
any other CPU that sees the effects of an RCU read-side critical section
in the interrupt handler (following the \co{rcu_irq_enter()}
invocation) will also see the increment of
\co{dynticks_progress_counter}.
\fi

\QuickQuiz{}
	왜 간단하게 \co{rcu_update_flag} 의 값을 증가시키고,
	\co{rcu_update_flag} 의 기존 값이 0일 경우에만
	\co{dynticks_progress_counter} 의 값을 증가시키는 방식을 사용하지
	않는거죠???
	\iffalse

	Why not simply increment \co{rcu_update_flag}, and then only
	increment \co{dynticks_progress_counter} if the old value
	of \co{rcu_update_flag} was zero???
	\fi
\QuickQuizAnswer{
	이는 NMI 의 존재 시에 실패합니다.
	이를 자세히 보기 위해, \co{rcu_irq_enter()} 가 \co{rcu_update_flag} 를
	증가시킨 직후, 하지만 \co{dynticks_progress_counter} 를 증가시키기 전에
	NMI 를 받았다고 생각해 봅시다.
	NMI 에 의해 호출된 \co{rcu_irq_enter()} 의 인스턴스는
	\co{rcu_update_flag} 의 원래 값이 0이 아닌 것으로 보게 될 것이고,
	따라서 \co{dynticks_progress_counter} 의 값을 증가시키지 않을 겁니다.
	이는 RCU grace-period 치가 이 CPU 에서 NMI 핸들러가 실행되었음을 알 수
	있는 증거를 남기지 않을 것이고, 따라서 이 NMI 핸들러 내에서의 모든 RCU
	red-side 크리티컬 섹션들은 RCU 보호를 깨트릴 겁니다.

	그 정의에 의해 마스킹 될 수 없는 NMI 핸들러들의 존재 가능성은 정말로 이
	코드를 복잡하게 만듭니다.
	\iffalse

	This fails in presence of NMIs.
	To see this, suppose an NMI was received just after
	\co{rcu_irq_enter()} incremented \co{rcu_update_flag},
	but before it incremented \co{dynticks_progress_counter}.
	The instance of \co{rcu_irq_enter()} invoked by the NMI
	would see that the original value of \co{rcu_update_flag}
	was non-zero, and would therefore refrain from incrementing
	\co{dynticks_progress_counter}.
	This would leave the RCU grace-period machinery no clue that the
	NMI handler was executing on this CPU, so that any RCU read-side
	critical sections in the NMI handler would lose their RCU protection.

	The possibility of NMI handlers, which, by definition cannot
	be masked, does complicate this code.
	\fi
} \QuickQuizEnd

\QuickQuiz{}
	하지만 line~7 이 우리가 가장 바깥의 인터럽트에 있음을 알게 된다면, 우린
	\emph{항상} \co{dynticks_progress_counter} 의 값을 증가시켜야 하는 것
	아닌가요?
	\iffalse

	But if line~7 finds that we are the outermost interrupt,
	wouldn't we \emph{always} need to increment
	\co{dynticks_progress_counter}?
	\fi
\QuickQuizAnswer{
	수행 중인 태스크를 인터럽트 했다면 그렇지 않습니다!
	그런 경우에는, \co{dynticks_progress_counter} 는 이미
	\co{rcu_exit_nohz()} 에 의해 값이 증가되었을 것이고, 이 경우에 대해서는
	값을 한번 더 증가시킬 필요가 없을 겁니다.
	\iffalse

	Not if we interrupted a running task!
	In that case, \co{dynticks_progress_counter} would
	have already been incremented by \co{rcu_exit_nohz()},
	and there would be no need to increment it again.
	\fi
} \QuickQuizEnd

인터럽트 종료는 \co{rcu_irq_exit()} 에 의해 비슷하게 처리됩니다:
\iffalse

Interrupt exit is handled similarly by
\co{rcu_irq_exit()}:
\fi

{ \scriptsize
\begin{verbatim}
  1 void rcu_irq_exit(void)
  2 {
  3   int cpu = smp_processor_id();
  4
  5   if (per_cpu(rcu_update_flag, cpu)) {
  6     if (--per_cpu(rcu_update_flag, cpu))
  7       return;
  8     WARN_ON(in_interrupt());
  9     smp_mb();
 10     per_cpu(dynticks_progress_counter, cpu)++;
 11     WARN_ON(per_cpu(dynticks_progress_counter,
 12                     cpu) & 0x1);
 13   }
 14 }
\end{verbatim}
}

Line~3 은 앞에서처럼 현재 CPU 의 번호를 가져옵니다.
Line~5 는 \co{rcu_update_flag} 의 값이 0이 아닌지 확인해보고, 그렇지 않다면
곧바로 (함수의 마지막으로 제어를 넘김으로써) 리턴합니다.
그렇지 않다면, line~6 부터 12 가 수행됩니다.
Line~6 은 \co{rcu_update_flag} 의 값을 감소시키고, 그 결과가 0이 아니라면
리턴합니다.
Line~8 은 우리가 정말로 중첩된 인터럽트들 가운데 가장 마지막 단계를 빠져나가고
있음을 검증하고, line~9 에서 메모리 배리어를 수행한 후, line~10 에서
\co{dynticks_progress_counter} 의 값을 증가시키고, line~11 과~12 에서 이 변수가
이제는 짝수임을 검증합니다.
\co{rcu_enter_nohz()} 에서와 같이, 메모리 배리어는
\co{dynticks_progress_counter} 의 값 증가를 본 다른 CPU 는 해당 인터럽트 핸들러
내에서의 (\co{rcu_irq_exit()} 호출을 앞서 수행된) RCU read-side 크리티컬 섹션의
결과 역시 보게 될것을 보장합니다.
\iffalse

Line~3 fetches the current CPU's number, as before.
Line~5 checks to see if the \co{rcu_update_flag} is
non-zero, returning immediately (via falling off the end of the
function) if not.
Otherwise, lines~6 through~12 come into play.
Line~6 decrements \co{rcu_update_flag}, returning
if the result is not zero.
Line~8 verifies that we are indeed leaving the outermost
level of nested interrupts, line~9 executes a memory barrier,
line~10 increments \co{dynticks_progress_counter},
and lines~11 and~12 verify that this variable is now even.
As with \co{rcu_enter_nohz()}, the memory barrier ensures that
any other CPU that sees the increment of
\co{dynticks_progress_counter}
will also see the effects of an RCU read-side critical section
in the interrupt handler (preceding the \co{rcu_irq_exit()}
invocation).
\fi

이 두 섹션들은 \co{dynticks_progress_counter} 변수가 태스크들과 인터럽트, NMI
에 의해 dynticks-idle 모드로 들어갈 때와 빠져나올 때에 어떻게 관리되는지를
설명했습니다.
다음 섹션은 이 변수가 preemption 가능한 RCU 의 grace-period 장치에 의해 어떻게
사용되는지를 설명합니다.
\iffalse

These two sections have described how the
\co{dynticks_progress_counter} variable is maintained during
entry to and exit from dynticks-idle mode, both by tasks and by
interrupts and NMIs.
The following section describes how this variable is used by
preemptible RCU's grace-period machinery.
\fi

\subsubsection{Grace-Period Interface}
\label{sec:formal:Grace-Period Interface}

\begin{figure}[htb]
\centering
\resizebox{3in}{!}{\includegraphics{formal/RCUpreemptStates}}
\caption{Preemptible RCU State Machine}
\label{fig:formal:Preemptible RCU State Machine}
\end{figure}

Figure~\ref{fig:formal:Preemptible RCU State Machine} 에 보인 네개의 preemption
가능한 RCU grace-period 상태등 가운데, \co{rcu_try_flip_waitack_state()} 와
\co{rcu_try_flip_waitmb_state()} 상태들만이 다른 CPU 들이 응답하길 기다려야
합니다.

물론, 한 CPU 가 dynticks-idle 상태에 있다면, 그걸 기다릴 수는 없습니다.
따라서, 이 두개의 상태들 중 하나에 들어가기 직전에, 앞의 상태는 각 CPU 들의
\co{dynticks_progress_counter} 변수의 값들의 스탭샷을 떠놓고, 그 스냅샷을
또다른 per-CPU 변수인 \co{rcu_dyntick_snapshot} 에 넣어둡니다.
이는 아래에 보여진 것과 같은 \co{dyntick_save_progress_counter()} 를
호출함으로써 이뤄집니다:
\iffalse

Of the four preemptible RCU grace-period states shown in
Figure~\ref{fig:formal:Preemptible RCU State Machine},
only the \co{rcu_try_flip_waitack_state()}
and \co{rcu_try_flip_waitmb_state()} states need to wait
for other CPUs to respond.

Of course, if a given CPU is in dynticks-idle state, we shouldn't
wait for it.
Therefore, just before entering one of these two states,
the preceding state takes a snapshot of each CPU's
\co{dynticks_progress_counter} variable, placing the
snapshot in another per-CPU variable,
\co{rcu_dyntick_snapshot}.
This is accomplished by invoking
\co{dyntick_save_progress_counter()}, shown below:
\fi

{ \scriptsize
\begin{verbatim}
  1 static void dyntick_save_progress_counter(int cpu)
  2 {
  3   per_cpu(rcu_dyntick_snapshot, cpu) =
  4     per_cpu(dynticks_progress_counter, cpu);
  5 }
\end{verbatim}
}

The \co{rcu_try_flip_waitack_state()} 상태는 아래에 보여진 것과 같은
\co{rcu_try_flip_waitack_needed()} 를 호출합니다:
\iffalse

The \co{rcu_try_flip_waitack_state()} state invokes
\co{rcu_try_flip_waitack_needed()}, shown below:
\fi

{ \scriptsize
\begin{verbatim}
  1 static inline int
  2 rcu_try_flip_waitack_needed(int cpu)
  3 {
  4   long curr;
  5   long snap;
  6
  7   curr = per_cpu(dynticks_progress_counter, cpu);
  8   snap = per_cpu(rcu_dyntick_snapshot, cpu);
  9   smp_mb();
 10   if ((curr == snap) && ((curr & 0x1) == 0))
 11     return 0;
 12   if ((curr - snap) > 2 || (snap & 0x1) == 0)
 13     return 0;
 14   return 1;
 15 }
\end{verbatim}
}

Line~7 과 8 은 \co{dynticks_progress_counter} 의 현재 버전과 스탭샷 버전을 각각
가져옵니다.
Line~9 에서의 메모리 배리어는 뒤의 \co{rcu_try_flip_waitzero_state()} 에서의
카운터 체크가 이 카운터들을 읽어오는 행위 뒤에 이뤄짐을 분명히
ㅎㅎㅎㅎㅎㅎ합니다.
Line~10 과 11 은 해당 CPU 가 스냅샷이 찍힌 후로 dynticks-idle 상태를 유지했다면
0을 리턴 (해당 CPU 와의 커뮤니케이션이 필요치 않음을 의미) 합니다.
비슷하게, Line~12 와 13 은 해당 CPU 가 초기에 dynticks-idle 상태였거나
dynticks-idle 상태를 완전히 지나왔다면 0을 리턴합니다.
이 두가지 경우 모두에 있어서, 해당 CPU 가 grace-period zk운터의 옜날 값을
가져왔을 수는 없습니다.
이런 조건들 가운데 하나도 해당되지 않는다면, line~14 에서 1을 리턴해서 해당 CPU
는 명시적인 응답을 필요로 함을 의미합니다.

Lines~7 and~8 pick up current and snapshot versions of
\co{dynticks_progress_counter}, respectively.
The memory barrier on line~9 ensures that the counter checks
in the later \co{rcu_try_flip_waitzero_state()} follow
the fetches of these counters.
Lines~10 and~11 return zero (meaning no communication with the
specified CPU is required) if that CPU has remained in dynticks-idle
state since the time that the snapshot was taken.
Similarly, lines~12 and~13 return zero if that CPU was initially
in dynticks-idle state or if it has completely passed through a
dynticks-idle state.
In both these cases, there is no way that that CPU could have retained
the old value of the grace-period counter.
If neither of these conditions hold, line~14 returns one, meaning
that the CPU needs to explicitly respond.

\co{rcu_try_flip_waitmb_state()} 상태는 아래에 보인 것과 같은
\co{rcu_try_flip_waitmb_needed()} 를 호출합니다:
\iffalse

For its part, the \co{rcu_try_flip_waitmb_state()} state
invokes \co{rcu_try_flip_waitmb_needed()}, shown below:
\fi

{ \scriptsize
\begin{verbatim}
  1 static inline int
  2 rcu_try_flip_waitmb_needed(int cpu)
  3 {
  4   long curr;
  5   long snap;
  6
  7   curr = per_cpu(dynticks_progress_counter, cpu);
  8   snap = per_cpu(rcu_dyntick_snapshot, cpu);
  9   smp_mb();
 10   if ((curr == snap) && ((curr & 0x1) == 0))
 11     return 0;
 12   if (curr != snap)
 13     return 0;
 14   return 1;
 15 }
\end{verbatim}
}

이는 \co{rcu_try_flip_waitack_needed()} 와 유사한데, 차이점은 line~12 와 13
뿐으로, dynticks-idle 상태로부터 또는 dynticks-idle 상태로의 상태 전환은 모두
\co{rcu_try_flip_waitmb_state()} 상태에 의해 필요한 메모리 배리어를 실행하기
때문입니다.

이제 우리는 RCU 와 dynticks-idle 상태 사이의 인터페이스에 관여된 모든 코드를
봤습니다.
다음 섹션은 이 코드를 검증하기 위핸 Promela 모델을 만들어 봅니다.
\iffalse

This is quite similar to \co{rcu_try_flip_waitack_needed()},
the difference being in lines~12 and~13, because any transition
either to or from dynticks-idle state executes the memory barrier
needed by the \co{rcu_try_flip_waitmb_state()} state.

We now have seen all the code involved in the interface between
RCU and the dynticks-idle state.
The next section builds up the Promela model used to verify this
code.
\fi

\QuickQuiz{}
	이 섹션에서 보인 모든 코드 가운데 버그들을 찾아냈나요?
	\iffalse

	Can you spot any bugs in any of the code in this section?
	\fi
\QuickQuizAnswer{
	당신이 맞았는지 보기 위해 다음 섹션을 읽어보세요.
	\iffalse

	Read the next section to see if you were correct.
	\fi
} \QuickQuizEnd

\subsection{Validating Preemptible RCU and dynticks}
\label{sec:formal:Validating Preemptible RCU and dynticks}

이 섹션은 dynticks 와 RCU 사이의 인터페이스를 위한 Promela 모델을 단계별로
개발해 보며, 뒤따르는 섹션들 각각은 하나의 단계씩을 설명하는데, 프로세스
단계부터 시작해서, 단정문, 인터럽트, 그리고 마지막으로 NMI 를 추가해 봅니다.
\iffalse

This section develops a Promela model for the interface between
dynticks and RCU step by step, with each of the following sections
illustrating one step, starting with the process-level code,
adding assertions, interrupts, and finally NMIs.
\fi

\subsubsection{Basic Model}
\label{sec:formal:Basic Model}

이 섹션은 프로세스 단계 dynticks 진입/종료 코드와 grace-period 처리를 Promela
로 변환해 봅니다~\cite{Holzmann03a}.
우리는 2.6.25-rc4 커널에서의 \co{rcu_exit_nohz()} 와 \co{rcu_enter_nohz()} 로
시작해서, 이것들을 dynticks-idle 모드를 들어가고 빠져나오는 과정을 모델링하는
Promela 프로세스를 다음의 루프와 같이 만들 겁니다:
\iffalse

This section translates the process-level dynticks entry/exit
code and the grace-period processing into
Promela~\cite{Holzmann03a}.
We start with \co{rcu_exit_nohz()} and
\co{rcu_enter_nohz()}
from the 2.6.25-rc4 kernel, placing these in a single Promela
process that models exiting and entering dynticks-idle mode in
a loop as follows:
\fi

{ \scriptsize
\begin{verbatim}
  1 proctype dyntick_nohz()
  2 {
  3   byte tmp;
  4   byte i = 0;
  5
  6   do
  7   :: i >= MAX_DYNTICK_LOOP_NOHZ -> break;
  8   :: i < MAX_DYNTICK_LOOP_NOHZ ->
  9     tmp = dynticks_progress_counter;
 10     atomic {
 11       dynticks_progress_counter = tmp + 1;
 12       assert((dynticks_progress_counter & 1) == 1);
 13     }
 14     tmp = dynticks_progress_counter;
 15     atomic {
 16       dynticks_progress_counter = tmp + 1;
 17       assert((dynticks_progress_counter & 1) == 0);
 18     }
 19     i++;
 20   od;
 21 }
\end{verbatim}
}

Line~6 과 20 은 루프를 정의합니다.
Line~7 은 \co{i} 의 값이 \co{MAX_DYNTICK_LOOP_NOHZ} 를 넘어서면 루프를
빠져나가게 합니다.
Line~8 은 루프의 각 패스가 line~9-19 를 수행하도록 구성합니다.
Line~7 과 8 의 조건들은 서로에게 배타적이기 때문에, 일반적인 Promela 의 옳은
조건에 대한 무작위 선택은 무효화 됩니다.
Line~9 와 11 은 \co{rcu_exit_nohz()} 의 어토믹하지 않은
\co{dynticks_progress_counter} 의 값 증가를 모델링 하며, line~12 는
\co{WARN_ON()} 을 모델링 합니다.
여기서의 \co{atomic}  은 \co{WARN_ON()} 이 엄밀히 말해선 알고리즘의 한 부분이
아닌 만큼 단순히 Promela 상태 공간의 크기를 줄여주는 역할을 합니다.
Line~14-18 은 비슷하게 \co{rcu_enter_nohz()} 의 값 증가와 \co{WARN_ON()} 을
모델링 합니다.
마지막으로, line~19 는 루프 카운터의 값을 증가시킵니다.
\iffalse

Lines~6 and~20 define a loop.
Line~7 exits the loop once the loop counter \co{i}
has exceeded the limit \co{MAX_DYNTICK_LOOP_NOHZ}.
Line~8 tells the loop construct to execute lines~9-19
for each pass through the loop.
Because the conditionals on lines~7 and~8 are exclusive of
each other, the normal Promela random selection of true conditions
is disabled.
Lines~9 and~11 model \co{rcu_exit_nohz()}'s non-atomic
increment of \co{dynticks_progress_counter}, while
line~12 models the \co{WARN_ON()}.
The \co{atomic} construct simply reduces the Promela state space,
given that the \co{WARN_ON()} is not strictly speaking part
of the algorithm.
Lines~14-18 similarly models the increment and
\co{WARN_ON()} for \co{rcu_enter_nohz()}.
Finally, line~19 increments the loop counter.
\fi

따라서 루프의 각 패스는 (예를 들어, 태스크를 시작하려고 해서) dynticks-idle
모드를 빠져나가는, 그리고 나서 다시 (예를 들어, 그 태스크가 블락 당해서)
dynticks-idle 모드로 들어가는 CPU 를 모델링합니다.
\iffalse

Each pass through the loop therefore models a CPU exiting
dynticks-idle mode (for example, starting to execute a task), then
re-entering dynticks-idle mode (for example, that same task blocking).
\fi

\QuickQuiz{}
	왜 \co{rcu_exit_nohz()} 와 \co{rcu_enter_nohz()}  사이의 메모리
	배리어는 Promela 에 모델링 되지 않은거죠?
	\iffalse

	Why isn't the memory barrier in \co{rcu_exit_nohz()}
	and \co{rcu_enter_nohz()} modeled in Promela?
	\fi
\QuickQuizAnswer{
	Promela 는 sequential consistency 를 가정하므로, 메모리 배리어를
	모델링할 필요가 없습니다.
	사실, 그대신
	page~\pageref{fig:analysis:QRCU Unordered Summation} 의
	Figure~\ref{fig:analysis:QRCU Unordered Summation}
	처럼 명시적으로 메모리 배리어의 부재를 모델링 해야 합니다.
	\iffalse

	Promela assumes sequential consistency, so
	it is not necessary to model memory barriers.
	In fact, one must instead explicitly model lack of memory barriers,
	for example, as shown in
	Figure~\ref{fig:analysis:QRCU Unordered Summation} on
	page~\pageref{fig:analysis:QRCU Unordered Summation}.
	\fi
} \QuickQuizEnd

\QuickQuiz{}
	\co{rcu_exit_nohz()} 에 이어서 \co{rcu_enter_nohz()} 가 뒤따르는 경우를
	모델링 하는건 좀 이상하지 않나요?
	그보다는 진입 후에 빠져나가는 상황을 모델링하는게 더 자연스럽지
	않을까요?
	\iffalse

	Isn't it a bit strange to model \co{rcu_exit_nohz()}
	followed by \co{rcu_enter_nohz()}?
	Wouldn't it be more natural to instead model entry before exit?
	\fi
\QuickQuizAnswer{
	그게 더 자연스러울 수 있습니다만, 우린 우리가 뒤에서 추가할 liveness
	체크를 위해 이 특정한 순서가 필요합니다.
	\iffalse

	It probably would be more natural, but we will need
	this particular order for the liveness checks that we will add later.
	\fi
} \QuickQuizEnd

다음 단계는 RCU 의 grace-period 처리를 위한 인터페이스의 모델링입니다.
이를 위해, 우리는 2.6.25-rc4 커널의
\co{dyntick_save_progress_counter()},
\co{rcu_try_flip_waitack_needed()},
\co{rcu_try_flip_waitmb_needed()},
뿐만 아니라
\co{rcu_try_flip_waitack()} 와
\co{rcu_try_flip_waitmb()} 를 모델링 해야 합니다.
다음의 \co{grace_period()} Promela 프로세스는 이 함수들을 그것들이 하나의
preemption 가능한 RCU 의 grace-period 처리 과정 사이에서 호출될 것처럼 모델링
합니다.
\iffalse

The next step is to model the interface to RCU's grace-period
processing.
For this, we need to model
\co{dyntick_save_progress_counter()},
\co{rcu_try_flip_waitack_needed()},
\co{rcu_try_flip_waitmb_needed()},
as well as portions of
\co{rcu_try_flip_waitack()} and
\co{rcu_try_flip_waitmb()}, all from the 2.6.25-rc4 kernel.
The following \co{grace_period()} Promela process models
these functions as they would be invoked during a single pass
through preemptible RCU's grace-period processing.
\fi

{ \scriptsize
\begin{verbatim}
  1 proctype grace_period()
  2 {
  3   byte curr;
  4   byte snap;
  5
  6   atomic {
  7     printf("MDLN = %d\n", MAX_DYNTICK_LOOP_NOHZ);
  8     snap = dynticks_progress_counter;
  9   }
 10   do
 11   :: 1 ->
 12     atomic {
 13       curr = dynticks_progress_counter;
 14       if
 15       :: (curr == snap) && ((curr & 1) == 0) ->
 16         break;
 17       :: (curr - snap) > 2 || (snap & 1) == 0 ->
 18         break;
 19       :: 1 -> skip;
 20       fi;
 21     }
 22   od;
 23   snap = dynticks_progress_counter;
 24   do
 25   :: 1 ->
 26     atomic {
 27       curr = dynticks_progress_counter;
 28       if
 29       :: (curr == snap) && ((curr & 1) == 0) ->
 30         break;
 31       :: (curr != snap) ->
 32         break;
 33       :: 1 -> skip;
 34       fi;
 35     }
 36   od;
 37 }
\end{verbatim}
}

Line~6-9 은 루프 리미트를 (에러가 났을 경우에 한해 .trail 파일에) 프린트하고
현재 CPU 의 \co{dynticks_progress_counter} 변수의 스탭샷을 얻어오는,
\co{rcu_try_flip_idle()} 의 코드 한줄과 \co{dyntick_save_progress_counter()}
호출을 모델링 합니다.
이 두 라인들은 상태 공간의 크기를 줄이기 위해 어토믹하게 수행됩니다.

Line~10-22 는 \co{rcu_try_flip_waitack()} 그 함수의
\co{rcu_try_flip_waitack_needed()} 함수 호출을 모델링합니다.
이 루프는 각 CPU 로부터 카운터 뒤집기 응답을 기다리는 grace-period 상태 머신을,
하지만 그 중에서도 dynticks-idle CPU 들과 상호작용하는 부분만을 모델링 합니다.
\iffalse

Lines~6-9 print out the loop limit (but only into the .trail file
in case of error) and models a line of code
from \co{rcu_try_flip_idle()} and its call to
\co{dyntick_save_progress_counter()}, which takes a
snapshot of the current CPU's \co{dynticks_progress_counter}
variable.
These two lines are executed atomically to reduce state space.

Lines~10-22 model the relevant code in
\co{rcu_try_flip_waitack()} and its call to
\co{rcu_try_flip_waitack_needed()}.
This loop is modeling the grace-period state machine waiting for
a counter-flip acknowledgement from each CPU, but only that part
that interacts with dynticks-idle CPUs.
\fi

Line~23 은 또다시 CPU 의 \co{dynticks_progress_counter} 변수의 스탭샷을
얻어오는 \co{rcu_try_flip_waitzero()} 의 코드 중 한줄과 그것의
\co{dyntick_save_progress_counter()} 함수 호출을 모델링 합니다.

마지막으로, line~24-36 은 \co{rcu_try_flip_waitack()} 과 그 함수의
\co{rcu_try_flip_waitack_needed()} 호출을 모델링합니다.
이 루프는 각각의 CPU 가 메모리 배리어를 실행하기를 기다리는, 하지만 여기서도
역시 dynticks-idle CPU 들과 상호작용하는 부분만을 모델링 합니다.
\iffalse

Line~23 models a line from \co{rcu_try_flip_waitzero()}
and its call to \co{dyntick_save_progress_counter()}, again
taking a snapshot of the CPU's \co{dynticks_progress_counter}
variable.

Finally, lines~24-36 model the relevant code in
\co{rcu_try_flip_waitack()} and its call to
\co{rcu_try_flip_waitack_needed()}.
This loop is modeling the grace-period state-machine waiting for
each CPU to execute a memory barrier, but again only that part
that interacts with dynticks-idle CPUs.
\fi

\QuickQuiz{}
	잠깐만요!
	이 리눅스 커널에서, \co{dynticks_progress_counter} 와
	\co{rcu_dyntick_snapshot} 은 per-CPU 변수들입니다.
	그런데 왜 그것들이 per-CPU 변수들이 아니라 하나의 글로벌 변수로 모델링
	된거죠?
	\iffalse

	Wait a minute!
	In the Linux kernel, both \co{dynticks_progress_counter} and
	\co{rcu_dyntick_snapshot} are per-CPU variables.
	So why are they instead being modeled as single global variables?
	\fi
\QuickQuizAnswer{
	해당 grace-period 코드는 각각의 CPU 의 \co{dynticks_progress_counter}
	와 \co{rcu_dyntick_snapshot} 을 분리해서 처리하기 때문에, 우린 이
	상태를 하나의 CPU 로 뭉칠 수 있습니다.
	만약 그 grace-period 코드가 특정 CPU 의 특정 값을 가지고 뭔ㄱ 타그별한
	일을 하려 했다면, 우린 정말로 여러개의 CPU 들을 모델링 해야 할겁니다.
	하지만 다행히도, 우린 하나의 CPU 는 grace-period 처리를 수행하고 있고
	다른 하나는 dynticks-idle 모드를 들어가고 빠져나오는 두개의 CPU 들에만
	우리를 안전히 국한시킬 수 있습니다.
	\iffalse

	Because the grace-period code processes each
	CPU's \co{dynticks_progress_counter} and
	\co{rcu_dyntick_snapshot} variables separately,
	we can collapse the state onto a single CPU.
	If the grace-period code were instead to do something special
	given specific values on specific CPUs, then we would indeed need
	to model multiple CPUs.
	But fortunately, we can safely confine ourselves to two CPUs, the
	one running the grace-period processing and the one entering and
	leaving dynticks-idle mode.
	\fi
} \QuickQuizEnd

최종적인 모델 (\path{dyntickRCU-base.spin}) 은 \path{runspin.sh} 스크립트를
통해 돌아갈 때, 691 개의 상태를 생성하고 에러 없이 수행되는데, 이것은 실패를
찾을 수 있는 단정문이 존재하지 않는다는 점을 생각해 보면 놀라울 일이 아닙니다.
따라서 다음 섹션에서는 안전성을 위한 단정문들을 추가해 봅니다.
\iffalse

The resulting model (\path{dyntickRCU-base.spin}),
when run with the
\path{runspin.sh} script,
generates 691 states and
passes without errors, which is not at all surprising given that
it completely lacks the assertions that could find failures.
The next section therefore adds safety assertions.
\fi

\subsubsection{Validating Safety}
\label{sec:formal:Validating Safety}

안전한 RCU 구현에서 grace period 는 그 grace period 의 시작 전에 시작된 모든
RCU 읽기 쓰레드들이 완료되기 전에는 절대 완료될 수 없어야 합니다.
이는 다음과 같은 세개의 상태를 취할 수 있는 \co{gp_state} 변수를 사용해 모델링
될 수 있습니다:
\iffalse

A safe RCU implementation must never permit a grace period to
complete before the completion of any RCU readers that started
before the start of the grace period.
This is modeled by a \co{gp_state} variable that
can take on three states as follows:
\fi

\vspace{5pt}
\begin{minipage}[t]{\columnwidth}
\begin{verbatim}
  1 #define GP_IDLE    0
  2 #define GP_WAITING  1
  3 #define GP_DONE    2
  4 byte gp_state = GP_DONE;
\end{verbatim}
\end{minipage}
\vspace{5pt}

\co{grace_period()} 프로세스는 이 변수를 아래에 보여진 것과 같이 grace-period
단계를 통해 진행되는 것처럼 설정합니다:
\iffalse

The \co{grace_period()} process sets this variable as it
progresses through the grace-period phases, as shown below:
\fi

{ \scriptsize
\begin{verbatim}
  1 proctype grace_period()
  2 {
  3   byte curr;
  4   byte snap;
  5
  6   gp_state = GP_IDLE;
  7   atomic {
  8     printf("MDLN = %d\n", MAX_DYNTICK_LOOP_NOHZ);
  9     snap = dynticks_progress_counter;
 10     gp_state = GP_WAITING;
 11   }
 12   do
 13   :: 1 ->
 14     atomic {
 15       curr = dynticks_progress_counter;
 16       if
 17       :: (curr == snap) && ((curr & 1) == 0) ->
 18         break;
 19       :: (curr - snap) > 2 || (snap & 1) == 0 ->
 20         break;
 21       :: 1 -> skip;
 22       fi;
 23     }
 24   od;
 25   gp_state = GP_DONE;
 26   gp_state = GP_IDLE;
 27   atomic {
 28     snap = dynticks_progress_counter;
 29     gp_state = GP_WAITING;
 30   }
 31   do
 32   :: 1 ->
 33     atomic {
 34       curr = dynticks_progress_counter;
 35       if
 36       :: (curr == snap) && ((curr & 1) == 0) ->
 37         break;
 38       :: (curr != snap) ->
 39         break;
 40       :: 1 -> skip;
 41       fi;
 42     }
 43   od;
 44   gp_state = GP_DONE;
 45 }
\end{verbatim}
}

Line~6, 10, 25, 26, 29, 그리고 44 는 기본적인 RCU 안전성을 검증하기 위해
\co{dyntick_nohz()} 프로세스를 허용할 수 있도록 이 변수를 (적당한 알고리즘적
오퍼레이션들을 어토믹하게 섞어서) 업데이트 합니다.
이 검증의 형태는 RCU 읽기 쓰레드들이 그럴듯하게 존재하는 시간동안 \co{gp_state}
변수의 값이 \co{GP_IDLE} 에서 \co{GP_DONE} 으로바뀔 수 없음을 단정하려는
것입니다.
\iffalse

Lines~6, 10, 25, 26, 29, and~44 update this variable (combining
atomically with algorithmic operations where feasible) to
allow the \co{dyntick_nohz()} process to verify the basic
RCU safety property.
The form of this verification is to assert that the value of the
\co{gp_state} variable cannot jump from
\co{GP_IDLE} to \co{GP_DONE} during a time period
over which RCU readers could plausibly persist.
\fi

\QuickQuiz{}
	Line~25 와 26 에 \co{gp_state} 의 연속된 두개의 변경이 있는데, 어떻게
	line~25 에서의 변경이 사라지지 않을거라고 확신할 수 있을까요?
	\iffalse

	Given there are a pair of back-to-back changes to
	\co{gp_state} on lines~25 and~26,
	how can we be sure that line~25's changes won't be lost?
	\fi
\QuickQuizAnswer{
	Promela 와 spin 이 모든 가능한 상태 변경의 시퀀스를 추적한다는 점을
	다시 상기하시기 바랍니다.
	따라서, 타이밍은 무의미합니다: Promela/spin 은 어떤 상태 변수가
	명시적으로 방해를 하지 않는 한은 그 두개의 상태들 사이의 모든 모델을
	적당히 섞어낼 겁니다.
	\iffalse

	Recall that Promela and spin trace out
	every possible sequence of state changes.
	Therefore, timing is irrelevant: Promela/spin will be quite
	happy to jam the entire rest of the model between those two
	statements unless some state variable specifically prohibits
	doing so.
	\fi
} \QuickQuizEnd

\co{dyntick_nohz()} Promela 프로세스는 이 검증을 아래와 같이 구현합니다:
\iffalse

The \co{dyntick_nohz()} Promela process implements
this verification as shown below:
\fi

{ \scriptsize
\begin{verbatim}
  1 proctype dyntick_nohz()
  2 {
  3   byte tmp;
  4   byte i = 0;
  5   bit old_gp_idle;
  6
  7   do
  8   :: i >= MAX_DYNTICK_LOOP_NOHZ -> break;
  9   :: i < MAX_DYNTICK_LOOP_NOHZ ->
 10     tmp = dynticks_progress_counter;
 11     atomic {
 12       dynticks_progress_counter = tmp + 1;
 13       old_gp_idle = (gp_state == GP_IDLE);
 14       assert((dynticks_progress_counter & 1) == 1);
 15     }
 16     atomic {
 17       tmp = dynticks_progress_counter;
 18       assert(!old_gp_idle ||
 19              gp_state != GP_DONE);
 20     }
 21     atomic {
 22       dynticks_progress_counter = tmp + 1;
 23       assert((dynticks_progress_counter & 1) == 0);
 24     }
 25     i++;
 26   od;
 27 }
\end{verbatim}
}

Line~13 은 \co{gp_state} 변수의 값이 태스크 수행 시작 시점에서 \co{GP_IDLE}
이라면 \co{old_gp_idle} 플래그를 설정하고, \co{gp_state} 변수가 태스크 수행
중간에 \co{GP_DONE} 으로 바뀌었다면 하나의 RCU read-side 크리티컬 섹션이 전체
시간동안 존재할 수 있다는 점에서 비합법적이므로 line~18 과 19 에서의 단정문이
실패합니다.

결과적으로 만들어지는 모델 (\path{dyntickRCU-base-s.spin}) 은 \path{runspin.sh}
스크립트를 통해 수행될 때, 964 개의 상태들을 생성하고 안심되게도 에러없이
돌아갑니다.
그렇다곤 하나, 안전성이 상당히 중요하긴 하지만, 불명확하게 지연되는 grace
period 를 막는 것 역시 굉장히 중요합니다.
따라서 다음 섹션에서는 liveness 의 검증을 다룹니다.
\iffalse

Line~13 sets a new \co{old_gp_idle} flag if the
value of the \co{gp_state} variable is
\co{GP_IDLE} at the beginning of task execution,
and the assertion at lines~18 and~19 fire if the \co{gp_state}
variable has advanced to \co{GP_DONE} during task execution,
which would be illegal given that a single RCU read-side critical
section could span the entire intervening time period.

The resulting
model (\path{dyntickRCU-base-s.spin}),
when run with the \path{runspin.sh} script,
generates 964 states and passes without errors, which is reassuring.
That said, although safety is critically important, it is also quite
important to avoid indefinitely stalling grace periods.
The next section therefore covers verifying liveness.
\fi

\subsubsection{Validating Liveness}
\label{sec:formal:Validating Liveness}

Liveness 는 증명되기에 어려울 수 있지만, 여기 적용할 수 있는 간단한 트릭이
있습니다.
첫번째 단계는 다음의 line~27 에서 보이는 것과 같이 \co{dyntick_nohz()} 가
\co{dyntick_nohz_done} 을 통해 할일을 마쳤음을 알리는 것입니다:
\iffalse

Although liveness can be difficult to prove, there is a simple
trick that applies here.
The first step is to make \co{dyntick_nohz()} indicate that
it is done via a \co{dyntick_nohz_done} variable, as shown on
line~27 of the following:
\fi

{ \scriptsize
\begin{verbatim}
  1 proctype dyntick_nohz()
  2 {
  3   byte tmp;
  4   byte i = 0;
  5   bit old_gp_idle;
  6
  7   do
  8   :: i >= MAX_DYNTICK_LOOP_NOHZ -> break;
  9   :: i < MAX_DYNTICK_LOOP_NOHZ ->
 10     tmp = dynticks_progress_counter;
 11     atomic {
 12       dynticks_progress_counter = tmp + 1;
 13       old_gp_idle = (gp_state == GP_IDLE);
 14       assert((dynticks_progress_counter & 1) == 1);
 15     }
 16     atomic {
 17       tmp = dynticks_progress_counter;
 18       assert(!old_gp_idle ||
 19              gp_state != GP_DONE);
 20     }
 21     atomic {
 22       dynticks_progress_counter = tmp + 1;
 23       assert((dynticks_progress_counter & 1) == 0);
 24     }
 25     i++;
 26   od;
 27   dyntick_nohz_done = 1;
 28 }
\end{verbatim}
}

이 변수를 두면, 불필요한 블록들을 체크하기 위해 다음과 같이 \co{grace_period()}
에 단정문들을 추가할 수 있습니다:
\iffalse

With this variable in place, we can add assertions to
\co{grace_period()} to check for unnecessary blockage
as follows:
\fi

{ \scriptsize
\begin{verbatim}
  1 proctype grace_period()
  2 {
  3   byte curr;
  4   byte snap;
  5   bit shouldexit;
  6
  7   gp_state = GP_IDLE;
  8   atomic {
  9     printf("MDLN = %d\n", MAX_DYNTICK_LOOP_NOHZ);
 10     shouldexit = 0;
 11     snap = dynticks_progress_counter;
 12     gp_state = GP_WAITING;
 13   }
 14   do
 15   :: 1 ->
 16     atomic {
 17       assert(!shouldexit);
 18       shouldexit = dyntick_nohz_done;
 19       curr = dynticks_progress_counter;
 20       if
 21       :: (curr == snap) && ((curr & 1) == 0) ->
 22         break;
 23       :: (curr - snap) > 2 || (snap & 1) == 0 ->
 24         break;
 25       :: else -> skip;
 26       fi;
 27     }
 28   od;
 29   gp_state = GP_DONE;
 30   gp_state = GP_IDLE;
 31   atomic {
 32     shouldexit = 0;
 33     snap = dynticks_progress_counter;
 34     gp_state = GP_WAITING;
 35   }
 36   do
 37   :: 1 ->
 38     atomic {
 39       assert(!shouldexit);
 40       shouldexit = dyntick_nohz_done;
 41       curr = dynticks_progress_counter;
 42       if
 43       :: (curr == snap) && ((curr & 1) == 0) ->
 44         break;
 45       :: (curr != snap) ->
 46         break;
 47       :: else -> skip;
 48       fi;
 49     }
 50   od;
 51   gp_state = GP_DONE;
 52 }
\end{verbatim}
}

우리는 line~10 에서 0으로 초기화되는 \co{shouldexit} 변수를 line~5 에서
추가했습니다.
Line~17 은 \co{shouldexit} 가 아직 0일 것을 단정하며, line~18 에서는
\co{shouldexit} 를 \co{dyntick_nohz()} 에 의해 관리되는 \co{dyntick_nohz_done}
의 값으로 설정합니다.
따라서 이 단정은 \co{dyntick_nohz()} 가 수행 완료된 후의 카운터가 뒤집히는
응답을 기다리는 과정에서 하나 이상의 패스를 시도하게 되면 실패할 것입니다.
무엇보다, \co{dyntick_nohz()} 가 완료되었다면, 더이상 우리를 해당 루프의
바깥으로 나가게 할 상태 변화는 더이상 없고, 따라서 이 상태를 두번 이상 수행하게
됨은 곧 무한루프를 의미해서, grace period 가 끝나지 않을 것임을 의미하게
됩니다.

Line~32, 39, 그리고 40 은 두번째 (메모리 배리어) 루프를 위해 비슷한 방식으로
수행됩니다.
\iffalse

We have added the \co{shouldexit} variable on line~5,
which we initialize to zero on line~10.
Line~17 asserts that \co{shouldexit} is not set, while
line~18 sets \co{shouldexit} to the \co{dyntick_nohz_done}
variable maintained by \co{dyntick_nohz()}.
This assertion will therefore trigger if we attempt to take more than
one pass through the wait-for-counter-flip-acknowledgement
loop after \co{dyntick_nohz()} has completed
execution.
After all, if \co{dyntick_nohz()} is done, then there cannot be
any more state changes to force us out of the loop, so going through twice
in this state means an infinite loop, which in turn means no end to the
grace period.

Lines~32, 39, and~40 operate in a similar manner for the
second (memory-barrier) loop.
\fi

하지만, 이 모델 (\path{dyntickRCU-base-sl-busted.spin}) 을 수행하는 것은 실패로
끝나게 되는데, line`23 이 잘못된 변수는 짝수라고 체크를 하기 때문입니다.
실패하게 되면, \co{spin} 은 ``trail'' 파일
(\path{dyntickRCU-base-sl-busted.spin.trail}) 을 쓰게 되는데, 이 파일은 실패로
이르게 된 상태들의 시퀀스를 기록합니다.
\co{spin} 이 이 상태들의 시퀀스를 재추적하게 하고 실행된 statement 들과
변수들의 값
(\path{dyntickRCU-base-sl-busted.spin.trail.txt})
을 프린트 하려면 {\tt spin -t -p -g -l dyntickRCU-base-sl-busted.spin} 커맨드를
사용하세요.
Spin 은 두 함수를 모두 하나의 파일에서 취할 것이기 때문에 위에 리스트된 코드와
라인 넘버가 맞지 않을 수 있음을 알아 두세요.
하지만, 라인 넘버들은 전체 모델 (\path{dyntickRCU-base-sl-busted.spin} 과는
\emph{맞아떨어질} 겁니다.
\iffalse

However, running this
model (\path{dyntickRCU-base-sl-busted.spin})
results in failure, as line~23 is checking that the wrong variable
is even.
Upon failure, \co{spin} writes out a
``trail'' file
(\path{dyntickRCU-base-sl-busted.spin.trail})
, which records the sequence of states that lead to the failure.
Use the {\tt spin -t -p -g -l dyntickRCU-base-sl-busted.spin}
command to cause \co{spin} to retrace this sequence of states,
printing the statements executed and the values of variables
(\path{dyntickRCU-base-sl-busted.spin.trail.txt}).
Note that the line numbers do not match the listing above due to
the fact that spin takes both functions in a single file.
However, the line numbers \emph{do} match the full
model (\path{dyntickRCU-base-sl-busted.spin}).
\fi

\co{dyntick_nohz()} 프로세스가 step 34 에서 (``34:'' 를 검색해 보세요)
완료되었지만, \co{grace_period()} 프로세스는 루프를 빠져나가는데 실패했음을 볼
수 있습니다.
\co{curr} 의 값은 \co{6} 이고 (step 35 를 보세요) \co{snap} 의 값은 \co{5}
(step 17 을 보세요) 입니다.
따라서 line~21 에서의 첫번째 조건은 \co{curr != snap} 이므로 성립되지 않고,
line~23 에서의 두번째 조건은 \co{snap} 이 홀수이고 \co{curr} 는 \co{snap} 보다
1 만큼 클 뿐이기 때문에 성립되지 않습니다.

따라서 이 두개의 조건들 중 하나는 올바르지 않습니다.
\co{rcu_try_flip_waitack_needed()} 의 첫번째 조건에 대한 코멘트를 따르면:
\iffalse

We see that the \co{dyntick_nohz()} process completed
at step 34 (search for ``34:''), but that the
\co{grace_period()} process nonetheless failed to exit the loop.
The value of \co{curr} is \co{6} (see step 35)
and that the value of \co{snap} is \co{5} (see step 17).
Therefore the first condition on line~21 above does not hold because
\co{curr != snap}, and the second condition on line~23
does not hold either because \co{snap} is odd and because
\co{curr} is only one greater than \co{snap}.

So one of these two conditions has to be incorrect.
Referring to the comment block in \co{rcu_try_flip_waitack_needed()}
for the first condition:
\fi

\begin{quote}
	CPU 가 전체 시간동안 dynticks 모드에 빠져있고 인터럽트, NMI, SNMI, 또는
	뭐든 받지 않았다면, 해당 CPU 는 \co{rcu_read_lock()} 의 중간에 있을 수
	없고, 따라서 그것이 수행하게 되는 다음의 \co{rcu_read_lock()} 는
	카운터의 새로운 값을 사용해야만 한다.  따라서 우리는 안전하게 이 CPU 가
	이미 그 카운터에 응답을 한 것처럼 행동을 한다.
	\iffalse

	If the CPU remained in dynticks mode for the entire time
	and didn't take any interrupts, NMIs, SMIs, or whatever,
	then it cannot be in the middle of an \co{rcu_read_lock()}, so
	the next \co{rcu_read_lock()} it executes must use the new value
	of the counter.  So we can safely pretend that this CPU
	already acknowledged the counter.
	\fi
\end{quote}

첫번째 조건은 실제로 이와 들어맞는데, \co{curr == snap} 이고 \co{curr} 가
짝수라면, 연관된 CPU 는 요구된 대로 dynticks-idle 모드에 전체 시간동안 존재한
것이기 때문입니다.
따라서 두번째 조건에 대한 코멘트를 보도록 합시다:
\iffalse

The first condition does match this, because if \co{curr == snap}
and if \co{curr} is even, then the corresponding CPU has been
in dynticks-idle mode the entire time, as required.
So let's look at the comment block for the second condition:
\fi

\begin{quote}
	CPU 가 dynticks idle 단계를 활성화된 irq 핸들러 없이 거쳐왔거나
	진입했다면, 앞에서와 같이, 우린 이 CPU 가 카운터에 응답을 이미 한것처럼
	안전하게 행동할 수 있다.
	\iffalse

	If the CPU passed through or entered a dynticks idle phase with
	no active irq handlers, then, as above, we can safely pretend
	that this CPU already acknowledged the counter.
	\fi
\end{quote}

조건의 첫번째 부분은 올바른데, \co{curr} 와 \co{snap} 이 2만큼 차이가 나면 그
사이에 최소 하나의 짝수가 존재할 수 있어서, 하나의 dynticks-idle 단계를 완전히
지나왔다고 볼 수 있기 때문입니다.
하지만, 조건의 두번째 부분은 dynticks-idle 모드가 \emph{시작되었}지만, 이
모드가 아직 \emph{끝나지} 않았음을 의미합니다.
따라서 우리는 짝수가 되어야 할 겂으로 \co{snap} 이 아니라 \co{curr} 를 테스트
해야합니다.

고쳐진 C 코드는 다음과 같습니다:
\iffalse

The first part of the condition is correct, because if \co{curr}
and \co{snap} differ by two, there will be at least one even
number in between, corresponding to having passed completely through
a dynticks-idle phase.
However, the second part of the condition corresponds to having
\emph{started} in dynticks-idle mode, not having \emph{finished}
in this mode.
We therefore need to be testing \co{curr} rather than
\co{snap} for being an even number.

The corrected C code is as follows:
\fi

{ \scriptsize
\begin{verbatim}
  1 static inline int
  2 rcu_try_flip_waitack_needed(int cpu)
  3 {
  4   long curr;
  5   long snap;
  6
  7   curr = per_cpu(dynticks_progress_counter, cpu);
  8   snap = per_cpu(rcu_dyntick_snapshot, cpu);
  9   smp_mb();
 10   if ((curr == snap) && ((curr & 0x1) == 0))
 11     return 0;
 12   if ((curr - snap) > 2 || (curr & 0x1) == 0)
 13     return 0;
 14   return 1;
 15 }
\end{verbatim}
}

Line~10-13 은 이제 다음과 같이 간단하게 결합될 수 있습니다.
비슷한 단순화가 \co{rcu_try_flip_waitmb_needed()} 에 적용될 수 있습니다.
\iffalse

Lines~10-13 can now be combined and simplified,
resulting in the following.
A similar simplification can be applied to
\co{rcu_try_flip_waitmb_needed()}.
\fi

{ \scriptsize
\begin{verbatim}
  1 static inline int
  2 rcu_try_flip_waitack_needed(int cpu)
  3 {
  4   long curr;
  5   long snap;
  6
  7   curr = per_cpu(dynticks_progress_counter, cpu);
  8   snap = per_cpu(rcu_dyntick_snapshot, cpu);
  9   smp_mb();
 10   if ((curr - snap) >= 2 || (curr & 0x1) == 0)
 11     return 0;
 12   return 1;
 13 }
\end{verbatim}
}

해당 모델에 연관된 수정을 가하면 (\path{dyntickRCU-base-sl.spin}) 에러 없이
통과되는 661 개의 상태들을 통한 올바른 검증을 얻을 수 있습니다.
하지만, 첫번째 버전의 liveness 검증은 liveness 검증 자체의 버그로 인해서, 이
버그를 잡는데 실패했음은 알려둘 가치가 있습니다.
이 liveness 증명 버그는 \co{grace_period()} 프로세스 안에 무한루프를
넣어둠으로써 발생했습니다!

우린 이제 안전성과 liveness 조건을 모두 성공적으로 검증했습니다만, 동작하고
블락킹 되는 프로세스들에 대해서만 그렇습니다.
인터럽트들 역시 처리해야 할텐데, 이는 다음 섹션에서 처리하도록 하겠습니다.
\iffalse

Making the corresponding correction in the
model (\path{dyntickRCU-base-sl.spin})
results in a correct verification with 661 states that passes without
errors.
However, it is worth noting that the first version of the liveness
verification failed to catch this bug, due to a bug in the liveness
verification itself.
This liveness-verification bug was located by inserting an infinite
loop in the \co{grace_period()} process, and noting that
the liveness-verification code failed to detect this problem!

We have now successfully verified both safety and liveness
conditions, but only for processes running and blocking.
We also need to handle interrupts, a task taken up in the next section.
\fi

\subsubsection{Interrupts}
\label{sec:formal:Interrupts}

Promela 에서 인터럽트를 모델링하는 두가지 방법이 있습니다:
\iffalse

There are a couple of ways to model interrupts in Promela:
\fi
\begin{enumerate}
\item	C 전처리기 트릭을 사용해서 모든 \co{dynticks_nohz()} 프로세스의
	statement 사이에 인터럽트 핸들러를 넣거나
\item	인터럽트 핸들러를 별도의 프로세스로 모델링 하는 겁니다.
\iffalse

\item	using C-preprocessor tricks to insert the interrupt handler
	between each and every statement of the \co{dynticks_nohz()}
	process, or
\item	modeling the interrupt handler with a separate process.
\fi
\end{enumerate}

두번째 방법이 더 작은 상태 공간을 갖게 해줄거라 생각할 수 있습니다만, 이는
인터럽트 핸들러가 어떻게든 \co{dynticks_nohz()} 와는 어토믹하게 동작해야 하지만
\co{grace_period()} 와는 어토믹하지 않게 동작해야 할 것을 필요로 합니다.

다행히도, Promela 는 어토믹 statement 들을 분기를 가를 수 있게 해줍니다.
이 트릭은 우리가 인터럽트 핸들러에서 플래그를 설정하고, \co{dynticks_nohz()} 가
어토믹하게 이 플래그를 체크한 후 그 플래그가 설정되지 않았을 때에만 실행되도록
코드를 수정할 수 있게 합니다.
이는 다음과 같이 라벨과 Promela statement 들을 사용하는 C 전처리기 매크로를
사용해 이뤄질 수 있습니다:
\iffalse

A bit of thought indicated that the second approach would have a
smaller state space, though it requires that the interrupt handler
somehow run atomically with respect to the \co{dynticks_nohz()}
process, but not with respect to the \co{grace_period()}
process.

Fortunately, it turns out that Promela permits you to branch
out of atomic statements.
This trick allows us to have the interrupt handler set a flag, and
recode \co{dynticks_nohz()} to atomically check this flag
and execute only when the flag is not set.
This can be accomplished with a C-preprocessor macro that takes
a label and a Promela statement as follows:
\fi

{ \scriptsize
\begin{verbatim}
  1 #define EXECUTE_MAINLINE(label, stmt) \
  2 label: skip; \
  3     atomic { \
  4       if \
  5       :: in_dyntick_irq -> goto label; \
  6       :: else -> stmt; \
  7       fi; \
  8     } \
\end{verbatim}
}

이 매크로를 다음과 같이 사용할 수 있을 겁니다:
\iffalse

One might use this macro as follows:
\fi

\vspace{5pt}
\begin{minipage}[t]{\columnwidth}
\scriptsize
\begin{verbatim}
EXECUTE_MAINLINE(stmt1,
                 tmp = dynticks_progress_counter)
\end{verbatim}
\end{minipage}
\vspace{5pt}

매그로의 line~2 는 특정 statement 라벨을 만듭니다.
Line~3-8 은 \co{in_dyntick_irq} 변수를 테스트 하는 어토믹 블락으로, 이 변수가
값이 설정되어 있다면 (인터럽트 핸들러가 활성 상태임을 의미), 어토믹 블락의
수행을 라벨로 되돌립니다.
그렇지 않다면, line~6 에서 요청된 statement 를 수행합니다.
전체 효과는 요청된대로 인터럽트가 활성화 될 때마다 메인 수행이 공회전을 하게
되는 것입니다.
\iffalse

Line~2 of the macro creates the specified statement label.
Lines~3-8 are an atomic block that tests the \co{in_dyntick_irq}
variable, and if this variable is set (indicating that the interrupt
handler is active), branches out of the atomic block back to the
label.
Otherwise, line~6 executes the specified statement.
The overall effect is that mainline execution stalls any time an interrupt
is active, as required.
\fi

\subsubsection{Validating Interrupt Handlers}
\label{sec:formal:Validating Interrupt Handlers}

The first step is to convert \co{dyntick_nohz()} to
\co{EXECUTE_MAINLINE()} form, as follows:

{ \scriptsize
\begin{verbatim}
  1 proctype dyntick_nohz()
  2 {
  3   byte tmp;
  4   byte i = 0;
  5   bit old_gp_idle;
  6
  7   do
  8   :: i >= MAX_DYNTICK_LOOP_NOHZ -> break;
  9   :: i < MAX_DYNTICK_LOOP_NOHZ ->
 10     EXECUTE_MAINLINE(stmt1,
 11       tmp = dynticks_progress_counter)
 12     EXECUTE_MAINLINE(stmt2,
 13       dynticks_progress_counter = tmp + 1;
 14       old_gp_idle = (gp_state == GP_IDLE);
 15       assert((dynticks_progress_counter & 1) == 1))
 16     EXECUTE_MAINLINE(stmt3,
 17       tmp = dynticks_progress_counter;
 18       assert(!old_gp_idle ||
 19              gp_state != GP_DONE))
 20     EXECUTE_MAINLINE(stmt4,
 21       dynticks_progress_counter = tmp + 1;
 22       assert((dynticks_progress_counter & 1) == 0))
 23     i++;
 24   od;
 25   dyntick_nohz_done = 1;
 26 }
\end{verbatim}
}

It is important to note that when a group of statements is passed
to \co{EXECUTE_MAINLINE()}, as in lines~11-14, all
statements in that group execute atomically.

\QuickQuiz{}
	But what would you do if you needed the statements in a single
	\co{EXECUTE_MAINLINE()} group to execute non-atomically?
\QuickQuizAnswer{
	The easiest thing to do would be to put
	each such statement in its own \co{EXECUTE_MAINLINE()}
	statement.
} \QuickQuizEnd

\QuickQuiz{}
	But what if the \co{dynticks_nohz()} process had
	``if'' or ``do'' statements with conditions,
	where the statement bodies of these constructs
	needed to execute non-atomically?
\QuickQuizAnswer{
	One approach, as we will see in a later section,
	is to use explicit labels and ``goto'' statements.
	For example, the construct:

	\vspace{5pt}
	\begin{minipage}[t]{\columnwidth}
	\scriptsize
	\begin{verbatim}
		if
		:: i == 0 -> a = -1;
		:: else -> a = -2;
		fi;
	\end{verbatim}
	\end{minipage}
	\vspace{5pt}

	could be modeled as something like:

	\vspace{5pt}
	\begin{minipage}[t]{\columnwidth}
	\scriptsize
	\begin{verbatim}
		EXECUTE_MAINLINE(stmt1,
				 if
				 :: i == 0 -> goto stmt1_then;
				 :: else -> goto stmt1_else;
				 fi)
		stmt1_then: skip;
		EXECUTE_MAINLINE(stmt1_then1, a = -1; goto stmt1_end)
		stmt1_else: skip;
		EXECUTE_MAINLINE(stmt1_then1, a = -2)
		stmt1_end: skip;
	\end{verbatim}
	\end{minipage}
	\vspace{5pt}

	However, it is not clear that the macro is helping much in the case
	of the ``if'' statement, so these sorts of situations will
	be open-coded in the following sections.
} \QuickQuizEnd

The next step is to write a \co{dyntick_irq()} process
to model an interrupt handler:

{ \scriptsize
\begin{verbatim}
  1 proctype dyntick_irq()
  2 {
  3   byte tmp;
  4   byte i = 0;
  5   bit old_gp_idle;
  6
  7   do
  8   :: i >= MAX_DYNTICK_LOOP_IRQ -> break;
  9   :: i < MAX_DYNTICK_LOOP_IRQ ->
 10     in_dyntick_irq = 1;
 11     if
 12     :: rcu_update_flag > 0 ->
 13        tmp = rcu_update_flag;
 14       rcu_update_flag = tmp + 1;
 15     :: else -> skip;
 16     fi;
 17     if
 18     :: !in_interrupt &&
 19       (dynticks_progress_counter & 1) == 0 ->
 20       tmp = dynticks_progress_counter;
 21       dynticks_progress_counter = tmp + 1;
 22       tmp = rcu_update_flag;
 23       rcu_update_flag = tmp + 1;
 24     :: else -> skip;
 25     fi;
 26     tmp = in_interrupt;
 27     in_interrupt = tmp + 1;
 28     old_gp_idle = (gp_state == GP_IDLE);
 29     assert(!old_gp_idle || gp_state != GP_DONE);
 30     tmp = in_interrupt;
 31     in_interrupt = tmp - 1;
 32     if
 33     :: rcu_update_flag != 0 ->
 34       tmp = rcu_update_flag;
 35       rcu_update_flag = tmp - 1;
 36       if
 37       :: rcu_update_flag == 0 ->
 38         tmp = dynticks_progress_counter;
 39         dynticks_progress_counter = tmp + 1;
 40       :: else -> skip;
 41       fi;
 42     :: else -> skip;
 43     fi;
 44     atomic {
 45       in_dyntick_irq = 0;
 46       i++;
 47     }
 48   od;
 49   dyntick_irq_done = 1;
 50 }
\end{verbatim}
}

The loop from lines~7-48 models up to \co{MAX_DYNTICK_LOOP_IRQ}
interrupts, with lines~8 and 9 forming the loop condition and line~45
incrementing the control variable.
Line~10 tells \co{dyntick_nohz()} that an interrupt handler
is running, and line~45 tells \co{dyntick_nohz()} that this
handler has completed.
Line~49 is used for liveness verification, just like the corresponding
line of \co{dyntick_nohz()}.

\QuickQuiz{}
	Why are lines~45 and 46 (the \co{in_dyntick_irq = 0;}
	and the \co{i++;}) executed atomically?
\QuickQuizAnswer{
	These lines of code pertain to controlling the
	model, not to the code being modeled, so there is no reason to
	model them non-atomically.
	The motivation for modeling them atomically is to reduce the size
	of the state space.
} \QuickQuizEnd

Lines~11-25 model \co{rcu_irq_enter()}, and
lines~26 and 27 model the relevant snippet of \co{__irq_enter()}.
Lines~28 and 29 verifies safety in much the same manner as do the
corresponding lines of \co{dynticks_nohz()}.
Lines~30 and 31 model the relevant snippet of \co{__irq_exit()},
and finally lines~32-43 model \co{rcu_irq_exit()}.

\QuickQuiz{}
	What property of interrupts is this \co{dynticks_irq()}
	process unable to model?
\QuickQuizAnswer{
	One such property is nested interrupts,
	which are handled in the following section.
} \QuickQuizEnd

The \co{grace_period} process then becomes as follows:

{ \scriptsize
\begin{verbatim}
  1 proctype grace_period()
  2 {
  3   byte curr;
  4   byte snap;
  5   bit shouldexit;
  6
  7   gp_state = GP_IDLE;
  8   atomic {
  9     printf("MDLN = %d\n", MAX_DYNTICK_LOOP_NOHZ);
 10     printf("MDLI = %d\n", MAX_DYNTICK_LOOP_IRQ);
 11     shouldexit = 0;
 12     snap = dynticks_progress_counter;
 13     gp_state = GP_WAITING;
 14   }
 15   do
 16   :: 1 ->
 17     atomic {
 18       assert(!shouldexit);
 19       shouldexit = dyntick_nohz_done && dyntick_irq_done;
 20       curr = dynticks_progress_counter;
 21       if
 22       :: (curr - snap) >= 2 || (curr & 1) == 0 ->
 23         break;
 24       :: else -> skip;
 25       fi;
 26     }
 27   od;
 28   gp_state = GP_DONE;
 29   gp_state = GP_IDLE;
 30   atomic {
 31     shouldexit = 0;
 32     snap = dynticks_progress_counter;
 33     gp_state = GP_WAITING;
 34   }
 35   do
 36   :: 1 ->
 37     atomic {
 38       assert(!shouldexit);
 39       shouldexit = dyntick_nohz_done && dyntick_irq_done;
 40       curr = dynticks_progress_counter;
 41       if
 42       :: (curr != snap) || ((curr & 1) == 0) ->
 43         break;
 44       :: else -> skip;
 45       fi;
 46     }
 47   od;
 48   gp_state = GP_DONE;
 49 }
\end{verbatim}
}

The implementation of \co{grace_period()} is very similar
to the earlier one.
The only changes are the addition of line~10 to add the new
interrupt-count parameter, changes to lines~19 and 39 to
add the new \co{dyntick_irq_done} variable to the liveness
checks, and of course the optimizations on lines~22 and 42.

This model (\path{dyntickRCU-irqnn-ssl.spin})
results in a correct verification with roughly half a million
states, passing without errors.
However, this version of the model does not handle nested
interrupts.
This topic is taken up in the next section.

\subsubsection{Validating Nested Interrupt Handlers}
\label{sec:formal:Validating Nested Interrupt Handlers}

Nested interrupt handlers may be modeled by splitting the body of
the loop in \co{dyntick_irq()} as follows:

{ \scriptsize
\begin{verbatim}
  1 proctype dyntick_irq()
  2 {
  3   byte tmp;
  4   byte i = 0;
  5   byte j = 0;
  6   bit old_gp_idle;
  7   bit outermost;
  8
  9   do
 10   :: i >= MAX_DYNTICK_LOOP_IRQ &&
 11      j >= MAX_DYNTICK_LOOP_IRQ -> break;
 12   :: i < MAX_DYNTICK_LOOP_IRQ ->
 13     atomic {
 14       outermost = (in_dyntick_irq == 0);
 15       in_dyntick_irq = 1;
 16     }
 17     if
 18     :: rcu_update_flag > 0 ->
 19       tmp = rcu_update_flag;
 20       rcu_update_flag = tmp + 1;
 21     :: else -> skip;
 22     fi;
 23     if
 24     :: !in_interrupt &&
 25        (dynticks_progress_counter & 1) == 0 ->
 26       tmp = dynticks_progress_counter;
 27       dynticks_progress_counter = tmp + 1;
 28       tmp = rcu_update_flag;
 29       rcu_update_flag = tmp + 1;
 30     :: else -> skip;
 31     fi;
 32     tmp = in_interrupt;
 33     in_interrupt = tmp + 1;
 34     atomic {
 35       if
 36       :: outermost ->
 37         old_gp_idle = (gp_state == GP_IDLE);
 38       :: else -> skip;
 39       fi;
 40     }
 41     i++;
 42   :: j < i ->
 43     atomic {
 44       if
 45       :: j + 1 == i ->
 46         assert(!old_gp_idle ||
 47                gp_state != GP_DONE);
 48       :: else -> skip;
 49       fi;
 50     }
 51     tmp = in_interrupt;
 52     in_interrupt = tmp - 1;
 53     if
 54     :: rcu_update_flag != 0 ->
 55       tmp = rcu_update_flag;
 56       rcu_update_flag = tmp - 1;
 57       if
 58       :: rcu_update_flag == 0 ->
 59         tmp = dynticks_progress_counter;
 60         dynticks_progress_counter = tmp + 1;
 61       :: else -> skip;
 62       fi;
 63     :: else -> skip;
 64     fi;
 65     atomic {
 66       j++;
 67       in_dyntick_irq = (i != j);
 68     }
 69   od;
 70   dyntick_irq_done = 1;
 71 }
\end{verbatim}
}

This is similar to the earlier \co{dynticks_irq()} process.
It adds a second counter variable \co{j} on line~5, so that
\co{i} counts entries to interrupt handlers and \co{j}
counts exits.
The \co{outermost} variable on line~7 helps determine
when the \co{gp_state} variable needs to be sampled
for the safety checks.
The loop-exit check on lines~10 and 11 is updated to require that the
specified number of interrupt handlers are exited as well as entered,
and the increment of \co{i} is moved to line~41, which is
the end of the interrupt-entry model.
Lines~13-16 set the \co{outermost} variable to indicate
whether this is the outermost of a set of nested interrupts and to
set the \co{in_dyntick_irq} variable that is used by the
\co{dyntick_nohz()} process.
Lines~34-40 capture the state of the \co{gp_state}
variable, but only when in the outermost interrupt handler.

Line~42 has the do-loop conditional for interrupt-exit modeling:
as long as we have exited fewer interrupts than we have entered, it is
legal to exit another interrupt.
Lines~43-50 check the safety criterion, but only if we are exiting
from the outermost interrupt level.
Finally, lines~65-68 increment the interrupt-exit count \co{j}
and, if this is the outermost interrupt level, clears
\co{in_dyntick_irq}.

This model (\path{dyntickRCU-irq-ssl.spin})
results in a correct verification with a bit more than half a million
states, passing without errors.
However, this version of the model does not handle NMIs,
which are taken up in the next section.

\subsubsection{Validating NMI Handlers}
\label{sec:formal:Validating NMI Handlers}

We take the same general approach for NMIs as we do for interrupts,
keeping in mind that NMIs do not nest.
This results in a \co{dyntick_nmi()} process as follows:

{ \scriptsize
\begin{verbatim}
  1 proctype dyntick_nmi()
  2 {
  3   byte tmp;
  4   byte i = 0;
  5   bit old_gp_idle;
  6
  7   do
  8   :: i >= MAX_DYNTICK_LOOP_NMI -> break;
  9   :: i < MAX_DYNTICK_LOOP_NMI ->
 10     in_dyntick_nmi = 1;
 11     if
 12     :: rcu_update_flag > 0 ->
 13       tmp = rcu_update_flag;
 14       rcu_update_flag = tmp + 1;
 15     :: else -> skip;
 16     fi;
 17     if
 18     :: !in_interrupt &&
 19        (dynticks_progress_counter & 1) == 0 ->
 20       tmp = dynticks_progress_counter;
 21       dynticks_progress_counter = tmp + 1;
 22       tmp = rcu_update_flag;
 23       rcu_update_flag = tmp + 1;
 24     :: else -> skip;
 25     fi;
 26     tmp = in_interrupt;
 27     in_interrupt = tmp + 1;
 28     old_gp_idle = (gp_state == GP_IDLE);
 29     assert(!old_gp_idle || gp_state != GP_DONE);
 30     tmp = in_interrupt;
 31     in_interrupt = tmp - 1;
 32     if
 33     :: rcu_update_flag != 0 ->
 34       tmp = rcu_update_flag;
 35       rcu_update_flag = tmp - 1;
 36       if
 37       :: rcu_update_flag == 0 ->
 38         tmp = dynticks_progress_counter;
 39         dynticks_progress_counter = tmp + 1;
 40       :: else -> skip;
 41       fi;
 42     :: else -> skip;
 43     fi;
 44     atomic {
 45       i++;
 46       in_dyntick_nmi = 0;
 47     }
 48   od;
 49   dyntick_nmi_done = 1;
 50 }
\end{verbatim}
}

Of course, the fact that we have NMIs requires adjustments in
the other components.
For example, the \co{EXECUTE_MAINLINE()} macro now needs to
pay attention to the NMI handler (\co{in_dyntick_nmi}) as well
as the interrupt handler (\co{in_dyntick_irq}) by checking
the \co{dyntick_nmi_done} variable as follows:

{ \scriptsize
\begin{verbatim}
  1 #define EXECUTE_MAINLINE(label, stmt) \
  2 label: skip; \
  3     atomic { \
  4       if \
  5       :: in_dyntick_irq || \
  6          in_dyntick_nmi -> goto label; \
  7       :: else -> stmt; \
  8       fi; \
  9     } \
\end{verbatim}
}

We will also need to introduce an \co{EXECUTE_IRQ()}
macro that checks \co{in_dyntick_nmi} in order to allow
\co{dyntick_irq()} to exclude \co{dyntick_nmi()}:

{ \scriptsize
\begin{verbatim}
  1 #define EXECUTE_IRQ(label, stmt) \
  2 label: skip; \
  3     atomic { \
  4       if \
  5       :: in_dyntick_nmi -> goto label; \
  6       :: else -> stmt; \
  7       fi; \
  8     } \
\end{verbatim}
}

It is further necessary to convert \co{dyntick_irq()}
to \co{EXECUTE_IRQ()} as follows:

{ \scriptsize
\begin{verbatim}
  1 proctype dyntick_irq()
  2 {
  3   byte tmp;
  4   byte i = 0;
  5   byte j = 0;
  6   bit old_gp_idle;
  7   bit outermost;
  8
  9   do
 10   :: i >= MAX_DYNTICK_LOOP_IRQ &&
 11      j >= MAX_DYNTICK_LOOP_IRQ -> break;
 12   :: i < MAX_DYNTICK_LOOP_IRQ ->
 13     atomic {
 14       outermost = (in_dyntick_irq == 0);
 15       in_dyntick_irq = 1;
 16     }
 17 stmt1: skip;
 18     atomic {
 19       if
 20       :: in_dyntick_nmi -> goto stmt1;
 21       :: !in_dyntick_nmi && rcu_update_flag ->
 22         goto stmt1_then;
 23       :: else -> goto stmt1_else;
 24       fi;
 25     }
 26 stmt1_then: skip;
 27     EXECUTE_IRQ(stmt1_1, tmp = rcu_update_flag)
 28     EXECUTE_IRQ(stmt1_2, rcu_update_flag = tmp + 1)
 29 stmt1_else: skip;
 30 stmt2: skip;  atomic {
 31       if
 32       :: in_dyntick_nmi -> goto stmt2;
 33       :: !in_dyntick_nmi &&
 34          !in_interrupt &&
 35          (dynticks_progress_counter & 1) == 0 ->
 36            goto stmt2_then;
 37       :: else -> goto stmt2_else;
 38       fi;
 39     }
 40 stmt2_then: skip;
 41     EXECUTE_IRQ(stmt2_1, tmp = dynticks_progress_counter)
 42     EXECUTE_IRQ(stmt2_2,
 43       dynticks_progress_counter = tmp + 1)
 44     EXECUTE_IRQ(stmt2_3, tmp = rcu_update_flag)
 45     EXECUTE_IRQ(stmt2_4, rcu_update_flag = tmp + 1)
 46 stmt2_else: skip;
 47     EXECUTE_IRQ(stmt3, tmp = in_interrupt)
 48     EXECUTE_IRQ(stmt4, in_interrupt = tmp + 1)
 49 stmt5: skip;
 50     atomic {
 51       if
 52       :: in_dyntick_nmi -> goto stmt4;
 53       :: !in_dyntick_nmi && outermost ->
 54         old_gp_idle = (gp_state == GP_IDLE);
 55       :: else -> skip;
 56       fi;
 57     }
 58     i++;
 59   :: j < i ->
 60 stmt6: skip;
 61     atomic {
 62       if
 63       :: in_dyntick_nmi -> goto stmt6;
 64       :: !in_dyntick_nmi && j + 1 == i ->
 65         assert(!old_gp_idle ||
 66                gp_state != GP_DONE);
 67       :: else -> skip;
 68       fi;
 69     }
 70     EXECUTE_IRQ(stmt7, tmp = in_interrupt);
 71     EXECUTE_IRQ(stmt8, in_interrupt = tmp - 1);
 72
 73 stmt9: skip;
 74     atomic {
 75       if
 76       :: in_dyntick_nmi -> goto stmt9;
 77       :: !in_dyntick_nmi && rcu_update_flag != 0 ->
 78         goto stmt9_then;
 79       :: else -> goto stmt9_else;
 80       fi;
 81     }
 82 stmt9_then: skip;
 83     EXECUTE_IRQ(stmt9_1, tmp = rcu_update_flag)
 84     EXECUTE_IRQ(stmt9_2, rcu_update_flag = tmp - 1)
 85 stmt9_3: skip;
 86     atomic {
 87       if
 88       :: in_dyntick_nmi -> goto stmt9_3;
 89       :: !in_dyntick_nmi && rcu_update_flag == 0 ->
 90         goto stmt9_3_then;
 91       :: else -> goto stmt9_3_else;
 92       fi;
 93     }
 94 stmt9_3_then: skip;
 95     EXECUTE_IRQ(stmt9_3_1,
 96       tmp = dynticks_progress_counter)
 97     EXECUTE_IRQ(stmt9_3_2,
 98       dynticks_progress_counter = tmp + 1)
 99 stmt9_3_else:
100 stmt9_else: skip;
101     atomic {
102       j++;
103       in_dyntick_irq = (i != j);
104     }
105   od;
106   dyntick_irq_done = 1;
107 }
\end{verbatim}
}

Note that we have open-coded the ``if'' statements
(for example, lines~17-29).
In addition, statements that process strictly local state
(such as line~58) need not exclude \co{dyntick_nmi()}.

Finally, \co{grace_period()} requires only a few changes:

{ \scriptsize
\begin{verbatim}
  1 proctype grace_period()
  2 {
  3   byte curr;
  4   byte snap;
  5   bit shouldexit;
  6
  7   gp_state = GP_IDLE;
  8   atomic {
  9     printf("MDLN = %d\n", MAX_DYNTICK_LOOP_NOHZ);
 10     printf("MDLI = %d\n", MAX_DYNTICK_LOOP_IRQ);
 11     printf("MDLN = %d\n", MAX_DYNTICK_LOOP_NMI);
 12     shouldexit = 0;
 13     snap = dynticks_progress_counter;
 14     gp_state = GP_WAITING;
 15   }
 16   do
 17   :: 1 ->
 18     atomic {
 19       assert(!shouldexit);
 20       shouldexit = dyntick_nohz_done &&
 21              dyntick_irq_done &&
 22              dyntick_nmi_done;
 23       curr = dynticks_progress_counter;
 24       if
 25       :: (curr - snap) >= 2 || (curr & 1) == 0 ->
 26         break;
 27       :: else -> skip;
 28       fi;
 29     }
 30   od;
 31   gp_state = GP_DONE;
 32   gp_state = GP_IDLE;
 33   atomic {
 34     shouldexit = 0;
 35     snap = dynticks_progress_counter;
 36     gp_state = GP_WAITING;
 37   }
 38   do
 39   :: 1 ->
 40     atomic {
 41       assert(!shouldexit);
 42       shouldexit = dyntick_nohz_done &&
 43              dyntick_irq_done &&
 44              dyntick_nmi_done;
 45       curr = dynticks_progress_counter;
 46       if
 47       :: (curr != snap) || ((curr & 1) == 0) ->
 48         break;
 49       :: else -> skip;
 50       fi;
 51     }
 52   od;
 53   gp_state = GP_DONE;
 54 }
\end{verbatim}
}

We have added the \co{printf()} for the new
\co{MAX_DYNTICK_LOOP_NMI} parameter on line~11 and
added \co{dyntick_nmi_done} to the \co{shouldexit}
assignments on lines~22 and 44.

The model (\path{dyntickRCU-irq-nmi-ssl.spin})
results in a correct verification with several hundred million
states, passing without errors.

\QuickQuiz{}
	Does Paul \emph{always} write his code in this painfully incremental
	manner?
\QuickQuizAnswer{
	Not always, but more and more frequently.
	In this case, Paul started with the smallest slice of code that
	included an interrupt handler, because he was not sure how best
	to model interrupts in Promela.
	Once he got that working, he added other features.
	(But if he was doing it again, he would start with a ``toy'' handler.
	For example, he might have the handler increment a variable twice and
	have the mainline code verify that the value was always even.)

	Why the incremental approach?
	Consider the following, attributed to Brian W. Kernighan:

	\begin{quote}
		Debugging is twice as hard as writing the code in the first
		place. Therefore, if you write the code as cleverly as possible,
		you are, by definition, not smart enough to debug it.
	\end{quote}

	This means that any attempt to optimize the production of code should
	place at least 66\% of its emphasis on optimizing the debugging process,
	even at the expense of increasing the time and effort spent coding.
	Incremental coding and testing is one way to optimize the debugging
	process, at the expense of some increase in coding effort.
	Paul uses this approach because he rarely has the luxury of
	devoting full days (let alone weeks) to coding and debugging.
} \QuickQuizEnd

\subsubsection{Lessons (Re)Learned}
\label{sec:formal:Lessons (Re)Learned}

{ \scriptsize
\begin{verbbox}
 static inline void rcu_enter_nohz(void)
 {
+       mb();
        __get_cpu_var(dynticks_progress_counter)++;
-       mb();
 }

 static inline void rcu_exit_nohz(void)
 {
-       mb();
        __get_cpu_var(dynticks_progress_counter)++;
+       mb();
 }
\end{verbbox}
}
\begin{figure}[tbp]
\centering
\theverbbox
\caption{Memory-Barrier Fix Patch}
\label{fig:formal:Memory-Barrier Fix Patch}
\end{figure}

{ \scriptsize
\begin{verbbox}
-       if ((curr - snap) > 2 || (snap & 0x1) == 0)
+       if ((curr - snap) > 2 || (curr & 0x1) == 0)
\end{verbbox}
}
\begin{figure}[tbp]
\centering
\theverbbox
\caption{Variable-Name-Typo Fix Patch}
\label{fig:formal:Variable-Name-Typo Fix Patch}
\end{figure}

This effort provided some lessons (re)learned:

\begin{enumerate}
\item	{\bf Promela and spin can verify interrupt/NMI-handler
	interactions}.
\item	{\bf Documenting code can help locate bugs}.
	In this case, the documentation effort located
	a misplaced memory barrier in
	\co{rcu_enter_nohz()} and \co{rcu_exit_nohz()},
	as shown by the patch in
	Figure~\ref{fig:formal:Memory-Barrier Fix Patch}.
\item	{\bf Validate your code early, often, and up to the point
	of destruction.}
	This effort located one subtle bug in
	\co{rcu_try_flip_waitack_needed()}
	that would have been quite difficult to test or debug, as
	shown by the patch in
	Figure~\ref{fig:formal:Variable-Name-Typo Fix Patch}.
\item	{\bf Always verify your verification code.}
	The usual way to do this is to insert a deliberate bug
	and verify that the verification code catches it.  Of course,
	if the verification code fails to catch this bug, you may also
	need to verify the bug itself, and so on, recursing infinitely.
	However, if you find yourself in this position,
	getting a good night's sleep
	can be an extremely effective debugging technique.
	You will then see that the obvious verify-the-verification
	technique is to deliberately insert bugs in the code being
	verified.
	If the verification fails to find them, the verification clearly
	is buggy.
\item	{\bf Use of atomic instructions can simplify verification.}
	Unfortunately, use of the \co{cmpxchg} atomic instruction
	would also slow down the critical irq fastpath, so they
	are not appropriate in this case.
\item	{\bf The need for complex formal verification often indicates
	a need to re-think your design.}
\end{enumerate}

To this last point, it turn out that there is a much simpler solution to
the dynticks problem, which is presented in the next section.

\subsubsection{Simplicity Avoids Formal Verification}
\label{sec:formal:Simplicity Avoids Formal Verification}

The complexity of the dynticks interface for preemptible RCU is primarily
due to the fact that both irqs and NMIs use the same code path and the
same state variables.
This leads to the notion of providing separate code paths and variables
for irqs and NMIs, as has been done for
hierarchical RCU~\cite{PaulEMcKenney2008HierarchicalRCU}
as indirectly suggested by
Manfred Spraul~\cite{ManfredSpraul2008StateMachineRCU}.

\subsubsection{State Variables for Simplified Dynticks Interface}
\label{sec:formal:State Variables for Simplified Dynticks Interface}

{ \scriptsize
\begin{verbbox}
  1 struct rcu_dynticks {
  2   int dynticks_nesting;
  3   int dynticks;
  4   int dynticks_nmi;
  5 };
  6
  7 struct rcu_data {
  8   ...
  9   int dynticks_snap;
 10   int dynticks_nmi_snap;
 11   ...
 12 };
\end{verbbox}
}
\begin{figure}[tbp]
\centering
\theverbbox
\caption{Variables for Simple Dynticks Interface}
\label{fig:formal:Variables for Simple Dynticks Interface}
\end{figure}

Figure~\ref{fig:formal:Variables for Simple Dynticks Interface}
shows the new per-CPU state variables.
These variables are grouped into structs to allow multiple independent
RCU implementations (e.g., \co{rcu} and \co{rcu_bh}) to conveniently
and efficiently share dynticks state.
In what follows, they can be thought of as independent per-CPU variables.

The \co{dynticks_nesting}, \co{dynticks}, and \co{dynticks_snap} variables
are for the irq code paths, and the \co{dynticks_nmi} and
\co{dynticks_nmi_snap} variables are for the NMI code paths, although
the NMI code path will also reference (but not modify) the
\co{dynticks_nesting} variable.
These variables are used as follows:

\begin{itemize}
\item	\co{dynticks_nesting}:
	This counts the number of reasons that the corresponding
	CPU should be monitored for RCU read-side critical sections.
	If the CPU is in dynticks-idle mode, then this counts the
	irq nesting level, otherwise it is one greater than the
	irq nesting level.
\item	\co{dynticks}:
	This counter's value is even if the corresponding CPU is
	in dynticks-idle mode and there are no irq handlers currently
	running on that CPU, otherwise the counter's value is odd.
	In other words, if this counter's value is odd, then the
	corresponding CPU might be in an RCU read-side critical section.
\item	\co{dynticks_nmi}:
	This counter's value is odd if the corresponding CPU is
	in an NMI handler, but only if the NMI arrived while this
	CPU was in dyntick-idle mode with no irq handlers running.
	Otherwise, the counter's value will be even.
\item	\co{dynticks_snap}:
	This will be a snapshot of the \co{dynticks} counter, but
	only if the current RCU grace period has extended for too
	long a duration.
\item	\co{dynticks_nmi_snap}:
	This will be a snapshot of the \co{dynticks_nmi} counter, but
	again only if the current RCU grace period has extended for too
	long a duration.
\end{itemize}

If both \co{dynticks} and \co{dynticks_nmi} have taken on an even
value during a given time interval, then the corresponding CPU has
passed through a quiescent state during that interval.

\QuickQuiz{}
	But what happens if an NMI handler starts running before
	an irq handler completes, and if that NMI handler continues
	running until a second irq handler starts?
\QuickQuizAnswer{
	This cannot happen within the confines of a single CPU.
	The first irq handler cannot complete until the NMI handler
	returns.
	Therefore, if each of the \co{dynticks} and \co{dynticks_nmi}
	variables have taken on an even value during a given time
	interval, the corresponding CPU really was in a quiescent
	state at some time during that interval.
} \QuickQuizEnd

\subsubsection{Entering and Leaving Dynticks-Idle Mode}
\label{sec:formal:Entering and Leaving Dynticks-Idle Mode}

{ \scriptsize
\begin{verbbox}
  1 void rcu_enter_nohz(void)
  2 {
  3   unsigned long flags;
  4   struct rcu_dynticks *rdtp;
  5
  6   smp_mb();
  7   local_irq_save(flags);
  8   rdtp = &__get_cpu_var(rcu_dynticks);
  9   rdtp->dynticks++;
 10   rdtp->dynticks_nesting--;
 11   WARN_ON(rdtp->dynticks & 0x1);
 12   local_irq_restore(flags);
 13 }
 14
 15 void rcu_exit_nohz(void)
 16 {
 17   unsigned long flags;
 18   struct rcu_dynticks *rdtp;
 19
 20   local_irq_save(flags);
 21   rdtp = &__get_cpu_var(rcu_dynticks);
 22   rdtp->dynticks++;
 23   rdtp->dynticks_nesting++;
 24   WARN_ON(!(rdtp->dynticks & 0x1));
 25   local_irq_restore(flags);
 26   smp_mb();
 27 }
\end{verbbox}
}
\begin{figure}[tbp]
\centering
\theverbbox
\caption{Entering and Exiting Dynticks-Idle Mode}
\label{fig:formal:Entering and Exiting Dynticks-Idle Mode}
\end{figure}

Figure~\ref{fig:formal:Entering and Exiting Dynticks-Idle Mode}
shows the \co{rcu_enter_nohz()} and \co{rcu_exit_nohz()},
which enter and exit dynticks-idle mode, also known as ``nohz'' mode.
These two functions are invoked from process context.

Line~6 ensures that any prior memory accesses (which might
include accesses from RCU read-side critical sections) are seen
by other CPUs before those marking entry to dynticks-idle mode.
Lines~7 and 12 disable and reenable irqs.
Line~8 acquires a pointer to the current CPU's \co{rcu_dynticks}
structure, and
line~9 increments the current CPU's \co{dynticks} counter, which
should now be even, given that we are entering dynticks-idle mode
in process context.
Finally, line~10 decrements \co{dynticks_nesting}, which should now be zero.

The \co{rcu_exit_nohz()} function is quite similar, but increments
\co{dynticks_nesting} rather than decrementing it and checks for
the opposite \co{dynticks} polarity.

\subsubsection{NMIs From Dynticks-Idle Mode}
\label{sec:formal:NMIs From Dynticks-Idle Mode}

{ \scriptsize
\begin{verbbox}
 1  void rcu_nmi_enter(void)
 2  {
 3    struct rcu_dynticks *rdtp;
 4 
 5    rdtp = &__get_cpu_var(rcu_dynticks);
 6    if (rdtp->dynticks & 0x1)
 7      return;
 8    rdtp->dynticks_nmi++;
 9    WARN_ON(!(rdtp->dynticks_nmi & 0x1));
10    smp_mb();
11  }
12 
13  void rcu_nmi_exit(void)
14  {
15    struct rcu_dynticks *rdtp;
16 
17    rdtp = &__get_cpu_var(rcu_dynticks);
18    if (rdtp->dynticks & 0x1)
19      return;
20    smp_mb();
21    rdtp->dynticks_nmi++;
22    WARN_ON(rdtp->dynticks_nmi & 0x1);
23  }
\end{verbbox}
}
\begin{figure}[tbp]
\centering
\theverbbox
\caption{NMIs From Dynticks-Idle Mode}
\label{fig:formal:NMIs From Dynticks-Idle Mode}
\end{figure}

Figure~\ref{fig:formal:NMIs From Dynticks-Idle Mode}
shows the \co{rcu_nmi_enter()} and \co{rcu_nmi_exit()} functions,
which inform RCU of NMI entry and exit, respectively, from dynticks-idle
mode.
However, if the NMI arrives during an irq handler, then RCU will already
be on the lookout for RCU read-side critical sections from this CPU,
so lines~6 and 7 of \co{rcu_nmi_enter} and lines~18 and 19
of \co{rcu_nmi_exit} silently return if \co{dynticks} is odd.
Otherwise, the two functions increment \co{dynticks_nmi}, with
\co{rcu_nmi_enter()} leaving it with an odd value and \co{rcu_nmi_exit()}
leaving it with an even value.
Both functions execute memory barriers between this increment
and possible RCU read-side critical sections on lines~11 and 21,
respectively.

\subsubsection{Interrupts From Dynticks-Idle Mode}
\label{sec:formal:Interrupts From Dynticks-Idle Mode}

{ \scriptsize
\begin{verbbox}
  1 void rcu_irq_enter(void)
  2 {
  3   struct rcu_dynticks *rdtp;
  4
  5   rdtp = &__get_cpu_var(rcu_dynticks);
  6   if (rdtp->dynticks_nesting++)
  7     return;
  8   rdtp->dynticks++;
  9   WARN_ON(!(rdtp->dynticks & 0x1));
 10   smp_mb();
 11 }
 12
 13 void rcu_irq_exit(void)
 14 {
 15   struct rcu_dynticks *rdtp;
 16
 17   rdtp = &__get_cpu_var(rcu_dynticks);
 18   if (--rdtp->dynticks_nesting)
 19     return;
 20   smp_mb();
 21   rdtp->dynticks++;
 22   WARN_ON(rdtp->dynticks & 0x1);
 23   if (__get_cpu_var(rcu_data).nxtlist ||
 24       __get_cpu_var(rcu_bh_data).nxtlist)
 25     set_need_resched();
 26 }
\end{verbbox}
}
\begin{figure}[tbp]
\centering
\theverbbox
\caption{Interrupts From Dynticks-Idle Mode}
\label{fig:formal:Interrupts From Dynticks-Idle Mode}
\end{figure}

Figure~\ref{fig:formal:Interrupts From Dynticks-Idle Mode}
shows \co{rcu_irq_enter()} and \co{rcu_irq_exit()}, which
inform RCU of entry to and exit from, respectively, irq context.
Line~6 of \co{rcu_irq_enter()} increments \co{dynticks_nesting},
and if this variable was already non-zero, line~7 silently returns.
Otherwise, line~8 increments \co{dynticks}, which will then have
an odd value, consistent with the fact that this CPU can now
execute RCU read-side critical sections.
Line~10 therefore executes a memory barrier to ensure that
the increment of \co{dynticks} is seen before any
RCU read-side critical sections that the subsequent irq handler
might execute.

Line~18 of \co{rcu_irq_exit()} decrements \co{dynticks_nesting}, and
if the result is non-zero, line~19 silently returns.
Otherwise, line~20 executes a memory barrier to ensure that the
increment of \co{dynticks} on line~21 is seen after any RCU
read-side critical sections that the prior irq handler might have executed.
Line~22 verifies that \co{dynticks} is now even, consistent with
the fact that no RCU read-side critical sections may appear in
dynticks-idle mode.
Lines~23-25 check to see if the prior irq handlers enqueued any
RCU callbacks, forcing this CPU out of dynticks-idle mode via
a reschedule API if so.

\subsubsection{Checking For Dynticks Quiescent States}
\label{sec:formal:Checking For Dynticks Quiescent States}

{ \scriptsize
\begin{verbbox}
 1  static int
 2  dyntick_save_progress_counter(struct rcu_data *rdp)
 3  {
 4    int ret;
 5    int snap;
 6    int snap_nmi;
 7 
 8    snap = rdp->dynticks->dynticks;
 9    snap_nmi = rdp->dynticks->dynticks_nmi;
10    smp_mb();
11    rdp->dynticks_snap = snap;
12    rdp->dynticks_nmi_snap = snap_nmi;
13    ret = ((snap & 0x1) == 0) &&
14          ((snap_nmi & 0x1) == 0);
15    if (ret)
16      rdp->dynticks_fqs++;
17    return ret;
18  }
\end{verbbox}
}
\begin{figure}[tbp]
\centering
\theverbbox
\caption{Saving Dyntick Progress Counters}
\label{fig:formal:Saving Dyntick Progress Counters}
\end{figure}

Figure~\ref{fig:formal:Saving Dyntick Progress Counters}
shows \co{dyntick_save_progress_counter()}, which takes a snapshot
of the specified CPU's \co{dynticks} and \co{dynticks_nmi}
counters.
Lines~8 and 9 snapshot these two variables to locals, line~10
executes a memory barrier to pair with the memory barriers in
the functions in
Figures~\ref{fig:formal:Entering and Exiting Dynticks-Idle Mode},
\ref{fig:formal:NMIs From Dynticks-Idle Mode}, and
\ref{fig:formal:Interrupts From Dynticks-Idle Mode}.
Lines~11 and 12 record the snapshots for later calls to
\co{rcu_implicit_dynticks_qs},
and lines~13 and~14 check to see if the CPU is in dynticks-idle mode with
neither irqs nor NMIs in progress (in other words, both snapshots
have even values), hence in an extended quiescent state.
If so, lines~15 and 16 count this event, and line~17 returns
true if the CPU was in a quiescent state.

{ \scriptsize
\begin{verbbox}
 1  static int
 2  rcu_implicit_dynticks_qs(struct rcu_data *rdp)
 3  {
 4    long curr;
 5    long curr_nmi;
 6    long snap;
 7    long snap_nmi;
 8 
 9    curr = rdp->dynticks->dynticks;
10    snap = rdp->dynticks_snap;
11    curr_nmi = rdp->dynticks->dynticks_nmi;
12    snap_nmi = rdp->dynticks_nmi_snap;
13    smp_mb();
14    if ((curr != snap || (curr & 0x1) == 0) &&
15        (curr_nmi != snap_nmi ||
16        (curr_nmi & 0x1) == 0)) {
17      rdp->dynticks_fqs++;
18      return 1;
19    }
20    return rcu_implicit_offline_qs(rdp);
21  }
\end{verbbox}
}
\begin{figure}[tbp]
\centering
\theverbbox
\caption{Checking Dyntick Progress Counters}
\label{fig:formal:Checking Dyntick Progress Counters}
\end{figure}

Figure~\ref{fig:formal:Checking Dyntick Progress Counters}
shows \co{dyntick_save_progress_counter}, which is called to check
whether a CPU has entered dyntick-idle mode subsequent to a call
to \co{dynticks_save_progress_counter()}.
Lines~9 and 11 take new snapshots of the corresponding CPU's
\co{dynticks} and \co{dynticks_nmi} variables, while lines~10 and 12
retrieve the snapshots saved earlier by
\co{dynticks_save_progress_counter()}.
Line~13 then
executes a memory barrier to pair with the memory barriers in
the functions in
Figures~\ref{fig:formal:Entering and Exiting Dynticks-Idle Mode},
\ref{fig:formal:NMIs From Dynticks-Idle Mode}, and
\ref{fig:formal:Interrupts From Dynticks-Idle Mode}.
Lines~14-16 then check to see if the CPU is either currently in
a quiescent state (\co{curr} and \co{curr_nmi} having even values) or
has passed through a quiescent state since the last call to
\co{dynticks_save_progress_counter()} (the values of
\co{dynticks} and \co{dynticks_nmi} having changed).
If these checks confirm that the CPU has passed through a dyntick-idle
quiescent state, then line~17 counts that fact and line~18 returns
an indication of this fact.
Either way, line~20 checks for race conditions that can result in RCU
waiting for a CPU that is offline.

\QuickQuiz{}
	This is still pretty complicated.
	Why not just have a \co{cpumask_t} that has a bit set for
	each CPU that is in dyntick-idle mode, clearing the bit
	when entering an irq or NMI handler, and setting it upon
	exit?
\QuickQuizAnswer{
	Although this approach would be functionally correct, it
	would result in excessive irq entry/exit overhead on
	large machines.
	In contrast, the approach laid out in this section allows
	each CPU to touch only per-CPU data on irq and NMI entry/exit,
	resulting in much lower irq entry/exit overhead, especially
	on large machines.
} \QuickQuizEnd

\subsubsection{Discussion}
\label{sec:formal:Discussion}

A slight shift in viewpoint resulted in a substantial simplification
of the dynticks interface for RCU.
The key change leading to this simplification was minimizing of
sharing between irq and NMI contexts.
The only sharing in this simplified interface is references from NMI
context to irq variables (the \co{dynticks} variable).
This type of sharing is benign, because the NMI functions never update
this variable, so that its value remains constant through the lifetime
of the NMI handler.
This limitation of sharing allows the individual functions to be
understood one at a time, in happy contrast to the situation
described in
Section~\ref{sec:formal:Promela Parable: dynticks and Preemptible RCU},
where an NMI might change shared state at any point during execution of
the irq functions.

Verification can be a good thing, but simplicity is even better.

% formal/ppcmem.tex
% mainfile: ../perfbook.tex
% SPDX-License-Identifier: CC-BY-SA-3.0

\section{Special-Purpose State-Space Search}
\label{sec:formal:Special-Purpose State-Space Search}
%
\epigraph{Jack of all trades, master of none.}{\emph{Unknown}}

Promela 와 Spin 이 여러분이 모든 (작은) 알고리즘을 얼마든지 검증할 수 있게
해주지만, 그것들의 큰 범용성은 가끔 문제가 될 수 있습니다.
예를 들어, Promela 는 메모리 모델이나 특정 종류의 순서 재배치 의미를 이해하지
못합니다.
따라서 이 섹션은 제품 단계 시스템에서 사용되는 메모리 모델을 이해해서 완화된
순서 코드의 검증을 크게 단순화 시키는 상태 공간 탐색 도구들을 소개합니다.

예를 들어,
\cref{sec:formal:Promela Example: QRCU}
는 완화된 메모리 순서규칙을 위한 처리를 위해 어떻게 Promela 를 다뤄야 하는지
보였습니다.
이 방법이 잘 동작하긴 하나, 이는 개발자가 그 시스템의 메모리 모델을 완전히
이해할 것을 필요로 합니다.
불행히도, 일부의 (존재한다면) 개발자들만이 현대 CPU 의 복잡한 메모리 모델을
완전히 이해합니다.

\iffalse

Although Promela and Spin allow you to verify pretty much any (smallish)
algorithm, their very generality can sometimes be a curse.
For example, Promela does not understand memory models or any sort
of reordering semantics.
This section therefore describes some state-space search tools that
understand memory models used by production systems, greatly simplifying the
verification of weakly ordered code.

For example,
\cref{sec:formal:Promela Example: QRCU}
showed how to convince Promela to account for weak memory ordering.
Although this approach can work well, it requires that the developer
fully understand the system's memory model.
Unfortunately, few (if any) developers fully understand the complex
memory models of modern CPUs.

\fi

따라서, 또다른 접근법은 Cambridge 대학의 \ppl{Peter}{Sewell} 와
\ppl{Susmit}{Sarkar}, INRIA 의 \ppl{Luc}{Maranget},
\ppl{Francesco Zappa}{Nardelli}, 그리고 \ppl{Pankaj}{Pawan} , 그리고 Oxford
대학의 \ppl{Jade}{Alglave} 가 IBM 의 \ppl{Derek}{Williams} 와 협업해 만들어낸
PPCMEM 도구와 같은, 이 메모리 순서 규칙을 이미 이해하고 있는 도구를 사용하는
것입니다.
이 연구 그룹은 Power, \ARM, x86 은 물론이고 C/C++11 표준의 메모리 모델을 정형화
시키고~\cite{RichardSmith2019N4800}, Power 와 \ARM\ 정형화에 기초에 PPCMEM
도구를 만들었습니다.

\iffalse

Therefore, another approach is to use a tool that already understands
this memory ordering, such as the PPCMEM tool produced by
\ppl{Peter}{Sewell} and \ppl{Susmit}{Sarkar} at the University of Cambridge,
\ppl{Luc}{Maranget}, \ppl{Francesco Zappa}{Nardelli}, and
\ppl{Pankaj}{Pawan} at INRIA, and \ppl{Jade}{Alglave} at Oxford University,
in cooperation with \ppl{Derek}{Williams} of
IBM~\cite{JadeAlglave2011ppcmem}.
This group formalized the memory models of Power, \ARM, x86, as well
as that of the C/C++11 standard~\cite{RichardSmith2019N4800}, and
produced the PPCMEM tool based on the Power and \ARM\ formalizations.

\fi

\QuickQuiz{
	하지만 x86 은 강한 메모리 규칙을 가지고 있는데 왜 그 메모리 모델을
	정형화 시키죠?

	\iffalse

	But x86 has strong memory ordering, so why formalize its memory
	model?

	\fi

}\QuickQuizAnswer{
	사실, 학계에서는 x86 메모리 모델을 완화된 형태로 생각하는데, 앞의
	쓰기가 뒤따르는 읽기와 재배치 되는 것을 허용할 수 있기 때문입니다.
	학계의 관점에서, 강한 메모리 모델은 어떤 재배치도 허용하지 않아서 모든
	쓰레드가 그것들에게 보이는 모든 오퍼레이션의 순서에 동의할 수 있는
	것입니다.

	또한, 이건 개발자들이 가끔 x86 메모리 순서규칙에 대해 혼란에 빠지는
	경우들입니다.

	\iffalse

	Actually, academics consider the x86 memory model to be weak
	because it can allow prior stores to be reordered with
	subsequent loads.
	From an academic viewpoint, a strong memory model is one
	that allows absolutely no reordering, so that all threads
	agree on the order of all operations visible to them.

	Plus it really is the case that developers are sometimes confused
	about x86 memory ordering.

	\fi

}\QuickQuizEnd

PPCMEM 도구는 \emph{리트머스 테스트} 를 입력으로 받습니다.
샘플 리트머스 테스트가
\cref{sec:formal:Anatomy of a Litmus Test} 에서 선보입니다.
\Cref{sec:formal:What Does This Litmus Test Mean?}
는 이 리트머스 테스트를 동일한 C-언어 프로그램으로 연관지어보고,
\cref{sec:formal:Running a Litmus Test} 은 이 리트머스 테스트에 PPCMEM 을
적용하는지 설명하며,
\cref{sec:formal:PPCMEM Discussion}
은 그 의미를 이야기 합니다.

\iffalse

The PPCMEM tool takes \emph{litmus tests} as input.
A sample litmus test is presented in
\cref{sec:formal:Anatomy of a Litmus Test}.
\Cref{sec:formal:What Does This Litmus Test Mean?}
relates this litmus test to the equivalent C-language program,
\cref{sec:formal:Running a Litmus Test} describes how to
apply PPCMEM to this litmus test, and
\cref{sec:formal:PPCMEM Discussion}
discusses the implications.

\fi

\subsection{Anatomy of a Litmus Test}
\label{sec:formal:Anatomy of a Litmus Test}

PPCMEM 을 위한 PowerPC 리트머스 테스트가
\cref{lst:formal:PPCMEM Litmus Test} 에 보여져 있습니다.
ARM 인터페이스도 같은 방식으로 동작하지만 \ARM\ 명령어들이 Power 명령어들로
대체되었고 시작 부분의 \qco{PPC} 도 \qco{ARM} 으로 교체되었습니다.

\iffalse

An example PowerPC litmus test for PPCMEM is shown in
\cref{lst:formal:PPCMEM Litmus Test}.
The ARM interface works the same way, but with \ARM\ instructions
substituted for the Power instructions and with the initial \qco{PPC}
replaced by \qco{ARM}.

\fi

\begin{listing}[tbp]
\begin{fcvlabel}[ln:formal:PPCMEM Litmus Test]
\begin{VerbatimL}[commandchars=\@\[\]]
PPC SB+lwsync-RMW-lwsync+isync-simple		@lnlbl[type]
""						@lnlbl[altname]
{						@lnlbl[init:b]
0:r2=x; 0:r3=2; 0:r4=y; 0:r10=0; 0:r11=0; 0:r12=z; @lnlbl[init:0]
1:r2=y; 1:r4=x;					@lnlbl[init:1]
}						@lnlbl[init:e]
 P0                 | P1           ;		@lnlbl[procid]
 li r1,1            | li r1,1      ;		@lnlbl[reginit]
 stw r1,0(r2)       | stw r1,0(r2) ;		@lnlbl[stw]
 lwsync             | sync         ; @lnlbl[P0lwsync] @lnlbl[P1sync]
                    | lwz r3,0(r4) ; @lnlbl[P0empty]  @lnlbl[P1lwz]
 lwarx  r11,r10,r12 | ;		@lnlbl[P0lwarx] @lnlbl[P1empty:b]
 stwcx. r11,r10,r12 | ;		@lnlbl[P0stwcx]
 bne Fail1          | ;		@lnlbl[P0bne]
 isync              | ;		@lnlbl[P0isync]
 lwz r3,0(r4)       | ;		@lnlbl[P0lwz]
 Fail1:             | ;		@lnlbl[P0fail1] @lnlbl[P1empty:e]

exists						@lnlbl[assert:b]
(0:r3=0 /\ 1:r3=0)				@lnlbl[assert:e]
\end{VerbatimL}
\end{fcvlabel}
\caption{PPCMEM Litmus Test}
\label{lst:formal:PPCMEM Litmus Test}
\end{listing}

\begin{fcvref}[ln:formal:PPCMEM Litmus Test]
이 예에서, \clnref{type} 은 시스템의 타입을 (\qco{ARM} 또는 \qco{PPC}) 알리며 
이 모델의 제목을 포함합니다.  \Clnref{altname} 은 이 테스트를 위한 대안적
이름을 위한 공간을 제공하는데, 여러분은 앞의 예에서처럼 빈 줄로 보통 놔둘
겁니다.
주석은 \clnref{altname,init:b} 사이에 Ocaml (또는 Pascal) 문법의 \nbco{(* *)}
를 사용해 삽입될 수 있습니다.

\Clnrefrange{init:b}{init:e} 는 모든 레지스터를 위한 초기 값을 제공합니다;
각각은 \co{P:R=V} 의 형태로, \co{P} 는 프로세스 지시어이고, \co{R} 은 레지스터
지시어이며, \co{V} 는 그 값입니다.
예를 들어, 프로세스~0 의 레지스터 \co{r3} 는 초기에 값 2를 가지고 있습니다.
만약 그 값이 변수라면 (이 예에서는 \co{x}, \co{y}, 또는 \co{z}) 그 레지스터는
그 변수의 주소로 초기화 되어 있습니다.
또한, 변수들의 내용물도 초기화가 가능한데, 예를 들어 \co{x=1} 은 \co{x} 의 값을
1로 초기화 시킵니다.
초기화 되지 않은 변수들은 기본적으로 값이 0이 되어서, 이 경우 \co{x}, \co{y},
그리고 \co{z} 는 모두 초기값 0을 갖습니다.

\iffalse

\begin{fcvref}[ln:formal:PPCMEM Litmus Test]
In the example, \clnref{type} identifies the type of system (\qco{ARM} or
\qco{PPC}) and contains the title for the model. \Clnref{altname}
provides a place for an
alternative name for the test, which you will usually want to leave
blank as shown in the above example. Comments can be inserted between
\clnref{altname,init:b} using the OCaml (or Pascal) syntax of \nbco{(* *)}.

\Clnrefrange{init:b}{init:e} give initial values for all registers;
each is of the form
\co{P:R=V}, where \co{P} is the process identifier, \co{R} is the register
identifier, and \co{V} is the value. For example, process~0's register
\co{r3} initially contains the value 2. If the value is a variable (\co{x},
\co{y}, or \co{z} in the example) then the register is initialized to the
address of the variable. It is also possible to initialize the contents
of variables, for example, \co{x=1} initializes the value of \co{x} to
1. Uninitialized variables default to the value zero, so that in the
example, \co{x}, \co{y}, and~\co{z} are all initially zero.

\fi

\Clnref{procid} 는 두 프로세스를 위한 식별자를 제공해서 \clnref{init:0} 의
\co{0:r3=2} 가 \co{P0:r3=2} 로 대신 쓰여질 수 있게 합니다.
\Clnref{procid} 는 필요하며, 이 지시어는 \co{Pn} 의 형태여야 하는데, \co{n} 은
열 수로, 가장 왼쪽의 열이 0으로 시작합니다.
이는 불필요하게 엄격해 보일 수 있겠으나, 실제 사용 시에 상당한 혼란을 방지해
줍니다.
\end{fcvref}

\iffalse

\Clnref{procid} provides identifiers for the two processes, so that
the \co{0:r3=2} on \clnref{init:0} could instead have been written
\co{P0:r3=2}. \Clnref{procid} is
required, and the identifiers must be of the form \co{Pn}, where \co{n}
is the column number, starting from zero for the left-most column. This
may seem unnecessarily strict, but it does prevent considerable confusion
in actual use.
\end{fcvref}

\fi

\QuickQuiz{
	\begin{fcvref}[ln:formal:PPCMEM Litmus Test]
	\Cref{lst:formal:PPCMEM Litmus Test} 의 \clnref{reginit} 는 왜
	레지스터를 초기화 시키나요?
	왜 그대신 \clnref{init:0,init:1} 에서 초기화 시키지 않죠?
	\end{fcvref}

	\iffalse

	\begin{fcvref}[ln:formal:PPCMEM Litmus Test]
	Why does \clnref{reginit} of \cref{lst:formal:PPCMEM Litmus Test}
	initialize the registers?
	Why not instead initialize them on \clnref{init:0,init:1}?
	\end{fcvref}

	\fi

}\QuickQuizAnswer{
	두 방법 모두 잘 동작합니다.
	그러나, 일반적으로는 명시적 명령보다 초기화를 사용하는게 낫습니다.
	명시적인 명령은 이 예에서 그 사용법을 보이기 위해 사용되었습니다.
	또한, 이 도구의 웹사이트에서
	(\url{https://www.cl.cam.ac.uk/~pes20/ppcmem/}) 얻을 수 있는 많은
	리트머스 테스트는 명시적 초기화 명령들을 생성하는 자동화 방법을 사용해
	만들어 졌습니다.

	\iffalse

	Either way works.
	However, in general, it is better to use initialization than
	explicit instructions.
	The explicit instructions are used in this example to demonstrate
	their use.
	In addition, many of the litmus tests available on the tool's
	web site (\url{https://www.cl.cam.ac.uk/~pes20/ppcmem/}) were
	automatically generated, which generates explicit
	initialization instructions.

	\fi

}\QuickQuizEnd

\begin{fcvref}[ln:formal:PPCMEM Litmus Test]
\Clnrefrange{reginit}{P0fail1} 은 각 프로세스를 위한 코드입니다.
특정 프로세스는 P0의 \clnref{P0empty} 와 P1 의
\clnrefrange{P1empty:b}{P1empty:e} 에서의 경우처럼 라인을 갖지 않을 수
있습니다.
라벨과 분기가 허용되는데, \clnref{P0bne} 에서 분기가, \clnref{P0fail1} 에
라벨이 선보여 있습니다.
그러나, 너무 자유로운 분기의 사용은 상태 공간을 폭증시킬 수 있습니다.
반복문의 사용은 여러분의 상태 공간을 폭증시키기 위한 특히 좋은 방법입니다.

\Clnrefrange{assert:b}{assert:e} 는 단정을 보이는데, 여기서는 우리가 P0 와 P1
의 \co{r3} 레지스터가 두 쓰레드가 모두 수행을 끝낸 후 모두 0이 될 수 있는지에
우리가 관심있음을 보입니다.
P0 와 P1 이 각자의 \co{r3} 레지스터에서 둘 다 0을 보게 된다면 비참한 실패를
유발할 수 있는 많은 사용 경우가 있기 때문에 중요합니다.

\iffalse

\begin{fcvref}[ln:formal:PPCMEM Litmus Test]
\Clnrefrange{reginit}{P0fail1} are the lines of code for each process.
A given process can have empty lines, as is the case for P0's
\clnref{P0empty} and P1's \clnrefrange{P1empty:b}{P1empty:e}.
Labels and branches are permitted, as demonstrated by the branch
on \clnref{P0bne} to the label on \clnref{P0fail1}.
That said, too-free use of branches
will expand the state space. Use of loops is a particularly good way to
explode your state space.

\Clnrefrange{assert:b}{assert:e} show the assertion, which in this case
indicates that we
are interested in whether P0's and P1's \co{r3} registers can both contain
zero after both threads complete execution. This assertion is important
because there are a number of use cases that would fail miserably if
both P0 and P1 saw zero in their respective \co{r3} registers.

\fi

이는 여러분이 간단한 리트머스 테스트를 만드는데 충분한 정보가 될겁니다.
추가적인 문서들을 구할 수 있습니다만, 그런 추가적 문서의 많은 부분은 실제
하드웨어에서 테스트를 수행하기 위한 다른 연구 도구를 위한 것입니다.
아마도 더 중요한 건, 온라인 도구를 통해
(\url{https://www.cl.cam.ac.uk/~pes20/ppcmem/} 의 ``Select ARM Test'' 와
``Select POWER Test'' 버튼을 통해 사용 가능합니다) 이미 존재하는 많은 수의
리트머스 테스트를 사용 가능하다는 것일 겁니다.
이런 이미 존재하는 리트머스 테스트들 중 하나는 여러분의 Power 또는 \ARM\ 메모리
순서규칙 질문에 대한 답을 줄 가능성이 상당할 겁니다.

\iffalse

This should give you enough information to construct simple litmus
tests. Some additional documentation is available, though much of this
additional documentation is intended for a different research tool that
runs tests on actual hardware. Perhaps more importantly, a large number of
pre-existing litmus tests are available with the online tool (available
via the ``Select ARM Test'' and ``Select POWER Test'' buttons at
\url{https://www.cl.cam.ac.uk/~pes20/ppcmem/}).
It is quite likely that one of these pre-existing litmus tests will
answer your Power or \ARM\ memory-ordering question.

\fi

\subsection{What Does This Litmus Test Mean?}
\label{sec:formal:What Does This Litmus Test Mean?}

P0 의 \clnref{reginit,stw} 는 C 명령문 \co{x=1} 과 동일한데 \clnref{init:0} 는
P0 의 레지스터 \co{r2} 가 \co{x} 의 주소가 되게 정의하기 때문입니다.
P0 의 \clnref{P0lwarx,P0stwcx} 는 load-linked (\ARM\ 용어에서의 ``load register
exclusive'' 이자 Power 용어에서의 ``load reserve'') 와 store-conditional (\ARM\
용어에서의 ``store register exclusive'') 의 기억을 각각 돕는 장치들입니다.
함께 사용되었을 때, 이것들은 하나의 어토믹 명령 시퀀스를 만드는데 x86
\co{lock;cmpxchg} 명령으로 예시될 수 있는 compare-and-swap 시퀀스와 대략적으로
비슷합니다.
더 높은 단계의 추상화 단계로 넘어가서 \clnrefrange{P0lwsync}{P0isync} 의
시퀀스는 리눅스 커널의 \co{atomic_add_return(&z, 0)} 와 동일합니다.
마지막으로, \clnref{P0lwz} 는 C 명령문 \co{r3=y} 와 대략적으로 동일합니다.

P1 의 \clnref{reginit,stw} 는 C 명령문 \co{y=1} 과 동일하며, \clnref{P1sync} 는
메모리 배리어로, 리눅스 커널 명령문 \co{smp_mb()} 와 동일하며 \clnref{P1lwz} 는
C 명령문 \co{r3=x} 와 동일합니다.
\end{fcvref}

\iffalse

P0's \clnref{reginit,stw} are equivalent to the C statement \co{x=1}
because \clnref{init:0} defines P0's register \co{r2} to be the address
of \co{x}. P0's \clnref{P0lwarx,P0stwcx} are the mnemonics for
load-linked (``load register
exclusive'' in \ARM\ parlance and ``load reserve'' in Power parlance)
and store-conditional (``store register exclusive'' in \ARM\ parlance),
respectively. When these are used together, they form an atomic
instruction sequence, roughly similar to the compare-and-swap sequences
exemplified by the x86 \co{lock;cmpxchg} instruction. Moving to a higher
level of abstraction, the sequence from \clnrefrange{P0lwsync}{P0isync}
is equivalent to the Linux kernel's \co{atomic_add_return(&z, 0)}.
Finally, \clnref{P0lwz} is
roughly equivalent to the C statement \co{r3=y}.

P1's \clnref{reginit,stw} are equivalent to the C statement \co{y=1},
\clnref{P1sync}
is a memory barrier, equivalent to the Linux kernel statement \co{smp_mb()},
and \clnref{P1lwz} is equivalent to the C statement \co{r3=x}.
\end{fcvref}

\fi

\QuickQuiz{
	\begin{fcvref}[ln:formal:PPCMEM Litmus Test]
	하지만 \cref{lst:formal:PPCMEM Litmus Test} 의 \clnref{P0fail1}, 즉
	\co{Fail1:} 라벨에 무언가 벌어지긴 할까요?
	\end{fcvref}

	\iffalse

	\begin{fcvref}[ln:formal:PPCMEM Litmus Test]
	But whatever happened to \clnref{P0fail1} of
	\cref{lst:formal:PPCMEM Litmus Test},
	the one that is the \co{Fail1:} label?
	\end{fcvref}

	\fi

}\QuickQuizAnswer{
	PowerPC 버전의 \co{atomic_add_return()} 구현은 \co{stwcx} 명령이
	실패했을 때 반복을 하게 되는데, 조건 코드 레지스터에 0이 아닌 상태를
	설정함으로써 이를 통신하며, 이는 결국 \co{bne} 명령에 의해 검사됩니다.
	실제로 반복문을 모델링 하는 것은 상태 공간 폭증을 일으킬 것이므로, 우린
	그 대신 \co{Fail1:} 라벨로 분기해서, P0 의 \co{r3} 레지스터의 초기값
	2를 가지고 이 모델을 종료시켜서 존재하는 단정문을 깨뜨리지 않게 합니다.

	이 속임수가 항상 적용 가능한지에 대해선 논란이 있습니다만, 이게
	실패하는 예를 전 아직 보지 못했습니다.

	\iffalse

	The implementation of PowerPC version of \co{atomic_add_return()}
	loops when the \co{stwcx} instruction fails, which it communicates
	by setting non-zero status in the condition-code register,
	which in turn is tested by the \co{bne} instruction. Because actually
	modeling the loop would result in state-space explosion, we
	instead branch to the \co{Fail1:} label, terminating the model with
	the initial value of 2 in P0's \co{r3} register, which
	will not trigger the exists assertion.

	There is some debate about whether this trick is universally
	applicable, but I have not seen an example where it fails.

	\fi

}\QuickQuizEnd

\begin{listing}[tbp]
\begin{VerbatimL}
void P0(void)
{
	int r3;

	x = 1; /* Lines 8 and 9 */
	atomic_add_return(&z, 0); /* Lines 10-15 */
	r3 = y; /* Line 16 */
}

void P1(void)
{
	int r3;

	y = 1; /* Lines 8-9 */
	smp_mb(); /* Line 10 */
	r3 = x; /* Line 11 */
}
\end{VerbatimL}
\caption{Meaning of PPCMEM Litmus Test}
\label{lst:formal:Meaning of PPCMEM Litmus Test}
\end{listing}

이를 모두 종합해서, 이 전체 리트머스 테스트의 C-언어 동일 버전이
\cref{lst:formal:Meaning of PPCMEM Litmus Test} 에 보여져 있습니다.
핵심은 \co{atomic_add_return()} 이 (리눅스 커널이 요구하듯) 완전한 메모리
배리어로 동작한다면 \co{P0} 와 \co{P1()} 의 \co{r3} 변수는 수행이 완료된 후 둘
다 0일 수 없다는 것입니다.

다음 섹션은 이 리트머스 테스트를 어떻게 수행하는지 설명합니다.

\iffalse

Putting all this together, the C-language equivalent to the entire litmus
test is as shown in
\cref{lst:formal:Meaning of PPCMEM Litmus Test}.
The key point is that if \co{atomic_add_return()} acts as a full
memory barrier (as the Linux kernel requires it to), 
then it should be impossible for \co{P0()}'s and \co{P1()}'s \co{r3}
variables to both be zero after execution completes.

The next section describes how to run this litmus test.

\fi

\subsection{Running a Litmus Test}
\label{sec:formal:Running a Litmus Test}

앞서 언급되었듯, 리트머스 테스트는 메모리 모델에 대한 이해를 도울 수 있는
\url{https://www.cl.cam.ac.uk/~pes20/ppcmem/} 를 통해 대화형태로 수행될 수
있습니다.
그러나, 이 방법은 사용자가 전체 상태공간 탐색을 일일이 진행할 것을 필요로
합니다.
여러분이 모든 가능한 이벤트 순서를 검사하는 것은 매우 어려운 게 분명하므로,
이를 위한 목적의 별도의 도구가 제공됩니다~\cite{PaulEMcKenney2011ppcmem}.

\iffalse

As noted earlier, litmus tests may be run interactively via
\url{https://www.cl.cam.ac.uk/~pes20/ppcmem/}, which can help build an
understanding of the memory model.
However, this approach requires that the user manually carry out the
full state-space search.
Because it is very difficult to be sure that you have checked every
possible sequence of events, a separate tool is provided for this
purpose~\cite{PaulEMcKenney2011ppcmem}.

\fi

\begin{listing}[tbp]
\begin{VerbatimL}[numbers=none,xleftmargin=0pt]
./ppcmem -model lwsync_read_block \
         -model coherence_points filename.litmus
...
States 6
0:r3=0; 1:r3=0;
0:r3=0; 1:r3=1;
0:r3=1; 1:r3=0;
0:r3=1; 1:r3=1;
0:r3=2; 1:r3=0;
0:r3=2; 1:r3=1;
Ok
Condition exists (0:r3=0 /\ 1:r3=0)
Hash=e2240ce2072a2610c034ccd4fc964e77
Observation SB+lwsync-RMW-lwsync+isync Sometimes 1
\end{VerbatimL}
\caption{PPCMEM Detects an Error}
\label{lst:formal:PPCMEM Detects an Error}
\end{listing}

\Cref{lst:formal:PPCMEM Litmus Test}
에 보인 리트머스 테스트는 read-modify-write 명령을 포함하므로, 우린 이 커맨드
라인에 \co{-model} 인자를 더해야 합니다.
이 리트머스 테스트가 \co{filename.litmus} 에 저장되어 있다면, 이는
\cref{lst:formal:PPCMEM Detects an Error} 에 보인 출력을 낼텐데, 여기서
\co{...} 는 큰 양의 진행을 알리는 출력물을 의미합니다.
상태들은 \co{0:r3=0; 1:r3=0;} 를 포함하는데, \co{atomic_add_return()} 의 구형
PowerPC 구현이 전체 배리어로 동작하지 않음을 다시 알립니다.
마지막 줄의 ``Sometimes'' 는 이를 알립니다: 이 단정문이 항상은 아니지만 일부
수행에서는 발동되었습니다.

\iffalse

Because the litmus test shown in
\cref{lst:formal:PPCMEM Litmus Test}
contains read-modify-write instructions, we must add \co{-model}
arguments to the command line.
If the litmus test is stored in \co{filename.litmus},
this will result in the output shown in
\cref{lst:formal:PPCMEM Detects an Error},
where the \co{...} stands for voluminous making-progress output.
The list of states includes \co{0:r3=0; 1:r3=0;}, indicating once again
that the old PowerPC implementation of \co{atomic_add_return()} does
not act as a full barrier.
The ``Sometimes'' on the last line confirms this: the assertion triggers
for some executions, but not all of the time.

\fi

\begin{listing}[tbp]
\begin{VerbatimL}[numbers=none,xleftmargin=0pt]
./ppcmem -model lwsync_read_block \
         -model coherence_points filename.litmus
...
States 5
0:r3=0; 1:r3=1;
0:r3=1; 1:r3=0;
0:r3=1; 1:r3=1;
0:r3=2; 1:r3=0;
0:r3=2; 1:r3=1;
No (allowed not found)
Condition exists (0:r3=0 /\ 1:r3=0)
Hash=77dd723cda9981248ea4459fcdf6097d
Observation SB+lwsync-RMW-lwsync+sync Never 0 5
\end{VerbatimL}
\caption{PPCMEM on Repaired Litmus Test}
\label{lst:formal:PPCMEM on Repaired Litmus Test}
\end{listing}

이 리눅스 커널 버그의 수정은 P0 의 \co{isync} 를 \co{sync} 로 바꾸는 것으로,
\cref{lst:formal:PPCMEM on Repaired Litmus Test} 에 보인 형태가 됩니다.
여기서 볼 수 있듯, \co{0:r3=0; 1:r3=0;} 는 상태 리스트에 나타나지 않으며,
마지막 행은 ``Never'' 라고 말합니다.
따라서, 이 모델은 공격 수행 시퀀스는 일어날 수 없다고 예측합니다.

\iffalse

The fix to this Linux-kernel bug is to replace P0's \co{isync} with
\co{sync}, which results in the output shown in
\cref{lst:formal:PPCMEM on Repaired Litmus Test}.
As you can see, \co{0:r3=0; 1:r3=0;} does not appear in the list of states,
and the last line calls out ``Never''.
Therefore, the model predicts that the offending execution sequence
cannot happen.

\fi

\QuickQuizSeries{%
\QuickQuizB{
	\ARM\ 리눅스 커널도 비슷한 버그를 가지고 있나요?

	\iffalse

	Does the \ARM\ Linux kernel have a similar bug?

	\fi

}\QuickQuizAnswer{
	\ARM\ 은 \co{atomic_add_return()} 함수의 어셈블리어 구현 전후에
	\co{smp_bm()} 를 위치하기 때문에 이 버그를 가지고 있지 않습니다.
	PowerPC 도 이 버그를 더이상 가지고 있지 않습니다; 이건 한참 전에
	고쳐졌습니다~\cite{BenjaminHerrenschmidt2011:powerpc:atomic_return}.

	\iffalse

	\ARM\ does not have this particular bug because it places
	\co{smp_mb()} before and after the \co{atomic_add_return()}
	function's assembly-language implementation.
	PowerPC no longer has this bug; it has long since been
	fixed~\cite{BenjaminHerrenschmidt2011:powerpc:atomic_return}.

	\fi

}\QuickQuizEndB
%
\QuickQuizE{
	\begin{fcvref}[ln:formal:PPCMEM Litmus Test]
	\Cref{lst:formal:PPCMEM Litmus Test} provide sufficient ordering
	의 \clnref{P0lwsync} 에 있는 \co{lwsync} 는 충분한 순서 규칙을
	제공하나요?
	\end{fcvref}

	\iffalse

	\begin{fcvref}[ln:formal:PPCMEM Litmus Test]
	Does the \co{lwsync} on \clnref{P0lwsync} in
	\cref{lst:formal:PPCMEM Litmus Test} provide sufficient ordering?
	\end{fcvref}

	\fi

}\QuickQuizAnswerE{
	필요한 의미에 따라 다릅니다.
	이 답의 나머지 부분은
	\cref{lst:formal:PPCMEM Litmus Test} 의 \co{P0} 를 위한 어셈블리어가
	값을 반환하는 어토믹 오퍼레이션을 구현할 것으로 여긴다고 가정합니다.

	\Cref{chp:Advanced Synchronization: Memory Ordering} 에서 이야기 했듯,
	리눅스 커널의 메모리 일관성 모델은 양쪽에서 모두 완전히 순서잡히기 위해
	값을 반환하는 어토믹 RMW 오퍼레이션을 필요로 합니다.
	\co{lwsync} 로 제공되는 순서는 이 목적에 불충분하며, 따라서 \co{sync}
	가 대신 사용되어야 합니다.
	이 변경은 다른 두개의 리트머스
	테스트를~\cite{Paulmck2015:powerpc:value-returning-atomics} 다뤘던
	이메일 쓰레드에 대한 대답의 일환으로
	만들어졌습니다~\cite{BoqunFeng2015:powerpc:value-returning-atomics}.
	리눅스 커널이 가지고 있을 수도 있는 모든 다른 버그의 반견은 독자
	여러분의 연습문제로 남겨둡니다.

	더 완화된 의미를 제공하는 다른 환경에서는 \co{lwsync} 가 충분할 수도
	있습니다.
	하지만 리눅스 커널의 값 반환 어토믹 오퍼레이션에서는 아닙니다!

	\iffalse

	It depends on the semantics required.
	The rest of this answer assumes that the assembly language
	for \co{P0} in
	\cref{lst:formal:PPCMEM Litmus Test}
	is supposed to implement a value-returning atomic operation.

	As is discussed in
	\cref{chp:Advanced Synchronization: Memory Ordering},
	Linux kernel's memory consistency model requires
	value-returning atomic RMW operations to be fully ordered
	on both sides.
	The ordering provided by \co{lwsync} is insufficient for this
	purpose, and so \co{sync} should be used instead.
	This change has since been
	made~\cite{BoqunFeng2015:powerpc:value-returning-atomics}
	in response to an email thread discussing a couple of other litmus
	tests~\cite{Paulmck2015:powerpc:value-returning-atomics}.
	Finding any other bugs that the Linux kernel might have is left
	as an exercise for the reader.

	In other enviroments providing weaker semantics, \co{lwsync}
	might be sufficient.
	But not for the Linux kernel's value-returning atomic operations!

	\fi

}\QuickQuizEndE
}

\subsection{PPCMEM Discussion}
\label{sec:formal:PPCMEM Discussion}

이 도구들은 \ARM\ 와 Power 에서 수행되는 저수준 병렬 기능을 작업하는 사람들에게
큰 도움이 될 것을 약속합니다.
이 도구들은 본질적인 한계도 가지고 있습니다:

\iffalse

These tools promise to be of great help to people working on low-level
parallel primitives that run on \ARM\ and on Power. These tools do have
some intrinsic limitations:

\fi

\begin{enumerate}
\item	이 도구들은 연구용 프로토타입이며, 따라서 지원되지 않는 경우들이
	있습니다.
\item	이 도구들은 IBM 이나 \ARM\ 에 의해 각각의 CPU 아키텍쳐에 대한 공식적
	성명을 갖추지 못했습니다.
	예를 들어, 두 회사 모두 이 도구들의 어떤 버전에 대해서든 언제든 버그를
	보고할 권리를 가지고 있습니다.
	따라서 이 도구들은 실제 하드웨어에서의 철저한 스트레스 테스트의
	대체물이 될 수 없습니다.
	더 나아가서, 이것들이 기반하고 있는 도구들과 모델 모두 여전히 개발
	중이며 언제든 바뀔 수 있습니다.
	다른 한편, 이 모델은 연관된 하드웨어 전문가의 자문 아래 개발되었으므로,
	이게 해당 아키텍쳐에 대한 충분한 표현이라고 자신을 가질 좋은 이유가
	있기도 합니다.
\item	이 도구들은 명령 집합의 부분집합만을 다룹니다.
	이 부분집ㅎ바은 많은 목적에 있어 충분했으나, 여러분의 목표는 다양할
	겁니다.
	특히, 이 도구는 워드 (word) 크기 (32 비트) 액세스만을 처리하며, 그
	워드는 올바르게 정렬되어 있어야만 합니다.\footnote{
		하지만 최근의 연구는 여러 크기가 혼재된 액세스도 주목하고
		있습니다~\cite{Flur:2017:MCA:3093333.3009839}.}
	또한, 이 도구는 더 완화된 \ARM\ 메모리 배리어 명령 변종들 일부도,
	산술도 다루지 않습니다.

\iffalse

\item	These tools are research prototypes, and as such are unsupported.
\item	These tools do not constitute official statements by IBM or \ARM\
	on their respective CPU architectures. For example, both
	corporations reserve the right to report a bug at any time against
	any version of any of these tools. These tools are therefore not a
	substitute for careful stress testing on real hardware. Moreover,
	both the tools and the model that they are based on are under
	active development and might change at any time. On the other
	hand, this model was developed in consultation with the relevant
	hardware experts, so there is good reason to be confident that
	it is a robust representation of the architectures.
\item	These tools currently handle a subset of the instruction set.
	This subset has been sufficient for my purposes, but your mileage
	may vary. In particular, the tool handles only word-sized accesses
	(32 bits), and the words accessed must be properly aligned.\footnote{
		But recent work focuses on mixed-size
		accesses~\cite{Flur:2017:MCA:3093333.3009839}.}
	In addition, the tool does not handle some of the weaker variants
	of the \ARM\ memory-barrier instructions, nor does it handle
	arithmetic.

\fi

\item	이 도구들은 작은 수의 쓰레드에서 돌아가는 작은 수의 반복문 없는 코드
	조각들에 제한되어 있습니다.
	더 큰 예제는 Promela 와 Spin 같은 비슷한 도구에서와 같이 상태 공간
	폭증을 야기합니다.
\item	전체 상태 공간 탐색은 어떻게 각 공격이 되는 상태에 이르렀는지를 알리지
	않습니다.
	그러나, 일단 그 상태에 닿는게 가능하다는 걸 알게 된다면 대화형 도구를
	사용해 그 상태를 찾는건 너무 어렵지 않은게 일반적입니다.
\item	이 도구들은 복잡한 데이터 구조에는 썩 좋지 못합니다만,
	\qco{x=y; y=z; z=43;} 형태의 초기화 명령들을 이용해 극단적으로 간단한
	링크드 리스트 순회를 만드는 건 가능합니다.
\item	이 도구들은 memory mapped I/O 나 기기 레지스터들을 다루지 않습니다.
	물론, 그런 것들을 처리하는 것은 그것들이 공식화 될 것을 필요로 하는데,
	아직은 공식화 되지 않은 것으로 보입니다.
\item	이 도구들은 여러분이 단정문을 짜넣는 문제들만 탐지합니다.
	이 약점은 모든 정형적 방법들에 공통된 것이며, 테스트가 여전히 중요한
	또다른 이유입니다.
	이 챕터의 시작에서 인용된 Donald Knuth 의 불멸의 명언을 인용하자면,
	``Beware of bugs in the above code; I have only proved it correct, not
	tried it.''

\iffalse

\item	The tools are restricted to small loop-free code fragments
	running on small numbers of threads. Larger examples result
	in state-space explosion, just as with similar tools such as
	Promela and Spin.
\item	The full state-space search does not give any indication of how
	each offending state was reached. That said, once you realize
	that the state is in fact reachable, it is usually not too hard
	to find that state using the interactive tool.
\item	These tools are not much good for complex data structures, although
	it is possible to create and traverse extremely simple linked
	lists using initialization statements of the form
	\qco{x=y; y=z; z=42;}.
\item	These tools do not handle memory mapped I/O or device registers.
	Of course, handling such things would require that they be
	formalized, which does not appear to be in the offing.
\item	The tools will detect only those problems for which you code an
	assertion. This weakness is common to all formal methods, and
	is yet another reason why testing remains important. In the
	immortal words of Donald Knuth quoted at the beginning of this
	chapter, ``Beware of bugs in the above
	code; I have only proved it correct, not tried it.''

\fi

\end{enumerate}

그러나, 이 도구들의 강점들 중 하나는 이 아키텍쳐들에 허용된 동작들 전체를
모델링 하도록 설계되었다는 것으로, 그 동작들은 합법적이지만 현재의 하드웨어
구현이 아직 부주의한 소프트웨어 개발자들에게 영향을 주지 않은 것들을
포함합니다.
따라서, 이 도구들에 의해 신뢰되는 알고리즘은 실제 하드웨어에서 수행될때 약간의
추가적 안전성 허용범위를 갖습니다.
더 나아가, 실제 하드웨어에서의 테스트는 버그를 발견할 수만 있습니다; 그런
테스트는 근본적으로 어떤 사용이 올바른지를 증명할 수는 없습니다.
이를 이해하려면, 연구자들이 그들의 모델을 검증하기 위해 실제 하드웨어에서
1000억개의 테스트를 주기적으로 수행했음을 생각해 보세요.
그 중 한 경우, 아키텍쳐상으로는 허용된 동작인 1760억 회의 수행에도 불구하고
발생하지 않았습니다~\cite{JadeAlglave2011ppcmem}.
대조적으로, 전체 상태 공간 탐색은 이 도구가 코드 조각의 올바름을 증명할 수 있게
합니다.

\iffalse

That said, one strength of these tools is that they are designed to
model the full range of behaviors allowed by the architectures, including
behaviors that are legal, but which current hardware implementations do
not yet inflict on unwary software developers. Therefore, an algorithm
that is vetted by these tools likely has some additional safety margin
when running on real hardware. Furthermore, testing on real hardware can
only find bugs; such testing is inherently incapable of proving a given
usage correct. To appreciate this, consider that the researchers
routinely ran in excess of 100 billion test runs on real hardware to
validate their model.
In one case, behavior that is allowed by the architecture did not occur,
despite 176 billion runs~\cite{JadeAlglave2011ppcmem}.
In contrast, the
full-state-space search allows the tool to prove code fragments correct.

\fi

정형적 방법론들과 도구들은 테스트의 대체물이 될 수 없음을 한번 더 반복할 가치가
있습니다.
중요한 사실은, 예를 들어 리눅스 커널과 같은 거대하며 신뢰성 있는 동시성
소프트웨어 작품을 만드는 것은 매우 어렵다는 것입니다.
따라서 개발자들은 이 목표를 위해 모든 도구를 적용할 준비를 해야 합니다.
이 챕터에서 선보인 도구들은 테스트를 통해 발견하기 (추적하기도) 매우 어려운
버그들을 찾을 수 있게 합니다.
다른 한편, 테스트는 이 챕터에 선보인 도구들이 영원히 처리하지 못할만큼 큰
몸체의 소프트웨어에 적용될 수 있습니다.
항상 그렇듯, 그 일에 맞는 도구를 사용하세요!

물론, 여러분의 일이 더 간단해지게끔 병렬 코드를 쉽게 분할되게끔 설계하고 더
높은 수준의 기능들을 (락, sequence counter, 어토믹 오퍼레이션, 그리고 RCU 같은)
사용해서 이 단계의 일을 회피하는게 최고입니다.
그리고 여러분이 정말로 저수준 메모리 배리어와 read-modify-write 명령들을
사용해야만 한다고 하더라도, 여러분의 더 보수적인 이런 날카로운 도구들의 사용이
여러분의 삶은 더 쉽게 만들어 줄 겁니다.

\iffalse

It is worth repeating that formal methods and tools are no substitute for
testing. The fact is that producing large reliable concurrent software
artifacts, the Linux kernel for example, is quite difficult. Developers
must therefore be prepared to apply every tool at their disposal towards
this goal. The tools presented in this chapter are able to locate bugs that
are quite difficult to produce (let alone track down) via testing. On the
other hand, testing can be applied to far larger bodies of software than
the tools presented in this chapter are ever likely to handle. As always,
use the right tools for the job!

Of course, it is always best to avoid the need to work at this level
by designing your parallel code to be easily partitioned and then
using higher-level primitives (such as locks, sequence counters, atomic
operations, and RCU) to get your job done more straightforwardly. And even
if you absolutely must use low-level memory barriers and read-modify-write
instructions to get your job done, the more conservative your use of
these sharp instruments, the easier your life is likely to be.

\fi

% formal/axiomatic.tex

\section{Axiomatic Approaches}
\label{sec:formal:Axiomatic Approaches}
\OriginallyPublished{Section}{sec:formal:Axiomatic Approaches}{Axiomatic Approaches}{Linux Weekly News}{PaulEMcKenney2014weakaxiom}

\begin{listing*}[tb]
{ \scriptsize
\begin{verbbox}
 1 PPC IRIW.litmus
 2 ""
 3 (* Traditional IRIW. *)
 4 {
 5 0:r1=1; 0:r2=x;
 6 1:r1=1;         1:r4=y;
 7         2:r2=x; 2:r4=y;
 8         3:r2=x; 3:r4=y;
 9 }
10 P0           | P1           | P2           | P3           ;
11 stw r1,0(r2) | stw r1,0(r4) | lwz r3,0(r2) | lwz r3,0(r4) ;
12              |              | sync         | sync         ;
13              |              | lwz r5,0(r4) | lwz r5,0(r2) ;
14 
15 exists
16 (2:r3=1 /\ 2:r5=0 /\ 3:r3=1 /\ 3:r5=0)
\end{verbbox}
}
\centering
\theverbbox
\caption{IRIW Litmus Test}
\label{lst:formal:IRIW Litmus Test}
\end{listing*}

PPCMEM 도구가
Figure~\ref{fig:formal:IRIW Litmus Test} 에 보여진,
유명한 ``independent reads of independent writes'' (IRIW) 리트머스 테스트를
해곃할 수 있긴 합니다만, 그러기 위해선 14 시간 이상의 CPU 시간과 10 기가바이트
이상의 상태공간을 필요로 합니다.
그렇다곤 하나, 이 상황은 이 문제를 풀기 위해선 커다란 레퍼런스 매뉴얼을
뒤져보고, 증명을 시도하고, 전문가와 토론을 하고, 마지막으로 내린 답에 대해서도
확신할 수 없었던, PPCMEM 이 나오기 전에 비하면 커다란 개선이 이뤄진 것입니다.
비록 14 시간은 긴 시간처럼 보일 수 있겠지만, 이는 수 주나 수 개월에 비하면
너무나도 짧은 시간입니다.

하지만, 두개의 쓰레드가 두개의 별도의 변수에 값을 쓰고 두개의 다른 쓰레드가 이
두개의 변수로부터 반대 순서로 값을 읽어들일 뿐인 해당 리트머스 테스트의
단순성을 놓고 보면 요구되는 해당 시간은 조금 놀랍습니다.
단정문은 두개의 값을 읽어들이는 쓰레드들이 두개의 값 저장의 순서에 대해 서로
다른 의견을 갖는다면 터집니다.
이 리트머스 테스트는 긴단한데, 표준적 메모리 순서 리트머스 테스트들을 놓고 봐도
그렇습니다.
\iffalse

Although the PPCMEM tool can solve the famous ``independent reads of
independent writes'' (IRIW) litmus test shown in
Listing~\ref{lst:formal:IRIW Litmus Test}, doing so requires no less than
fourteen CPU hours and generates no less than ten gigabytes of state space.
That said, this situation is a great improvement over that before the advent
of PPCMEM, where solving this problem required perusing volumes of
reference manuals, attempting proofs, discussing with experts, and
being unsure of the final answer.
Although fourteen hours can seem like a long time, it is much shorter
than weeks or even months.

However, the time required is a bit surprising given the simplicity
of the litmus test, which has two threads storing to two separate variables
and two other threads loading from these two variables in opposite
orders.
The assertion triggers if the two loading threads disagree on the order
of the two stores.
This litmus test is simple, even by the standards of memory-order litmus
tests.
\fi

소모되는 시간과 공간의 양에 대한 한가지 이유는 PPCMEM 이 추적 기반의 전체 상태
공간 탐색을 한다는 것으로, 이는 아키텍쳐 단계에서 이벤트들의 모든 가능한 순서와
조합을 만들어내고 수행해 봐야 함을 의미합니다.
이 단계에서, 화려한 이벤트와 액션들의 시퀀스에 연관된 로드와 스토어는 모두 모두
탐색되어야만 하는 매우 커다란 상태 공간을 초래하게 되고, 이는 커다란 메모리와
CPU 소모로 이어지게 됩니다.

물론, 그런 추적들 가운데 많은 것들은 다른 것들과 상당히 유사해서, 비슷한
추적들을 하나로 취급하는 것이 성능을 개선시킬 수도 있을 것임을 시사합니다.
그런 한가지 방법이 Alglave 등의 공리적 집합론
방법~\cite{Alglave:2014:HCM:2594291.2594347} 으로, 여기선 메모리 모델을
나타내기 위한 공리 집합을 만들어내고 리트머스 테스트들을 이 공리들의 집합으로
증명되거나 반증될 수 있는 정리로 변환시킵니다.
이렇게 만들어진, ``herd'' 라 불리는 도구는 편리하게 PPCMEM 에서와 같은 리트머스
테스트들을 입력으로 받는데,
Figure~\ref{fig:formal:IRIW Litmus Test} 에 보인 IRIW 리트머스 테스트도
포함됩니다.
\iffalse

One reason for the amount of time and space consumed is that PPCMEM does
a trace-based full-state-space search, which means that it must generate
and evaluate all possible orders and combinations of events at the
architectural level.
At this level, both loads and stores correspond to ornate sequences
of events and actions, resulting in a very large state space that must
be completely searched, in turn resulting in large memory and CPU
consumption.

Of course, many of the traces are quite similar to one another, which
suggests that an approach that treated similar traces as one might
improve performace.
One such approach is the axiomatic approach of
Alglave et al.~\cite{Alglave:2014:HCM:2594291.2594347},
which creates a set of axioms to represent the memory model and then
converts litmus tests to theorems that might be proven or disproven
over this set of axioms.
The resulting tool, called ``herd'',  conveniently takes as input the
same litmus tests as PPCMEM, including the IRIW litmus test shown in
Listing~\ref{lst:formal:IRIW Litmus Test}.
\fi

\begin{listing*}[tb]
{ \scriptsize
\begin{verbbox}
 1 PPC IRIW5.litmus
 2 ""
 3 (* Traditional IRIW, but with five stores instead of just one. *)
 4 {
 5 0:r1=1; 0:r2=x;
 6 1:r1=1;         1:r4=y;
 7         2:r2=x; 2:r4=y;
 8         3:r2=x; 3:r4=y;
 9 }
10 P0           | P1           | P2           | P3           ;
11 stw r1,0(r2) | stw r1,0(r4) | lwz r3,0(r2) | lwz r3,0(r4) ;
12 addi r1,r1,1 | addi r1,r1,1 | sync         | sync         ;
13 stw r1,0(r2) | stw r1,0(r4) | lwz r5,0(r4) | lwz r5,0(r2) ;
14 addi r1,r1,1 | addi r1,r1,1 |              |              ;
15 stw r1,0(r2) | stw r1,0(r4) |              |              ;
16 addi r1,r1,1 | addi r1,r1,1 |              |              ;
17 stw r1,0(r2) | stw r1,0(r4) |              |              ;
18 addi r1,r1,1 | addi r1,r1,1 |              |              ;
19 stw r1,0(r2) | stw r1,0(r4) |              |              ;
20 
21 exists
22 (2:r3=1 /\ 2:r5=0 /\ 3:r3=1 /\ 3:r5=0)
\end{verbbox}
}
\centering
\theverbbox
\caption{Expanded IRIW Litmus Test}
\label{lst:formal:Expanded IRIW Litmus Test}
\end{listing*}

IRIW 를 푸는데에 PPCMEM 이 14 CPU 시간을 필요로 하는 반면, herd 는 17
밀리세컨드만에 IRIW 를 푸는데, 이는 백만배 이상의 속도 향상을 의미합니다.
그렇다곤 하나, 문제는 근본적으로 기하급수적이므로, 더 커다란 문제들에 있어서는
herd 역시 기하급수적으로 느려질 것을 예상해야 합니다.
그리고 이는 실제로 일어나는 일인데, 예를 들어 우리가
Figure~\ref{fig:formal:Expanded IRIW Litmus Test} 에 보여진 것처럼 쓰기를 하는
CPU 마다 네개의 쓰기를 추가하면, herd 는 50,000 배 이상 느려져서 15 \emph{분}
이상의 CPU 시간을 필요로 하게 됩니다.
쓰레드를 추가하는 것 역시 기하급수적인 속도저하를
초래합니다~\cite{PaulEMcKenney2014weakaxiom}.
\iffalse

However, where PPCMEM requires 14 CPU hours to solve IRIW, herd does so
in 17 milliseconds, which represents a speedup of more than six orders
of magnitude.
That said, the problem is exponential in nature, so we should expect
herd to exhibit exponential slowdowns for larger problems.
And this is exactly what happens, for example, if we add four more writes
per writing CPU as shown in
Listing~\ref{lst:formal:Expanded IRIW Litmus Test},
herd slows down by a factor of more than 50,000, requiring more than
15 \emph{minutes} of CPU time.
Adding threads also results in exponential
slowdowns~\cite{PaulEMcKenney2014weakaxiom}.
\fi

이런 근본적 기하급수적 성질에도 불구하고, PPCMEM 과 herd 는 x86 시스템에서의
queued-lock handoff 를 포함해서 핵심적 병렬 알고리즘을 체크하는데에 상당히
유용한 것으로 증명되었습니다.
Herd 의 약점은
Section~\ref{sec:formal:PPCMEM Discussion} 에서 이야기한 PPCMEM 의 그것과
유사합니다.
PPCMEM 과 herd 가 동의하지 않게 되는 불분명한 (하지만 매우 현실적인) 경우들이
존재하는데 2014년 말 현재까지는 이 이견문제를 해결하는 노력이 진행 중입니다.

장기적으로, 희망은 공리적 방법이 더 높은 단계의 소프트웨어의 것들을 설명하는
공리들을 포함하는 것입니다.
이는 잠재적으로 훨씬 더 커다란 소프트웨어 시스템의 공리적 증명을 가능하게
할것입니다.
또다른 대안은 다음 섹션에서 설명하는 것처럼 이진 논리의 공리들을 제공하는
것으로, 다음 섹션에서 다룹니다.
\iffalse

Despite their exponential nature, both PPCMEM and herd have proven quite
useful for checking key parallel algorithms, including the queued-lock
handoff on x86 systems.
The weaknesses of the herd tool are similar to those of PPCMEM, which
were described in
Section~\ref{sec:formal:PPCMEM Discussion}.
There are some obscure (but very real) cases for which the PPCMEM and
herd tools disagree, and as of late 2014 resolving these disagreements
was ongoing.

Longer term, the hope is that the axiomatic approaches incorporate
axioms describing higher-level software artifacts.
This could potentially allow axiomatic verification of much larger
software systems.
Another alternative is to press the axioms of boolean logic into service,
as described in the next section.
\fi

% formal/sat.tex
% mainfile: ../perfbook.tex
% SPDX-License-Identifier: CC-BY-SA-3.0

\section{SAT Solvers}
\label{sec:formal:SAT Solvers}
%
\epigraph{Live by the heuristic, die by the heuristic.}{\emph{Unknown}}

경계가 정해진 반복문과 재귀를 갖는 유한한 프로그램은 논리 표현으로 변환될 수
있는데, 이는 이 프로그램의 단정들을 그 입력으로 표현할 수도 있습니다.
그런 논리적 표현을 가지면 어떤 가능한 입력의 조합이 그런 단정 중 하나가 터지게
할 수 있는지 알아보는게 꽤 흥미로울 겁니다.
그 입력이 이진 변수들의 조합으로 표현된다면 이는 satisfiability 문제라고도
알려져 있는, 단순한 SAT 입니다.
SAT 풀이기는 하드웨어 검증에 널리 사용되며, 상당한 진보의 모티베이션이
되었습니다.
1990년대 초의 세계급 SAT 풀이기는 100개의 이진 변수들의 논리 표현을 처리할 수
있을 수도 있지만, 2010년대 초에는 100만개 변수의 SAT 풀이기도 사용
가능해졌습니다~\cite{Kroening:2008:DPA:1391237}.

\iffalse

Any finite program with bounded loops and recursion can be converted
into a logic expression, which might express that program's assertions
in terms of its inputs.
Given such a logic expression, it would be quite interesting to know
whether any possible combinations of inputs could result in one of
the assertions triggering.
If the inputs are expressed as combinations of boolean variables,
this is simply SAT, also known as the satisfiability problem.
SAT solvers are heavily used in verification of hardware, which has
motivated great advances.
A world-class early 1990s SAT solver might be able to handle a logic
expression with 100 distinct boolean variables, but by the early 2010s
million-variable SAT solvers were readily
available~\cite{Kroening:2008:DPA:1391237}.

\fi

\begin{figure}[tbp]
\centering
\resizebox{2in}{!}{\includegraphics{formal/cbmc}}
\caption{CBMC Processing Flow}
\label{fig:formal:CBMC Processing Flow}
\end{figure}

또한, SAT 풀이기를 위한 프론트엔드 프로그램은 C 코드를 자동으로 논리 표현으로
변환할 수 있어서, 단정들을 받아서 array-bounds 오류 같은 오류 조건들을 위한
단정들을 만들 수 있습니다.
한 예는 C bounded model check, 또는 \co{cbmc} 라 불리는 것으로, 많은 리눅스
배포판에서 사용 가능합니다.
이 도구는 사용하기가 무척 쉬운데, \co{cbmc test.c} 가 \path{test.c} 를
검사하는데 충분하며, Figure~\ref{fig:formal:CBMC Processing Flow} 에 보인 처리
플로우를 만듭니다.
이 쉬운 사용성은 정형적 검증이 재귀 테스트 프레임웍에 포함되는 것을 가능하게
하므로 무척이나 중요합니다.
대조적으로, 특수 목적 언어로의 간단핮 ㅣ않은 변환을 필요로 하는 전통적 도구들은
설계 시점 검증에만 제한됩니다.

\iffalse

In addition, front-end programs for SAT solvers can automatically translate
C code into logic expressions, taking assertions into account and generating
assertions for error conditions such as array-bounds errors.
One example is the C bounded model checker, or \co{cbmc}, which is
available as part of many Linux distributions.
This tool is quite easy to use, with \co{cbmc test.c} sufficing to
validate \path{test.c}, resulting in the processing flow shown in
Figure~\ref{fig:formal:CBMC Processing Flow}.
This ease of use is exceedingly important because it opens the door
to formal verification being incorporated into regression-testing
frameworks.
In contrast, the traditional tools that require non-trivial translation
to a special-purpose language are confined to design-time verification.

\fi

더 최근 들어, SAT 풀이기는 병렬 코드를 처리하기 시작했습니다.
이 풀이기들은 입력 코드를 single static assignment (SSA) 형태로 변환하고, 모든
허용된 액세스 순서들을 생성합니다.
이 접근법은 잘 동작할 것 같습니다만, 실전에서 얼마나 잘 사용될지는 두고봐야
하겠습니다.
한가지 좋은 신호는 \co{cbmc} 를 리눅스 커널 RCU 에 적용한 2016년의
작업물입니다~\cite{LihaoLiang2016VerifyTreeRCU,Liang:2018:VTB,LanceRoy2017CBMC-SRCU}.
이 작업에서는 RCU 의 최소 설정을 사용하고 작은 수의 쓰레드를 사용하는
시나리오를 검증했지만, 성공적으로 리눅스 커널 C 코드를 사용했고 유용한 결과를
만들어냈습니다.
이 C 코드로부터 만들어진 논리 표현은 9000만개의 변수와 4500만개의 절을 가져서
수십 기가바이트의 메모리를 사용했으며 이 SAT 풀이기가 올바른 결과를 만들기까지
CPU 시간으로 80 시간을 요했습니다.

\iffalse

More recently, SAT solvers have appeared that handle parallel code.
These solvers operate by converting the input code into single static
assignment (SSA) form, then generating all permitted access orders.
This approach seems promising, but it remains to be seen how well
it works in practice.
One encouraging sign is work in 2016 applying \co{cbmc} to Linux-kernel
RCU~\cite{LihaoLiang2016VerifyTreeRCU,Liang:2018:VTB,LanceRoy2017CBMC-SRCU}.
This work used minimal configurations of RCU, and verified scenarios
using small numbers of threads, but nevertheless successfully ingested
Linux-kernel C code and produced a useful result.
The logic expressions generated from the C code had up to 90~million
variables, 450~million clauses, occupied tens of gigabytes of memory,
and required up to 80~hours of CPU time for the SAT solver to produce
the correct result.

\fi

그러나, 리눅스 커널 해커는 그들의 코드가 자동으로 검증되었다는 주장에 대해
회의적인 느낌을 받을 수도 있으며, 그런 해커는 수십년 전부터 있어온 많은 다른
회의적 시각을 찾을 수 있을 겁니다~\cite{DeMillo:1979:SPP:359104.359106}.
그런 회의적 시각을 건설적으로 표현하는 한가지 방법은 검증되었다고 주장되는
코드의 버그가 내포된 버전을 제공하는 겁니다.
이 정형적 검증 도구가 그렇게 추가된 버그를 모두 찾는다면, 우리의 해커는 이
도구의 기능에 좀 더 자신을 얻을 수도 있습니다.
물론, 그 해커가 아직 인지하지 못하고 있는 버그를 찾는 도구는 그보다도 더한
만족을 발생시킬 겁니다.
그리고 이게 왜 \co{git} 이 20개의 다른 브랜치를 가지며 각 브랜치는 리눅스 커널
RCU 에 추가된 버그를 잠재적으로 갖는
이유입니다~\cite{PaulEMcKenney2017VerificationChallenge6}.
누구든 정형 검증 도구를 가지고 이 검증 도전 문제들에 도전해 보시기를 진심으로
환영하는 바입니다.

현재, \co{cbmc} 는 여러 추가된 버그를 찾을 수 있지만 RCU 메인테이너가 알고 있지
못하는 버그를 아직은 찾지 못했습니다.
그러나, SAT 풀이기가 언젠가 병렬 코드의 동시성 버그를 찾는데 유용해질 날을
희망할 이유가 여럿 있습니다.

\iffalse

Nevertheless, a Linux-kernel hacker might be justified in feeling skeptical
of a claim that his or her code had been automatically verified, and
such hackers would find many fellow skeptics going back
decades~\cite{DeMillo:1979:SPP:359104.359106}.
One way to productively express such skepticism is to provide bug-injected
versions of the allegedly verified code.
If the formal-verification tool finds all the injected bugs, our hacker
might gain more confidence in the tool's capabilities.
Of course, tools that find valid bugs of which the hacker was not yet aware
will likely engender even more confidence.
And this is exactly why there is a \co{git} archive with a 20-branch set
of mutations, with each branch potentially containing a bug injected
into Linux-kernel RCU~\cite{PaulEMcKenney2017VerificationChallenge6}.
Anyone with a formal-verification tool is cordially invited to try that
tool out on this set of verification challenges.

Currently, \co{cbmc} is able to find a number of injected bugs,
however, it has not yet been able to locate a bug that RCU's
maintainer was not already aware of.
Nevertheless, there is some reason to hope that SAT solvers will someday
be useful for finding concurrency bugs in parallel code.

\fi

% formal/stateless.tex
% mainfile: ../perfbook.tex
% SPDX-License-Identifier: CC-BY-SA-3.0

\section{Stateless Model Checkers}
\label{sec:formal:Stateless Model Checkers}
%
\epigraph{He's making a list, he's permuting it twice\dots}
	{\emph{with apologies to Haven Gillespie and J. Fred Coots}}

앞 섹션에서 설명한 SAT 풀이기 접근법은 상당히 편리하고 강력하지만, 상태를
포함한 모든 가능한 수행을 완전히 따라가는 것은 상당한 오버헤드를 일으킵니다.
실제로, 이 메모리와 CPU 시간 오버헤드는 적당하게 검증될 수 있는 프로그램의
크기를 급격하게 제한할 수 있는데, 이는 더 큰 프로그램에서는 덜 정확한 접근법이
버그를 찾을 수도 있을까 하는 질문을 일으킵니다.

\iffalse

The SAT-solver approaches described in the previous section are quite
convenient and powerful, but the full tracking of all possible
executions, including state, can incur substantial overhead.
In fact, the memory and CPU-time overheads can sharply limit the size
of programs that can be feasibly verified, which raises the question
of whether less-exact approaches might find bugs in larger programs.

\fi

\begin{figure}[tbp]
\centering
\resizebox{2.1in}{!}{\includegraphics{formal/nidhugg}}
\caption{Nidhugg Processing Flow}
\label{fig:formal:Nidhugg Processing Flow}
\end{figure}

여전히 배심원들은 이 질문에 대해 고민 중이지만,
Nidhugg~\cite{CarlLeonardsson2014Nidhugg} 와 같은 stateless 모델 검사기들은
Figure~\ref{fig:formal:Nidhugg Processing Flow} 에 보인 것처럼 몇몇 경우에
비슷한 수준의 쉬운 사용성을 가지고 더 큰 프로그램들을
처리했습니다~\cite{SMC-TreeRCU}.
또한, Nidhugg 는 리눅스 커널 RCU 검증 시나리오에서 \co{cbmc} 보다 수십배
빨랐습니다.
물론, Nidhugg 의 속도와 확장성 이득은 그게 데이터 비결정성을 처리하지 않는다는
사실에서 나옵니다만, 이 특정 검증 시나리오에서는 중요한 게 아니었습니다.

그렇다고 하나, \co{cbmc} 에서와 같이 Nidhugg 는 리눅스 커널 RCU 메인테이너가
아직 몰랐던 버그를 찾지는 못했습니다.
그러나, 리눅스 커널 RCU 의 한 역사적 버그는 그 메인테이너의 생각과 다른 커밋에
의해 수정되었음을 보이는 게 가능했는데, 이는 Nidhugg 와 같은 stateless 모델
검사기가 언젠가는 병렬 코드의 동시성 버그를 찾는데 유용해질 것이라는 추가적
희망을 갖게 해줍니다.

\iffalse

Although the jury is still out on this question, stateless model
checkers such as Nidhugg~\cite{CarlLeonardsson2014Nidhugg} have in
some cases handled larger programs~\cite{SMC-TreeRCU}, and with
similar ease of use, as illustrated by
Figure~\ref{fig:formal:Nidhugg Processing Flow}.
In addition, Nidhugg was more than an order of magnitude faster than
was \co{cbmc} for some Linux-kernel RCU verification scenarios.
Of course, Nidhugg's speed and scalability advantages are tied to
the fact that it does not handle data non-determinism, but this
was not a factor in these particular verification scenarios.

Nevertheless, as with \co{cbmc}, Nidhugg has not yet been able to
locate a bug that Linux-kernel RCU's maintainer was not already
aware of.
However, it was able to demonstrate that one historical bug in
Linux-kernel RCU was fixed by a different commit than the maintainer
thought, which gives some additional hope that stateless model checkers
like Nidhugg might someday be useful for finding concurrency bugs in
parallel code.

\fi


\section{Summary}
\label{sec:formal:Summary}
%
\epigraph{Western thought has focused on True-False;
	  it is high time to shift to Robust-Fragile.}
	 {\emph{summarized from Nassim Nicholas Taleb}}
% Full quote:
% Since Plato, Western thought and the theory of knowledge has focused on
% the notions of True-False; as commendable as that was, it is high time
% to shift the concern to Robust-Fragile, and social epistemology to the
% more serious problem of Sucker-Nonsucker.

이 챕터에서 설명한 형식적 검증 테크닉은 작은 병렬 알고리즘들을 검증하는데에는
매우 강력한 도구입니다만, 당신의 도구상자에 그것들만 있어선 안됩니다.
지난 수십년간의 형식적 검증에 대한 집중에도 불구하고, 테스트는 커다란 병렬
소프트웨어 시스템을 위한 대표적 검증 도구로
남아있습니다~\cite{JonathanCorbet2006lockdep,DaveJones2011Trinity,PaulEMcKenney2016Formal}.
\iffalse

The formal-verification techniques described in this chapter
are very powerful tools for validating small
parallel algorithms, but they should not be the only tools in your toolbox.
Despite decades of focus on formal verification, testing remains the
validation workhorse for large parallel software
systems~\cite{JonathanCorbet2006lockdep,DaveJones2011Trinity,PaulEMcKenney2016Formal}.
\fi

물론 이는 항상 그렇지는 않을 수도 있는 것이 사실입니다.
이를 확실히 하기 위해, 2017년 기준으로 200억개가 넘는 리눅스 커널의
인스턴스가 존재한다고 추정된다는 점을 생각해 보세요.
리눅스 커널에 평균적으로 백만년의 수행 중 한번 발생할 수 있는 버그가 있다고
생각해 봅시다.
앞의 챕터의 마지막에서 설명했다시피, 이 버그는 이 전체 설치된 환경에서
\emph{하루에} 50 번씩 나타날 겁니다.
하지만 대부분의 형식적 검증 테크닉은 매우 작은 코드에 대해서만 사용될 수 있다는
사실도 여전합니다.
그런 상황에서 동시적 코드를 짜는 사람들은 뭘 해야 할까요?
\iffalse

It is nevertheless quite possible that this will not always be the case.
To see this, consider that there is estimated to be more than twenty
billion instances of the Linux kernel as of 2017.
Suppose that the Linux kernel has a bug that manifests on average every million
years of runtime.
As noted at the end of the preceding chapter, this bug will be appearing
more than 50 times \emph{per day} across the installed base.
But the fact remains that most formal validation techniques can be used
only on very small code bases.
So what is a concurrency coder to do?
\fi

한가지 방법은 첫번째 버그, 첫번째로 관련있는 버그, 마지막으로 관련있는 버그,
그리고 마지막 버그를 찾는 것에 대해 생각해 보는 겁니다.

첫번째 버그는 코드 검사난 컴파일러의 조사를 통해 발견되어집니다.
최신의 컴파일러에 의해 제공되는, 지속적으로 세련되어지는 진단 기능이 가벼운
형태의 형식적 검증으로 여겨질 수도 있긴 하지만, 그것들을 그런 용어로 부르는건
흔하지 않습니다.
이는 부분적으로는 ``내가 그걸 사용하고 있다면, 그것은 형식적 검증일 수 없다''
라고 말하는, 이상한 실무자의 선입견 때문이기도 하고, 컴파일러 진단기능과 검증에
대한 연구 사이의 서로 다른 철학 때문이기도 합니다.
\iffalse

One approach is to think in terms of finding the first bug, the first
relevant bug, the last relevant bug, and the last bug.

The first bug is normally found via inspection or compiler diagnostics.
Although the increasingly sophisticated diagnostics provided by modern
compilers might be considered to be a lightweight sort of formal
verification, it is not common to think of them in those terms.
This is in part due to an odd practitioner prejudice which says
``If I am using it, it cannot be formal verification'' on the one
hand, and the large difference in sophistication between compiler
diagnostics and verification research on the other.
\fi

첫번째로 관련있는 버그는 코드 검사나 컴파일러 진단 기능으로 발견되어질 수도
있지만, 이 두 단계가 오타와 거짓 양성 반응만을 찾게 되는 것도 흔하지 않은 일은
아닙니다.
어떤 방식으로든, 실제 제품에서마주하게 될 버그들인 관련된 많은 버그들은 많은
경우 테스트를 통해 발견되어질 겁니다.

테스트가 예측된 사용 예에서 만들어졌든 진짜 사용 예에서 만들어졌든, 마지막
연관된 버그가 테스트를 통해 발견되어지는 것은 흔치 않은 일이 아닙니다.
이 상황은 형식적 검증에 대한 완전한 거부의 동기가 될 수도 있겠습니다만,
관계없는 버그들은 black-hat 공격의 덕에 최소한의 알맞은 사이즈에 있어서는
갑자기 관계있는 것이 되어버리는 안좋은 습관을 가지고 있습니다.
전체 소프트웨어 가운데 그 분포도를 지속적으로 높여가는 보안에 관련된
소프트웨어에 있어서는 마지막 버그를 찾아내고 고쳐내려는 강한 동기가 존재할 수
있습니다.
테스트는 마지막 버그를 찾아내는 것은 명백히 불가능하므로 형식적 검증에게만
가능한 역할이 존재합니다.
형식적 검증이 그 안으로 적용될 수 있다고만 하면 그런 역할이 있다는
이야기입니다.
이 챕터에서 보였듯이, 현재의 형식적 검증 시스템들은 상당히 제한적입니다.
\iffalse

Although the first relevant bug might be located via inspection or
compiler diagnostics, it is not unusual for these two steps to find
only typos and false positives.
Either way, the bulk of the relevant bugs, that is, those bugs that
might actually be encountered in production, will often be found via testing.

When testing is driven by anticipated or real use cases, it is not
uncommon for the last relevant bug to be located by testing.
This situation might motivate a complete rejection of formal verification,
however, irrelevant bugs have an annoying habit of suddenly becoming relevant
at the least convenient moment possible, courtesy of black-hat attacks.
For security-critical software, which appears to be a continually
increasing fraction of the total, there can thus be strong motivation
to find and fix the last bug.
Testing is demonstrably unable to find the last bug, so there is a
possible role for formal verification.
That is, there is such a role if and only if formal verification
proves capable of growing into it.
As this chapter has shown, current formal verification systems are
extremely limited.
\fi

\QuickQuiz{}
	하지만 충분히 낮은 단계의 소프트웨어는 모든 의도와 목적에 있어
	블랙햇으로부터의 공격에 내성이 있지 않나요?
	\iffalse

	But shouldn't sufficiently low-level software be for all intents
	and purposes immune to being exploited by black hats?
	\fi
\QuickQuizAnswer{
	불행히도, 그렇지 않습니다.

	한번은, Paul E. McKenney 는 리눅스 커널 RCU 가 그런 공격에 대해 내성이
	있다고 생각했는데, Row Hanner 의 발전은 그렇지 않음을 보였습니다.
	무엇보다도, 그 블랙햇들이 시스템의 DRAM 을 건들 수 있다면, RCU 를 건들
	수 있습니다.
	\iffalse

	Unfortunately, no.

	At one time, Paul E. McKenney felt that Linux-kernel RCU
	was immune to such exploits, but the advent of Row Hammer
	showed him otherwise.
	After all, if the black hats can hit the system's DRAM,
	they can hit RCU.
	\fi
} \QuickQuizEnd

또다른 방법은 형식적 검증이 많은 경우 테스트보다 적용하기가 어렵다는 점을
고려하는 겁니다.
물론 이는 문화적인 측면에서의 이야기일 수 있고, 형식적 검증이 더 많은
사람들에게 친숙해 질수록 더 쉽게 전파될 것이라는 희망을 가져볼 수도 있습니다.
그렇다고는 하나, 매우 간단한 테스트 사용은 임의의 거대한 소프트웨어 시스템에서
심각한 버그들을 찾아낼 수 있습니다.
반면에, 형식적 검증을 적용하기 위해 필요한 노력은 시스템의 크기가 증가할수록
극적으로 증가합니다.

전 20년이 넘도록 형식적 검증을 형식적 검증이 효력을 발휘하는 곳에서 필요할
때에만 사용했을 뿐인데, 설계 시점에서의, 무엇보다 중요한 소프트웨어 구성물의
작고 복잡한 부분의 검증에 있어 그랬습니다.
무엇보다 중요하지만 더 커다란 소프트웨어 구성물들은 물론 테스트로 검증했습니다.
\iffalse

Another approach is to consider that
formal verification is often much harder to use than is testing.
This is of course in part a cultural statement, and there is every reason
to hope that formal verification will be perceived to be easier as more
people become familiar with it.
That said, very simple test harnesses can find significant bugs in arbitrarily
large software systems.
In contrast, the effort required to apply formal verification seems to
increase dramatically as the system size increases.

I have nevertheless made occasional use of formal verification
for more than 20 years, playing to formal verification's strengths,
namely design-time verification of small complex portions of the overarching
software construct.
The larger overarching software construct is of course validated by testing.
\fi

\QuickQuiz{}
	L4 마이크로커널의 전체 검증을 생각해 보면, 이런 형식적 검증에 대한
	제한적인 시각은 약간 시대에 뒤쳐진 것 아닌가요?
	\iffalse

	In light of the full verification of the L4 microkernel,
	isn't this limited view of formal verification just a little
	bit obsolete?
	\fi
\QuickQuizAnswer{
	안타깝지만, 그렇지 않습니다.

	L4 마이크로커널의 첫번째 전체 검증은 많은 수의 Ph.D.~학생들 학생마다
	매우 느린 속도로 진행한, 손으로 하는 코드 검증으로 이루어진
	역작이었습니다.
	이런 수준의 노력은 대부분의 소프트웨어 프로젝트들에는 적용될 수가
	없는데, 변화의 비율이 너무 거대하기 때문입니다.
	더 나아가서, 비록 L4 마이크로커널이 형식적 검증의 시점에서 보기에는
	커다란 소프트웨어 작품이긴 합니다만, LLVM, \GCC, 리눅스 커널, Hadoop,
	MongoDB, 그 외의 커다란 많은 것들 등과 같은 많은 수의 프로젝트들에
	비교하면 매우 작은 것입니다.
	또한, 이 검증은 일부 연구자들이 인정했듯이, 한계를 가지고 있습니다:
	\url{https://wiki.sel4.systems/FrequentlyAskedQuestions#Does_seL4_have_zero_bugs.3F}.
	\iffalse

	Unfortunately, no.

	The first full verification of the L4 microkernel was a tour de force,
	with a large number of Ph.D.~students hand-verifying code at a
	very slow per-student rate.
	This level of effort could not be applied to most software projects
	because the rate of change is just too great.
	Furthermore, although the L4 microkernel is a large software
	artifact from the viewpoint of formal verification, it is tiny
	compared to a great number of projects, including LLVM,
	\GCC, the Linux kernel, Hadoop, MongoDB, and a great many others.
	In addition, this verification did have limits, as the researchers
	freely admit, to their credit:
	\url{https://wiki.sel4.systems/FrequentlyAskedQuestions#Does_seL4_have_zero_bugs.3F}.
	\fi

	형식적 검증이 마침내 더 최근의, 커다란 수준의 자동화에 관련된 L4 검증을
	포함해서 어떤 전망을 내놓기 시작하고 있기는 하지만, 전망 가능한 미래에
	테스트를 완전히 대체할 기회는 현재로써는 보이지 않습니다.
	그리고 이 점에 있어서 제가 틀렸다고 증명되면 전 좋아할 겁니다만, 그런
	증명은 실제 소프트웨어를 검증하는 실제 도구를 통한 형태여야 하지,
	자극적인 미사여구를 통한 형태가 되어선 안될 겁니다.

	언젠가는 형식적 검증이 지금은 회귀 검증으로 알려진 영역을 포함해 많은
	검증에 사용될 수도 있을겁니다.
	Section~\ref{sec:future:Formal Regression Testing?} 에서는 이 가능성이
	현실이 되기 위해 무엇이 필요한지 알아봅니다.

	\iffalse

	Although formal verification is finally starting to show some
	promise, including more-recent L4 verifications involving greater
	levels of automation, it currently has no chance of completely
	displacing testing in the foreseeable future.
	And although I would dearly love to be proven wrong on this point,
	please note that such a proof will be in the form of a real tool
	that verifies real software, not in the form of a large body of
	rousing rhetoric.

	Perhaps someday formal verification will be used heavily for
	validation, including for what is now known as regression testing.
	Section~\ref{sec:future:Formal Regression Testing?} looks at
	what would be required to make this possibility a reality.
	\fi
} \QuickQuizEnd

마지막 방법은 다음의 두개의 정의와 그것들이 암시하는 결론에 대해서 고려해 보는
것입니다:
\iffalse

One final approach is to consider the following two definitions and the
consequence that they imply:
\fi

\begin{description}[itemsep=0pt,labelindent=1em]
\item[정의:]	버그가 없는 프로그램들은 사소한 프로그램들이다.
\item[정의:]	신뢰할 수 있는 프로그램은 알려진 버그가 없다.
\item[결론:]	모든 사소하지 않으며 신뢰할 수 있는 프로그램에는 최소 하나의
		아직 알려지지 않은 버그가 있다.
\iffalse

\item[Definition:]	Bug-free programs are trivial programs.
\item[Definition:]	Reliable programs have no known bugs.
\item[Consequence:]	Any non-trivial reliable program contains at least
			one as-yet-unknown bug.
\fi
\end{description}

이 시점에서 보면, 모든 검증 영역에서의 진보는 두가지 영향을 가질 수밖에 없을
겁니다: (1)~사소한 프로그램들의 갯수의 증가 또는 (2)~신뢰할 수 있는
프로그램들의 수의 감소.
물론, 인류의 멀티코어 시스템과 소프트웨어에 대한 증가하는 의존도는 사소한
프로그램들의 갯수의 가파른 증가에 대한 커다란 동기가 될겁니다!
\iffalse

From this viewpoint, any advances in validation and verification can
have but two effects: (1)~An increase in the number of trivial programs or
(2)~A decrease in the number of reliable programs.
Of course, the human race's increasing reliance on multicore systems and
software provides extreme motivation for a very sharp increase in the
number of trivial programs!
\fi

하지만, 만약 여러분의 코드가 너무 복잡해서 여러분이 형식적 검증 도구에 너무
과하게 의존하고 있다는 점을 알게 된다면, 여러분은 여러분의 설계를 다시 세심하게
생각해 봐야 하는데, 당신의 형식적 검증 도구들이 당신의 코드가 특정 목적 언어로
손으로 변환해야 하게 되는 상황이라면 특히 그렇습니다.
예를 들어,
Section~\ref{sec:formal:Promela Parable: dynticks and Preemptible RCU}
에서 보인 preemption 가능한 RCU 의 복잡한 dynticks 인터페이스 구현에 있어서는
Section~\ref{sec:formal:Simplicity Avoids Formal Verification} 에서
이야기한대로 훨씬 간단한 대안적 구현이 존재함이 드러났습니다.
다른게 모두 동일하다면, 복잡한 구현을 위한 올바름의 증명보다 간단한 구현이 훨씬
낫습니다!

그리고 형식적 검증 테크닉들과 시스템들에 대한 열려있는 도전은 이 요약 내용이
틀리다고 증명하는 것입니다!
이 일을 돕기 위해, Verification Challenge~6 가 사용
가능합니다~\cite{PaulEMcKenney2017VerificationChallenge6}.
한번 보세요!!!
\iffalse

However, if your code is so complex that you find yourself
relying too heavily on formal-verification
tools, you should carefully rethink your design, especially if your
formal-verification tools require your code to be hand-translated
to a special-purpose language.
For example, a complex implementation of the dynticks interface for
preemptible RCU that was presented in
Section~\ref{sec:formal:Promela Parable: dynticks and Preemptible RCU}
turned out to
have a much simpler alternative implementation, as discussed in
Section~\ref{sec:formal:Simplicity Avoids Formal Verification}.
All else being equal, a simpler implementation is much better than
a proof of correctness for a complex implementation!

And the open challenge to those working on formal verification techniques
and systems is to prove this summary wrong!
To assist in this task, Verification Challenge~6 is now
available~\cite{PaulEMcKenney2017VerificationChallenge6}.
Have at it!!!
\fi
