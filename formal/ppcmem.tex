% formal/ppcmem.tex
% mainfile: ../perfbook.tex
% SPDX-License-Identifier: CC-BY-SA-3.0

\section{Special-Purpose State-Space Search}
\label{sec:formal:Special-Purpose State-Space Search}
%
\epigraph{Jack of all trades, master of none.}{\emph{Unknown}}

Promela 와 Spin 이 여러분이 모든 (작은) 알고리즘을 얼마든지 검증할 수 있게
해주지만, 그것들의 큰 범용성은 가끔 문제가 될 수 있습니다.
예를 들어, Promela 는 메모리 모델이나 특정 종류의 순서 재배치 의미를 이해하지
못합니다.
따라서 이 섹션은 제품 단계 시스템에서 사용되는 메모리 모델을 이해해서 완화된
순서 코드의 검증을 크게 단순화 시키는 상태 공간 탐색 도구들을 소개합니다.

예를 들어,
\cref{sec:formal:Promela Example: QRCU}
는 완화된 메모리 순서규칙을 위한 처리를 위해 어떻게 Promela 를 다뤄야 하는지
보였습니다.
이 방법이 잘 동작하긴 하나, 이는 개발자가 그 시스템의 메모리 모델을 완전히
이해할 것을 필요로 합니다.
불행히도, 일부의 (존재한다면) 개발자들만이 현대 CPU 의 복잡한 메모리 모델을
완전히 이해합니다.

\iffalse

Although Promela and Spin allow you to verify pretty much any (smallish)
algorithm, their very generality can sometimes be a curse.
For example, Promela does not understand memory models or any sort
of reordering semantics.
This section therefore describes some state-space search tools that
understand memory models used by production systems, greatly simplifying the
verification of weakly ordered code.

For example,
\cref{sec:formal:Promela Example: QRCU}
showed how to convince Promela to account for weak memory ordering.
Although this approach can work well, it requires that the developer
fully understand the system's memory model.
Unfortunately, few (if any) developers fully understand the complex
memory models of modern CPUs.

\fi

따라서, 또다른 접근법은 Cambridge 대학의 \ppl{Peter}{Sewell} 와
\ppl{Susmit}{Sarkar}, INRIA 의 \ppl{Luc}{Maranget},
\ppl{Francesco Zappa}{Nardelli}, 그리고 \ppl{Pankaj}{Pawan} , 그리고 Oxford
대학의 \ppl{Jade}{Alglave} 가 IBM 의 \ppl{Derek}{Williams} 와 협업해 만들어낸
PPCMEM 도구와 같은, 이 메모리 순서 규칙을 이미 이해하고 있는 도구를 사용하는
것입니다.
이 연구 그룹은 Power, \ARM, x86 은 물론이고 C/C++11 표준의 메모리 모델을 정형화
시키고~\cite{RichardSmith2019N4800}, Power 와 \ARM\ 정형화에 기초에 PPCMEM
도구를 만들었습니다.

\iffalse

Therefore, another approach is to use a tool that already understands
this memory ordering, such as the PPCMEM tool produced by
\ppl{Peter}{Sewell} and \ppl{Susmit}{Sarkar} at the University of Cambridge,
\ppl{Luc}{Maranget}, \ppl{Francesco Zappa}{Nardelli}, and
\ppl{Pankaj}{Pawan} at INRIA, and \ppl{Jade}{Alglave} at Oxford University,
in cooperation with \ppl{Derek}{Williams} of
IBM~\cite{JadeAlglave2011ppcmem}.
This group formalized the memory models of Power, \ARM, x86, as well
as that of the C/C++11 standard~\cite{RichardSmith2019N4800}, and
produced the PPCMEM tool based on the Power and \ARM\ formalizations.

\fi

\QuickQuiz{
	하지만 x86 은 강한 메모리 규칙을 가지고 있는데 왜 그 메모리 모델을
	정형화 시키죠?

	\iffalse

	But x86 has strong memory ordering, so why formalize its memory
	model?

	\fi

}\QuickQuizAnswer{
	사실, 학계에서는 x86 메모리 모델을 완화된 형태로 생각하는데, 앞의
	쓰기가 뒤따르는 읽기와 재배치 되는 것을 허용할 수 있기 때문입니다.
	학계의 관점에서, 강한 메모리 모델은 어떤 재배치도 허용하지 않아서 모든
	쓰레드가 그것들에게 보이는 모든 오퍼레이션의 순서에 동의할 수 있는
	것입니다.

	또한, 이건 개발자들이 가끔 x86 메모리 순서규칙에 대해 혼란에 빠지는
	경우들입니다.

	\iffalse

	Actually, academics consider the x86 memory model to be weak
	because it can allow prior stores to be reordered with
	subsequent loads.
	From an academic viewpoint, a strong memory model is one
	that allows absolutely no reordering, so that all threads
	agree on the order of all operations visible to them.

	Plus it really is the case that developers are sometimes confused
	about x86 memory ordering.

	\fi

}\QuickQuizEnd

PPCMEM 도구는 \emph{리트머스 테스트} 를 입력으로 받습니다.
샘플 리트머스 테스트가
\cref{sec:formal:Anatomy of a Litmus Test} 에서 선보입니다.
\Cref{sec:formal:What Does This Litmus Test Mean?}
는 이 리트머스 테스트를 동일한 C-언어 프로그램으로 연관지어보고,
\cref{sec:formal:Running a Litmus Test} 은 이 리트머스 테스트에 PPCMEM 을
적용하는지 설명하며,
\cref{sec:formal:PPCMEM Discussion}
은 그 의미를 이야기 합니다.

\iffalse

The PPCMEM tool takes \emph{litmus tests} as input.
A sample litmus test is presented in
\cref{sec:formal:Anatomy of a Litmus Test}.
\Cref{sec:formal:What Does This Litmus Test Mean?}
relates this litmus test to the equivalent C-language program,
\cref{sec:formal:Running a Litmus Test} describes how to
apply PPCMEM to this litmus test, and
\cref{sec:formal:PPCMEM Discussion}
discusses the implications.

\fi

\subsection{Anatomy of a Litmus Test}
\label{sec:formal:Anatomy of a Litmus Test}

PPCMEM 을 위한 PowerPC 리트머스 테스트가
\cref{lst:formal:PPCMEM Litmus Test} 에 보여져 있습니다.
ARM 인터페이스도 같은 방식으로 동작하지만 \ARM\ 명령어들이 Power 명령어들로
대체되었고 시작 부분의 \qco{PPC} 도 \qco{ARM} 으로 교체되었습니다.

\iffalse

An example PowerPC litmus test for PPCMEM is shown in
\cref{lst:formal:PPCMEM Litmus Test}.
The ARM interface works the same way, but with \ARM\ instructions
substituted for the Power instructions and with the initial \qco{PPC}
replaced by \qco{ARM}.

\fi

\begin{listing}[tbp]
\begin{fcvlabel}[ln:formal:PPCMEM Litmus Test]
\begin{VerbatimL}[commandchars=\@\[\]]
PPC SB+lwsync-RMW-lwsync+isync-simple		@lnlbl[type]
""						@lnlbl[altname]
{						@lnlbl[init:b]
0:r2=x; 0:r3=2; 0:r4=y; 0:r10=0; 0:r11=0; 0:r12=z; @lnlbl[init:0]
1:r2=y; 1:r4=x;					@lnlbl[init:1]
}						@lnlbl[init:e]
 P0                 | P1           ;		@lnlbl[procid]
 li r1,1            | li r1,1      ;		@lnlbl[reginit]
 stw r1,0(r2)       | stw r1,0(r2) ;		@lnlbl[stw]
 lwsync             | sync         ; @lnlbl[P0lwsync] @lnlbl[P1sync]
                    | lwz r3,0(r4) ; @lnlbl[P0empty]  @lnlbl[P1lwz]
 lwarx  r11,r10,r12 | ;		@lnlbl[P0lwarx] @lnlbl[P1empty:b]
 stwcx. r11,r10,r12 | ;		@lnlbl[P0stwcx]
 bne Fail1          | ;		@lnlbl[P0bne]
 isync              | ;		@lnlbl[P0isync]
 lwz r3,0(r4)       | ;		@lnlbl[P0lwz]
 Fail1:             | ;		@lnlbl[P0fail1] @lnlbl[P1empty:e]

exists						@lnlbl[assert:b]
(0:r3=0 /\ 1:r3=0)				@lnlbl[assert:e]
\end{VerbatimL}
\end{fcvlabel}
\caption{PPCMEM Litmus Test}
\label{lst:formal:PPCMEM Litmus Test}
\end{listing}

\begin{fcvref}[ln:formal:PPCMEM Litmus Test]
이 예에서, \clnref{type} 은 시스템의 타입을 (\qco{ARM} 또는 \qco{PPC}) 알리며 
이 모델의 제목을 포함합니다.  \Clnref{altname} 은 이 테스트를 위한 대안적
이름을 위한 공간을 제공하는데, 여러분은 앞의 예에서처럼 빈 줄로 보통 놔둘
겁니다.
주석은 \clnref{altname,init:b} 사이에 Ocaml (또는 Pascal) 문법의 \nbco{(* *)}
를 사용해 삽입될 수 있습니다.

\Clnrefrange{init:b}{init:e} 는 모든 레지스터를 위한 초기 값을 제공합니다;
각각은 \co{P:R=V} 의 형태로, \co{P} 는 프로세스 지시어이고, \co{R} 은 레지스터
지시어이며, \co{V} 는 그 값입니다.
예를 들어, 프로세스~0 의 레지스터 \co{r3} 는 초기에 값 2를 가지고 있습니다.
만약 그 값이 변수라면 (이 예에서는 \co{x}, \co{y}, 또는 \co{z}) 그 레지스터는
그 변수의 주소로 초기화 되어 있습니다.
또한, 변수들의 내용물도 초기화가 가능한데, 예를 들어 \co{x=1} 은 \co{x} 의 값을
1로 초기화 시킵니다.
초기화 되지 않은 변수들은 기본적으로 값이 0이 되어서, 이 경우 \co{x}, \co{y},
그리고 \co{z} 는 모두 초기값 0을 갖습니다.

\iffalse

\begin{fcvref}[ln:formal:PPCMEM Litmus Test]
In the example, \clnref{type} identifies the type of system (\qco{ARM} or
\qco{PPC}) and contains the title for the model. \Clnref{altname}
provides a place for an
alternative name for the test, which you will usually want to leave
blank as shown in the above example. Comments can be inserted between
\clnref{altname,init:b} using the OCaml (or Pascal) syntax of \nbco{(* *)}.

\Clnrefrange{init:b}{init:e} give initial values for all registers;
each is of the form
\co{P:R=V}, where \co{P} is the process identifier, \co{R} is the register
identifier, and \co{V} is the value. For example, process~0's register
\co{r3} initially contains the value 2. If the value is a variable (\co{x},
\co{y}, or \co{z} in the example) then the register is initialized to the
address of the variable. It is also possible to initialize the contents
of variables, for example, \co{x=1} initializes the value of \co{x} to
1. Uninitialized variables default to the value zero, so that in the
example, \co{x}, \co{y}, and~\co{z} are all initially zero.

\fi

\Clnref{procid} 는 두 프로세스를 위한 식별자를 제공해서 \clnref{init:0} 의
\co{0:r3=2} 가 \co{P0:r3=2} 로 대신 쓰여질 수 있게 합니다.
\Clnref{procid} 는 필요하며, 이 지시어는 \co{Pn} 의 형태여야 하는데, \co{n} 은
열 수로, 가장 왼쪽의 열이 0으로 시작합니다.
이는 불필요하게 엄격해 보일 수 있겠으나, 실제 사용 시에 상당한 혼란을 방지해
줍니다.
\end{fcvref}

\iffalse

\Clnref{procid} provides identifiers for the two processes, so that
the \co{0:r3=2} on \clnref{init:0} could instead have been written
\co{P0:r3=2}. \Clnref{procid} is
required, and the identifiers must be of the form \co{Pn}, where \co{n}
is the column number, starting from zero for the left-most column. This
may seem unnecessarily strict, but it does prevent considerable confusion
in actual use.
\end{fcvref}

\fi

\QuickQuiz{
	\begin{fcvref}[ln:formal:PPCMEM Litmus Test]
	\Cref{lst:formal:PPCMEM Litmus Test} 의 \clnref{reginit} 는 왜
	레지스터를 초기화 시키나요?
	왜 그대신 \clnref{init:0,init:1} 에서 초기화 시키지 않죠?
	\end{fcvref}

	\iffalse

	\begin{fcvref}[ln:formal:PPCMEM Litmus Test]
	Why does \clnref{reginit} of \cref{lst:formal:PPCMEM Litmus Test}
	initialize the registers?
	Why not instead initialize them on \clnref{init:0,init:1}?
	\end{fcvref}

	\fi

}\QuickQuizAnswer{
	두 방법 모두 잘 동작합니다.
	그러나, 일반적으로는 명시적 명령보다 초기화를 사용하는게 낫습니다.
	명시적인 명령은 이 예에서 그 사용법을 보이기 위해 사용되었습니다.
	또한, 이 도구의 웹사이트에서
	(\url{https://www.cl.cam.ac.uk/~pes20/ppcmem/}) 얻을 수 있는 많은
	리트머스 테스트는 명시적 초기화 명령들을 생성하는 자동화 방법을 사용해
	만들어 졌습니다.

	\iffalse

	Either way works.
	However, in general, it is better to use initialization than
	explicit instructions.
	The explicit instructions are used in this example to demonstrate
	their use.
	In addition, many of the litmus tests available on the tool's
	web site (\url{https://www.cl.cam.ac.uk/~pes20/ppcmem/}) were
	automatically generated, which generates explicit
	initialization instructions.

	\fi

}\QuickQuizEnd

\begin{fcvref}[ln:formal:PPCMEM Litmus Test]
\Clnrefrange{reginit}{P0fail1} 은 각 프로세스를 위한 코드입니다.
특정 프로세스는 P0의 \clnref{P0empty} 와 P1 의
\clnrefrange{P1empty:b}{P1empty:e} 에서의 경우처럼 라인을 갖지 않을 수
있습니다.
라벨과 분기가 허용되는데, \clnref{P0bne} 에서 분기가, \clnref{P0fail1} 에
라벨이 선보여 있습니다.
그러나, 너무 자유로운 분기의 사용은 상태 공간을 폭증시킬 수 있습니다.
반복문의 사용은 여러분의 상태 공간을 폭증시키기 위한 특히 좋은 방법입니다.

\Clnrefrange{assert:b}{assert:e} 는 단정을 보이는데, 여기서는 우리가 P0 와 P1
의 \co{r3} 레지스터가 두 쓰레드가 모두 수행을 끝낸 후 모두 0이 될 수 있는지에
우리가 관심있음을 보입니다.
P0 와 P1 이 각자의 \co{r3} 레지스터에서 둘 다 0을 보게 된다면 비참한 실패를
유발할 수 있는 많은 사용 경우가 있기 때문에 중요합니다.

\iffalse

\begin{fcvref}[ln:formal:PPCMEM Litmus Test]
\Clnrefrange{reginit}{P0fail1} are the lines of code for each process.
A given process can have empty lines, as is the case for P0's
\clnref{P0empty} and P1's \clnrefrange{P1empty:b}{P1empty:e}.
Labels and branches are permitted, as demonstrated by the branch
on \clnref{P0bne} to the label on \clnref{P0fail1}.
That said, too-free use of branches
will expand the state space. Use of loops is a particularly good way to
explode your state space.

\Clnrefrange{assert:b}{assert:e} show the assertion, which in this case
indicates that we
are interested in whether P0's and P1's \co{r3} registers can both contain
zero after both threads complete execution. This assertion is important
because there are a number of use cases that would fail miserably if
both P0 and P1 saw zero in their respective \co{r3} registers.

\fi

이는 여러분이 간단한 리트머스 테스트를 만드는데 충분한 정보가 될겁니다.
추가적인 문서들을 구할 수 있습니다만, 그런 추가적 문서의 많은 부분은 실제
하드웨어에서 테스트를 수행하기 위한 다른 연구 도구를 위한 것입니다.
아마도 더 중요한 건, 온라인 도구를 통해
(\url{https://www.cl.cam.ac.uk/~pes20/ppcmem/} 의 ``Select ARM Test'' 와
``Select POWER Test'' 버튼을 통해 사용 가능합니다) 이미 존재하는 많은 수의
리트머스 테스트를 사용 가능하다는 것일 겁니다.
이런 이미 존재하는 리트머스 테스트들 중 하나는 여러분의 Power 또는 \ARM\ 메모리
순서규칙 질문에 대한 답을 줄 가능성이 상당할 겁니다.

\iffalse

This should give you enough information to construct simple litmus
tests. Some additional documentation is available, though much of this
additional documentation is intended for a different research tool that
runs tests on actual hardware. Perhaps more importantly, a large number of
pre-existing litmus tests are available with the online tool (available
via the ``Select ARM Test'' and ``Select POWER Test'' buttons at
\url{https://www.cl.cam.ac.uk/~pes20/ppcmem/}).
It is quite likely that one of these pre-existing litmus tests will
answer your Power or \ARM\ memory-ordering question.

\fi

\subsection{What Does This Litmus Test Mean?}
\label{sec:formal:What Does This Litmus Test Mean?}

P0's \clnref{reginit,stw} are equivalent to the C statement \co{x=1}
because \clnref{init:0} defines P0's register \co{r2} to be the address
of \co{x}. P0's \clnref{P0lwarx,P0stwcx} are the mnemonics for
load-linked (``load register
exclusive'' in \ARM\ parlance and ``load reserve'' in Power parlance)
and store-conditional (``store register exclusive'' in \ARM\ parlance),
respectively. When these are used together, they form an atomic
instruction sequence, roughly similar to the compare-and-swap sequences
exemplified by the x86 \co{lock;cmpxchg} instruction. Moving to a higher
level of abstraction, the sequence from \clnrefrange{P0lwsync}{P0isync}
is equivalent to the Linux kernel's \co{atomic_add_return(&z, 0)}.
Finally, \clnref{P0lwz} is
roughly equivalent to the C statement \co{r3=y}.

P1's \clnref{reginit,stw} are equivalent to the C statement \co{y=1},
\clnref{P1sync}
is a memory barrier, equivalent to the Linux kernel statement \co{smp_mb()},
and \clnref{P1lwz} is equivalent to the C statement \co{r3=x}.
\end{fcvref}

\QuickQuiz{
	\begin{fcvref}[ln:formal:PPCMEM Litmus Test]
	But whatever happened to \clnref{P0fail1} of
	\cref{lst:formal:PPCMEM Litmus Test},
	the one that is the \co{Fail1:} label?
	\end{fcvref}
}\QuickQuizAnswer{
	The implementation of PowerPC version of \co{atomic_add_return()}
	loops when the \co{stwcx} instruction fails, which it communicates
	by setting non-zero status in the condition-code register,
	which in turn is tested by the \co{bne} instruction. Because actually
	modeling the loop would result in state-space explosion, we
	instead branch to the \co{Fail1:} label, terminating the model with
	the initial value of 2 in P0's \co{r3} register, which
	will not trigger the exists assertion.

	There is some debate about whether this trick is universally
	applicable, but I have not seen an example where it fails.
}\QuickQuizEnd

\begin{listing}[tbp]
\begin{VerbatimL}
void P0(void)
{
	int r3;

	x = 1; /* Lines 8 and 9 */
	atomic_add_return(&z, 0); /* Lines 10-15 */
	r3 = y; /* Line 16 */
}

void P1(void)
{
	int r3;

	y = 1; /* Lines 8-9 */
	smp_mb(); /* Line 10 */
	r3 = x; /* Line 11 */
}
\end{VerbatimL}
\caption{Meaning of PPCMEM Litmus Test}
\label{lst:formal:Meaning of PPCMEM Litmus Test}
\end{listing}

Putting all this together, the C-language equivalent to the entire litmus
test is as shown in
\cref{lst:formal:Meaning of PPCMEM Litmus Test}.
The key point is that if \co{atomic_add_return()} acts as a full
memory barrier (as the Linux kernel requires it to), 
then it should be impossible for \co{P0()}'s and \co{P1()}'s \co{r3}
variables to both be zero after execution completes.

The next section describes how to run this litmus test.

\subsection{Running a Litmus Test}
\label{sec:formal:Running a Litmus Test}

As noted earlier, litmus tests may be run interactively via
\url{https://www.cl.cam.ac.uk/~pes20/ppcmem/}, which can help build an
understanding of the memory model.
However, this approach requires that the user manually carry out the
full state-space search.
Because it is very difficult to be sure that you have checked every
possible sequence of events, a separate tool is provided for this
purpose~\cite{PaulEMcKenney2011ppcmem}.

\begin{listing}[tbp]
\begin{VerbatimL}[numbers=none,xleftmargin=0pt]
./ppcmem -model lwsync_read_block \
         -model coherence_points filename.litmus
...
States 6
0:r3=0; 1:r3=0;
0:r3=0; 1:r3=1;
0:r3=1; 1:r3=0;
0:r3=1; 1:r3=1;
0:r3=2; 1:r3=0;
0:r3=2; 1:r3=1;
Ok
Condition exists (0:r3=0 /\ 1:r3=0)
Hash=e2240ce2072a2610c034ccd4fc964e77
Observation SB+lwsync-RMW-lwsync+isync Sometimes 1
\end{VerbatimL}
\caption{PPCMEM Detects an Error}
\label{lst:formal:PPCMEM Detects an Error}
\end{listing}

Because the litmus test shown in
\cref{lst:formal:PPCMEM Litmus Test}
contains read-modify-write instructions, we must add \co{-model}
arguments to the command line.
If the litmus test is stored in \co{filename.litmus},
this will result in the output shown in
\cref{lst:formal:PPCMEM Detects an Error},
where the \co{...} stands for voluminous making-progress output.
The list of states includes \co{0:r3=0; 1:r3=0;}, indicating once again
that the old PowerPC implementation of \co{atomic_add_return()} does
not act as a full barrier.
The ``Sometimes'' on the last line confirms this: the assertion triggers
for some executions, but not all of the time.

\begin{listing}[tbp]
\begin{VerbatimL}[numbers=none,xleftmargin=0pt]
./ppcmem -model lwsync_read_block \
         -model coherence_points filename.litmus
...
States 5
0:r3=0; 1:r3=1;
0:r3=1; 1:r3=0;
0:r3=1; 1:r3=1;
0:r3=2; 1:r3=0;
0:r3=2; 1:r3=1;
No (allowed not found)
Condition exists (0:r3=0 /\ 1:r3=0)
Hash=77dd723cda9981248ea4459fcdf6097d
Observation SB+lwsync-RMW-lwsync+sync Never 0 5
\end{VerbatimL}
\caption{PPCMEM on Repaired Litmus Test}
\label{lst:formal:PPCMEM on Repaired Litmus Test}
\end{listing}

The fix to this Linux-kernel bug is to replace P0's \co{isync} with
\co{sync}, which results in the output shown in
\cref{lst:formal:PPCMEM on Repaired Litmus Test}.
As you can see, \co{0:r3=0; 1:r3=0;} does not appear in the list of states,
and the last line calls out ``Never''.
Therefore, the model predicts that the offending execution sequence
cannot happen.

\QuickQuizSeries{%
\QuickQuizB{
	Does the \ARM\ Linux kernel have a similar bug?
}\QuickQuizAnswer{
	\ARM\ does not have this particular bug because it places
	\co{smp_mb()} before and after the \co{atomic_add_return()}
	function's assembly-language implementation.
	PowerPC no longer has this bug; it has long since been
	fixed~\cite{BenjaminHerrenschmidt2011:powerpc:atomic_return}.
}\QuickQuizEndB
%
\QuickQuizE{
	\begin{fcvref}[ln:formal:PPCMEM Litmus Test]
	Does the \co{lwsync} on \clnref{P0lwsync} in
	\cref{lst:formal:PPCMEM Litmus Test} provide sufficient ordering?
	\end{fcvref}
}\QuickQuizAnswerE{
	It depends on the semantics required.
	The rest of this answer assumes that the assembly language
	for \co{P0} in
	\cref{lst:formal:PPCMEM Litmus Test}
	is supposed to implement a value-returning atomic operation.

	As is discussed in
	\cref{chp:Advanced Synchronization: Memory Ordering},
	Linux kernel's memory consistency model requires
	value-returning atomic RMW operations to be fully ordered
	on both sides.
	The ordering provided by \co{lwsync} is insufficient for this
	purpose, and so \co{sync} should be used instead.
	This change has since been
	made~\cite{BoqunFeng2015:powerpc:value-returning-atomics}
	in response to an email thread discussing a couple of other litmus
	tests~\cite{Paulmck2015:powerpc:value-returning-atomics}.
	Finding any other bugs that the Linux kernel might have is left
	as an exercise for the reader.

	In other enviroments providing weaker semantics, \co{lwsync}
	might be sufficient.
	But not for the Linux kernel's value-returning atomic operations!
}\QuickQuizEndE
}

\subsection{PPCMEM Discussion}
\label{sec:formal:PPCMEM Discussion}

These tools promise to be of great help to people working on low-level
parallel primitives that run on \ARM\ and on Power. These tools do have
some intrinsic limitations:

\begin{enumerate}
\item	These tools are research prototypes, and as such are unsupported.
\item	These tools do not constitute official statements by IBM or \ARM\
	on their respective CPU architectures. For example, both
	corporations reserve the right to report a bug at any time against
	any version of any of these tools. These tools are therefore not a
	substitute for careful stress testing on real hardware. Moreover,
	both the tools and the model that they are based on are under
	active development and might change at any time. On the other
	hand, this model was developed in consultation with the relevant
	hardware experts, so there is good reason to be confident that
	it is a robust representation of the architectures.
\item	These tools currently handle a subset of the instruction set.
	This subset has been sufficient for my purposes, but your mileage
	may vary. In particular, the tool handles only word-sized accesses
	(32 bits), and the words accessed must be properly aligned.\footnote{
		But recent work focuses on mixed-size
		accesses~\cite{Flur:2017:MCA:3093333.3009839}.}
	In addition, the tool does not handle some of the weaker variants
	of the \ARM\ memory-barrier instructions, nor does it handle
	arithmetic.
\item	The tools are restricted to small loop-free code fragments
	running on small numbers of threads. Larger examples result
	in state-space explosion, just as with similar tools such as
	Promela and spin.
\item	The full state-space search does not give any indication of how
	each offending state was reached. That said, once you realize
	that the state is in fact reachable, it is usually not too hard
	to find that state using the interactive tool.
\item	These tools are not much good for complex data structures, although
	it is possible to create and traverse extremely simple linked
	lists using initialization statements of the form
	``\co{x=y; y=z; z=42;}''.
\item	These tools do not handle memory mapped I/O or device registers.
	Of course, handling such things would require that they be
	formalized, which does not appear to be in the offing.
\item	The tools will detect only those problems for which you code an
	assertion. This weakness is common to all formal methods, and
	is yet another reason why testing remains important. In the
	immortal words of Donald Knuth quoted at the beginning of this
	chapter, ``Beware of bugs in the above
	code; I have only proved it correct, not tried it.''
\end{enumerate}

That said, one strength of these tools is that they are designed to
model the full range of behaviors allowed by the architectures, including
behaviors that are legal, but which current hardware implementations do
not yet inflict on unwary software developers. Therefore, an algorithm
that is vetted by these tools likely has some additional safety margin
when running on real hardware. Furthermore, testing on real hardware can
only find bugs; such testing is inherently incapable of proving a given
usage correct. To appreciate this, consider that the researchers
routinely ran in excess of 100 billion test runs on real hardware to
validate their model.
In one case, behavior that is allowed by the architecture did not occur,
despite 176 billion runs~\cite{JadeAlglave2011ppcmem}.
In contrast, the
full-state-space search allows the tool to prove code fragments correct.

It is worth repeating that formal methods and tools are no substitute for
testing. The fact is that producing large reliable concurrent software
artifacts, the Linux kernel for example, is quite difficult. Developers
must therefore be prepared to apply every tool at their disposal towards
this goal. The tools presented in this chapter are able to locate bugs that
are quite difficult to produce (let alone track down) via testing. On the
other hand, testing can be applied to far larger bodies of software than
the tools presented in this chapter are ever likely to handle. As always,
use the right tools for the job!

Of course, it is always best to avoid the need to work at this level
by designing your parallel code to be easily partitioned and then
using higher-level primitives (such as locks, sequence counters, atomic
operations, and RCU) to get your job done more straightforwardly. And even
if you absolutely must use low-level memory barriers and read-modify-write
instructions to get your job done, the more conservative your use of
these sharp instruments, the easier your life is likely to be.
