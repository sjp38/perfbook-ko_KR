% owned/owned.tex
% mainfile: ../perfbook.tex
% SPDX-License-Identifier: CC-BY-SA-3.0

\QuickQuizChapter{chp:Data Ownership}{Data Ownership}{qqzowned}
%
\Epigraph{It is mine, I tell you. My own. My precious. Yes, my precious.}
	 {\emph{Gollum in ``The Fellowship of the Ring'', J.R.R.~Tolkien}}

락킹과 함께 나타나는 동기화 오버헤드를 막는 가장 간단한 한가지 방법은 데이터를
쓰레드들 사이에 (또는, 커널의 경우라면, CPU 사이에) 배분해서 데이터의 특정
부분은 하나의 쓰레드에 의해서만 액세스 되고 수정되게끔 하는 것입니다.
흥미롭게도, 데이터 소유권은 병렬 설계 기술의 ``커다란 세가지'' 각각을
커버합니다:
쓰레드 사이로 (또는, 경우에 따라선 CPU 사이에) 파티셔닝을 하고,
모든 로컬 오퍼레이션을 배치로 처리하며,
동기화 오퍼레이션의 제거가 극한의 로직으로 이르는 것을 완화시킵니다.
따라서 데이터 소유권이 널리 사용된다는 것은 놀라운 일이 아닙니다:
심지어 초심자들도 본능적으로 이를 사용합니다.
사실, 데이터 소유권은 워낙 널리 사용되어서 이 챕터에서는 새로운 예제를
소개하지는 않고, 앞의 챕터들에서 다루었던 것들을 다시 한번 다뤄 보겠습니다.

\iffalse

One of the simplest ways to avoid the synchronization overhead that
comes with locking is to parcel the data out among the threads (or,
in the case of kernels, CPUs)
so that a given piece of data is accessed and modified by only one
of the threads.
Interestingly enough, data ownership covers each of the ``big three''
parallel design techniques:
It partitions over threads (or CPUs, as the case may be),
it batches all local operations,
and its elimination of synchronization operations is weakening
carried to its logical extreme.
It should therefore be no surprise that data ownership is heavily used:
Even novices use it almost instinctively.
In fact, it is so heavily used that this chapter will not introduce any
new examples, but will instead refer back to those of previous chapters.

\fi

\QuickQuiz{
	C 또는 C++ 로 공유 메모리 병렬 프로그램들을 (예를 들면, pthread 를
	사용해서) 만들 때 어떤 형태의 데이터 소유권이 제거하기 엄청 어렵나요?

	\iffalse

	What form of data ownership is extremely difficult
	to avoid when creating shared-memory parallel programs
	(for example, using pthreads) in C or C++?

	\fi

}\QuickQuizAnswer{
	함수에서의 auto 변수의 사용입니다.
	기본적으로, 이것들은 현재 함수를 수행하는 쓰레드에 사유화 되어
	있습니다.

	\iffalse

	Use of auto variables in functions.
	By default, these are private to the thread executing the
	current function.

	\fi

}\QuickQuizEnd

데이터 소유권을 위한 여러 접근법들이 있습니다.
Section~\ref{sec:owned:Multiple Processes} 은 데이터 소유권에서의 논리적 극한을
선보이는데, 각 쓰레드가 각자의 사적 주소 공간을 갖는 경우입니다.
Section~\ref{sec:owned:Partial Data Ownership and pthreads} 은 그 반대의 극한을
보는데, 데이터가 공유되지만 다른 쓰레드들은 이 데이터로의 다른 접근 권한을
갖습니다.
Section~\ref{sec:owned:Function Shipping} 은 함수 배달을 설명하는데, 이는 다른
쓰레드들이 특정 쓰레드에 의해 소유된 데이터로의 간접 액세스를 하게 허용하는
방법입니다.
Section~\ref{sec:owned:Designated Thread} 은 어떻게 지정된 쓰레드들이 특정
함수와 연관 데이터의 소유권을 할당받을 수 있는지 설명합니다.
Section~\ref{sec:owned:Privatization} 은 공유 데이터를 사용하는 알고리즘을
데이터 소유권을 사용하도록 변화시킴으로써 성능을 개선시키는 과정을 논합니다.
마지막으로, Section~\ref{sec:owned:Other Uses of Data Ownership} 은 데이터
소유권을 일등시민으로 사용하는 소프트웨어 환경 몇가지를 보입니다.

\iffalse

There are a number of approaches to data ownership.
Section~\ref{sec:owned:Multiple Processes} presents the logical extreme
in data ownership, where each thread has its own private address space.
Section~\ref{sec:owned:Partial Data Ownership and pthreads} looks at
the opposite extreme, where the data is shared, but different threads
own different access rights to the data.
Section~\ref{sec:owned:Function Shipping} describes function shipping,
which is a way of allowing other threads to have indirect access to
data owned by a particular thread.
Section~\ref{sec:owned:Designated Thread} describes how designated
threads can be assigned ownership of a specified function and the
related data.
Section~\ref{sec:owned:Privatization} discusses improving performance
by transforming algorithms with shared data to instead use data ownership.
Finally, Section~\ref{sec:owned:Other Uses of Data Ownership} lists
a few software environments that feature data ownership as a
first-class citizen.

\fi

\section{Multiple Processes}
\label{sec:owned:Multiple Processes}
%
\epigraph{A man's home is his castle}{\emph{Ancient Laws of England}}

Section~\ref{sec:toolsoftrade:Scripting Languages}
은 다음 예를 소개했습니다:

\iffalse

Section~\ref{sec:toolsoftrade:Scripting Languages}
introduced the following example:

\fi

\begin{VerbatimN}[samepage=true]
compute_it 1 > compute_it.1.out &
compute_it 2 > compute_it.2.out &
wait
cat compute_it.1.out
cat compute_it.2.out
\end{VerbatimN}

이 예는 \co{compute_it} 프로그램의 메모리를 공유하지 않는 별개의 프로세스로서
두개의 인스턴스를 병렬로 수행합니다.
따라서, 특정 프로세스의 모든 데이터는 그 프로세스에 의해 소유되며, 따라서 앞의
예에서의 데이터의 거의 모든 것이 소유되어 있습니다.
이 방법은 동기화 오버헤드를 거의 없앱니다.
그로 인한 극단적 단순성과 최적의 성능의 조합은 분명 무척 매력적입니다.

\iffalse

This example runs two instances of the \co{compute_it} program in
parallel, as separate processes that do not share memory.
Therefore, all data in a given process is owned by that process,
so that almost the entirety of data in the above example is owned.
This approach almost entirely eliminates synchronization overhead.
The resulting combination of extreme simplicity and optimal performance
is obviously quite attractive.

\fi

\QuickQuizSeries{%
\QuickQuizB{
	Section~\ref{sec:owned:Multiple Processes}
	에서 보인 이 예에 남아 있는 동기화는 무엇인가요?

	\iffalse

	What synchronization remains in the example shown in
	Section~\ref{sec:owned:Multiple Processes}?

	\fi

}\QuickQuizAnswerB{
	\co{sh} \co{&} 오퍼레이터를 통한 쓰레드의 생성과 \co{sh} \co{wait}
	커맨드를 통한 쓰레드 기다리기입니다.

	물론, 프로세스가 예를 들어 \co{shmget()} 이나 \co{mmap()} 시스템 콜을
	이용해 명시적으로 메모리를 공유한다면, 이 공유된 메모리를 접근하거나
	업데이트 할 때 명시적 동기화가 필요할 겁니다.
	프로세스들은 또한 다음 프로세스간 통신 메커니즘 중 무엇을 이용해서든
	동기화를 할수도 있을 겁니다:
	\begin{enumerate}
	\item	System V 세마포어.
	\item	System V 메세지 큐.
	\item	유닉스 도메인 소켓.
	\item	TCP/IP, UDP, 그리고 다른 것들의 모든 호스트를 포함한네트워킹
		프로토콜.
	\item	파일 락킹.
	\item	\co{O_CREAT} 와 \co{O_EXCL} 플래그를 사용하는 \co{open()}
		시스템콜.
	\item	\co{rename()} 시스템콜의 사용.
	\end{enumerate}
	가능한 동기화 메커니즘의 완전한 리스트는 독자 여러분의 연습문제로
	남겨두는데, 그것은 무척 긴 리스트가 될 것이라는 경고를 남겨둡니다.
	예상치 못한 시스템 콜들의 놀랄만큼 큰 수는 서비스로서의 동기화
	메커니즘으로 압착될 수 있습니다.

	\iffalse

	The creation of the threads via the \co{sh} \co{&} operator
	and the joining of thread via the \co{sh} \co{wait}
	command.

	Of course, if the processes explicitly share memory, for
	example, using the \co{shmget()} or \co{mmap()} system
	calls, explicit synchronization might well be needed when
	acccessing or updating the shared memory.
	The processes might also synchronize using any of the following
	interprocess communications mechanisms:
	\begin{enumerate}
	\item	System V semaphores.
	\item	System V message queues.
	\item	UNIX-domain sockets.
	\item	Networking protocols, including TCP/IP, UDP, and a whole
		host of others.
	\item	File locking.
	\item	Use of the \co{open()} system call with the
		\co{O_CREAT} and \co{O_EXCL} flags.
	\item	Use of the \co{rename()} system call.
	\end{enumerate}
	A complete list of possible synchronization mechanisms is left
	as an exercise to the reader, who is warned that it will be
	an extremely long list.
	A surprising number of unassuming system calls can be pressed
	into service as synchronization mechanisms.

	\fi

}\QuickQuizEndB
%
\QuickQuizE{
	Section~\ref{sec:owned:Multiple Processes}
	에 보인 예에 어떤 공유되는 데이터가 있나요?

	\iffalse

	Is there any shared data in the example shown in
	Section~\ref{sec:owned:Multiple Processes}?

	\fi

}\QuickQuizAnswerE{
	이건 철학적 질문입니다.

	``아니오'' 라는 대답을 원하는 사람들은 프로세스들이 정의대로 메모리를
	공유하지 않는다고 주장할 수도 있겠습니다.

	``네'' 라는 대답을 원하는 사람들은 공유 메모리를 필요로 하지 않는 많은
	수의 동기화 메커니즘을 나열할 수 있겠고, 커널은 어떤 공유된 상태를 가질
	것을 이야기할 수 있으며, 어쩌면 프로세스 ID (PID) 의 할당은 공유
	데이터로 간주될 수 있다고까지 주장할 수도 있겠습니다.

	그런 논쟁은 무척 지적인 행위이고, 지적이라 느끼고 불행한 학우나
	동료에게서 점수를 따기에 좋은 방법입니다만, 유용한 무엇인가가 되는 것을
	막는 방법인 경우가 대부분입니다.

	\iffalse

	That is a philosophical question.

	Those wishing the answer ``no'' might argue that processes by
	definition do not share memory.

	Those wishing to answer ``yes'' might list a large number of
	synchronization mechanisms that do not require shared memory,
	note that the kernel will have some shared state, and perhaps
	even argue that the assignment of process IDs (PIDs) constitute
	shared data.

	Such arguments are excellent intellectual exercise, and are
	also a wonderful way of feeling intelligent and scoring points
	against hapless classmates or colleagues, but are mostly a way
	of avoiding getting anything useful done.

	\fi

}\QuickQuizEndE
}

이와 똑같은 패턴은
Listing~\ref{lst:toolsoftrade:Using the fork() Primitive}
와~\ref{lst:toolsoftrade:Using the wait() Primitive}
에 보인 것처럼 \co{sh} 과 같이 C 로도 작성될 수 있습니다.

이 병렬성의 간단한 형태를 반복하는 것은 속임수나 책임을 회피하는 행위가 아니고,
오히려 여러분의 코드를 더 빨리 돌아가게 만드는 간단하고 우아한 방법입니다.
이는 빠르고, 잘 확장되며, 프로그래밍하기 쉽고, 유지하기 쉬우며, 일이 되게
만듭니다.
또한, (가용한 곳에) 이 방법을 취하는 것은 개발자가 정교한 싱글쓰레드 최적화를
\co{compute_it} 에 적용하거나, 이 방법이 적용될 수 없는 부분의 코드에 정교한
병렬 프로그래밍 패턴을 적용하는 것과 같은 다른 곳에 집중할 시간을 벌어줍니다.
좋아하지 않을 이유가 있나요?

다음 섹션은 공유 메모리 병렬 프로그램에서의 데이터 소유권의 사용에 대해 이야기
합니다.

\iffalse

This same pattern can be written in C as well as in \co{sh}, as illustrated by
Listings~\ref{lst:toolsoftrade:Using the fork() Primitive}
and~\ref{lst:toolsoftrade:Using the wait() Primitive}.

It bears repeating that these trivial forms of parallelism are not in
any way cheating or ducking responsibility, but are rather simple and
elegant ways to make your code run faster.
It is fast, scales well, is easy to program, easy to maintain, and
gets the job done.
In addition, taking this approach (where applicable) allows the developer
more time to focus on other things whether these things might involve
applying sophisticated single-threaded optimizations to \co{compute_it}
on the one hand, or applying sophisticated parallel-programming patterns
to portions of the code where this approach is inapplicable.
What is not to like?

The next section discusses the use of data ownership in shared-memory
parallel programs.

\fi

\section{Partial Data Ownership and pthreads}
\label{sec:owned:Partial Data Ownership and pthreads}
%
\epigraph{Give thy mind more to what thou hast than to what thou hast not.}
	 {\emph{Marcus Aurelius Antoninus}}

동시적 카운팅 (\cref{chp:Counting} 을 참고하세요) 은 데이터 소유권을 무척 많이
사용합니다만, 살짝 꼬여 있습니다.
쓰레드들은 다른 쓰레드에 의해 소유된 데이터를 수정하지 못합니다만, 그걸 읽을 수
있습니다.
짧게 말해, 공유 메모리의 사용은 소유권과 접근 권한의 미묘한 차이가 있는 개념을
허용합니다.

예를 들어,
page~\pageref{lst:count:Per-Thread Statistical Counters} 의
Listing~\ref{lst:count:Per-Thread Statistical Counters}
에서 보인 쓰레드별 통계적 카운터 구현을 생각해 봅시다.
여기서, \co{inc_count()} 는 연관된 쓰레드의 \co{counter} 의 인스턴스만을
업데이트하는 반면, \co{read_count()} 는 모든 쓰레드의 \co{counter} 의
인스턴스를 접근하지만 수정하지는 않습니다.

\iffalse

Concurrent counting (see \cref{chp:Counting}) uses data ownership heavily,
but adds a twist.
Threads are not allowed to modify data owned by other threads,
but they are permitted to read it.
In short, the use of shared memory allows more nuanced notions
of ownership and access rights.

For example, consider the per-thread statistical counter implementation
shown in
Listing~\ref{lst:count:Per-Thread Statistical Counters} on
page~\pageref{lst:count:Per-Thread Statistical Counters}.
Here, \co{inc_count()} updates only the corresponding thread's
instance of \co{counter},
while \co{read_count()} accesses, but does not modify, all
threads' instances of \co{counter}.

\fi

\QuickQuiz{
	각 쓰레드가 각자의 쓰레드별 변수 인스턴스를 읽지만 다른 쓰레드의
	인스턴스에 쓰기를 하는 부분적 데이터 소유권이 말이 되긴 할까요?

	\iffalse

	Does it ever make sense to have partial data ownership where
	each thread reads only its own instance of a per-thread variable,
	but writes to other threads' instances?

	\fi

}\QuickQuizAnswer{
	놀랍게도, 그렇습니다.
	한가지 예는 쓰레드들이 다른 쓰레드의 우편함에 메세지를 보내고 각
	쓰레드는 그 메세지가 처리되고 나면 그 메세지를 제거할 책임을 갖는
	경우와 같은 간단한 메세지 전달 시스템이 되겠습니다.
	그런 알고리즘의 구현은 비슷한 소유권 패턴의 다른 알고리즘을 알아보는
	것과 함께 독자 여러분의 연습문제로 남겨둡니다.

	\iffalse

	Amazingly enough, yes.
	One example is a simple message-passing system where threads post
	messages to other threads' mailboxes, and where each thread
	is responsible for removing any message it sent once that message
	has been acted on.
	Implementation of such an algorithm is left as an exercise for
	the reader, as is identifying other algorithms with similar
	ownership patterns.

	\fi

}\QuickQuizEnd

부분적 데이터 소유권은 리눅스 커널 내에서도 흔합니다.
예를 들어, 특정 CPU 는 자신의 CPU 별 변수를 인터럽트가 불능화 된 상태에서만
읽는게 허용될 수도 있고, 다른 CPU 는 이 첫번째 CPU 의 CPU 별 변수를 연관된 CPU
별 락을 잡은 샅태에서만 읽기가 허용될 수도 있습니다.
그러면 이 CPU 는 자신이 소유한 CPU 별 변수의 집합을 인터럽트를 불능화 했고
자신의 CPU 별 락을 잡고 있는 상태에서라면 업데이트 하는게 허용될 겁니다.
이 조정은 각 CPU 의 매우 오버헤드가 낮은 스스로가 소유한 CPU 별 변수 집합으로의
액세스를 허용하는 reader-writer 락으로 생각될 수 있습니다.
이 테마에는 무척 많은 변종들이 있습니다.

그 자신의 부분에서, 순수한 데이터 소유권은 또한 일반적이고 유용한데, 예를 들어
Section~\ref{sec:SMPdesign:Resource Allocator Caches}
의
page~\pageref{sec:SMPdesign:Resource Allocator Caches} 에서부터 설명된 쓰레드별
메모리 할당자 캐쉬가 한 예가 되겠습니다.
이 알고리즘에서, 각 쓰레드의 캐쉬는 해당 쓰레드에 완전히 사유되어 있습니다.

\iffalse

Partial data ownership is also common within the Linux kernel.
For example, a given CPU might be permitted to read a given set of its
own per-CPU variables only with interrupts disabled, another CPU might
be permitted to read that same set of the first CPU's per-CPU variables
only when holding the corresponding per-CPU lock.
Then that given CPU would be permitted to update this set of its own
per-CPU variables if it both has interrupts disabled and holds its
per-CPU lock.
This arrangement can be thought of as a reader-writer lock that allows
each CPU very low-overhead access to its own set of per-CPU variables.
There are a great many variations on this theme.

For its own part, pure data ownership is also both common and useful,
for example, the per-thread memory-allocator caches discussed in
Section~\ref{sec:SMPdesign:Resource Allocator Caches}
starting on
page~\pageref{sec:SMPdesign:Resource Allocator Caches}.
In this algorithm, each thread's cache is completely private to that
thread.

\fi

\section{Function Shipping}
\label{sec:owned:Function Shipping}
%
\epigraph{If the mountain will not come to Muhammad, then Muhammad must
	  go to the mountain.}
	 {\emph{Essays, Francis Bacon}}

앞의 섹션은 쓰레드들이 다른 쓰레드의 데이터에 접근하는 약화된 형태의 데이터
소유권을 이야기 했습니다.
이는 데이터를 그것을 필요로 하는 함수에게 가져다 주는 것으로 생각될 수
있습니다.
하나의 대안적 방법은 함수를 데이터에게 보내는 것입니다.

그런 방법이
page~\pageref{sec:count:Signal-Theft Limit Counter Design}
에서 시작하는
Section~\ref{sec:count:Signal-Theft Limit Counter Design}
에 그려져 있는데, 특히
page~\pageref{lst:count:Signal-Theft Limit Counter Value-Migration Functions}
의
Listing~\ref{lst:count:Signal-Theft Limit Counter Value-Migration Functions}
에 있는 \co{flush_local_count_sig()} 와 \co{flush_local_count()} 함수가
그것입니다.

\iffalse

The previous section described a weak form of data ownership where
threads reached out to other threads' data.
This can be thought of as bringing the data to the functions that
need it.
An alternative approach is to send the functions to the data.

Such an approach is illustrated in
Section~\ref{sec:count:Signal-Theft Limit Counter Design}
beginning on
page~\pageref{sec:count:Signal-Theft Limit Counter Design},
in particular the \co{flush_local_count_sig()} and
\co{flush_local_count()} functions in
Listing~\ref{lst:count:Signal-Theft Limit Counter Value-Migration Functions}
on
page~\pageref{lst:count:Signal-Theft Limit Counter Value-Migration Functions}.

\fi

\co{flush_local_count_sig()} 함수는 보내진 함수처럼 동작하는 시그널
핸들러입니다.
\co{flush_local_count()} 의 \co{pthread_kill()} 함수는 시그널을 보내고---함수를
배송하고--- 이 보내진 함수가 수행될 때까지 기다립니다.
이 보내진 함수는 동시에 수행되는 \co{add_count()} 나 \co{sub_count()} 들과의
(page~\pageref{lst:count:Signal-Theft Limit Counter Add Function} 의
Listing~\ref{lst:count:Signal-Theft Limit Counter Add Function} 와
page~\pageref{lst:count:Signal-Theft Limit Counter Subtract Function} 의
Listing~\ref{lst:count:Signal-Theft Limit Counter Subtract Function} 를
참고하세요) 상호작요이라는 드물지만은 않은 추가적 복잡도를 갖습니다.

\iffalse

The \co{flush_local_count_sig()} function is a signal handler that
acts as the shipped function.
The \co{pthread_kill()} function in \co{flush_local_count()}
sends the signal---shipping the function---and then waits until
the shipped function executes.
This shipped function has the not-unusual added complication of
needing to interact with any concurrently executing \co{add_count()}
or \co{sub_count()} functions (see
Listing~\ref{lst:count:Signal-Theft Limit Counter Add Function}
on
page~\pageref{lst:count:Signal-Theft Limit Counter Add Function} and
Listing~\ref{lst:count:Signal-Theft Limit Counter Subtract Function}
on
page~\pageref{lst:count:Signal-Theft Limit Counter Subtract Function}).

\fi

\QuickQuiz{
	POSIX 시그널 외에 어떤 다른 메커니즘이 함수 보내기를 위해 사용될 수
	있을까요?

	\iffalse

	What mechanisms other than POSIX signals may be used for function
	shipping?

	\fi

}\QuickQuizAnswer{
	그런 메커니즘들이 매우 많은데, 다음을 포함합니다:
	\begin{enumerate}
	\item	System V 메세지 큐.
	\item	공유 메모리 디큐
		(Section~\ref{sec:SMPdesign:Double-Ended Queue} 를 참고하세요).
	\item	공유 메모리 메일함.
	\item	유닉스 도메인 소켓.
	\item	TCP/IP 또는 UDP, 그리고 RPC, HTTP, XML, SOAP, 그 외에도 여러 더
		높은 단계에서의 프로토콜과 결함된 형태의 것들.
	\end{enumerate}
	완전한 리스트를 만드는 것은 그 리스트는 굉장히 길 것이라 경고받은,
	그럼에도 그것을 원하는 독자 여러분의 연습문제로 남겨둡니다.

	\iffalse

	There is a very large number of such mechanisms, including:
	\begin{enumerate}
	\item	System V message queues.
	\item	Shared-memory dequeue (see
		Section~\ref{sec:SMPdesign:Double-Ended Queue}).
	\item	Shared-memory mailboxes.
	\item	UNIX-domain sockets.
	\item	TCP/IP or UDP, possibly augmented by any number of
		higher-level protocols, including RPC, HTTP,
		XML, SOAP, and so on.
	\end{enumerate}
	Compilation of a complete list is left as an exercise to
	sufficiently single-minded readers, who are warned that the
	list will be extremely long.

	\fi

}\QuickQuizEnd

\section{Designated Thread}
\label{sec:owned:Designated Thread}
%
\epigraph{Let a man practice the profession which he best knows.}
	 {\emph{Cicero}}

앞의 섹션들은 각 쓰레드가 각자의 복사본을 또는 각자의 데이터 일부를 가질 수
있게 하는 방법들을 설명했습니다.
대조적으로, 이 섹션은 함수적 해제 방법을 설명하는데, 특별한 특정 쓰레드가 그
일을 하는데 필요한 데이터로의 권리를 갖는 형태입니다.
\begin{fcvref}[ln:count:count_stat_eventual:whole:eventual]
Section~\ref{sec:count:Eventually Consistent Implementation} 의 결과적으로
일관적인 카운터 구현이 한 예를 제공합니다.
이 구현은
Listing~\ref{lst:count:Array-Based Per-Thread Eventually Consistent Counters}
의 \clnrefrange{b}{e} 에 보인 \co{eventual()} 함수를 수행하는 특별한 쓰레드를
갖습니다.
이 \co{eventual()} 쓰레드는 주기적으로 쓰레드별 카운트를 글로벌 카운터로
끌어와서, 글로벌 카운터로의 액세스는 그 이름이 말하듯 결과적으로 실제 값으로
수렴하게 만듭니다.
\end{fcvref}

\iffalse

The earlier sections describe ways of allowing each thread to keep its
own copy or its own portion of the data.
In contrast, this section describes a functional-decomposition approach,
where a special designated thread owns the rights to the data
that is required to do its job.
\begin{fcvref}[ln:count:count_stat_eventual:whole:eventual]
The eventually consistent counter implementation described in
Section~\ref{sec:count:Eventually Consistent Implementation} provides an example.
This implementation has a designated thread that runs the
\co{eventual()} function shown on \clnrefrange{b}{e} of
Listing~\ref{lst:count:Array-Based Per-Thread Eventually Consistent Counters}.
This \co{eventual()} thread periodically pulls the per-thread counts
into the global counter, so that accesses to the global counter will,
as the name says, eventually converge on the actual value.
\end{fcvref}

\fi

\QuickQuiz{
	\begin{fcvref}[ln:count:count_stat_eventual:whole:eventual]
	하지만
	Listing~\ref{lst:count:Array-Based Per-Thread Eventually Consistent Counters}
	의 \clnrefrange{b}{e} 에 있는 \co{eventual()} 함수의 어느 부분도 실제로
	\co{eventual()} 쓰레드에 소유되어 있지 않습니다!
	이게 어떻게 데이터 소유권인가요???
	\end{fcvref}

	\iffalse

	\begin{fcvref}[ln:count:count_stat_eventual:whole:eventual]
	But none of the data in the \co{eventual()} function shown on
	\clnrefrange{b}{e} of
	Listing~\ref{lst:count:Array-Based Per-Thread Eventually Consistent Counters}
	is actually owned by the \co{eventual()} thread!
	In just what way is this data ownership???
	\end{fcvref}

	\fi

}\QuickQuizAnswer{
	\begin{fcvref}[ln:count:count_stat_eventual:whole]
	핵심 문구는 ``데이터로의 권한을 소유한다'' 입니다.
	이 경우, 이 권한은 이 리스트의 \clnref{per_thr_cnt} 에 정의된 쓰레드별
	\co{counter} 변수로의 접근 권한입니다.
	이 상황은
	Section~\ref{sec:owned:Partial Data Ownership and pthreads}
	에서 설명한 것과 비슷합니다.

	하지만, \co{eventual()} 에 소유되는 데이터가 실제로 있는데, 이 리스트의
	\clnref{t,sum} 에 정의된 \co{t} 와 \co{sum} 변수들입니다.

	특수 쓰레드의 다른 예들을 위해서는, 리눅스 커널의 커널 쓰레드들을 보실
	수 있겠는데, 예를 들면 \co{kthread_create()} 와 \co{kthread_run()} 에
	의해 만들어지는 것들입니다.
	\end{fcvref}

	\iffalse

	\begin{fcvref}[ln:count:count_stat_eventual:whole]
	The key phrase is ``owns the rights to the data''.
	In this case, the rights in question are the rights to access
	the per-thread \co{counter} variable defined on \clnref{per_thr_cnt}
	of the listing.
	This situation is similar to that described in
	Section~\ref{sec:owned:Partial Data Ownership and pthreads}.

	However, there really is data that is owned by the \co{eventual()}
	thread, namely the \co{t} and \co{sum} variables defined on
	\clnref{t,sum} of the listing.

	For other examples of designated threads, look at the kernel
	threads in the Linux kernel, for example, those created by
	\co{kthread_create()} and \co{kthread_run()}.
	\end{fcvref}

	\fi

}\QuickQuizEnd

\section{Privatization}
\label{sec:owned:Privatization}
%
\epigraph{There is, of course, a difference between what a man seizes
	  and what he really possesses.}
	 {\emph{Pearl S.~Buck}}

One way of improving the performance and scalability of a shared-memory
parallel program is to transform it so as to convert shared data to
private data that is owned by a particular thread.

An excellent example of this is shown in the answer to one of the
Quick Quizzes in
Section~\ref{sec:SMPdesign:Dining Philosophers Problem},
which uses privatization to produce a solution to the
Dining Philosophers problem with much better performance and scalability
than that of the standard textbook solution.
The original problem has five philosophers sitting around the table
with one fork between each adjacent pair of philosophers, which permits
at most two philosophers to eat concurrently.

We can trivially privatize this problem by providing an additional five
forks, so that each philosopher has his or her own private pair of forks.
This allows all five philosophers to eat concurrently, and also offers
a considerable reduction in the spread of certain types of disease.

In other cases, privatization imposes costs.
For example, consider the simple limit counter shown in
\cref{lst:count:Simple Limit Counter Add; Subtract; and Read} on
\cpageref{lst:count:Simple Limit Counter Add; Subtract; and Read}.
This is an example of an algorithm where threads can read each others'
data, but are only permitted to update their own data.
A quick review of the algorithm shows that the only cross-thread
accesses are in the summation loop in \co{read_count()}.
If this loop is eliminated, we move to the more-efficient pure
data ownership, but at the cost of a less-accurate result
from \co{read_count()}.

\QuickQuiz{
	Is it possible to obtain greater accuracy while still
	maintaining full privacy of the per-thread data?
}\QuickQuizAnswer{
	Yes.
	One approach is for \co{read_count()} to add the value
	of its own per-thread variable.
	This maintains full ownership and performance, but only
	a slight improvement in accuracy, particularly on systems
	with very large numbers of threads.

	Another approach is for \co{read_count()} to use function
	shipping, for example, in the form of per-thread signals.
	This greatly improves accuracy, but at a significant performance
	cost for \co{read_count()}.

	However, both of these methods have the advantage of eliminating
	cache thrashing for the common case of updating counters.
}\QuickQuizEnd

Partial privatization is also possible, with some synchronization
requirements, but less than in the fully shared case.
Some partial-privatization possibilities were explored in
\cref{sec:toolsoftrade:Avoiding Data Races}.
\Cref{chp:Deferred Processing} will introduce a temporal component
to data ownership by providing ways of safely taking public data
structures private.

In short, privatization is a powerful tool in the parallel programmer's
toolbox, but it must nevertheless be used with care.
Just like every other synchronization primitive, it has the potential
to increase complexity while decreasing performance and scalability.

\section{Other Uses of Data Ownership}
\label{sec:owned:Other Uses of Data Ownership}
%
\epigraph{Everything comes to us that belongs to us if we create the
	  capacity to receive it.}
	 {\emph{Rabindranath Tagore}}

Data ownership works best when the data can be partitioned so that there
is little or no need for cross thread access or update.
Fortunately, this situation is reasonably common, and in a wide variety
of parallel-programming environments.

Examples of data ownership include:

\begin{enumerate}
\item	All message-passing environments, such as MPI~\cite{MPIForum2008}
	and BOINC~\cite{BOINC2008}.
\item	Map-reduce~\cite{MapReduce2008MIT}.
\item	Client-server systems, including RPC, web services, and
	pretty much any system with a back-end database server.
\item	Shared-nothing database systems.
\item	Fork-join systems with separate per-process address spaces.
\item	Process-based parallelism, such as the Erlang language.
\item	Private variables, for example, C-language on-stack auto variables,
	in threaded environments.
\item	Many parallel linear-algebra algorithms, especially those
	well-suited for GPGPUs.\footnote{
		But note that a great many other classes of applications
		have also been ported to
		GPGPUs~\cite{NormMatloff2017ParProcBook,AMD2020ROCm,NVidia2017GPGPU,NVidia2017GPGPU-university}.}
\item	Operating-system kernels adapted for networking, where each connection
	(also called \emph{flow}~\cite{Shenker89,ZhangPhD,McKenney90})
	is assigned to a specific thread.
	One recent example of this approach is the IX operating
	system~\cite{Belay:2016:IOS:3014162.2997641}.
	IX does have some shared data structures, which use synchronization
	mechanisms to be described in
	Section~\ref{sec:defer:Read-Copy Update (RCU)}.
\end{enumerate}

Data ownership is perhaps the most underappreciated synchronization
mechanism in existence.
When used properly, it delivers unrivaled simplicity, performance,
and scalability.
Perhaps its simplicity costs it the respect that it deserves.
Hopefully a greater appreciation for the subtlety and power of data ownership
will lead to greater level of respect, to say nothing of leading to
greater performance and scalability coupled with reduced complexity.

% populate with problems showing benefits of coupling data ownership with
% other approaches. For example, work-stealing schedulers. Perhaps also
% move memory allocation here, though its current location is quite good.

\QuickQuizAnswersChp{qqzowned}
