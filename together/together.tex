% together/together.tex
% mainfile: ../perfbook.tex
% SPDX-License-Identifier: CC-BY-SA-3.0

\QuickQuizChapter{chp:Putting It All Together}{Putting It All Together}{qqztogether}
%
\Epigraph{You don't learn how to shoot and then learn how to launch
	  and then learn to do a controlled spin---you learn to
	  launch-shoot-spin.}{\emph{``Ender's Shadow'', Orson Scott Card}}

% And the paragraph preceding this is also instructive:
% ``I may be pissed off, but that doesn't mean I can't learn.''

이 \lcnamecref{chp:Putting It All Together} 는 동시성 프로그래밍 퍼즐들에 대한
힌트를 약간 제공합니다.
\Cref{sec:together:Counter Conundrums}
는 카운터 수수께끼를 생각해 보고,
\cref{sec:together:Refurbish Reference Counting}
는 레퍼런스 카운팅을 다시 들여다보며,
\cref{sec:together:Hazard-Pointer Helpers}
는 해저드 포인터를 돕고,
\cref{sec:together:Sequence-Locking Specials}
는 시퀀스 락킹의 특수한 경우들을 요약하며, 마지막으로
\cref{sec:together:RCU Rescues}
는 RCU 의 구조를 알아봅니다.

\iffalse

This \lcnamecref{chp:Putting It All Together}
gives some hints on concurrent-programming puzzles.
\Cref{sec:together:Counter Conundrums}
considers counter conundrums,
\cref{sec:together:Refurbish Reference Counting}
refurbishes reference counting,
\cref{sec:together:Hazard-Pointer Helpers}
helps with hazard pointers,
\cref{sec:together:Sequence-Locking Specials}
surmises on sequence-locking specials,
and finally
\cref{sec:together:RCU Rescues}
reflects on RCU rescues.

\fi

% together/count.tex
% mainfile: ../perfbook.tex
% SPDX-License-Identifier: CC-BY-SA-3.0

\section{Counter Conundrums}
\label{sec:together:Counter Conundrums}
%
\epigraph{Ford carried on counting quietly.
	  This is about the most aggressive thing you can do to a
	  computer, the equivalent of going up to a human being and saying
	  ``Blood \dots blood \dots blood \dots blood \dots''}
	 {\emph{Douglas Adams}}

이 \lcnamecref{sec:together:Counter Conundrums} 는 카운터 수수께끼들에 대한
해결책들을 정리해 봅니다.

\iffalse

This \lcnamecref{sec:together:Counter Conundrums}
outlines solutions to counter conundrums.

\fi

\subsection{Counting Updates}
\label{sec:together:Counting Updates}

Schr\"odinger
(\cref{sec:datastruct:Motivating Application} 참고) 가 각 동물의 업데이트
횟수를 세고 싶어하며, 이 업데이트는 데이터 원소별 락을 이용해 동기화 된다고
해봅시다.
이 카운팅은 어떻게 해야 가장 잘 할 수 있을까요?

물론, \cref{chp:Counting} 에서의 카운팅 알고리즘들이 얼마든지 충분할 수도
있지만, 최적의 방법은 상당히 간단합니다.
각 데이터 원소에 카운터를 놓고, 그 원소의 락의 보호 아래 그 카운터를 증가시키는
겁니다!

읽기 쓰레드가 락 없이 이 숫자에 접근하면, 업데이트 쓰레드는 이 카운터를
업데이트 하는데에 \co{WRITE_ONCE()} 를 사용하고 락 없는 읽기 쓰레드는 이를 읽기
위해 \co{READ_ONCE()} 를 사용해야 합니다.

\iffalse

Suppose that Schr\"odinger (see
\cref{sec:datastruct:Motivating Application})
wants to count the number of updates for each animal,
and that these updates are synchronized using a per-data-element lock.
How can this counting best be done?

Of course, any number of counting algorithms from \cref{chp:Counting}
might qualify, but the optimal approach is quite simple.
Just place a counter in each data element, and increment it under the
protection of that element's lock!

If readers access the count locklessly, then updaters should use
\co{WRITE_ONCE()} to update the counter and lockless readers should
use \co{READ_ONCE()} to load it.

\fi

\subsection{Counting Lookups}
\label{sec:together:Counting Lookups}

Suppose that Schr\"odinger also wants to count the number of lookups for
each animal, where lookups are protected by RCU\@.
How can this counting best be done?

One approach would be to protect a lookup counter with the per-element
lock, as discussed in \cref{sec:together:Counting Updates}.
Unfortunately, this would require all lookups to acquire this lock,
which would be a severe bottleneck on large systems.

Another approach is to ``just say no'' to counting, following the example
of the \co{noatime} mount option.
If this approach is feasible, it is clearly the best:  After all, nothing
is faster than doing nothing.
If the lookup count cannot be dispensed with, read on!

Any of the counters from \cref{chp:Counting}
could be pressed into service, with the statistical counters described in
\cref{sec:count:Statistical Counters} being perhaps the most common choice.
However, this results in a large memory footprint: The number of counters
required is the number of data elements multiplied by the number of
threads.

If this memory overhead is excessive, then one approach is to keep
per-core or even per-socket counters rather than per-CPU counters,
with an eye to the hash-table performance results depicted in
\cref{fig:datastruct:Read-Only Hash-Table Performance For Schroedinger's Zoo; 448 CPUs}.
This will require that the counter increments be atomic operations,
especially for user-mode execution where a given thread could migrate
to another CPU at any time.

If some elements are looked up very frequently, there are a number
of approaches that batch updates by maintaining a per-thread log,
where multiple log entries for a given element can be merged.
After a given log entry has a sufficiently large increment or after
sufficient time has passed, the log entries may be applied to the
corresponding data elements.
Silas Boyd-Wickizer has done some work formalizing this
notion~\cite{SilasBoydWickizerPhD}.

% together/refcnt.tex

\section{Refurbish Reference Counting}
\label{sec:together:Refurbish Reference Counting}

레퍼런스 카운팅이 개념적으로는 간단한 테크닉이지만, 동시적 소프트웨어에
적용되면 그 디테일 안에 많은 악마들이 숨어있습니다.
무엇보다도, 오브젝트가 지나치게 일찍 폐기되는 일이 없는 오브젝트라면, 애초에
레퍼런스 카운터가 필요하지도 않았을 겁니다.
하지만 오브젝트가 폐기될 수 있다면, 레퍼런스를 얻어오는 프로세스 그 자체가
진행되는 동안 폐기되는 것을 무엇으로 막을 수 있을까요?

동시적 소프트웨어에서 사용되기 위해 레퍼런스 카운터를 재정비하는 몇가지
방법들이 있는데, 다음과 같은 것들이 포함됩니다:
\iffalse

Although reference counting is a conceptually simple technique,
many devils hide in the details when it is applied to concurrent
software.
After all, if the object was not subject to premature disposal,
there would be no need for the reference counter in the first place.
But if the object can be disposed of, what prevents disposal during
the reference-acquisition process itself?

There are a number of ways to refurbish reference counters for
use in concurrent software, including:
\fi

\begin{enumerate}
\item	레퍼런스 카운트를 조정하는 동안에는 바깥에 존재하는 오브젝트의 락을
	잡아야만 하기.
\item	오브젝트는 0이 아닌 레퍼런스 카운트와 함께 생성되고, 새로운
	레퍼런스들은 레퍼런스 카운터의 현재 값이 0이 아닐 때에만 획득될 수 있게
		하기.
	어떤 레퍼런스가 특정 오브젝트로의 레퍼런스를 가지고 있지 않다면, 이미
		레퍼런스를 가지고 있는 다른 쓰레드의 도움으로 해당 오브젝트로의
		레퍼런스를 얻을 수 있음.
\item	해당 오브젝트를 위한 존재 보장이 제공되어서, 해당 오브젝트가 다른
	무언가가 레퍼런스를 얻으려 시도하는 동안은 해제되지 않도록 하기.
	존재 보장은 자동화된 garbage collector 를 통해서, 또는
	Section~\ref{sec:defer:Read-Copy Update (RCU)} 에서 보여지듯이 RCU 를
	통해서 제공되곤 합니다.
\item	오브젝트를 위한 타입 안정성을 제공하기.
	레퍼런스가 일단 획득되면 추가적인 아이덴티티 체크가 이뤄져야만 합니다.
	타입 안정성 보장은 예를 들어,
	Section~\ref{sec:defer:Read-Copy Update (RCU)} 에서 보여진 것과 같이
	리눅스 커널 안에서 \co{SLAB_DESTROY_BY_RCU} 기능과 같이 특수 목적
	메모리 할당자를 통해 제공될 수 있습니다.
\iffalse

\item	A lock residing outside of the object must be held while
	manipulating the reference count.
\item	The object is created with a non-zero reference count, and new
	references may be acquired only when the current value of
	the reference counter is non-zero.
	If a thread does not have a reference to a given object,
	it may obtain one with the help of another thread that
	already has a reference.
\item	An existence guarantee is provided for the object, preventing
	it from being freed while some other
	entity might be attempting to acquire a reference.
	Existence guarantees are often provided by automatic
	garbage collectors, and, as will be seen in
	Section~\ref{sec:defer:Read-Copy Update (RCU)}, by RCU.
\item	A type-safety guarantee is provided for the object.
	An additional identity check must be performed once
	the reference is acquired.
	Type-safety guarantees can be provided by special-purpose
	memory allocators, for example, by the
	\co{SLAB_DESTROY_BY_RCU} feature within the Linux kernel,
	as will be seen in Section~\ref{sec:defer:Read-Copy Update (RCU)}.
\fi
\end{enumerate}

물론, 존재 보장을 제공하는 모든 메커니즘은 그 정의에 의해 타입 안정성 보장도
제공합니다.
따라서 이 섹션은 마지막의 두 답들을 RCU 의 전례 하에 레퍼런스 획득 보호에
일반적으로 사용되는 세개의 카테고리로 그룹지어 보겠습니다.: 레퍼런스 카운팅,
시퀀스 락킹, 그리고 RCU.
\iffalse

Of course, any mechanism that provides existence guarantees
by definition also provides type-safety guarantees.
This section will therefore group the last two answers together under the
rubric of RCU, leaving us with three general categories of
reference-acquisition protection: Reference counting, sequence
locking, and RCU.
\fi

\QuickQuiz{}
	레퍼런스 획득을 단순히 레퍼런스 카운터의 값이 0이 아닌 경우에만
	레퍼런스를 획득하는 compare-and-swap 오퍼레이션으로 간단하게 만들지는
	않는거죠?
	\iffalse

	Why not implement reference-acquisition using
	a simple compare-and-swap operation that only
	acquires a reference if the reference counter is
	non-zero?
	\fi
\QuickQuizAnswer{
	이 방법이 마지막 레퍼런스의 해제와 새로운 레퍼런스 획득 사이의 경주를
	해결할 수 는 있겠지만, 데이터 구조체가 해제되고 전혀 다른 종류의
	구조체로 재할당 되는 것을 방지하는데에 있어서는 어떤 일도 해주지
	않습니다.
	``간단한 compare-and-swap 오퍼레이션'' 이 다른 종류의 구조체에 적용되면
	정의되지 않은 결과를 낼 가능성이 상당히 큽니다.

	짧게 말해서, compare-and-swap 과 같은 어토믹 오퍼레이션의 사용은 타입
	안정성이나 존재 보장을 필요로 합니다.
	\iffalse

	Although this can resolve the race between the release of
	the last reference and acquisition of a new reference,
	it does absolutely nothing to prevent the data structure
	from being freed and reallocated, possibly as some completely
	different type of structure.
	It is quite likely that the ``simple compare-and-swap
	operation'' would give undefined results if applied to the
	differently typed structure.

	In short, use of atomic operations such as compare-and-swap
	absolutely requires either type-safety or existence guarantees.
	\fi
} \QuickQuizEnd

\begin{table}[tb]
\footnotesize
\centering
\begin{tabular}{l||c|c|c}
	& \multicolumn{3}{c}{Release Synchronization} \\
	\cline{2-4}
	Acquisition     &         & Reference &     \\
	Synchronization & Locking & Counting  & RCU \\
	\hline
	\hline
	Locking		& -	  & CAM	      & CA  \\
	\hline
	Reference	& A	  & AM	      & A   \\
	Counting	&  	  &   	      &     \\
	\hline
	RCU		& CA	  & MCA	      & CA  \\
\end{tabular}
\caption{Reference Counting and Synchronization Mechanisms}
\label{tab:together:Reference Counting and Synchronization Mechanisms}
\end{table}

레퍼런스 카운팅의 핵심 이슈는 레퍼런스의 획득과 오브젝트의 해제 사이의 동기화란
점을 놓고 보면, 우린
Table~\ref{tab:together:Reference Counting and Synchronization Mechanisms} 에
보인 것과 같은 아홉개의 메커니즘 조합들이 존재 가능합니다.
이 표는 레퍼런스 카운팅 메커니즘들을 다음과 같이 넓은 카테고리들로 나눕니다:
\iffalse

Given that the key reference-counting issue
is synchronization between acquisition
of a reference and freeing of the object, we have nine possible
combinations of mechanisms, as shown in
Table~\ref{tab:together:Reference Counting and Synchronization Mechanisms}.
This table
divides reference-counting mechanisms into the following broad categories:
\fi
\begin{enumerate}
\item	어토믹 오퍼레이션, 메모리 베리어, 정렬 제약도 없는 간단한 카운팅
	\makebox{(``-'')}.
\item	메모리 배리어 없이 진행되는 어토믹 카운팅 (``A'').
\item	릴리즈 시에만 메모리 배리어를 사용하는 어토믹 카운팅 (``AM'').
\item	어토믹 획득 오퍼레이션과 릴리즈 시에만 필요시 되는 메모리 배리어들과
	조합된 체크를 하는 어토믹 카운팅 (``CAM'').
\item	어토믹 획득 오퍼레이션과 조합된 체크를 하는 어토믹 카운팅 (``CA'').
\item	어토믹 획득 오퍼레이션과 레퍼런스 획득 때에도 필요시 되는 메모리
	배리어들과 조합된 체크를 사용하는 어토믹 카운팅 (``MCA'').
\iffalse

\item	Simple counting with neither atomic operations, memory
	barriers, nor alignment constraints \makebox{(``-'')}.
\item	Atomic counting without memory barriers (``A'').
\item	Atomic counting, with memory barriers required only on release
	(``AM'').
\item	Atomic counting with a check combined with the atomic acquisition
	operation, and with memory barriers required only on release
	(``CAM'').
\item	Atomic counting with a check combined with the atomic acquisition
	operation (``CA'').
\item	Atomic counting with a check combined with the atomic acquisition
	operation, and with memory barriers also required on acquisition
	(``MCA'').
\fi
\end{enumerate}
하지만, 값을 리턴하는 모든 리눅스 커널 어토믹 오퍼레이션들은 메모리 배리어를
포함하도록 정의되어 있으므로,\footnote{
	\co{atomic_read()} 와 \co{ATOMIC_INIT()} 는 이 규칙에 예외가 됩니다.}
모든 릴리즈 오퍼레이션들은 메모리 배리어를 포함하고, 모든 체크되는 레퍼런스
획득 오퍼레이션들 또한 메모리 배리어를 포함하게 됩니다.
따라서, ``CA'' 와 ``MCA'' 는 ``CAM'' 과 동일해서, 앞의 네개의 경우들을 위한
섹션들만이 남게 됩니다:
\makebox{``-''}, ``A'', ``AM'', 그리고 ``CAM''.
레퍼런스 카운팅을 지원하는 리눅스의 기능들은
Section~\ref{sec:together:Linux Primitives Supporting Reference Counting} 에
소개되어 있습니다.
뒤의 섹션들은 레퍼런스 획득과 해제가 매우 빈번할 때, 그리고 레퍼런스 카운트가
매우 가끔씩만 0인지 여부를 체크해야 하는 경우에 대해 성능을 개선할 수 있는
최적화 방법들을 인용합니다.
\iffalse

However, because all Linux-kernel atomic operations that return a
value are defined to contain memory barriers,\footnote{
	With \co{atomic_read()} and \co{ATOMIC_INIT()} being the
	exceptions that prove the rule.}
all release operations
contain memory barriers, and all checked acquisition operations also
contain memory barriers.
Therefore, cases ``CA'' and ``MCA'' are equivalent to ``CAM'', so that
there are sections below for only the first four cases:
\makebox{``-''}, ``A'', ``AM'', and ``CAM''.
The Linux primitives that support reference counting are presented in
Section~\ref{sec:together:Linux Primitives Supporting Reference Counting}.
Later sections cite optimizations that can improve performance
if reference acquisition and release is very frequent, and the
reference count need be checked for zero only very rarely.
\fi

\subsection{Implementation of Reference-Counting Categories}
\label{sec:together:Implementation of Reference-Counting Categories}

락킹으로 보호되는 간단한 카운팅 (\makebox{``-''}) 이
Section~\ref{sec:together:Simple Counting} 에 설명되고,
메모리 배리어 없이 수행되는 어토믹한 카운팅 (``A'') 이
Section~\ref{sec:together:Atomic Counting} 에 설명되고,
레퍼런스 획득 시의 메모리 배리어와 함께 수행되는 어토믹한 카운팅 (``AM'') 이
Section~\ref{sec:together:Atomic Counting With Release Memory Barrier} 에 설명되며,
레퍼런스 획득시의 메모리 배리어와 체크와 함께 수행되는 어토믹한 카운팅 (``CAM'') 이
Section~\ref{sec:together:Atomic Counting With Check and Release Memory Barrier}
에서 설명됩니다.
\iffalse

Simple counting protected by locking (\makebox{``-''}) is described in
Section~\ref{sec:together:Simple Counting},
atomic counting with no memory barriers (``A'') is described in
Section~\ref{sec:together:Atomic Counting},
atomic counting with acquisition memory barrier (``AM'') is described in
Section~\ref{sec:together:Atomic Counting With Release Memory Barrier},
and
atomic counting with check and release memory barrier (``CAM'') is described in
Section~\ref{sec:together:Atomic Counting With Check and Release Memory Barrier}.
\fi

\subsubsection{Simple Counting}
\label{sec:together:Simple Counting}

어토믹 오퍼레이션도 메모리 배리어도 사용하지 않는 간단한 카운팅은 레퍼런스
카운터 획득과 해제가 모두 같은 락으로 보호될 때에 사용될 수 있습니다.
이 경우에 레퍼런스 카운터 자체는 어토믹하지 않게 조정될 것이란 점이 분명한데,
락은 모든 필요한 배타성, 메모리 배리어, 어토믹 인스트럭션, 그리고 컴파일러
최적화에 대한 방지를 제공하기 때문입니다.
이는 해당 락이 레퍼런스 카운트 만이 아니라 다른 오퍼레이션들도 보호해야 하지만
해당 오브젝트로의 레퍼런스가 해당 락을 해제한 후에 잡아야만 할 때에 선택될 수
있는 방법입니다.
Figure~\ref{fig:together:Simple Reference-Count API} 는 간단한 어토믹하지 않은
레퍼런스 카운팅을 구현하는데 사용될 수 있는 간단한 API 를 보입니다---간단한
레퍼런스 카운팅은 거의 항상 인터페이스 없이 곧바로 코딩되곤 하지만요.
\iffalse

Simple counting, with neither atomic operations nor memory barriers,
can be used when the reference-counter acquisition and release are
both protected by the same lock.
In this case, it should be clear that the reference count itself
may be manipulated non-atomically, because the lock provides any
necessary exclusion, memory barriers, atomic instructions, and disabling
of compiler optimizations.
This is the method of choice when the lock is required to protect
other operations in addition to the reference count, but where
a reference to the object must be held after the lock is released.
Figure~\ref{fig:together:Simple Reference-Count API} shows a simple
API that might be used to implement simple non-atomic reference
counting---although simple reference counting is almost always
open-coded instead.
\fi

\begin{figure}[tbp]
{ \scriptsize
\begin{verbbox}
  1 struct sref {
  2   int refcount;
  3 };
  4
  5 void sref_init(struct sref *sref)
  6 {
  7   sref->refcount = 1;
  8 }
  9
 10 void sref_get(struct sref *sref)
 11 {
 12   sref->refcount++;
 13 }
 14
 15 int sref_put(struct sref *sref,
 16              void (*release)(struct sref *sref))
 17 {
 18   WARN_ON(release == NULL);
 19   WARN_ON(release == (void (*)(struct sref *))kfree);
 20
 21   if (--sref->refcount == 0) {
 22     release(sref);
 23     return 1;
 24   }
 25   return 0;
 26 }
\end{verbbox}
}
\centering
\theverbbox
\caption{Simple Reference-Count API}
\label{fig:together:Simple Reference-Count API}
\end{figure}

\subsubsection{Atomic Counting}
\label{sec:together:Atomic Counting}

간단한 어토믹 카운팅은 레퍼런스를 획득하는 모든 CPU 는 레퍼런스를 이미 잡고
있어야 하는 경우에 사용될 수 있습니다.
이 스타일은 하나의 CPU 가 자신의 사용을 위해 오브젝트를 생성하지만 나중에
태어난 다른 CPU, 태스크, 타이머 핸들러, 또는 I/O 완료 핸들러들이 그 오브젝트에
접근할 수 있도록 해야만 할 때 사용됩니다.
이 오브젝트를 넘겨주는 CPU 는 받게되는 오브젝트를 위해 새로운 레퍼런스를 먼저
획득해야만 합니다.
리눅스 커널에서는, \co{kref} 기능이 이런 스타일의 레퍼런스 카운팅을 구현하기
위해 사용되는데,
Figure~\ref{fig:together:Linux Kernel kref API} 에 보여져 있습니다.
\iffalse

Simple atomic counting may be used in cases where any CPU acquiring
a reference must already hold a reference.
This style is used when a single CPU creates an object for its
own private use, but must allow other CPU, tasks, timer handlers,
or I/O completion handlers that it later spawns to also access this object.
Any CPU that hands the object off must first acquire a new reference
on behalf of the recipient object.
In the Linux kernel, the \co{kref} primitives are used to implement
this style of reference counting, as shown in
Figure~\ref{fig:together:Linux Kernel kref API}.
\fi

모든 레퍼런스 카운팅 오퍼레이션들이 락킹으로 보호되지는 않는데, 이는 두개의
다른 CPU 들이 동시에 레퍼런스 카운트를 조정할 수 있음을 의미하기 때문에 어토믹
카운팅이 필요해집니다.
평범한 값 증가와 감소 오퍼레이션이 사용된다면, 한쌍의 CPU 들이 둘 다 레퍼런스
카운트를 동시에 가져와서, ``3'' 이라는 값을 읽을 수 있을 겁니다.
둘 다 그 값을 증가시키려 한다면, 이들은 모두 ``4'' 를 얻게 될테고, 둘 다 이
값을 카운터에 다시 써넣을 겁니다.
카운터의 새로운 값은 ``5'' 가 되었어야 하므로, 두개의 값 증가 오퍼레이션 중
하나는 없어진 셈입니다.
따라서, 카운터의 값 증가에도 값 감소에도 어토믹 오퍼레이션이 사용되어야만
합니다.
\iffalse

Atomic counting is required
because locking is not used to protect all reference-count operations,
which means that it is possible for two different CPUs to concurrently
manipulate the reference count.
If normal increment and decrement were used, a pair of CPUs might both
fetch the reference count concurrently, perhaps both obtaining
the value ``3''.
If both of them increment their value, they will both obtain ``4'',
and both will store this value back into the counter.
Since the new value of the counter should instead be ``5'', one
of the two increments has been lost.
Therefore, atomic operations must be used both for counter increments
and for counter decrements.
\fi

레퍼런스 릴리즈가 락킹이나 RCU 로 보호된다면, 메모리 배리어는 필요하지
\emph{않을테지만}, 다른 이유로 인해서입니다.
락킹의 경우, 락은 모든 필요한 메모리 배리어들을 제공하고 (그리고 컴파일러
최적화를 불능화 시킵니다),락들은 또한 두개의 레퍼런스 릴리즈가 동시에 수행되는
것을 막습니다.
RCU 의 경우, 현재 수행중인 RCU read-side 크리티컬 섹션들이 모두 완료되기
전까지는 정리 작업이 유예되어야만 하고, 모든 필요한 메모리 배리어들이나
컴파일러 최적화의 불능화가 RCU 기능으로 제공될 겁니다.
따라서, 두개의 CPU 들이 두개의 레퍼런스를 동시에 릴리즈 시킨다면, 실제
정리작업은 두개의 CPU 들이 모두 RCU read-side 크리티컬 섹션들을 빠져나갈 때까지
유예될 겁니다.
\iffalse

If releases are guarded by locking or RCU,
memory barriers are \emph{not} required, but for different reasons.
In the case of locking, the locks provide any needed memory barriers
(and disabling of compiler optimizations), and the locks also
prevent a pair of releases from running concurrently.
In the case of RCU, cleanup must be deferred until all currently
executing RCU read-side critical sections have completed, and
any needed memory barriers or disabling of compiler optimizations
will be provided by the RCU infrastructure.
Therefore, if two CPUs release the final two references concurrently,
the actual cleanup will be deferred until both CPUs exit their
RCU read-side critical sections.
\fi

\QuickQuiz{}
	한 CPU 가 마지막 레퍼런스를 해제한 직후에 다른 CPU 가 레퍼런스를
	획득하는 경우에 대해서도 보호를 해줘야 하지 않나요?
	\iffalse

	Why isn't it necessary to guard against cases where one CPU
	acquires a reference just after another CPU releases the last
	reference?
	\fi
\QuickQuizAnswer{
	CPU 는 합법적으로 다른 레퍼런스를 얻기 위해서는 이미 레퍼런스를 가지고
	있어야 하기 때문입니다.
	따라서, 한 CPU 가 마지막 레퍼런스를 해제했다면, 새로운 레퍼런스를
	획득할 수 있는 CPU 는 존재할 수가 없습니다.
	이와 같은 사실이
	Figure~\ref{fig:together:Linux Kernel kref API} 의 line~22 의
	어토믹하지 않은 검사를 가능하게 합니다.
	\iffalse

	Because a CPU must already hold a reference in order
	to legally acquire another reference.
	Therefore, if one CPU releases the last reference,
	there cannot possibly be any CPU that is permitted
	to acquire a new reference.
	This same fact allows the non-atomic check in line~22
	of Figure~\ref{fig:together:Linux Kernel kref API}.
	\fi
} \QuickQuizEnd

\begin{figure}[tbp]
{ \scriptsize
\begin{verbbox}
  1 struct kref {
  2   atomic_t refcount;
  3 };
  4 
  5 void kref_init(struct kref *kref)
  6 {
  7   atomic_set(&kref->refcount, 1);
  8 }
  9 
 10 void kref_get(struct kref *kref)
 11 {
 12   WARN_ON(!atomic_read(&kref->refcount));
 13   atomic_inc(&kref->refcount);
 14 }
 15 
 16 static inline int
 17 kref_sub(struct kref *kref, unsigned int count,
 18          void (*release)(struct kref *kref))
 19 {
 20   WARN_ON(release == NULL);
 21 
 22   if (atomic_sub_and_test((int) count,
 23                           &kref->refcount)) {
 24     release(kref);
 25     return 1;
 26   }
 27   return 0;
 28 }
\end{verbbox}
}
\centering
\theverbbox
\caption{Linux Kernel kref API}
\label{fig:together:Linux Kernel kref API}
\end{figure}

\co{kref} 구조체 자체는 하나의 어토믹 데이터 아이템으로 구성되며,
Figure~\ref{fig:together:Linux Kernel kref API} 의 line~1-3 에 보여져 있습니다.
Line~5-8 의 \co{kref_init()} 함수는 이 카운터를 값 ``1'' 로 초기화 시킵니다.
\co{atomic_set()} 기능은 단순한 값 할당으로, 그 이름은 \co{atomic_t} 의 데이터
타입에서 왔지, 그 동작에서 온것이 아님을 알아두시기 바랍니다.
\co{kref_init()} 함수는 오브젝트 생성 때에, 해당 오브젝트가 어떤 다른 CPU 들에
의해 접근될 수 있게 되기 전에 실행되어야만 합니다.
\iffalse

The \co{kref} structure itself, consisting of a single atomic
data item, is shown in lines~1-3 of
Figure~\ref{fig:together:Linux Kernel kref API}.
The \co{kref_init()} function on lines~5-8 initializes the counter
to the value ``1''.
Note that the \co{atomic_set()} primitive is a simple
assignment, the name stems from the data type of \co{atomic_t}
rather than from the operation.
The \co{kref_init()} function must be invoked during object creation,
before the object has been made available to any other CPU.
\fi

Line~10-14 의 \co{kref_get()} 함수는 무조건적으로 카운터를 어토믹하게
증가시킵니다.
\co{atomic_inc()} 기능은 모든 플랫폼에서 컴파일러 최적화를 명시적으로 불능화
시켜야만 하지는 않습니다만, \co{kref} 이 별개의 모듈에 존재하고 리눅스 커널
빌드 프로세스는 모듈간 최적화를 하지 않는다는 사실이 똑같은 효과를 냅니다.

Line~16-28 의 \co{kref_sub()} 함수는 어토믹하게 카운터를 감소시키고, 만약 그
결과가 0이라면, line~24 에서 명시된 \co{release()} 함수를 호출하고 line~25 에서
호출자에게 \co{release()} 가 호출되었음을 알리면서 리턴합니다.
그렇지 않다면, \co{kref_sub()} 은 0을 리턴해서 호출자에게 \co{release()} 가
호출되지 않았음을 알립니다.
\iffalse

The \co{kref_get()} function on lines~10-14 unconditionally atomically
increments the counter.
The \co{atomic_inc()} primitive does not necessarily explicitly
disable compiler
optimizations on all platforms, but the fact that the \co{kref}
primitives are in a separate module and that the Linux kernel build
process does no cross-module optimizations has the same effect.

The \co{kref_sub()} function on lines~16-28 atomically decrements the
counter, and if the result is zero, line~24 invokes the specified
\co{release()} function and line~25 returns, informing the caller
that \co{release()} was invoked.
Otherwise, \co{kref_sub()} returns zero, informing the caller that
\co{release()} was not called.
\fi

\QuickQuiz{}
	Figure~\ref{fig:together:Linux Kernel kref API} 의 line~22 에서
	\co{atomic_sub_and_test()} 가 호출된 직후에, 어떤 다른 CPU 가
	\co{kref_get()} 을 호출했다고 생각해 봅시다.
	이는 이 다른 CPU 가 이제 비합법적으로 해제된 오브젝트로의 레퍼런스를
	갖게 된 거 아닌가요?
	\iffalse

	Suppose that just after the \co{atomic_sub_and_test()}
	on line~22 of
	Figure~\ref{fig:together:Linux Kernel kref API} is invoked,
	that some other CPU invokes \co{kref_get()}.
	Doesn't this result in that other CPU now having an illegal
	reference to a released object?
	\fi
\QuickQuizAnswer{
	그런 일은 이 함수가 올바르게 사용된다면 일어날 수 없는 일입니다.
	이미 레퍼런스를 가지고 있지 않다면 \co{kref_get()} 을 호출하는 것은
	비합법적인 일이어서 \co{kref_sub()} 는 카운터의 값을 0으로 감소시킬 수
	없었을 겁니다.
	\iffalse

	This cannot happen if these functions are used correctly.
	It is illegal to invoke \co{kref_get()} unless you already
	hold a reference, in which case the \co{kref_sub()} could
	not possibly have decremented the counter to zero.
	\fi
} \QuickQuizEnd

\QuickQuiz{}
	\co{kref_sub()} 가 0을 리턴해서 \co{release()} 함수가 호출되지 않았음을
	알렸다고 생각해 봅시다.
	어떤 조건에서 호출자는 이 오브젝트의 존재의 지속에 의존할 수 있을까요?
	\iffalse

	Suppose that \co{kref_sub()} returns zero, indicating that
	the \co{release()} function was not invoked.
	Under what conditions can the caller rely on the continued
	existence of the enclosing object?
	\fi
\QuickQuizAnswer{
	호출자는 최소 하나의 레퍼런스가 계속해서 존재할 것임을 알지 못한다면
	오브젝트의 존재 지속에 의존할 수 없습니다.
	일반적으로, 호출자는 이를 알 방법이 없을 것이고, 따라서 \co{kref_sub()}
	후에 오브젝트를 참조하는 것을 주의깊게 피해야만 합니다.
	\iffalse

	The caller cannot rely on the continued existence of the
	object unless it knows that at least one reference will
	continue to exist.
	Normally, the caller will have no way of knowing this, and
	must therefore carefullly avoid referencing the object after
	the call to \co{kref_sub()}.
	\fi
} \QuickQuizEnd

\QuickQuiz{}
	왜 그냥 해제 함수로 \co{kfree()} 를 넘기지 않는거죠?
	\iffalse

	Why not just pass \co{kfree()} as the release function?
	\fi
\QuickQuizAnswer{
	일반적으로 \co{kref} 구조체는 더 커다란 구조체에 포함되어 있게 되므로,
	\co{kref} 필드만이 아니라 전체 구조체를 해제시킬 필요가 있습니다.
	이는 일반적으로 \co{container_of()} 와 \co{kfree()} 를 호출하는 wrapper
	함수를 정의하는 것으로 이뤄질 수 있습니다.
	\iffalse

	Because the \co{kref} structure normally is embedded in
	a larger structure, and it is necessary to free the entire
	structure, not just the \co{kref} field.
	This is normally accomplished by defining a wrapper function
	that does a \co{container_of()} and then a \co{kfree()}.
	\fi
} \QuickQuizEnd

\subsubsection{Atomic Counting With Release Memory Barrier}
\label{sec:together:Atomic Counting With Release Memory Barrier}

이런 스타일의 레퍼런스는 리눅스 커널의 네트워킹 레이어에서 패킷 라우팅에
사용되는 목적지 캐시를 추적하는데에 사용됩니다.
실제 구현은 훨씬 더 관련되어 있습니다; 이 섹션은 이 사용예에 적합하며
Figure~\ref{fig:together:Linux Kernel dst-clone API} 에 보인
\co{struct dst_entry} 레퍼런스 카운트 핸들링 부분에 집중하고 있습니다.
\iffalse

This style of reference is used in the Linux kernel's networking
layer to track the destination caches that are used in packet routing.
The actual implementation is quite a bit more involved; this section
focuses on the aspects of \co{struct dst_entry} reference-count
handling that matches this use case,
shown in Figure~\ref{fig:together:Linux Kernel dst-clone API}.
\fi

\begin{figure}[tbp]
{ \scriptsize
\begin{verbbox}
  1 static inline
  2 struct dst_entry * dst_clone(struct dst_entry * dst)
  3 {
  4   if (dst)
  5     atomic_inc(&dst->__refcnt);
  6   return dst;
  7 }
  8
  9 static inline
 10 void dst_release(struct dst_entry * dst)
 11 {
 12   if (dst) {
 13     WARN_ON(atomic_read(&dst->__refcnt) < 1);
 14     smp_mb__before_atomic_dec();
 15     atomic_dec(&dst->__refcnt);
 16   }
 17 }
\end{verbbox}
}
\centering
\theverbbox
\caption{Linux Kernel dst\_clone API}
\label{fig:together:Linux Kernel dst-clone API}
\end{figure}

\co{dst_clone()} 함수는 호출자가 이미 명시된 \co{dst_entry} 에 대한 레퍼런스를
가지고 있을 때 사용될 수 있는데, 이는 커널 내의 다른 존재에게 넘겨줄 수도 있는
또다른 레퍼런스를 획득하는 경우입니다.
호출자에 의해 이미 레퍼런스가 하나 잡혀 있기 때문에, \co{dst_clone()} 은 어떤
메모리 배리어도 실행할 필요가 없습니다.
어떤 다른 존재에게 \co{dst_entry} 를 넘겨주는 행위는 메모리 배리어를 필요로
할수도, 필요로 하지 않을 수도 있습니다만, 그런 메모리 배리어가 필요하다면, 그
메모리 배리어는 \co{dst_entry} 를 넘기는 메커니즘 내에 내장되어야 합니다.
\iffalse

The \co{dst_clone()} primitive may be used if the caller
already has a reference to the specified \co{dst_entry},
in which case it obtains another reference that may be handed off
to some other entity within the kernel.
Because a reference is already held by the caller, \co{dst_clone()}
need not execute any memory barriers.
The act of handing the \co{dst_entry} to some other entity might
or might not require a memory barrier, but if such a memory barrier
is required, it will be embedded in the mechanism used to hand the
\co{dst_entry} off.
\fi

\co{dst_release()} 함수는 어떤 환경에서도 호출될 수 있고, 호출자는
\co{dst_release()} 호출에 앞서 \co{dst_entry} 구조체의 원소들을 직접 참조할
수도 있습니다.
따라서 \co{dst_release()} 함수는 line~14 에서 메모리 배리어를 포함해서
컴파일러나 CPU 가 접근 순서를 잘못 재배치하지 않도록 합니다.

\co{dst_clone()} 과 \co{dst_release()} 를 사용하는 프로그래머는 메모리 배리어에
대해서 신경쓸 필요가 없고, 단지 이 두개의 함수들의 사용규칙만을 알면 된다는
점을 알아 두시기 바랍니다.
\iffalse

The \co{dst_release()} primitive may be invoked from any environment,
and the caller might well reference elements of the \co{dst_entry}
structure immediately prior to the call to \co{dst_release()}.
The \co{dst_release()} primitive therefore contains a memory
barrier on line~14 preventing both the compiler and the CPU
from misordering accesses.

Please note that the programmer making use of \co{dst_clone()} and
\co{dst_release()} need not be aware of the memory barriers, only
of the rules for using these two primitives.
\fi

\subsubsection{Atomic Counting With Check and Release Memory Barrier}
\label{sec:together:Atomic Counting With Check and Release Memory Barrier}

호출자가 현재는 레퍼런스를 가지고 있지 않은 오브젝트로의 레퍼런스를 획득해야만
하는 경우를 생각해 봅시다.
최초의 레퍼런스 카운트 획득은 레퍼런스 카운트 해제와 동시에 이뤄질 수 있다는
사실이 복잡도를 더욱 증가시킵니다.
레퍼런스 카운트 해제가 레퍼런스 카운트의 해제 후 값이 0이 되게 됨을 발견했고 그
레퍼런스에 연관된 오브젝트가 해제되어도 안전하다는 신호를 날렸다고 생각해
봅시다.
그런 해제작업이 개시된 후에 레퍼런스 카운트 획득을 허가할 수 없는 것은
분명하므로, 레퍼런스 획득은 레퍼런스 카운트가 0인지에 대한 검사를 포함해야만
합니다.
이런 검사는 다음에 보인 것과 같이 어토믹한 값 증가 오퍼레이션이 한 부분이
되어야만 합니다.
\iffalse

Consider a situation where the caller must be able to acquire a new
reference to an object to which it does not already hold a reference.
The fact that initial reference-count acquisition can now run concurrently
with reference-count release adds further complications.
Suppose that a reference-count release finds that the new
value of the reference count is zero, signalling that it is
now safe to clean up the reference-counted object.
We clearly cannot allow a reference-count acquisition to
start after such clean-up has commenced, so the acquisition
must include a check for a zero reference count.
This check must be part of the atomic increment operation,
as shown below.
\fi

\QuickQuiz{}
	레퍼런스 카운트의 값이 0인지에 대한 검사는 왜 간단히 어토믹 값 증가
	오퍼레이션을 갖는 ``if'' 문의 ``then'' 절에 들어갈 수 없는거죠?
	\iffalse

	Why can't the check for a zero reference count be
	made in a simple ``if'' statement with an atomic
	increment in its ``then'' clause?
	\fi
\QuickQuizAnswer{
	``if'' 조건이 완료되었고 레퍼런스 카운터의 값이 1이라는 걸 알게
	되었다고 해봅시다.
	이어서 레퍼런스 해제 오퍼레이션이 실행되어서 이 레퍼런스 카운터의 값을
	0으로 바꾸었고 오브젝트 해제 오퍼레이션을 시작시킵니다.
	하지만 이제 ``then'' 절은 이 카운터의 값을 다시 1로 증가시킬 수가
	있어서 오브젝트가 해제된 후에 사용되는 상황을 가능하게 해버리고 맙니다.
	\iffalse

	Suppose that the ``if'' condition completed, finding
	the reference counter value equal to one.
	Suppose that a release operation executes, decrementing
	the reference counter to zero and therefore starting
	cleanup operations.
	But now the ``then'' clause can increment the counter
	back to a value of one, allowing the object to be
	used after it has been cleaned up.
	\fi
} \QuickQuizEnd

리눅스 커널의 \co{fget()} 과 \co{fput()} 기능들이 이런 스타일의 레퍼런스
카운팅을 사용합니다.
이 함수들의 간료화된 버전이
Figure~\ref{fig:together:Linux Kernel fget/fput API} 에 보여져 있습니다.
\iffalse

The Linux kernel's \co{fget()} and \co{fput()} primitives
use this style of reference counting.
Simplified versions of these functions are shown in
Figure~\ref{fig:together:Linux Kernel fget/fput API}.
\fi

\begin{figure}[tbp]
{ \fontsize{6.5pt}{7.5pt}\selectfont
\begin{verbbox}
  1 struct file *fget(unsigned int fd)
  2 {
  3   struct file *file;
  4   struct files_struct *files = current->files;
  5
  6   rcu_read_lock();
  7   file = fcheck_files(files, fd);
  8   if (file) {
  9     if (!atomic_inc_not_zero(&file->f_count)) {
 10       rcu_read_unlock();
 11       return NULL;
 12     }
 13   }
 14   rcu_read_unlock();
 15   return file;
 16 }
 17
 18 struct file *
 19 fcheck_files(struct files_struct *files, unsigned int fd)
 20 {
 21   struct file * file = NULL;
 22   struct fdtable *fdt = rcu_dereference((files)->fdt);
 23
 24   if (fd < fdt->max_fds)
 25     file = rcu_dereference(fdt->fd[fd]);
 26   return file;
 27 }
 28
 29 void fput(struct file *file)
 30 {
 31   if (atomic_dec_and_test(&file->f_count))
 32     call_rcu(&file->f_u.fu_rcuhead, file_free_rcu);
 33 }
 34
 35 static void file_free_rcu(struct rcu_head *head)
 36 {
 37   struct file *f;
 38
 39   f = container_of(head, struct file, f_u.fu_rcuhead);
 40   kmem_cache_free(filp_cachep, f);
 41 }
\end{verbbox}
}
\centering
\theverbbox
\caption{Linux Kernel fget/fput API}
\label{fig:together:Linux Kernel fget/fput API}
\end{figure}

\co{fget()} 의 Line~4 는, 다른 프로세스들과 공유될 수도 있는 현재 프로세스의
file-descriptor 테이블을 가져옵니다.
Line~6 는 RCU read-side 크리티컬 섹션을 들어가는 \co{rcu_read_lock()} 을
호출합니다.
뒤따르는 \co{call_rcu()} 에 전달되는 콜백 함수의 수행은 매칭되는
\co{rcu_read_unlock()} 이 호출될 때까지 (이 경우 line~10 또는 14) 연기됩니다.
Line~7 은 \co{fd} 인자에 명시된 파일 디스크립터에 연관되는 file 구조체를
탐색하는데, 이에 대해서는 뒤에서 설명하겠습니다.
명시된 파일 디스크립터에 연관된 열린 파일이 존재한다면 line~9 에서는 어토믹하게
레퍼런스 카운트를 획득하려 시도합니다.
만약 그렇게 하는데 실패한다면, line~10-11 은 RCU read-side 크리티컬 섹션을
빠져나오고 실패했음을 알립니다.
그렇지 않고 시도가 성공했다면, line~14-15 는 이 read-side 크리티컬 섹션을
빠져나오고 해당 file 구조체로의 포인터를 리턴합니다.
\iffalse

Line~4 of \co{fget()} fetches the pointer to the current
process's file-descriptor table, which might well be shared
with other processes.
Line~6 invokes \co{rcu_read_lock()}, which
enters an RCU read-side critical section.
The callback function from any subsequent \co{call_rcu()} primitive
will be deferred until a matching \co{rcu_read_unlock()} is reached
(line~10 or~14 in this example).
Line~7 looks up the file structure corresponding to the file
descriptor specified by the \co{fd} argument, as will be
described later.
If there is an open file corresponding to the specified file descriptor,
then line~9 attempts to atomically acquire a reference count.
If it fails to do so, lines~10-11 exit the RCU read-side critical
section and report failure.
Otherwise, if the attempt is successful, lines~14-15 exit the read-side
critical section and return a pointer to the file structure.
\fi

\co{fcheck_files()} 기능은 \co{fget()} 을 위한 도움을 주는 함수입니다.
해당 함수는 나중의 디레퍼런싱을 위해 (이는 데이터 종속성이 메모리 순서 규칙을
강제하지 않는 DEC Alpha 와 같은 CPU 들에서는 메모리 배리어를 실행합니다)
안전하게 RCU 로 보호되는 포인터를 가져오기 위해 사용됩니다.
Line~22 는 이 태스크의 현재 file-descriptor 테이블로의 포인터를 가져오기 위해
\co{rcu_dereference()} 를 사용하고, line~24 는 명시된 파일 디스크립터가 범위
안에 있는지 체크합니다.
만약 그렇다면 line~25 는 해당 file 구조체로의 포인터를 가져오는데, 여기서도
\co{rcu_dereference()} 기능이 사용됩니다.
그러고나서 Line~26 은 성공했다면 해당 file 구조체로의 포인터를 리턴하고,
실패했다면 \co{NULL} 을 리턴하게 됩니다.
\iffalse

The \co{fcheck_files()} primitive is a helper function for
\co{fget()}.
It uses the \co{rcu_dereference()} primitive to safely fetch an
RCU-protected pointer for later dereferencing (this emits a
memory barrier on CPUs such as DEC Alpha in which data dependencies
do not enforce memory ordering).
Line~22 uses \co{rcu_dereference()} to fetch a pointer to this
task's current file-descriptor table,
and line~24 checks to see if the specified file descriptor is in range.
If so, line~25 fetches the pointer to the file structure, again using
the \co{rcu_dereference()} primitive.
Line~26 then returns a pointer to the file structure or \co{NULL}
in case of failure.
\fi

\co{fput()} 기능은 file 구조체로의 레퍼런스를 해제합니다.
Line~31 은 어토믹하게 레퍼런스 카운트를 감소시키고, 만약 그 감소 결과가 0이라면
line~32 에서 해당 file 구조체를 모든 현재 수행중인 RCU read-side 크리티컬
섹션들이 완료된 후에 (\co{call_rcu()} 의 두번째 인자로 전달되는
\co{file_free_rcu()} 를 통해) 해제하기 위해 \co{call_rcu()} 기능을
실행시킵니다.
모든 현재 수행중인 RCU read-side 크리티컬 섹션들이 완료되기까지 필요한 시간은
``grace period'' 라고 불립니다.
\co{atomic_dec_and_test()} 기능은 메모리 배리어를 포함하고 있음을 알아두시기
바랍니다.
이 메모리 배리어는 이 예제에서는 필요치 않은데, 이 구조체는 RCU read-side
크리티컬 섹션이 완료되기 전까지는 해제될 수 없기 때문입니다만 리눅스에서는 그
결과를 리턴하는 모든 어토믹 오퍼레이션들은 정의에 의해 메모리 배리어를
포함해야만 합니다.
\iffalse

The \co{fput()} primitive releases a reference to a file structure.
Line~31 atomically decrements the reference count, and, if the result
was zero, line~32 invokes the \co{call_rcu()} primitives in order to
free up the file structure (via the \co{file_free_rcu()} function
specified in \co{call_rcu()}'s second argument),
but only after all currently-executing
RCU read-side critical sections complete.
The time period required for all currently-executing RCU read-side
critical sections to complete is termed a ``grace period''.
Note that the \co{atomic_dec_and_test()} primitive contains
a memory barrier.
This memory barrier is not necessary in this example, since the structure
cannot be destroyed until the RCU read-side critical section completes,
but in Linux, all atomic operations that return a result must
by definition contain memory barriers.
\fi

일단 grace period 가 완료되면, \co{file_free_rcu()} 함수가 line~39 에서 file
구조체로의 포인터를 가져오고 line~40 에서 해제시킵니다.

이런 방법은 리눅스의 가상 메모리 시스템에서도 사용되는데, 페이지 구조체를
위해선 \co{get_page_unless_zero()} 와 \co{put_page_testzero()} 를, 메모리 맵
구조체를 위해서는 \co{try_to_unuse()} 와 \co{mmput()} 을 참고하시기 바랍니다.
\iffalse

Once the grace period completes, the \co{file_free_rcu()} function
obtains a pointer to the file structure on line~39, and frees it
on line~40.

This approach is also used by Linux's virtual-memory system,
see \co{get_page_unless_zero()} and \co{put_page_testzero()} for
page structures as well as \co{try_to_unuse()} and \co{mmput()}
for memory-map structures.
\fi

\subsection{Linux Primitives Supporting Reference Counting}
\label{sec:together:Linux Primitives Supporting Reference Counting}

앞의 예제들에서 사용된 리눅스 커널의 기능들이 다음의 리스트에 요약되어
있습니다.
\IfInBook{}{RCU 기능들은 일부 독자들에게는 친숙하지 않을 수도 있으므로,
	    Section~\ref{sec:together:Background on RCU} 에 인용과 간략한
	    개요가 있습니다.}
\iffalse

The Linux-kernel primitives used in the above examples are
summarized in the following list.
\IfInBook{}{The RCU primitives may be unfamiliar to some readers,
	    so a brief overview with citations is included in
	    Section~\ref{sec:together:Background on RCU}.}
\fi

\begin{itemize}
\item	\co{atomic_t}
	어토믹하게 조정되는 32-bit 크기의 것을 위한 타입 정의.
\item	\co{void atomic_dec(atomic_t *var);}
	메모리 배리어를 요청하거나 컴파일러 최적화를 불능화시키지 않으며
	참조된 변수의 값을 어토믹하게 감소.
\item	\co{int atomic_dec_and_test(atomic_t *var);}
	어토믹하게 참조된 변수의 값을 감소시키고 그 감소 결과값이 0이라면
	\co{true} (0 이 아닌 값) 을 리턴.
	이 기능의 앞뒤로 메모리 참조가 옮겨가지 못하도록 메모리 배리어와
	컴파일러 최적화 불능화를 수행.
\item	\co{void atomic_inc(atomic_t *var);}
	메모리 배리어를 요청하거나 컴파일러 최적화를 불능화시키지 않으며 참조된
	변수의 값을 어토믹하게 증가.
\item	\co{int atomic_inc_not_zero(atomic_t *var);}
	참조된 변수의 값이 0이 아니라면 그 값을 어토믹하게 증가시키고 값 증가가
	수행되었다면 \co{true} (0이 아닌 값) 을 리턴함.
	이 기능의 앞뒤로 메모리 참조가 옮겨가지 못하도록 메모리 배리어와
	컴파일러 최적화 불능화를 수행.
\iffalse

\item	\co{atomic_t}
	Type definition for 32-bit quantity to be manipulated atomically.
\item	\co{void atomic_dec(atomic_t *var);}
	Atomically decrements the referenced variable without necessarily
	issuing a memory barrier or disabling compiler optimizations.
\item	\co{int atomic_dec_and_test(atomic_t *var);}
	Atomically decrements the referenced variable, returning
	\co{true} (non-zero) if the result is zero.
	Issues a memory barrier and disables compiler optimizations that
	might otherwise move memory references across this primitive.
\item	\co{void atomic_inc(atomic_t *var);}
	Atomically increments the referenced variable without necessarily
	issuing a memory barrier or disabling compiler optimizations.
\item	\co{int atomic_inc_not_zero(atomic_t *var);}
	Atomically increments the referenced variable, but only if the
	value is non-zero, and returning \co{true} (non-zero) if the
	increment occurred.
	Issues a memory barrier and disables compiler optimizations that
	might otherwise move memory references across this primitive.
\fi
\item	\co{int atomic_read(atomic_t *var);}
	참조된 변수의 정수 값을 리턴.
	이는 어토믹 오퍼레이션이어야 할 필요가 없고, 메모리 배리어 명령어를
	요청할 필요도 없음.
	``어토믹한 읽기'' 라고 생각하기 보다는 ``어토믹 변수로부터의 평범한
	읽기'' 라고 생각할 것.
\item	\co{void atomic_set(atomic_t *var, int val);}
	참조된 어토믹 변수의 값을 ``val'' 로 설정.
	이는 어토믹 오퍼레이션이어야 할 필요가 없으며, 메모리 배리어나 컴파일러
	최적화의 불능화를 수행할 필요도 없음.
	이를 ``어토믹한 값 설정'' 으로 생각하기 보다는 ``어토믹 변수에의 평범한
	값 설정'' 으로 생각할 것.
\iffalse

\item	\co{int atomic_read(atomic_t *var);}
	Returns the integer value of the referenced variable.
	This need not be an atomic operation, and it need not issue any
	memory-barrier instructions.
	Instead of thinking of as ``an atomic read'', think of it as
	``a normal read from an atomic variable''.
\item	\co{void atomic_set(atomic_t *var, int val);}
	Sets the value of the referenced atomic variable to ``val''.
	This need not be an atomic operation, and it is not required
	to either issue memory
	barriers or disable compiler optimizations.
	Instead of thinking of as ``an atomic set'', think of it as
	``a normal set of an atomic variable''.
\fi
\item	\co{void call_rcu(struct rcu_head *head, void (*func)(struct rcu_head *head));}
	일부 시간이 지나서 현재 수행중인 RCU read-side 크리티컬 섹션들이 완료된
	후에 \co{func(head)} 를 수행하되 \co{call_rcu()} 기능 자체는 즉시
	리턴함.
	\co{head} 는 일반적으로 RCU 로 보호되는 데이터 구조체이고, \co{func} 는
	일반적으로 이 데이터 구조체를 메모리에서 해제시키는 함수임을 알아둘 것.
	\co{call_rcu()} 의 호출과 \co{func} 호출 사이의 시간 간격은 ``grace
	period'' 라 불리움.
	Grace period 를 포함하는 모든 시간 간격은 그 자체로 grace period 임.
\item	\co{type *container_of(p, type, f);}
	명시된 타입의 구조체의 필드 \co{f} 로의 포인터 \co{p} 를 받고 해당
	구조체로의 포인터를 리턴.
\item	\co{void rcu_read_lock(void);}
	RCU read-side 크리티컬 섹션의 시작을 표시.
\item	\co{void rcu_read_unlock(void);}
	RCU read-side 크리티컬 섹션의 종료를 표시.
	RCU read-side 크리티컬 섹션은 중첩될 수 있음.
\item	\co{void smp_mb__before_atomic_dec(void);}
	해당 플랫폼의 \co{atomic_dec()} 기능이 그러지 않는다면 코드를 움직일 수
	있는 컴파일러 최적화의 불능화와 메모리 배리어를 요청.
\item	\co{struct rcu_head}
	Grace period 를 기다리는 오브젝트들을 추적하기 위해 RCU 설비에서
	사용되는 데이터 구조체.
	일반적으로 RCU 로 보호되는 데이터 구조체 내에 하나의 필드로 포함되어
	있음.
\iffalse

\item	\co{void call_rcu(struct rcu_head *head, void (*func)(struct rcu_head *head));}
	Invokes \co{func(head)} some time after all currently executing RCU
	read-side critical sections complete, however, the \co{call_rcu()}
	primitive returns immediately.
	Note that \co{head} is normally a field within an RCU-protected
	data structure, and that \co{func} is normally a function that
	frees up this data structure.
	The time interval between the invocation of \co{call_rcu()} and
	the invocation of \co{func} is termed a ``grace period''.
	Any interval of time containing a grace period is itself a
	grace period.
\item	\co{type *container_of(p, type, f);}
	Given a pointer \co{p} to a field \co{f} within a structure
	of the specified type, return a pointer to the structure.
\item	\co{void rcu_read_lock(void);}
	Marks the beginning of an RCU read-side critical section.
\item	\co{void rcu_read_unlock(void);}
	Marks the end of an RCU read-side critical section.
	RCU read-side critical sections may be nested.
\item	\co{void smp_mb__before_atomic_dec(void);}
	Issues a memory barrier and disables code-motion compiler
	optimizations only if the platform's \co{atomic_dec()}
	primitive does not already do so.
\item	\co{struct rcu_head}
	A data structure used by the RCU infrastructure to track
	objects awaiting a grace period.
	This is normally included as a field within an RCU-protected
	data structure.
\fi
\end{itemize}

\QuickQuiz{}
	어토믹하지 않은 \co{atomic_read()} 와 \co{atomic_set()} 이라구요?
	이건 또 무슨 농담이죠???
	\iffalse

	An \co{atomic_read()} and an \co{atomic_set()} that are
	non-atomic?
	Is this some kind of bad joke???
	\fi
\QuickQuizAnswer{
	그렇게 보일수도 있겠습니다만, 현재 문제시 되는 어토믹 변수에 어떤 다른
	CPU 도 접근을 하지 않는 상황에서는 실제 어토믹 인스트럭션의 오버헤드는
	낭비가 될 수 있습니다.
	다른 CPU 가 접근하지 안는 경우의 두가지 예는 초기화와 정리 작업 입니다.
	\iffalse

	It might well seem that way, but in situations where no other
	CPU has access to the atomic variable in question, the overhead
	of an actual atomic instruction would be wasteful.
	Two examples where no other CPU has access are
	during initialization and cleanup.
	\fi
} \QuickQuizEnd

\subsection{Counter Optimizations}
\label{sec:together:Counter Optimizations}

카운터 증가와 감소가 흔하게 일어나지만 카운터 값이 0인지에 대한 검사는 가끔만
이뤄지는 일부 경우들에 있어서는,
Chapter~\ref{chp:Counting} 에서 이야기한 것처럼 per-CPU 또는 per-task
카운터들을 두는게 말이 될겁니다.
이런 테크닉이 RCU 에 적용된 예를 보기 위해 sleepable read-copy update (SRCU) 에
대한 논문~\cite{PaulEMcKenney2006c} 을 참고하세요.
이 방법은 카운터 값 증가와 감소 기능들에서의 어토믹 명령어나 메모리 배리어의
사용의 필요를 없애줍니다만, 코드의 위치를 움직이는 컴파일러 최적화의 제거는
여전히 필요합니다.
또한, 합산된 레퍼런스 카운트가 0이 되었는지를 체크하게 되는
\co{synchronize_srcu()} 와 같은 기능들은 상당히 느려집니다.
이는 이런 테크닉들은 레퍼런스들이 빈번히 획득되고 해제되지만 레퍼런스 카운트가
0이 되었는지에 대한 검사는 가끔만 일어나는 상황을 위해서 설계된 것이란 점을
강조합니다.
\iffalse

In some cases where increments and decrements are common, but checks
for zero are rare, it makes sense to maintain per-CPU or per-task
counters, as was discussed in Chapter~\ref{chp:Counting}.
See the paper on sleepable read-copy update
(SRCU) for an example of this technique applied to
RCU~\cite{PaulEMcKenney2006c}.
This approach eliminates the need for atomic instructions or memory
barriers on the increment and decrement primitives, but still requires
that code-motion compiler optimizations be disabled.
In addition, the primitives such as \co{synchronize_srcu()}
that check for the aggregate reference
count reaching zero can be quite slow.
This underscores the fact that these techniques are designed
for situations where the references are frequently acquired and
released, but where it is rarely necessary to check for a zero
reference count.
\fi

% @@@ Difficulty of managing reference counts: leaks, premature freeing.

하지만, 레퍼런스 카운트가 사용되지 않았다면 읽기 전용이었을 수 있는 데이터
구조에 레퍼런스 카운트의 사용은 (대부분의 경우 어토믹한) 쓰기를 필요로 한다는
점이 문제인 경우가 많습니다.
이런 경우, 레퍼런스 카운트는 읽기를 하는 쓰레드들에게 비싼 캐시 미스를
부과합니다.

따라서 읽기를 하는 쓰레드들이 횡단하게 되는 데이터 구조체에 쓰기를 하지 않아도
되게 하는 동기화 메커니즘들을 찾아볼 가치가 있습니다.
그런 것들 중 하나는
Section~\ref{sec:defer:Hazard Pointers} 에서 다룬 해저드 포인터이고, 또 다른
하나는
Section~\ref{sec:defer:Read-Copy Update (RCU)} 에서 다룬 RCU 입니다.
\iffalse

However, it is usually the case that use of reference counts requires
writing (often atomically) to a data structure that is otherwise
read only.
In this case, reference counts are imposing expensive cache misses
on readers.

It is therefore worthwhile to look into synchronization mechanisms
that do not require readers to write to the data structure being
traversed.
One possibility is the hazard pointers covered in
Section~\ref{sec:defer:Hazard Pointers}
and another is RCU, which is covered in
Section~\ref{sec:defer:Read-Copy Update (RCU)}.
\fi

% together/hazptr.tex
% mainfile: ../perfbook.tex
% SPDX-License-Identifier: CC-BY-SA-3.0

\section{Hazard-Pointer Helpers}
\label{sec:together:Hazard-Pointer Helpers}
%
\epigraph{It's the little things that count, hundreds of them.}
	 {\emph{Cliff Shaw}}

이 섹션은 해쉬 테이블을 다룰 때 생길 수 있는 문제들을 알아봅니다.
이 문제들은 많은 다른 탐색 구조체들에서도 발생할 수 있음을 유의하시기 바랍니다.

\iffalse

This section looks at some issues that can arise when dealing with
hash tables.
Please note that these issues also apply to many other search structures.

\fi

\subsection{Scalable Reference Count}
\label{sec:together:Scalable Reference Count}

레퍼런스 카운트가 성능이나 확장성의 병목이 된다고 해봅시다.
뭘 할 수 있을까요?

한가지 방법은 해저드 포인터를 대신 사용하는 겁니다.

해저드 포인터에는 몇가지 차이가 있는데, 그중 가장 눈여겨 볼만한 건 언제 연관된
레퍼런스 카운트가 0이 되었는지 확인하는 비용이 굉장히 높다는 겁니다.

이 문제를 해결하는 한가지 방법은 레퍼런스 카운트와 해저드 포인터 사이에 부하를
쪼개는 겁니다.
각 데이터 원소는 이 원소를 참조하는 다른 데이터 원소의 수를 추적하는 동안 읽기
쓰레드는 해저드 포인터를 사용하는 겁니다.

이 방법을 효율적이면서 올바르게 사용하기는 상당한 노력이 필요하며, 따라서 관심
있는 독자 여러분은 Folly 오픈소스 라이브러리에 구현되어 있는 UnboundedQueue 와
ConcurrentHashMap 데이터 구조를 살펴보시기 바랍니다.\footnote{
	\url{https://github.com/facebook/folly}}

\iffalse

Suppose a reference count is becoming a performance or scalability
bottleneck.
What can you do?

One approach is to instead use hazard pointers.

There are some differences, perhaps most notably that with
hazard pointers it is extremely expensive to determine when
the corresponding reference count has reached zero.

One way to work around this problem is to split the load between
reference counters and hazard pointers.
Each data element has a reference counter that tracks the number
of other data elements referencing this element on the one hand,
and readers use hazard pointers on the other.

Making this arrangement work both efficiently and correctly can be
quite challenging, and so interested readers are invited to examine
the UnboundedQueue and ConcurrentHashMap data structures implemented in
Folly open-source library.\footnote{
	\url{https://github.com/facebook/folly}}

\fi

% @@@ papers to maybe cite: OrcGC, ThreadScan, Fast and Robust Memory...

% @@@ Generalized hazard-pointer link counts, if and when.

% @@@ Representative hazard pointer for list, so that nothing
% @@@ in list gets freed until list's hazard pointer is released.
% @@@ Midpoint between hazard pointers and RCU, in fact, you
% @@@ could argue that Tasks Trace RCU with read-side memory
% @@@ barriers is sort of a per-CPU hazard pointers implementing RCU.
% @@@ Except no re-checking because CPUs cannot be freed.

% together/seqlock.tex
% mainfile: ../perfbook.tex
% SPDX-License-Identifier: CC-BY-SA-3.0

\section{Sequence-Locking Specials}
\label{sec:together:Sequence-Locking Specials}
%
\epigraph{The girl who can't dance says the band can't play.}
	 {\emph{Yiddish proverb}}

이 섹션은 시퀀스 락의 특별한 사용처들을 알아봅니다.

\iffalse

This section looks at some special uses of sequence locks.

\fi

\subsection{Correlated Data Elements}
\label{sec:together:Correlated Data Elements}

두개 이상의 원소들을 연관지어 봐야 하는 해쉬 테이블을 가지고 있다고 해봅시다.
이 원소들은 함께 업데이트 되며, 첫번째 원소의 기존 버전을 다른 원소의 새 버전과
함께 보고 싶지 않습니다.
예를 들어, Schr\"odinger 는 그의 in-memory 데이터베이스에 그의 동물들에 더해
확장된 가족을 넣고 싶습니다.
Schr\"odinger 는 결혼과 이혼이 급작스럽게 일어나지는 않음을 알지만, 그는 또한
전통주의자이기도 합니다.
따라서, 그는 그의 데이터베이스가 신부는 결혼했는데 신랑은 그렇지 않은, 또는 그
반대의 경우를 보이기를 원치 않습니다.
또한, Schr\"odinger 가 전통주의자라 생각한다면, 여러분은 그의 가족 구성원들 중
일부와 대화해 보세요!
달리 말하자면, Schr\"odinger 는 그의 데이터베이스가 결혼에 있어 일관적이길
원합니다.

\iffalse

Suppose we have a hash table where we need correlated views of two or
more of the elements.
These elements are updated together, and we do not want to see an old
version of the first element along with new versions of the other
elements.
For example, Schr\"odinger decided to add his extended family to his
in-memory database along with all his animals.
Although Schr\"odinger understands that marriages and divorces do not
happen instantaneously, he is also a traditionalist.
As such, he absolutely does not want his database ever to show that the
bride is now married, but the groom is not, and vice versa.
Plus, if you think Schr\"odinger is a traditionalist, you just
try conversing with some of his family members!
In other words, Schr\"odinger wants to be able to carry out a
wedlock-consistent traversal of his database.

\fi

한가지 방법은 시퀀스 락을 사용해서
(\cref{sec:defer:Sequence Locks} 를 참고하세요),
결혼에 관련된 업데이트가 \co{write_seqlock()} 의 보호 아래 진행되고 결혼
일관성이 피료한 읽기는 \co{read_seqbegin()} / \co{read_seqretry()} 반복문
아래에서 진행되게 하는 겁니다.
시퀀스 락은 RCU 보호의 대체제가 아님에 유의하세요:
시퀀스 락은 동시의 수정으로부터 보호를 해줍니다만, 동시의 삭제로부터의 보호를
위해선 여전히 RCU 가 필요합니다.

이 방법은 연관된 원소의 수가 작고 원소들에의 읽기 시간이 짧으며, 업데이트
비율이 낮을 때 상당히 잘 작동합니다.
그렇지 않다면, 읽기 쓰레드가 영원히 완료되지 못할 수도 있게끔 업데이트가
빈번하게 이러날 수도 있습니다.
Schr\"odinger 는 이런 문제가 일어날 만큼 그의 가장 덜 정상적인 지인들이
결혼하고 곧바로 이혼할 거라 생각하지 않지만, 그는 이 문제가 다른 환경에서는
일어날 수 있음을 알고 있습니다.
이 읽기 쓰레드 starvation 문제를 해결하는 한가지 방법은 읽기 쓰레드가 너무 많은
재시도를 했다면 업데이트 쪽 기능을 사용하게 하는 것입니다만, 이는 성능과
확장성을 모두 악화시킬 수 있습니다.
Starvation 을 막는 다른 방법은 여러 시퀀스 락을 상요하는 것으로, 이
Schr\"odinger 의 경우엔 종당 하나가 될 수 있겠습니다.

\iffalse

One approach is to use sequence locks
(see \cref{sec:defer:Sequence Locks}),
so that wedlock-related updates are carried out under the
protection of \co{write_seqlock()}, while reads requiring
wedlock consistency are carried out within
a \co{read_seqbegin()} / \co{read_seqretry()} loop.
Note that sequence locks are not a replacement for RCU protection:
Sequence locks protect against concurrent modifications, but RCU
is still needed to protect against concurrent deletions.

This approach works quite well when the number of correlated elements is
small, the time to read these elements is short, and the update rate is
low.
Otherwise, updates might happen so quickly that readers might never complete.
Although Schr\"odinger does not expect that even his least-sane relatives
will marry and divorce quickly enough for this to be a problem,
he does realize that this problem could well arise in other situations.
One way to avoid this reader-starvation problem is to have the readers
use the update-side primitives if there have been too many retries,
but this can degrade both performance and scalability.
Another way to avoid starvation is to have multiple sequence locks,
in Schr\"odinger's case, perhaps one per species.

\fi

또한, 만약 업데이트 쪽 기능이 너무 자주 사용된다면, 락 컨텐션으로 인해 낮은
성능과 확장성이 초래될 겁니다.
이를 막는 한가지 방법은 원소별 시퀀스 락을 두고 부부의 결혼 상태를 업데이트 할
때 부부 두명의 락을 모두 잡는 겁니다.
읽기 쓰레드는 이 한쌍의 멤버들의 결혼 상태에 대한 모든 변화에 대해 안정적인
읽기를 위해 이 부부의 락들 중 하나만 가지고 재시도 반복을 할 수 있습니다.
이는 높은 결혼과 이혼율에 의한 컨텐션을 막을 수 있으나, 데이터베이스의 한번의
스캔 동안의 모든 결혼 상태에 대한 일관적 시각을 얻기를 복잡하게 만듭니다.

결혼 상태가 그러길 바라듯이 원소 그룹 짓기가 잘 정의되었고 영구적이라면, 가능한
한가지 방법은 데이터 원소로의 포인터들을 더해서 특정 그룹의 멤버들을 함께
연결짓는 겁니다.
그러면 읽기 쓰레드는 첫번째 원소가 발견되면 같은 그룹의 데이터 원소들을
접근하기 위해 이 포인터들을 따라갈 수 있습니다.

이 기법은 리눅스 커널에서 널리 사용되는데, dcache subsystem
에서~\cite{NeilBrown2015RCUwalk} 특히 그렇습니다.
비슷한 방법이 해저드 포인터를 통해서도 동작할 수 있음을 알아 두시기 바랍니다.

\iffalse

In addition, if the update-side primitives are used too frequently,
poor performance and scalability will result due to lock contention.
One way to avoid this is to maintain a per-element sequence lock,
and to hold both spouses' locks when updating their marital status.
Readers can do their retry looping on either of the spouses' locks
to gain a stable view of any change in marital status involving both
members of the pair.
This avoids contention due to high marriage and divorce rates, but
complicates gaining a stable view of all marital statuses during a
single scan of the database.

If the element groupings are well-defined and persistent, which marital
status is hoped to be,
then one approach is to add pointers to the data elements to link
together the members of a given group.
Readers can then traverse these pointers to access all the data elements
in the same group as the first one located.

This technique is used heavily in the Linux kernel, perhaps most
notably in the dcache subsystem~\cite{NeilBrown2015RCUwalk}.
Note that it is likely that similar schemes also work with hazard
pointers.

\fi

또다른 방법은 데이터 원소들을 파편화 하고 각 업데이트가 그 업데이트로 영향 받는
모든 데이터 원소를 위한 모든 시퀀스 락에 대해 쓰기 권한 획득을 하게 하는
겁니다.
물론, 이 쓰기 권한 획득은 데드락을 막기 위해 신중히 가해져야 합니다.
읽기 쓰레드는 여러 시퀀스 락에 대한 읽기 권한 획득을 필요로 하겠습니다만 읽기
쓰레드가 하나의 데이터 원소만 읽어야 하는 놀랍도록 흔한 상황에서는 하나의
시퀀스 락에 대해서만 읽기 권한 획득이 필요합니다.

이 방법은 성공한 읽기 쓰레드에게 sequential consistency 를 제공하여, 각자는
특정 업데이트의 효과를 보거나 보지 못하며, 모든 중간의 업데이트는 읽기 쪽
재시도를 하게 합니다.
Sequential consistency 는 극단적으로 강력한 보장으로, 동일하게 강한 제한과 높은
오버헤드를 수반합니다.
이 경우, 우린 읽기 쓰레드가 starvation 에 빠질 수 있거나 업데이트 쪽 락을
획득해야 할수도 있음을 보았습니다.
업데이트가 흔하지 않은 경우에 이는 잘 동작하지만, 이는 업데이트가 재시도되는
읽기 쓰레드가 액세스하는 데이터에는 여향을 주지도 않은 업데이트에조차 재시도를
불필요하게 강제합니다.
따ㅓ라서 \cref{sec:together:Correlated Fields} 는 읽기 쓰레드의 starvation 을
막을 뿐 아니라 모든 형태의 읽기 쪽 재시도를 막는 완화된 형태의 일관성을
다룹니다.

\iffalse

Another approach is to shard the data elements, and then have each update
write-acquire all the sequence locks needed to cover the data elements
affected by that update.
Of course, these write acquisitions must be done carefully in order to
avoid deadlock.
Readers would also need to read-acquire multiple sequence locks, but
in the surprisingly common case where readers only look up one data
element, only one sequence lock need be read-acquired.

This approach provides sequential consistency to successful readers,
each of which will either see the effects of a given update or not,
with any partial updates resulting in a read-side retry.
Sequential consistency is an extremely strong guarantee, incurring equally
strong restrictions and equally high overheads.
In this case, we saw that readers might be starved on the one hand, or
might need to acquire the update-side lock on the other.
Although this works very well in cases where updates are infrequent,
it unnecessarily forces read-side retries even when the update does not
affect any of the data that a retried reader accesses.
\Cref{sec:together:Correlated Fields} therefore covers a much weaker form
of consistency that not only avoids reader starvation, but also avoids
any form of read-side retry.

\fi

\subsection{Upgrade to Writer}
\label{sec:together:Upgrade to Writer}

\Cref{sec:defer:RCU is a Reader-Writer Lock Replacement} 에서 이야기 된 것과
같이, RCU 는 읽기 쓰레드가 쓰기 쓰레드로 업그레이드 되는 것을 허용합니다.
이 능력은 RCU 로 보호되는 데이터 구조를 읽는 읽기 쓰레드가 현재 원소에
업데이트가 필요함을 알았을 때 상당히 유용합니다.
이걸 시퀀스 락을 가지고 하려면 어떻게 될까요?

이런 시퀀스 락 기법은 리눅스 커널에서 사용되고 있는데, 예를 들어
\path{drivers/infiniband/hw/hfi1/sdma.c} 의 \co{sdma_flush()} 함수에서
그렇습니다.
그 효과는 읽기 쓰레드의 재시도를 막는 겁니다.
따라서 이 기법은 읽기 쓰레드가 재시도를 필요로 하는 어떤 조건을 파악했을 때
사용됩니다.

\iffalse

As discussed in
\cref{sec:defer:RCU is a Reader-Writer Lock Replacement},
RCU permits readers to upgrade to writers.
This capability can be quite useful when a reader scanning an
RCU-protected data structure notices that the current element
needs to be updated.
What happens when you try this trick with sequence locking?

It turns out that this sequence-locking trick is actually used in
the Linux kernel, for example, by the \co{sdma_flush()} function in
\path{drivers/infiniband/hw/hfi1/sdma.c}.
The effect is to doom the enclosing reader to retry.
This trick is therefore used when the reader detects some condition
that requires a retry.

\fi

% together/applyrcu.tex

\section{RCU Rescues}
\label{sec:together:RCU Rescues}

이 섹션은 이 책의 앞부분에서 이야기한 몇가지 예제들에 RCU 를 어떻게
적용하는지를 보입니다.
일부 경우들에 있어서는 RCU 는 간단한 코드를 제공하고, 어떤 경우에는 더 나은
성능과 확장성을 제공하며, 또다른 경우에는 두가지를 모두 제공합니다.
\iffalse

This section shows how to apply RCU to some examples discussed earlier
in this book.
In some cases, RCU provides simpler code, in other cases better
performance and scalability, and in still other cases, both.
\fi

\subsection{RCU and Per-Thread-Variable-Based Statistical Counters}
\label{sec:together:RCU and Per-Thread-Variable-Based Statistical Counters}

Section~\ref{sec:count:Per-Thread-Variable-Based Implementation}
는 대략적으로 평범한 값 증가 연산 (C \co{++} 오퍼레이터) 과 같은---하지만
\co{inc_count()}를 통해서만 값을 증가시키는---훌륭한 성능과 선형적 확장성을
보이는 통계적 카운터들의 구현을 설명했습니다.
불행히도, \co{read_count()} 를 통해 값을 읽어와야 하는 쓰레드들은 글로벌 락을 잡아야만 했고, 따라서 높은 오버헤드를 일으키고 낮은 확장성으로 고통받아야 했습니다.
락 기반의 구현 코드는
Page~\pageref{fig:count:Per-Thread Statistical Counters} 의
Figure~\ref{fig:count:Per-Thread Statistical Counters} 에 보여져 있습니다.
\iffalse

Section~\ref{sec:count:Per-Thread-Variable-Based Implementation}
described an implementation of statistical counters that provided
excellent
performance, roughly that of simple increment (as in the C \co{++}
operator), and linear scalability---but only for incrementing
via \co{inc_count()}.
Unfortunately, threads needing to read out the value via \co{read_count()}
were required to acquire a global
lock, and thus incurred high overhead and suffered poor scalability.
The code for the lock-based implementation is shown in
Figure~\ref{fig:count:Per-Thread Statistical Counters} on
Page~\pageref{fig:count:Per-Thread Statistical Counters}.
\fi

\QuickQuiz{}
	대체 왜 그런 글로벌 락이 필요했던 거지요?
	\iffalse

	Why on earth did we need that global lock in the first place?
	\fi
\QuickQuizAnswer{
	특정 쓰레드의 \co{__thread} 변수들은 그 쓰레드가 종료될 때 없어집니다.
	따라서 다른 쓰레드의 \co{__thread} 변수들을 접근하는 모든 오퍼레이션은
	쓰레드 종료와 동기화 되어야 할 필요가 있습니다.
	그런 동기화가 없다면, 방금 종료된 쓰레드의 \co{__thread} 변수로의
	접근은 segmentation fault 를 초래할 겁니다.
	\iffalse

	A given thread's \co{__thread} variables vanish when that
	thread exits.
	It is therefore necessary to synchronize any operation that
	accesses other threads' \co{__thread} variables with
	thread exit.
	Without such synchronization, accesses to \co{__thread} variable
	of a just-exited thread will result in segmentation faults.
	\fi
} \QuickQuizEnd

\subsubsection{Design}

원하는건 \co{inc_count()} 만이 아니라 \co{read_count()} 에서도 훌륭한 성능과
확장성을 얻기 위해 \co{read_count()} 의 쓰레드 횡단을 보호하는 데에
\co{final_mutex} 대신에 RCU 를 사용하는 것입니다.
하지만, 계산된 합계의 정확성을 포기하지도 않고 싶습니다.
자세히 말하자면, 특정 스레드가 종료될 때에, 우린 종료되는 쓰레드의 카운트를
잃어버릴 수도, 그걸 두번씩 세서도 안됩니다.
그런 에러는 결과의 전체 정확성과 동일한 정도의 비정확성을 초래할 수 있는데,
달리 말하자면 그런 에러는 결과값이 완전히 쓸모없게 만들 수 있습니다.
그리고 사실, \co{final_mutex} 의 목적들 중 하나는 쓰레드들이 \co{read_count()}
의 실행 사이에 들어왔다 나갔다 하지 않음을 분명히 하는 것입니다.
\iffalse

The hope is to use RCU rather than \co{final_mutex} to protect the
thread traversal in \co{read_count()} in order to obtain excellent
performance and scalability from \co{read_count()}, rather than just
from \co{inc_count()}.
However, we do not want to give up any accuracy in the computed sum.
In particular, when a given thread exits, we absolutely cannot
lose the exiting thread's count, nor can we double-count it.
Such an error could result in inaccuracies equal to the full
precision of the result, in other words, such an error would
make the result completely useless.
And in fact, one of the purposes of \co{final_mutex} is to
ensure that threads do not come and go in the middle of \co{read_count()}
execution.
\fi

\QuickQuiz{}
	어쨌든, \co{read_count()} 의 정확성을 대체 뭔가요?
	\iffalse

	Just what is the accuracy of \co{read_count()}, anyway?
	\fi
\QuickQuizAnswer{
	Page~\pageref{fig:count:Per-Thread Statistical Counters} 의
	Figure~\ref{fig:count:Per-Thread Statistical Counters} 를 참고하세요.
	동시적인 \co{inc_count()} 의 실행이 존재하지 않는다면,
	\co{read_count()} 는 분명한 결과를 내놓을 것은 분명합니다.
	하지만, \co{inc_count()} 의 동시적인 실행이 \emph{존재한다면}, 그
	합계값은 \co{read_count()} 가 그 합산을 진행함에 따라 실제로 달라질
	것입니다.
	그렇다곤 하나, 쓰레드의 생성과 종료는 \co{final_mutex} 에 의해
	배제되므로, \co{counterp} 안의 포인터들은 상수로 유지될 것입니다.
	\iffalse

	Refer to
	Figure~\ref{fig:count:Per-Thread Statistical Counters} on
	Page~\pageref{fig:count:Per-Thread Statistical Counters}.
	Clearly, if there are no concurrent invocations of \co{inc_count()},
	\co{read_count()} will return an exact result.
	However, if there \emph{are} concurrent invocations of
	\co{inc_count()}, then the sum is in fact changing as
	\co{read_count()} performs its summation.
	That said, because thread creation and exit are excluded by
	\co{final_mutex}, the pointers in \co{counterp} remain constant.
	\fi

	메모리의 즉석 스냅샷을 얻어올 수 있는 가상의 기계를 상상해 봅시다.
	이 기계가 \co{read_count()} 의 실행 시작지점의 스냅샷과
	\co{read_count()} 의 실행 종료 시점의 스냅샷을 만들어낸다고 생각해
	봅시다.
	그렇다면 \co{read_count()} 는 이 두 스냡샷들 사이의 어떤 시점에서의 각
	쓰레드의 카운터에 접근을 할 것이고, 따라서 이 두개의 스냡샷들에 의해
	값이 포괄적으로 한정지어지는 결과를 얻어오게 될 것입니다.
	따라서, 전체 합계는 이 두개의 스냡샷으로부터 각각 얻어와질 수 있는
	합계들의 쌍에 의해 그 값이 한정지어질 것입니다 (다시 말하지만,
	포괄적으로).

	따라서 예상되는 에러는 이 두개의 스냅샷으로부터 얻어올 수 있는 두개의
	합계의 쌍들 사이의 차이의 절반일 것이고, 이는 \co{read_count()} 의 실행
	시간에 단위 시간당 \co{inc_count()} 의 예상되는 호출 횟수를 곱한 값의
	절반입니다.
	\iffalse

	Let's imagine a mythical machine that is able to take an
	instantaneous snapshot of its memory.
	Suppose that this machine takes such a snapshot at the
	beginning of \co{read_count()}'s execution, and another
	snapshot at the end of \co{read_count()}'s execution.
	Then \co{read_count()} will access each thread's counter
	at some time between these two snapshots, and will therefore
	obtain a result that is bounded by those of the two snapshots,
	inclusive.
	The overall sum will therefore be bounded by the pair of sums that
	would have been obtained from each of the two snapshots (again,
	inclusive).

	The expected error is therefore half of the difference between
	the pair of sums that would have been obtained from each of the
	two snapshots, that is to say, half of the execution time of
	\co{read_count()} multiplied by the number of expected calls to
	\co{inc_count()} per unit time.
	\fi

	또는, 수식을 선호하는 분들을 위해 표시하면:
	\begin{equation}
	\epsilon = \frac{T_r R_i}{2}
	\end{equation}
	로, $\epsilon$ 는 \co{read_count()} 의 리턴 값에 예측되는 에러이고,
	$T_r$ 은 \co{read_count()} 가 실행되는데 걸리는 시간이고, $R_i$ 는 단위
	시간당 \co{inc_count()} 호출 횟수의 비율입니다.
	(그리고 당연하지만, $T_r$ 과 $R_i$ 는 같은 단위 시간을 사용해야 합니다:
	마이크로세컨드와 마이크로세컨드당 호출 횟수, 초와 초당 호출 횟수, 뭐가
	됐든, 같은 단위를 사용하기만 한다면.)
	\iffalse

	Or, for those who prefer equations:
	\begin{equation}
	\epsilon = \frac{T_r R_i}{2}
	\end{equation}
	where $\epsilon$ is the expected error in \co{read_count()}'s
	return value,
	$T_r$ is the time that \co{read_count()} takes to execute,
	and $R_i$ is the rate of \co{inc_count()} calls per unit time.
	(And of course, $T_r$ and $R_i$ should use the same units of
	time: microseconds and calls per microsecond, seconds and calls
	per second, or whatever, as long as they are the same units.)
	\fi
} \QuickQuizEnd

따라서, 우리가 \co{final_mutex} 를 없애려 한다면, 우리는 일관성을 보장하기 위한
어떤 다른 방법을 사용해야 합니다.
한가지 방법은 앞서 종료된 쓰레드들 전체를 위한 전체 카운트와 쓰레드별
카운터로의 포인터들의 배열을 하나의 구조체에 넣는 것입니다.
\co{read_count()} 에 의해 접근될 수 있는 그런 구조체는 상수가 되므로,
\co{read_count()} 가 일관적인 데이터를 보게 될 것을 보장합니다.
\iffalse

Therefore, if we are to dispense with \co{final_mutex}, we will need
to come up with some other method for ensuring consistency.
One approach is to place the total count for all previously exited
threads and the array of pointers to the per-thread counters into a single
structure.
Such a structure, once made available to \co{read_count()}, is
held constant, ensuring that \co{read_count()} sees consistent data.
\fi

\subsubsection{Implementation}

\begin{figure}[bp]
{ \scriptsize
\begin{verbbox}
  1 struct countarray {
  2   unsigned long total;
  3   unsigned long *counterp[NR_THREADS];
  4 };
  5 
  6 long __thread counter = 0;
  7 struct countarray *countarrayp = NULL;
  8 DEFINE_SPINLOCK(final_mutex);
  9 
 10 void inc_count(void)
 11 {
 12   counter++;
 13 }
 14 
 15 long read_count(void)
 16 {
 17   struct countarray *cap;
 18   unsigned long sum;
 19   int t;
 20 
 21   rcu_read_lock();
 22   cap = rcu_dereference(countarrayp);
 23   sum = cap->total;
 24   for_each_thread(t)
 25     if (cap->counterp[t] != NULL)
 26       sum += *cap->counterp[t];
 27   rcu_read_unlock();
 28   return sum;
 29 }
 30 
 31 void count_init(void)
 32 {
 33   countarrayp = malloc(sizeof(*countarrayp));
 34   if (countarrayp == NULL) {
 35     fprintf(stderr, "Out of memory\n");
 36     exit(-1);
 37   }
 38   memset(countarrayp, '\0', sizeof(*countarrayp));
 39 }
 40 
 41 void count_register_thread(void)
 42 {
 43   int idx = smp_thread_id();
 44 
 45   spin_lock(&final_mutex);
 46   countarrayp->counterp[idx] = &counter;
 47   spin_unlock(&final_mutex);
 48 }
 49 
 50 void count_unregister_thread(int nthreadsexpected)
 51 {
 52   struct countarray *cap;
 53   struct countarray *capold;
 54   int idx = smp_thread_id();
 55 
 56   cap = malloc(sizeof(*countarrayp));
 57   if (cap == NULL) {
 58     fprintf(stderr, "Out of memory\n");
 59     exit(-1);
 60   }
 61   spin_lock(&final_mutex);
 62   *cap = *countarrayp;
 63   cap->total += counter;
 64   cap->counterp[idx] = NULL;
 65   capold = countarrayp;
 66   rcu_assign_pointer(countarrayp, cap);
 67   spin_unlock(&final_mutex);
 68   synchronize_rcu();
 69   free(capold);
 70 }
\end{verbbox}
}
\centering
\theverbbox
\caption{RCU and Per-Thread Statistical Counters}
\label{fig:together:RCU and Per-Thread Statistical Counters}
\end{figure}

Figure~\ref{fig:together:RCU and Per-Thread Statistical Counters}
의 line~1-4 는 \co{countarray} 구조체를 보이고 있는데, 이 구조체는 앞서 종료된
쓰레드들의 카운트를 위한 \co{->total} 필드와 현재 돌아가고 있는 각각의 쓰레드를
위한 per-thread \co{counter} 로의 포인터들의 배열인 \co{counterp[]} 를
갖습니다.
이 구조체는 \co{read_count()} 의 한 수행이 수행중인 쓰레드의 알려진 집합과
일관적인 전체값을 볼 수 있도록 합니다.

Line~6-8 은 per-thread \co{counter} 변수, 현재의 \co{countarray} 구조체를
가리키는 \co{countarrayp} 글로벌 포인터, 그리고 \co{final_mutex} 스핀락의
정의를 담고 있습니다.

Line~10-13 은
Figure~\ref{fig:count:Per-Thread Statistical Counters}
로부터 달라지지 않은 \co{inc_count()} 를 보입니다.
\iffalse

Lines~1-4 of
Figure~\ref{fig:together:RCU and Per-Thread Statistical Counters}
show the \co{countarray} structure, which contains a
\co{->total} field for the count from previously exited threads,
and a \co{counterp[]} array of pointers to the per-thread
\co{counter} for each currently running thread.
This structure allows a given execution of \co{read_count()}
to see a total that is consistent with the indicated set of running
threads.

Lines~6-8 contain the definition of the per-thread \co{counter}
variable, the global pointer \co{countarrayp} referencing
the current \co{countarray} structure, and
the \co{final_mutex} spinlock.

Lines~10-13 show \co{inc_count()}, which is unchanged from
Figure~\ref{fig:count:Per-Thread Statistical Counters}.
\fi

Line~15-29 는 상당히 많이 바뀐 \co{read_count()} 를 보입니다.
Line~21 과 27 은 \co{rcu_read_lock()} 과 \co{rcu_read_unlock()} 으로
\co{final_mutex} 의 획득과 해제를 대신합니다.
Line~22 는 현재의 \co{countarray} 구조체를 로컬 변수 \co{cap} 으로 스냅샷을
뜨기 위해 \co{rcu_dereference()} 를 사용합니다.
RCU 가 올바르게 사용된다면 이 \co{countarray} 구조체가 적어도 line~27 에서의
현재 RCU read-side 크리티컬 섹션의 종료까지는 유지될 것이 보장될 겁니다.
Line~23 은 \co{sum} 을 \co{cap->total} 로 초기화 시키는데, 이는 앞서 종료된
쓰레드들의 카운트의 합입니다.
Line~24-26 은 현재 수행중인 쓰레드들과 연관되어 있는 per-thread 카운터들을
합하고, 마지막으로 line~28 은 그 합을 리턴합니다.
\iffalse

Lines~15-29 show \co{read_count()}, which has changed significantly.
Lines~21 and~27 substitute \co{rcu_read_lock()} and
\co{rcu_read_unlock()} for acquisition and release of \co{final_mutex}.
Line~22 uses \co{rcu_dereference()} to snapshot the
current \co{countarray} structure into local variable \co{cap}.
Proper use of RCU will guarantee that this \co{countarray} structure
will remain with us through at least the end of the current RCU
read-side critical section at line~27.
Line~23 initializes \co{sum} to \co{cap->total}, which is the
sum of the counts of threads that have previously exited.
Lines~24-26 add up the per-thread counters corresponding to currently
running threads, and, finally, line~28 returns the sum.
\fi

\co{countarrayp} 의 초기값은 line~31-39 의 \co{count_init()} 에 의해
주어집니다.
이 함수는 첫번째 쓰레드가 생성되기 전에 수행되는데, 이 함수의 일은 초기의
구조체를 할당하고 그 값을 0으로 세팅한 후에 \co{countarrayp} 를 할당하는
것입니다.

Line~41-48 은 \co{count_register_thread()} 함수를 보이는데, 이 함수는 각
쓰레드가 새로이 생성될 때마다 호출됩니다.
Line~43 은 현재 쓰레드의 인덱스를 가져오고, line~45 는 \co{final_mutex} 를
획득하며, line~46 은 이 쓰레드의 \co{counter} 로의 포인터를 설치하며, line~47
에서 \co{final_mutex} 를 놓습니다.
\iffalse

The initial value for \co{countarrayp} is
provided by \co{count_init()} on lines~31-39.
This function runs before the first thread is created, and its job
is to allocate
and zero the initial structure, and then assign it to \co{countarrayp}.

Lines~41-48 show the \co{count_register_thread()} function, which
is invoked by each newly created thread.
Line~43 picks up the current thread's index, line~45 acquires
\co{final_mutex}, line~46 installs a pointer to this thread's
\co{counter}, and line~47 releases \co{final_mutex}.
\fi

\QuickQuiz{}
	이봐요!!!
	Figure~\ref{fig:together:RCU and Per-Thread Statistical Counters}
	의 line~46 은 앞서 존재한 \co{countarray} 구조체의 값을 수정하잖아요!
	이 구조체는 일단 한번 \co{read_count()} 에 접근 가능하게 되면 상수로
	남게 된다고 하지 않았어요???
	\iffalse

	Hey!!!
	Line~46 of
	Figure~\ref{fig:together:RCU and Per-Thread Statistical Counters}
	modifies a value in a pre-existing \co{countarray} structure!
	Didn't you say that this structure, once made available to
	\co{read_count()}, remained constant???
	\fi
\QuickQuizAnswer{
	실제로 전 그렇게 말했습니다.
	그리고 \co{count_unregister_thread()} 가 현재 그렇듯이
	\co{count_register_thread()} 가 새로운 구조체를 할당하도록 하는 것이
	가능할 겁니다.

	하지만 그건 불필요한 일입니다.
	에러가 메모리의 스냅샷에 기반을 둔 \co{read_count()} 에 의해 최대값이
	제한된다는 유도를 다시 생각해 보세요.
	새로운 쓰레드들은 그 값이 0인 \co{counter} 값과 함께 시작되므로, 이
	유도는 우리가 \co{read_count()} 의 실행 중간에 새로운 쓰레드를
	추가한다고 해도 지켜집니다.
	따라서, 흥미롭게도, 새로운 쓰레드를 추가할 때에, 이 구현은 새로운
	구조체를 할당하는 효과를 실제로는 할당을 하지 않으면서도 얻을 수 있게
	됩니다.
	\iffalse

	Indeed I did say that.
	And it would be possible to make \co{count_register_thread()}
	allocate a new structure, much as \co{count_unregister_thread()}
	currently does.

	But this is unnecessary.
	Recall the derivation of the error bounds of \co{read_count()}
	that was based on the snapshots of memory.
	Because new threads start with initial \co{counter} values of
	zero, the derivation holds even if we add a new thread partway
	through \co{read_count()}'s execution.
	So, interestingly enough, when adding a new thread, this
	implementation gets the effect of allocating a new structure,
	but without actually having to do the allocation.
	\fi
} \QuickQuizEnd

Line~50-70 은 각 쓰레드가 종료하기 직전에 실행되는
\co{count_unregister_thread()} 를 보입니다.
Line~56-60 은 새로운 \co{countarray} 구조체를 할당하고, line~61 에서
\co{final_mutex} 를 획득하며 line~67 에서 이를 해제합니다.
Line~62 는 현재 \co{countarray} 의 내용을 새로 할당된 버전에 복사하고, line~63
은 종료되는 쓰레드의 \co{counter} 를 새로운 구조체의 \co{->total} 에 더하고,
line~66 은 \co{countarray} 구조체의 새로운 버전을 설치하는데에
\co{rcu_assign_pointer()} 를 사용합니다.
Line~68 은 동시적으로 \co{read_count()} 를 수행하고 있을 수 있는, 따라서 기존의
\co{countarray} 구조체로의 레퍼런스를 가지고 있을 수 있는 모든 쓰레드가 이들의
RCU read-side 크리티컬 섹션을 종료해서 그런 레퍼런스를 모두 버려버리도록 grace
period 가 하나 지나가길 기다립니다.
그러고 나서 line~69 는 이제 기존의 \co{countarray} 구조체를 안전하게 해제시킬
수 있습니다.
\iffalse

Lines~50-70 shows \co{count_unregister_thread()}, which is invoked
by each thread just before it exits.
Lines~56-60 allocate a new \co{countarray} structure,
line~61 acquires \co{final_mutex} and line~67 releases it.
Line~62 copies the contents of the current \co{countarray} into
the newly allocated version, line~63 adds the exiting thread's \co{counter}
to new structure's \co{->total}, and line~64 \co{NULL}s the exiting thread's
\co{counterp[]} array element.
Line~65 then retains a pointer to the current (soon to be old)
\co{countarray} structure, and line~66 uses \co{rcu_assign_pointer()}
to install the new version of the \co{countarray} structure.
Line~68 waits for a grace period to elapse, so that any threads that
might be concurrently executing in \co{read_count()}, and thus might
have references to the old \co{countarray} structure, will be allowed
to exit their RCU read-side critical sections, thus dropping any such
references.
Line~69 can then safely free the old \co{countarray} structure.
\fi

\subsubsection{Discussion}

\QuickQuiz{}
	우와!
	Figure~\ref{fig:count:Per-Thread Statistical Counters} 는 라인수가 42
	밖에 되지 않는데 반해 Figure~\ref{fig:together:RCU and Per-Thread
	Statistical Counters} 는 69 라인이나 되는군요.
	이 추가적인 복잡도가 정말로 가치가 있는 건가요?
	\iffalse

	Wow!
	Figure~\ref{fig:together:RCU and Per-Thread Statistical Counters}
	contains 69 lines of code, compared to only 42 in
	Figure~\ref{fig:count:Per-Thread Statistical Counters}.
	Is this extra complexity really worth it?
	\fi
\QuickQuizAnswer{
	이는 당연하게도 경우에 따라 다르게 결정되어야 합니다.
	선형적으로 확장되는 \co{read_count()} 구현이 필요하다면,
	Figure~\ref{fig:count:Per-Thread Statistical Counters}
	에 보인 락 기반의 구현은 당신을 위한 동작을 하지 않을 겁니다.
	반면에, 만약 \co{count_read()} 의 호출이 충분히 드물다면, 락 기반
	버전이 더 간단하고 따라서 더 나을 수도 있습니다, 대부분의 크기 차이는
	구조체 정의, 메모리 할당, 그리고 \co{NULL} 리턴 체크로 인한 것이니
	하지만 말입니다.

	물론, 더 나은 질문은 ``왜 이 언어는 \co{__thread} 변수들로의 쓰레드를
	넘어서는 접근을 구현하지 않는거죠?'' 가 될겁니다.
	무엇보다, 그런 구현은 락킹과 RCU 의 사용을 불필요하게 만들 겁니다.
	이는 결국
	Figure~\ref{fig:together:RCU and Per-Thread Statistical Counters} 에
	보여진 구현의 확장성과 성능상의 이득에 더해서
	Figure~\ref{fig:count:Per-Thread Statistical Counters} 에 보여진
	것보다도 더 간단한 구현이 가능하게 할 겁니다!
	\iffalse

	This of course needs to be decided on a case-by-case basis.
	If you need an implementation of \co{read_count()} that
	scales linearly, then the lock-based implementation shown in
	Figure~\ref{fig:count:Per-Thread Statistical Counters}
	simply will not work for you.
	On the other hand, if calls to \co{count_read()} are sufficiently
	rare, then the lock-based version is simpler and might thus be
	better, although much of the size difference is due
	to the structure definition, memory allocation, and \co{NULL}
	return checking.

	Of course, a better question is ``Why doesn't the language
	implement cross-thread access to \co{__thread} variables?''
	After all, such an implementation would make both the locking
	and the use of RCU unnecessary.
	This would in turn enable an implementation that
	was even simpler than the one shown in
	Figure~\ref{fig:count:Per-Thread Statistical Counters}, but
	with all the scalability and performance benefits of the
	implementation shown in
	Figure~\ref{fig:together:RCU and Per-Thread Statistical Counters}!
	\fi
} \QuickQuizEnd

RCU 의 사용은 종료되는 쓰레드가 다른 쓰레드들이 이 종료되는 쓰레드들의
\co{__thread} 변수들을 사용하는 것을 마칠 때까지 기다릴 수 있도록 해줍니다.
이는 \co{read_count()} 함수가 락킹을 필요없게 해서 \co{inc_count()} 와
\co{read_count()} 함수 둘 다에 훌륭한 성능과 확장성을 제공합니다.
하지만, 이 성능과 확장성은 약간의 코드 복잡도의 증가를 비용으로 지불합니다.
컴파일러와 라이브러리를 작성하는 사람들이 안전한 쓰레드간 \co{__thread}
변수들로의 접근을 제공하기 위해 유저 레벨 RCU~\cite{MathieuDesnoyers2009URCU}
를 제공해서 \co{__thread} 변수들의 사용자간에 보여지게 되는 복잡도를 줄일 수
있게 된다면 좋을 겁니다.
\iffalse

Use of RCU enables exiting threads to wait until other threads are
guaranteed to be done using the exiting threads' \co{__thread} variables.
This allows the \co{read_count()} function to dispense with locking,
thereby providing
excellent performance and scalability for both the \co{inc_count()}
and \co{read_count()} functions.
However, this performance and scalability come at the cost of some increase
in code complexity.
It is hoped that compiler and library writers employ user-level
RCU~\cite{MathieuDesnoyers2009URCU} to provide safe cross-thread
access to \co{__thread} variables, greatly reducing the
complexity seen by users of \co{__thread} variables.
\fi

\subsection{RCU and Counters for Removable I/O Devices}
\label{sec:together:RCU and Counters for Removable I/O Devices}

Section~\ref{sec:count:Applying Specialized Parallel Counters}
는 제거 가능한 디바이스들로의 I/O 액세스의 카운팅을 다루는 한쌍의 코드 조각을
보였습니다.
이 코드 조각들은 reader-writer 락을 잡아야 하는 필요로 인해 빠른 수행 경로 (I/O
를 시작하는 것) 에서의 높은 오버헤드로 힘들어했습니다.

이 섹션은 RCU 가 이 오버헤드를 어떻게 없앨 수 있는지 보입니다.

I/O 를 수행하는 코드는 원래의 것과 상당히 유사한데, RCU read-side 크리티컬
섹션이 원래 것의 reader-writer 락의 read-side 크리티컬 섹션을 대체합니다:
\iffalse

Section~\ref{sec:count:Applying Specialized Parallel Counters}
showed a fanciful pair of code fragments for dealing with counting
I/O accesses to removable devices.
These code fragments suffered from high overhead on the fastpath
(starting an I/O) due to the need to acquire a reader-writer
lock.

This section shows how RCU may be used to avoid this overhead.

The code for performing an I/O is quite similar to the original, with
a RCU read-side critical section being substituted for the reader-writer
lock read-side critical section in the original:
\fi

\vspace{5pt}
\begin{minipage}[t]{\columnwidth}
\small
\begin{verbatim}
  1 rcu_read_lock();
  2 if (removing) {
  3   rcu_read_unlock();
  4   cancel_io();
  5 } else {
  6   add_count(1);
  7   rcu_read_unlock();
  8   do_io();
  9   sub_count(1);
 10 }
\end{verbatim}
\end{minipage}
\vspace{5pt}

RCU read-side 기능들은 최소한의 오버헤드만을 가지고 있으므로, 원했던대로 빠른
수행 경로의 속도를 높입니다.

디바이스 제거 부분의 업데이트된 코드 조각은 다음과 같습니다:
\iffalse

The RCU read-side primitives have minimal overhead, thus speeding up
the fastpath, as desired.

The updated code fragment removing a device is as follows:
\fi

\vspace{5pt}
\begin{minipage}[t]{\columnwidth}
\small
\begin{verbatim}
  1 spin_lock(&mylock);
  2 removing = 1;
  3 sub_count(mybias);
  4 spin_unlock(&mylock);
  5 synchronize_rcu();
  6 while (read_count() != 0) {
  7   poll(NULL, 0, 1);
  8 }
  9 remove_device();
\end{verbatim}
\end{minipage}
\vspace{5pt}

여기서 우린 reader-writer 락을 배타적 스핀락으로 바꾸고 모든 RCU read-side
크리티컬 섹션들이 완료되길 기다리기 위해 \co{synchronize_rcu()} 를
추가했습니다.
\co{synchronize_rcu()} 때문에, 일단 우리가 line~6 에 도달한다면, 우린 모든
남아있는 I/O 들이 처리되었음을 알 수 있습니다.

물론, \co{synchronize_rcu()} 의 오버헤드는 클 수 있습니다만, 디바이스의 제거는
상당히 드문 일임을 고려한다면 이는 괜찮은 트레이드오프입니다.
\iffalse

Here we replace the reader-writer lock with an exclusive spinlock and
add a \co{synchronize_rcu()} to wait for all of the RCU read-side
critical sections to complete.
Because of the \co{synchronize_rcu()},
once we reach line~6, we know that all remaining I/Os have been accounted
for.

Of course, the overhead of \co{synchronize_rcu()} can be large,
but given that device removal is quite rare, this is usually a good
tradeoff.
\fi

\subsection{Array and Length}
\label{sec:together:Array and Length}

\begin{figure}[tbp]
{ \scriptsize
\begin{verbbox}
 1 struct foo {
 2   int length;
 3   char *a;
 4 };
\end{verbbox}
}
\centering
\theverbbox
\caption{RCU-Protected Variable-Length Array}
\label{fig:together:RCU-Protected Variable-Length Array}
\end{figure}

Figure~\ref{fig:together:RCU-Protected Variable-Length Array} 에 보인 것과 같이
RCU 로 보호되는 가변길이의 배열을 가지고 있다고 생각해 봅시다.
배열 \co{->a[]} 의 길이는 동적으로 언제든 변할 수 있고, 그 길이는 \co{->length}
를 통해 제공됩니다.
물론, 이는 다음과 같은 경주 조건을 만들어냅니다:
\iffalse

Suppose we have an RCU-protected variable-length array, as shown in
Figure~\ref{fig:together:RCU-Protected Variable-Length Array}.
The length of the array \co{->a[]} can change dynamically, and at any
given time, its length is given by the field \co{->length}.
Of course, this introduces the following race condition:
\fi

\begin{enumerate}
\item	배열의 길이가 초기에는 16 character 만큼의 길이이고, 따라서
	\co{->length} 는 16의 값을 갖습니다.
\item	CPU~0 가 \co{->length} 의 값을 읽어와서 16을 얻게 됩니다.
\item	CPU~1 이 이 배열의 길이를 8로 줄이게 되고, 메모리의 8-charcter 블락
	구간으로의 포인터를 \co{->a[]} 에 할당합니다.
\item	CPU~0 가 \co{->a[]} 의 새로운 포인터를 가져오고, 그 12번째 원소에
	새로운 값을 저장합니다.
	이 배열은 8 개 character 만을 가지고 있으므로, 이는 SEGV 또는 (더 나쁜
	경우인) 메모리 오염을 초래하게 됩니다.
\iffalse

\item	The array is initially 16 characters long, and thus \co{->length}
	is equal to 16.
\item	CPU~0 loads the value of \co{->length}, obtaining the value 16.
\item	CPU~1 shrinks the array to be of length 8, and assigns a pointer
	to a new 8-character block of memory into \co{->a[]}.
\item	CPU~0 picks up the new pointer from \co{->a[]}, and stores a
	new value into element 12.
	Because the array has only 8 characters, this results in
	a SEGV or (worse yet) memory corruption.
\fi
\end{enumerate}

이를 어떻게 막을 수 있을까요?

한가지 방법은
Section~\ref{sec:advsync:Memory Barriers} 에서 소개된 메모리 배리어를
조심스럽게 사용하는 것입니다.
이는 제대로 동작합니다만, 읽는 쪽의 오버헤드를 초래하고, 아마도 더 나쁠 수도
있게도, 명시적인 메모리 배리어의 사용을 필요로 합니다.
\iffalse

How can we prevent this?

One approach is to make careful use of memory barriers, which are
covered in Section~\ref{sec:advsync:Memory Barriers}.
This works, but incurs read-side overhead and, perhaps worse, requires
use of explicit memory barriers.
\fi

\begin{figure}[tbp]
{ \scriptsize
\begin{verbbox}
 1 struct foo_a {
 2   int length;
 3   char a[0];
 4 };
 5 
 6 struct foo {
 7   struct foo_a *fa;
 8 };
\end{verbbox}
}
\centering
\theverbbox
\caption{Improved RCU-Protected Variable-Length Array}
\label{fig:together:Improved RCU-Protected Variable-Length Array}
\end{figure}

더 나은 전략은
Figure~\ref{fig:together:Improved RCU-Protected Variable-Length Array}
에 보인 것과 같이 해당 배열과 값을 같은 구조체에 놓는 것입니다.
이제 새로운 배열 (\co{foo_a} 구조체) 를 할당하는 일은 자동으로 배열 공간을 위한
새로운 공간을 제공합니다.
이는 어떤 CPU 가 \co{->fa} 로의 레퍼런스를 가져가면 \co{->length} 는 \co{->a[]}
의 길이와 들어맞을 것임을 보장합니다.
\iffalse

A better approach is to put the value and the array into the same structure,
as shown in
Figure~\ref{fig:together:Improved RCU-Protected Variable-Length Array}.
Allocating a new array (\co{foo_a} structure) then automatically provides
a new place for the array length.
This means that if any CPU picks up a reference to \co{->fa}, it is
guaranteed that the \co{->length} will match the \co{->a[]}
length~\cite{Arcangeli03}.
\fi

\begin{enumerate}
\item	배열의 길이가 초기에는 16 character 길이이고, 따라서 \co{->length} 는
	16의 값을 갖습니다.
\item	CPU~0 이 \co{->fa} 의 값을 읽어오게 되어서 값 16과 16-byte 배열을 담고
	있는 구조체를 가져오게 됩니다.
\item	CPU~0 가 \co{->fa->length} 의 값을 읽어와서 16이라는 값을 얻게 됩니다.
\item	CPU~1 이 배열의 길이를 8로 줄이고, 메모리의 8-character 블록 구간을
	포함하는, 새로운 \co{foo_a} 구조체로의 포인터를 \co{->fa} 에
	할당합니다.
\item	CPU~0 이 \co{->a[]} 로부터 새로운 포인터를 가져가고 새로운 값을 12번째
	원소에 저장합니다.
	하지만 CPU~0 는 여전히 16-byte 배열을 담고 있는 기존의 \co{foo_a} 를
	레퍼런스하고 있으므로, 모두 문제 없습니다.
\iffalse

\item	The array is initially 16 characters long, and thus \co{->length}
	is equal to 16.
\item	CPU~0 loads the value of \co{->fa}, obtaining a pointer to
	the structure containing the value 16 and the 16-byte array.
\item	CPU~0 loads the value of \co{->fa->length}, obtaining the value 16.
\item	CPU~1 shrinks the array to be of length 8, and assigns a pointer
	to a new \co{foo_a} structure containing an 8-character block
	of memory into \co{->fa}.
\item	CPU~0 picks up the new pointer from \co{->a[]}, and stores a
	new value into element 12.
	But because CPU~0 is still referencing the old \co{foo_a}
	structure that contains the 16-byte array, all is well.
\fi
\end{enumerate}

물론, 두 경우 모두, CPU~1 은 기존의 배열을 메모리에서 해제하기 전에 하나의
grace period 를 기다려야만 합니다.

이 방법의 더 일반적인 버전이 다음 섹션에서 제공됩니다.
\iffalse

Of course, in both cases, CPU~1 must wait for a grace period before
freeing the old array.

A more general version of this approach is presented in the next section.
\fi

\subsection{Correlated Fields}
\label{sec:together:Correlated Fields}

\begin{figure}[tbp]
{ \scriptsize
\begin{verbbox}
 1 struct animal {
 2   char name[40];
 3   double age;
 4   double meas_1;
 5   double meas_2;
 6   double meas_3;
 7   char photo[0]; /* large bitmap. */
 8 };
\end{verbbox}
}
\centering
\theverbbox
\caption{Uncorrelated Measurement Fields}
\label{fig:together:Uncorrelated Measurement Fields}
\end{figure}

각각의 Sch\"odinger 의 동물들이
Figure~\ref{fig:together:Uncorrelated Measurement Fields}
에 보여진 데이터 원소로 표현된다고 생각해 봅시다.
\co{meas_1}, \co{meas_2}, 그리고 \co{meas_3} 필드는 주기적으로 업데이트되는
연관된 측정치의 집합입니다.
읽기를 하는 쓰레드들은 이 세개의 값들을 하나의 측정치 업데이트로부터 본다는
점이 특히 중요합니다: 만약 한 읽기 쓰레드가 \co{meas_1} 의 예전 값을 읽지만
\co{meas_2} 와 \co{meas_3} 의 새로운 값을 읽게 되다면, 그 읽기 쓰레드는 완전히
혼란에 빠질 겁니다.
읽기 쓰레드들이 이 세개의 값들의 통합된 집합을 볼 수 있도록 보장하려면 어떻게
해야 할까요?
\iffalse

Suppose that each of Sch\"odinger's animals is represented by the
data element shown in
Figure~\ref{fig:together:Uncorrelated Measurement Fields}.
The \co{meas_1}, \co{meas_2}, and \co{meas_3} fields are a set
of correlated measurements that are updated periodically.
It is critically important that readers see these three values from
a single measurement update: If a reader sees an old value of
\co{meas_1} but new values of \co{meas_2} and \co{meas_3}, that
reader will become fatally confused.
How can we guarantee that readers will see coordinated sets of these
three values?
\fi

한가지 방법은 새로운 \co{animal} 구조체를 할당하고, 기존 구조체를 새로운
구조체로 복사하고, 새로운 구조체의 \co{meas_1}, \co{meas_2}, 그리고 \co{meas_3}
필드들을 업데이트 한 후, 기존의 구조체를 새로운 구조체로 포인터를 업데이트하는
방법으로 교체하는 것입니다.
이는 모든 읽기 쓰레드들이 측정 값들의 통합된 집합을 볼 수 있을 것을
보장합니다만, 이는 \co{->photo[]} 필드 때문에 커다란 구조체의 복사를 필요로
하게 됩니다.
이 복사 작업은 받아들일 수 없을 만큼 커다란 오버헤드를 일으킬 수 있습니다.
\iffalse

One approach would be to allocate a new \co{animal} structure,
copy the old structure into the new structure, update the new
structure's \co{meas_1}, \co{meas_2}, and \co{meas_3} fields,
and then replace the old structure with a new one by updating
the pointer.
This does guarantee that all readers see coordinated sets of
measurement values, but it requires copying a large structure due
to the \co{->photo[]} field.
This copying might incur unacceptably large overhead.
\fi

\begin{figure}[tbp]
{ \scriptsize
\begin{verbbox}
 1 struct measurement {
 2   double meas_1;
 3   double meas_2;
 4   double meas_3;
 5 };
 6 
 7 struct animal {
 8   char name[40];
 9   double age;
10   struct measurement *mp;
11   char photo[0]; /* large bitmap. */
12 };
\end{verbbox}
}
\centering
\theverbbox
\caption{Correlated Measurement Fields}
\label{fig:together:Correlated Measurement Fields}
\end{figure}

또다른 방법은 하나의 간접적 단계를 추가하는 것으로,
Figure~\ref{fig:together:Correlated Measurement Fields} 보여진 것과 같습니다.
새로운 측정이 취해진다면, 새로운 \co{measurement} 구조체를 할당하고, 이를
새로운 측정치로 채우고, \co{animal} 구조체의 \co{->mp} 필드가 이 새로운
\co{measurement} 구조체의 포인터를 가리키도록 \co{rcu_assign_pointer()} 를
사용해서 업데이트 하는 것입니다.
하나의 grace period 가 지나간 후에, 기존의 \co{measurement} 구조체는 메모리에서
해제될 수 있습니다.
\iffalse

Another approach is to insert a level of indirection, as shown in
Figure~\ref{fig:together:Correlated Measurement Fields}.
When a new measurement is taken, a new \co{measurement} structure
is allocated, filled in with the measurements, and the \co{animal}
structure's \co{->mp} field is updated to point to this new
\co{measurement} structure using \co{rcu_assign_pointer()}.
After a grace period elapses, the old \co{measurement} structure
can be freed.
\fi

\QuickQuiz{}
	하지만
	Figure~\ref{fig:together:Correlated Measurement Fields}
	에 보인 방법은 추가적인 캐시 미스를 초래할 수 있어서, 추가적인 읽기
	쪽의 오버헤드를 초래할 수 있지 않나요?
	\iffalse

	But cant't the approach shown in
	Figure~\ref{fig:together:Correlated Measurement Fields}
	result in extra cache misses, in turn resulting in additional
	read-side overhead?
	\fi
\QuickQuizAnswer{
	실제로 그럴 수 있습니다.
	\iffalse

	Indeed it can.
	\fi

\begin{figure}[tbp]
{ \scriptsize
\begin{verbbox}
 1 struct measurement {
 2   double meas_1;
 3   double meas_2;
 4   double meas_3;
 5 };
 6 
 7 struct animal {
 8   char name[40];
 9   double age;
10   struct measurement *mp;
11   struct measurement meas;
12   char photo[0]; /* large bitmap. */
13 };
\end{verbbox}
}
\centering
\theverbbox
\caption{Localized Correlated Measurement Fields}
\label{fig:together:Localized Correlated Measurement Fields}
\end{figure}

	이 캐시 미스 오버헤드를 없애기 위한 한가지 방법이
	Figure~\ref{fig:together:Localized Correlated Measurement Fields} 에
	보여져 있습니다:
	간단히 \co{measurement} 구조체의 인스턴스 하나를 \co{meas} 라는
	이름으로 \co{animal} 구조체 내에 내장시키고, \co{->mp} 필드를 통해 이
	\co{->meas} 필드를 가리키는 것이죠.

	측정치 업데이트는 이제 다음과 같이 진행될 수 있습니다:
	\iffalse

	One way to avoid this cache-miss overhead is shown in
	Figure~\ref{fig:together:Localized Correlated Measurement Fields}:
	Simply embed an instance of a \co{measurement} structure
	named \co{meas}
	into the \co{animal} structure, and point the \co{->mp}
	field at this \co{->meas} field.

	Measurement updates can then be carried out as follows:
	\fi

	\begin{enumerate}
	\item	새로운 \co{measurement} 구조체를 할당하고 새로운 측정치를 그
		안에 집어넣습니다.
	\item	\co{rcu_assign_pointer()} 를 사용해서 \co{->mp} 가 이 새로운
		구조체를 가리키도록 합니다.
	\item	하나의 grace period 가 지나가길 기다리는데, 예를들어
		\co{synchronize_rcu()} 나 \co{call_rcu()} 를 사용합니다.
	\item	새로운 \co{measurement} 구조체의 측정치들을 내장된 \co{->meas}
		필드로 복사합니다.
	\item	\co{rcu_assign_pointer()} 를 사용해서 \co{->mp} 가 다시 이
		기존의 내장된 \co{->meas} 필드를 가리키도록 합니다.
	\item	또다른 하나의 grace period 가 지나간 후, 새로운
		\co{measurement} 구조체를 메모리에서 해제시킵니다.
	\iffalse

	\item	Allocate a new \co{measurement} structure and place
		the new measurements into it.
	\item	Use \co{rcu_assign_pointer()} to point \co{->mp} to
		this new structure.
	\item	Wait for a grace period to elapse, for example using
		either \co{synchronize_rcu()} or \co{call_rcu()}.
	\item	Copy the measurements from the new \co{measurement}
		structure into the embedded \co{->meas} field.
	\item	Use \co{rcu_assign_pointer()} to point \co{->mp}
		back to the old embedded \co{->meas} field.
	\item	After another grace period elapses, free up the
		new \co{measurement} structure.
	\fi
	\end{enumerate}

	이 방법은 일반적인 경우에서의 추가적인 캐시 미스를 제거하기 위해 더
	무거운 업데이트 절치를 사용합니다.
	이 추가적인 캐시 미스는 업데이트가 실제로 진행중일 때에만 일어날
	겁니다.
	\iffalse

	This approach uses a heavier weight update procedure to eliminate
	the extra cache miss in the common case.
	The extra cache miss will be incurred only while an update is
	actually in progress.
	\fi
} \QuickQuizEnd

이 방법은 읽기 쓰레드들이 선택된 필드들에 대해 연관된 값들을 최소한의 읽기 쪽
오버헤드만을 가지고 볼 수 있도록 해줍니다.
\iffalse

This approach enables readers to see correlated values for selected
fields with minimal read-side overhead.
\fi

% Birthstone/tombstone for moving records when readers cannot be permitted
% to see extraneous records.

% Flag for deletion (if not already covered in the defer chapter).

% @@@ Later add section on updates: hashed arrays of locks, fifos/streaming,
% batching to trade off latency for perf/scale.

\QuickQuizAnswersChp{qqztogether}
