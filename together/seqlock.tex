% together/seqlock.tex
% mainfile: ../perfbook.tex
% SPDX-License-Identifier: CC-BY-SA-3.0

\section{Sequence-Locking Specials}
\label{sec:together:Sequence-Locking Specials}
%
\epigraph{The girl who can't dance says the band can't play.}
	 {\emph{Yiddish proverb}}

이 섹션은 시퀀스 락의 특별한 사용처들을 알아봅니다.

\iffalse

This section looks at some special uses of sequence locks.

\fi

\subsection{Correlated Data Elements}
\label{sec:together:Correlated Data Elements}

두개 이상의 원소들을 연관지어 봐야 하는 해쉬 테이블을 가지고 있다고 해봅시다.
이 원소들은 함께 업데이트 되며, 첫번째 원소의 기존 버전을 다른 원소의 새 버전과
함께 보고 싶지 않습니다.
예를 들어, Schr\"odinger 는 그의 in-memory 데이터베이스에 그의 동물들에 더해
확장된 가족을 넣고 싶습니다.
Schr\"odinger 는 결혼과 이혼이 급작스럽게 일어나지는 않음을 알지만, 그는 또한
전통주의자이기도 합니다.
따라서, 그는 그의 데이터베이스가 신부는 결혼했는데 신랑은 그렇지 않은, 또는 그
반대의 경우를 보이기를 원치 않습니다.
또한, Schr\"odinger 가 전통주의자라 생각한다면, 여러분은 그의 가족 구성원들 중
일부와 대화해 보세요!
달리 말하자면, Schr\"odinger 는 그의 데이터베이스가 결혼에 있어 일관적이길
원합니다.

\iffalse

Suppose we have a hash table where we need correlated views of two or
more of the elements.
These elements are updated together, and we do not want to see an old
version of the first element along with new versions of the other
elements.
For example, Schr\"odinger decided to add his extended family to his
in-memory database along with all his animals.
Although Schr\"odinger understands that marriages and divorces do not
happen instantaneously, he is also a traditionalist.
As such, he absolutely does not want his database ever to show that the
bride is now married, but the groom is not, and vice versa.
Plus, if you think Schr\"odinger is a traditionalist, you just
try conversing with some of his family members!
In other words, Schr\"odinger wants to be able to carry out a
wedlock-consistent traversal of his database.

\fi

한가지 방법은 시퀀스 락을 사용해서
(\cref{sec:defer:Sequence Locks} 를 참고하세요),
결혼에 관련된 업데이트가 \co{write_seqlock()} 의 보호 아래 진행되고 결혼
일관성이 피료한 읽기는 \co{read_seqbegin()} / \co{read_seqretry()} 반복문
아래에서 진행되게 하는 겁니다.
시퀀스 락은 RCU 보호의 대체제가 아님에 유의하세요:
시퀀스 락은 동시의 수정으로부터 보호를 해줍니다만, 동시의 삭제로부터의 보호를
위해선 여전히 RCU 가 필요합니다.

이 방법은 연관된 원소의 수가 작고 원소들에의 읽기 시간이 짧으며, 업데이트
비율이 낮을 때 상당히 잘 작동합니다.
그렇지 않다면, 읽기 쓰레드가 영원히 완료되지 못할 수도 있게끔 업데이트가
빈번하게 이러날 수도 있습니다.
Schr\"odinger 는 이런 문제가 일어날 만큼 그의 가장 덜 정상적인 지인들이
결혼하고 곧바로 이혼할 거라 생각하지 않지만, 그는 이 문제가 다른 환경에서는
일어날 수 있음을 알고 있습니다.
이 읽기 쓰레드 starvation 문제를 해결하는 한가지 방법은 읽기 쓰레드가 너무 많은
재시도를 했다면 업데이트 쪽 기능을 사용하게 하는 것입니다만, 이는 성능과
확장성을 모두 악화시킬 수 있습니다.
Starvation 을 막는 다른 방법은 여러 시퀀스 락을 상요하는 것으로, 이
Schr\"odinger 의 경우엔 종당 하나가 될 수 있겠습니다.

\iffalse

One approach is to use sequence locks
(see \cref{sec:defer:Sequence Locks}),
so that wedlock-related updates are carried out under the
protection of \co{write_seqlock()}, while reads requiring
wedlock consistency are carried out within
a \co{read_seqbegin()} / \co{read_seqretry()} loop.
Note that sequence locks are not a replacement for RCU protection:
Sequence locks protect against concurrent modifications, but RCU
is still needed to protect against concurrent deletions.

This approach works quite well when the number of correlated elements is
small, the time to read these elements is short, and the update rate is
low.
Otherwise, updates might happen so quickly that readers might never complete.
Although Schr\"odinger does not expect that even his least-sane relatives
will marry and divorce quickly enough for this to be a problem,
he does realize that this problem could well arise in other situations.
One way to avoid this reader-starvation problem is to have the readers
use the update-side primitives if there have been too many retries,
but this can degrade both performance and scalability.
Another way to avoid starvation is to have multiple sequence locks,
in Schr\"odinger's case, perhaps one per species.

\fi

또한, 만약 업데이트 쪽 기능이 너무 자주 사용된다면, 락 컨텐션으로 인해 낮은
성능과 확장성이 초래될 겁니다.
이를 막는 한가지 방법은 원소별 시퀀스 락을 두고 부부의 결혼 상태를 업데이트 할
때 부부 두명의 락을 모두 잡는 겁니다.
읽기 쓰레드는 이 한쌍의 멤버들의 결혼 상태에 대한 모든 변화에 대해 안정적인
읽기를 위해 이 부부의 락들 중 하나만 가지고 재시도 반복을 할 수 있습니다.
이는 높은 결혼과 이혼율에 의한 컨텐션을 막을 수 있으나, 데이터베이스의 한번의
스캔 동안의 모든 결혼 상태에 대한 일관적 시각을 얻기를 복잡하게 만듭니다.

결혼 상태가 그러길 바라듯이 원소 그룹 짓기가 잘 정의되었고 영구적이라면, 가능한
한가지 방법은 데이터 원소로의 포인터들을 더해서 특정 그룹의 멤버들을 함께
연결짓는 겁니다.
그러면 읽기 쓰레드는 첫번째 원소가 발견되면 같은 그룹의 데이터 원소들을
접근하기 위해 이 포인터들을 따라갈 수 있습니다.

이 기법은 리눅스 커널에서 널리 사용되는데, dcache subsystem
에서~\cite{NeilBrown2015RCUwalk} 특히 그렇습니다.
비슷한 방법이 해저드 포인터를 통해서도 동작할 수 있음을 알아 두시기 바랍니다.

\iffalse

In addition, if the update-side primitives are used too frequently,
poor performance and scalability will result due to lock contention.
One way to avoid this is to maintain a per-element sequence lock,
and to hold both spouses' locks when updating their marital status.
Readers can do their retry looping on either of the spouses' locks
to gain a stable view of any change in marital status involving both
members of the pair.
This avoids contention due to high marriage and divorce rates, but
complicates gaining a stable view of all marital statuses during a
single scan of the database.

If the element groupings are well-defined and persistent, which marital
status is hoped to be,
then one approach is to add pointers to the data elements to link
together the members of a given group.
Readers can then traverse these pointers to access all the data elements
in the same group as the first one located.

This technique is used heavily in the Linux kernel, perhaps most
notably in the dcache subsystem~\cite{NeilBrown2015RCUwalk}.
Note that it is likely that similar schemes also work with hazard
pointers.

\fi

또다른 방법은 데이터 원소들을 파편화 하고 각 업데이트가 그 업데이트로 영향 받는
모든 데이터 원소를 위한 모든 시퀀스 락에 대해 쓰기 권한 획득을 하게 하는
겁니다.
물론, 이 쓰기 권한 획득은 데드락을 막기 위해 신중히 가해져야 합니다.
읽기 쓰레드는 여러 시퀀스 락에 대한 읽기 권한 획득을 필요로 하겠습니다만 읽기
쓰레드가 하나의 데이터 원소만 읽어야 하는 놀랍도록 흔한 상황에서는 하나의
시퀀스 락에 대해서만 읽기 권한 획득이 필요합니다.

이 방법은 성공한 읽기 쓰레드에게 sequential consistency 를 제공하여, 각자는
특정 업데이트의 효과를 보거나 보지 못하며, 모든 중간의 업데이트는 읽기 쪽
재시도를 하게 합니다.
Sequential consistency 는 극단적으로 강력한 보장으로, 동일하게 강한 제한과 높은
오버헤드를 수반합니다.
이 경우, 우린 읽기 쓰레드가 starvation 에 빠질 수 있거나 업데이트 쪽 락을
획득해야 할수도 있음을 보았습니다.
업데이트가 흔하지 않은 경우에 이는 잘 동작하지만, 이는 업데이트가 재시도되는
읽기 쓰레드가 액세스하는 데이터에는 여향을 주지도 않은 업데이트에조차 재시도를
불필요하게 강제합니다.
따ㅓ라서 \cref{sec:together:Correlated Fields} 는 읽기 쓰레드의 starvation 을
막을 뿐 아니라 모든 형태의 읽기 쪽 재시도를 막는 완화된 형태의 일관성을
다룹니다.

\iffalse

Another approach is to shard the data elements, and then have each update
write-acquire all the sequence locks needed to cover the data elements
affected by that update.
Of course, these write acquisitions must be done carefully in order to
avoid deadlock.
Readers would also need to read-acquire multiple sequence locks, but
in the surprisingly common case where readers only look up one data
element, only one sequence lock need be read-acquired.

This approach provides sequential consistency to successful readers,
each of which will either see the effects of a given update or not,
with any partial updates resulting in a read-side retry.
Sequential consistency is an extremely strong guarantee, incurring equally
strong restrictions and equally high overheads.
In this case, we saw that readers might be starved on the one hand, or
might need to acquire the update-side lock on the other.
Although this works very well in cases where updates are infrequent,
it unnecessarily forces read-side retries even when the update does not
affect any of the data that a retried reader accesses.
\Cref{sec:together:Correlated Fields} therefore covers a much weaker form
of consistency that not only avoids reader starvation, but also avoids
any form of read-side retry.

\fi

\subsection{Upgrade to Writer}
\label{sec:together:Upgrade to Writer}

\Cref{sec:defer:RCU is a Reader-Writer Lock Replacement} 에서 이야기 된 것과
같이, RCU 는 읽기 쓰레드가 쓰기 쓰레드로 업그레이드 되는 것을 허용합니다.
이 능력은 RCU 로 보호되는 데이터 구조를 읽는 읽기 쓰레드가 현재 원소에
업데이트가 필요함을 알았을 때 상당히 유용합니다.
이걸 시퀀스 락을 가지고 하려면 어떻게 될까요?

이런 시퀀스 락 기법은 리눅스 커널에서 사용되고 있는데, 예를 들어
\path{drivers/infiniband/hw/hfi1/sdma.c} 의 \co{sdma_flush()} 함수에서
그렇습니다.
그 효과는 읽기 쓰레드의 재시도를 막는 겁니다.
따라서 이 기법은 읽기 쓰레드가 재시도를 필요로 하는 어떤 조건을 파악했을 때
사용됩니다.

\iffalse

As discussed in
\cref{sec:defer:RCU is a Reader-Writer Lock Replacement},
RCU permits readers to upgrade to writers.
This capability can be quite useful when a reader scanning an
RCU-protected data structure notices that the current element
needs to be updated.
What happens when you try this trick with sequence locking?

It turns out that this sequence-locking trick is actually used in
the Linux kernel, for example, by the \co{sdma_flush()} function in
\path{drivers/infiniband/hw/hfi1/sdma.c}.
The effect is to doom the enclosing reader to retry.
This trick is therefore used when the reader detects some condition
that requires a retry.

\fi
