% together/hazptr.tex
% mainfile: ../perfbook.tex
% SPDX-License-Identifier: CC-BY-SA-3.0

\section{Hazard-Pointer Helpers}
\label{sec:together:Hazard-Pointer Helpers}
%
\epigraph{It's the little things that count, hundreds of them.}
	 {\emph{Cliff Shaw}}

이 섹션은 해쉬 테이블을 다룰 때 생길 수 있는 문제들을 알아봅니다.
이 문제들은 많은 다른 탐색 구조체들에서도 발생할 수 있음을 유의하시기 바랍니다.

\iffalse

This section looks at some issues that can arise when dealing with
hash tables.
Please note that these issues also apply to many other search structures.

\fi

\subsection{Scalable Reference Count}
\label{sec:together:Scalable Reference Count}

레퍼런스 카운트가 성능이나 확장성의 병목이 된다고 해봅시다.
뭘 할 수 있을까요?

한가지 방법은 해저드 포인터를 대신 사용하는 겁니다.

해저드 포인터에는 몇가지 차이가 있는데, 그중 가장 눈여겨 볼만한 건 언제 연관된
레퍼런스 카운트가 0이 되었는지 확인하는 비용이 굉장히 높다는 겁니다.

이 문제를 해결하는 한가지 방법은 레퍼런스 카운트와 해저드 포인터 사이에 부하를
쪼개는 겁니다.
각 데이터 원소는 이 원소를 참조하는 다른 데이터 원소의 수를 추적하는 동안 읽기
쓰레드는 해저드 포인터를 사용하는 겁니다.

이 방법을 효율적이면서 올바르게 사용하기는 상당한 노력이 필요하며, 따라서 관심
있는 독자 여러분은 Folly 오픈소스 라이브러리에 구현되어 있는 UnboundedQueue 와
ConcurrentHashMap 데이터 구조를 살펴보시기 바랍니다.\footnote{
	\url{https://github.com/facebook/folly}}

\iffalse

Suppose a reference count is becoming a performance or scalability
bottleneck.
What can you do?

One approach is to instead use hazard pointers.

There are some differences, perhaps most notably that with
hazard pointers it is extremely expensive to determine when
the corresponding reference count has reached zero.

One way to work around this problem is to split the load between
reference counters and hazard pointers.
Each data element has a reference counter that tracks the number
of other data elements referencing this element on the one hand,
and readers use hazard pointers on the other.

Making this arrangement work both efficiently and correctly can be
quite challenging, and so interested readers are invited to examine
the UnboundedQueue and ConcurrentHashMap data structures implemented in
Folly open-source library.\footnote{
	\url{https://github.com/facebook/folly}}

\fi

% @@@ papers to maybe cite: OrcGC, ThreadScan, Fast and Robust Memory...

% @@@ Generalized hazard-pointer link counts, if and when.

% @@@ Representative hazard pointer for list, so that nothing
% @@@ in list gets freed until list's hazard pointer is released.
% @@@ Midpoint between hazard pointers and RCU, in fact, you
% @@@ could argue that Tasks Trace RCU with read-side memory
% @@@ barriers is sort of a per-CPU hazard pointers implementing RCU.
% @@@ Except no re-checking because CPUs cannot be freed.
