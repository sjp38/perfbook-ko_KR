% together/refcnt.tex
% mainfile: ../perfbook.tex
% SPDX-License-Identifier: CC-BY-SA-3.0

\section{Refurbish Reference Counting}
\label{sec:together:Refurbish Reference Counting}
%
\epigraph{Counting is the religion of this generation.  It is its
	  hope and its salvation.}
	 {\emph{Gertrude Stein}}

레퍼런스 카운팅이 컨셉상으로는 간단한 기법이지만, 실제로 동시성 소프트웨어에
적용하기에는 많은 어려움이 숨어있습니다.
어쨌건, 어떤 객체가 이른 제거를 당하지 않는다면, 그걸 위한 레퍼런스 카운터를
가질 이유 자체가 없습니다.
하지만 그 객체가 제거될 수 있다면, 레퍼런스 획득 프로세스 자체를 제거하는 건
뭘로 막나요?

동시성 소프트웨어에서의 레퍼런스 카운터 사용을 재고하기 위한 여러 방법들이 있는데, 다음의 것들이 포함됩니다:

\iffalse

Although reference counting is a conceptually simple technique,
many devils hide in the details when it is applied to concurrent
software.
After all, if the object was not subject to premature disposal,
there would be no need for the reference counter in the first place.
But if the object can be disposed of, what prevents disposal during
the reference-acquisition process itself?

There are a number of ways to refurbish reference counters for
use in concurrent software, including:

\fi

\begin{enumerate}
\item	레퍼런스 카운트 조정 중에는 객체의 밖에 있는 락을 잡아야 합니다.
\item	이 객체는 0이 아닌 레퍼런스 카운트를 가지고 생성되며, 새 레퍼런스는 이
	레퍼런스 카운터의 값이 0이 아닐 때에만 획득될 수 있습니다.
	어떤 쓰레드가 특정 객체로의 참조를 가지고 있지 않다면, 참조를 이미
	가지고 있는 다른 쓰레드의 도움이 필요할 수 있습니다.
\item	어떤 경우, 해저드 포인터가 레퍼런스 카운터의 자체 교체를 위해 사용될 수
	있습니다.
\item	객체를 위한 존재 보장 (existance guarantee) 이 제공되어서 어떤 다른
	것이 참조를 얻으려 하는 사이에 이 객체가 메모리 해제되는 것을 막습니다.
	존재 보장은 자동화된 garbage collector 에 의해 종종 제공되며,
	\cref{sec:defer:Hazard Pointers,sec:defer:Read-Copy Update (RCU)}
	에서 보았듯 해저드 포인터와 RCU 에 의해 각각 제공되기도 합니다.
\item	타입 안전성 보장 (type-safety guarantee) 이 객체에 제공됩니다.
	참조가 획득된 후에는 정체성 검사가 반드시 추가로 수행되어야 합니다.
	타입 안전성은 특수 목적 메모리 할당자에 의해 제공될 수 있는데,
	\cref{sec:defer:Read-Copy Update (RCU)} 에서 보인, 리눅스 커널에 있는
	\co{SLAB_TYPESAFE_BY_RCU} 가 한 예가 되겠습니다.

\iffalse

\item	A lock residing outside of the object must be held while
	manipulating the reference count.
\item	The object is created with a non-zero reference count, and new
	references may be acquired only when the current value of
	the reference counter is non-zero.
	If a thread does not have a reference to a given object, it
	might seek help from another thread that already has a reference.
\item	In some cases, hazard pointers may be used as a drop-in
	replacement for reference counters.
\item	An existence guarantee is provided for the object, thus preventing
	it from being freed while some other entity might be attempting
	to acquire a reference.
	Existence guarantees are often provided by automatic
	garbage collectors, and, as is seen in
	\cref{sec:defer:Hazard Pointers,sec:defer:Read-Copy Update (RCU)},
	by hazard pointers and RCU, respectively.
\item	A type-safety guarantee is provided for the object.
	An additional identity check must be performed once
	the reference is acquired.
	Type-safety guarantees can be provided by special-purpose
	memory allocators, for example, by the
	\co{SLAB_TYPESAFE_BY_RCU} feature within the Linux kernel,
	as is seen in \cref{sec:defer:Read-Copy Update (RCU)}.

\fi

\end{enumerate}

물론, 존재 보장을 제공하는 모든 메커니즘은 그 정의에 따라 타입 안전성 보장도
제공합니다.
이는 레퍼런스 획득 보호를 위한 네개의 일반적 카테고리를 형성합니다:
레퍼런스 카운팅, 해저드 포인터, 시퀀스 락킹, 그리고 RCU.

\iffalse

Of course, any mechanism that provides existence guarantees
by definition also provides type-safety guarantees.
This results in four general categories of reference-acquisition
protection:
Reference counting, hazard pointers, sequence locking, and RCU\@.

\fi

\QuickQuiz{
	레퍼런스 카운터가 0이 아닐 때에만 참조를 획득하는 간단한
	compare-and-swap 오퍼레이션을 사용해 구현하는 건 어떤가요?

	\iffalse

	Why not implement reference-acquisition using
	a simple compare-and-swap operation that only
	acquires a reference if the reference counter is
	non-zero?

	\fi

}\QuickQuizAnswer{
	이게 마지막 참조의 해제와 새 참조의 획득 사이의 경주는 해결할 수
	있지만, 데이터 구조가 메모리 해제되고 어쩌면 심지어 어떤 완전 다른
	타입의 구조로 재할당 되는 것을 막기 위한 일은 전혀 하지 않습니다.
	``간단한 compare-and-swap 오퍼레이션'' 이 다른 타입의 구조에 적용되었을
	때에는 정의되지 않은 결과를 낼 수 있을 겁니다.

	요약하자면, compare-and-swap 같은 어토믹 오퍼레이션의 사용은 타입
	안전성 또는 존재 보장을 필요로 합니다.

	하지만 타입 변경을 허용할 필요가 있다면 어떨까요?

	한가지 방법은 그런 타입 각자가 같은 위치에 레퍼런스 카운터를 갖게 해서,
	재할당이 이 타입의 객체 그룹에서 수행된다면 모든게 안전하게 합니다.
	이걸 C 에서 하려면 각 구조의 레퍼런스 카운터를 사용될 때마다 주석을
	다시기 바랍니다.
	C++ 에서는 상속과 템플릿을 사용하세요.

	\iffalse

	Although this can resolve the race between the release of
	the last reference and acquisition of a new reference,
	it does absolutely nothing to prevent the data structure
	from being freed and reallocated, possibly as some completely
	different type of structure.
	It is quite likely that the ``simple compare-and-swap
	operation'' would give undefined results if applied to the
	differently typed structure.

	In short, use of atomic operations such as compare-and-swap
	absolutely requires either type-safety or existence guarantees.

	But what if it is absolutely necessary to let the type change?

	One approach is for each such type to have the reference counter
	at the same location, so that as long as the reallocation results
	in an object from this group of types, all is well.
	If you do this in C, make sure you comment the reference counter
	in each structure in which it appears.
	In C++, use inheritance and templates.

	\fi

}\QuickQuizEnd

\begin{table}[tb]
\renewcommand*{\arraystretch}{1.25}
\rowcolors{3}{}{lightgray}
\small
\centering
\begin{tabular}{lcccc}
	\toprule
	& \multicolumn{4}{c}{Release} \\
	\cmidrule(l){2-5}
	Acquisition	& Locks
				& \parbox[c]{.5in}{Reference\\Counts}
					& \parbox[c]{.5in}{Hazard\\Pointers}
						& RCU \\
	\cmidrule{1-1} \cmidrule(l){2-5}
	Locks		& $-$	& CAM	& M	& CA  \\
	\parbox[c][6ex]{.6in}{Reference\\Counts}
			& A	& AM    & M	& A   \\
	\parbox[c][6ex]{.6in}{Hazard\\Pointers}
			& M	& M	& M	& M   \\
	RCU		& CA	& MCA	& M	& CA  \\
	\bottomrule
\end{tabular}
\caption{Synchronizing Reference Counting}
\label{tab:together:Synchronizing Reference Counting}
\end{table}

핵심 레퍼런스 카운팅 문제는 참조의 획득과 객체의 메모리 해제 사이의
동기화이므로,
\cref{tab:together:Synchronizing Reference Counting}
에 보인 것처럼 아홉개의 가능한 메코니즘 조합이 존재합니다.
이 표는 레퍼런스 카운팅 메커니즘들을 다음의 넓은 카테고리들로 나눕니다:
\begin{enumerate}
\item	어토믹 오퍼레이션, 메모리 배리어, 정렬 제한 무엇도 없는 간단한 카운팅
	(``$-$'').
\item	메모리 배리어 없이 하는 원자적 카운팅 (``A'').
\item	해제 시에만 메모리 배리어를 필요로 하는 원자적 카운팅 (``AM'').
\item	원자적 획득과 결합된 검사를 가지며 해제 시에만 메모리 배리어를 필요로
	하는 원자적 카운팅 (``CAM'').
\item	원자적 획득 오퍼레이션과 검사가 결합된 원자적 카운팅 (``CA'').
\item	전체 메모리 배리어와 검사가 결합된 간단한 카운팅 (``M'').
\item	원자적 획득과 검사가 결합되고 획득 시에 메모리 배리어도 요구되는 원자적
	카운팅 (``MCA'').
\end{enumerate}

\iffalse

Given that the key reference-counting issue
is synchronization between acquisition
of a reference and freeing of the object, we have nine possible
combinations of mechanisms, as shown in
\cref{tab:together:Synchronizing Reference Counting}.
This table
divides reference-counting mechanisms into the following broad categories:
\begin{enumerate}
\item	Simple counting with neither atomic operations, memory
	barriers, nor alignment constraints (``$-$'').
\item	Atomic counting without memory barriers (``A'').
\item	Atomic counting, with memory barriers required only on release
	(``AM'').
\item	Atomic counting with a check combined with the atomic acquisition
	operation, and with memory barriers required only on release
	(``CAM'').
\item	Atomic counting with a check combined with the atomic acquisition
	operation (``CA'').
\item	Simple counting with a check combined with full memory barriers
	(``M'').
\item	Atomic counting with a check combined with the atomic acquisition
	operation, and with memory barriers also required on acquisition
	(``MCA'').
\end{enumerate}

\fi

그러나, 값을 반환하는 모든 리눅스 커널의 원자적 오퍼레이션들은 메모리 배리어를
포함하게 정의되어 있으므로,\footnote{
	\co{atomic()} 와 \co{ATOMIC_INIT()} 는 예외입니다.}
모든 해제 오퍼레이션은 메모리 배리어를 포함하며, 모든 검사되는 획득 오퍼레이션
역시 메모리 배리어를 포함합니다.
따라서, ``CA'' 와 ``MCA'' 는 ``CAM'' 과 동일해서, 첫번째 네개와 여섯번째의
경우들에 대한 것들만 남습니다: ``$-$'', ``A'', ``AM'', ``CAM'', and ``M''.
다음 섹션들은 참조 획득과 해제가 매우 빈번하고 레퍼런스 카운트가 0인지는 매우
드물게 검사될 필요가 있을 때 성능을 개선할 수 있는 최적화 기법들을 소개합니다.

\iffalse

However, because all Linux-kernel atomic operations that return a
value are defined to contain memory barriers,\footnote{
	With \co{atomic_read()} and \co{ATOMIC_INIT()} being the
	exceptions that prove the rule.}
all release operations
contain memory barriers, and all checked acquisition operations also
contain memory barriers.
Therefore, cases ``CA'' and ``MCA'' are equivalent to ``CAM'', so that
there are sections below for only the first four cases and the sixth case:
``$-$'', ``A'', ``AM'', ``CAM'', and ``M''.
Later sections describe optimizations that can improve performance
if reference acquisition and release is very frequent, and the
reference count need be checked for zero only very rarely.

\fi

\subsection{Implementation of Reference-Counting Categories}
\label{sec:together:Implementation of Reference-Counting Categories}

락킹으로 보호되는 간단한 카운팅이 (``$-$'')
\cref{sec:together:Simple Counting} 에 소개되고,
메모리 배리어 없는 원자적 카운팅이 (``A'')
\cref{sec:together:Atomic Counting} 에 설명되며,
획득시 메모리 배리어를 갖는 원자적 카운팅이 (``AM'')
\cref{sec:together:Atomic Counting With Release Memory Barrier} 에,
검사와 해제시 메모리 배리어를 갖는 원자적 카운팅이 (``CAM'')
\cref{sec:together:Atomic Counting With Check and Release Memory Barrier} 에
설명됩니다.
해저드 포인터의 사용은
page~\ref{sec:defer:Hazard Pointers} 의
\cref{sec:defer:Hazard Pointers} 와
\cref{sec:together:Hazard-Pointer Helpers}
에 설명됩니다.

\iffalse

Simple counting protected by locking (``$-$'') is described in
\cref{sec:together:Simple Counting},
atomic counting with no memory barriers (``A'') is described in
\cref{sec:together:Atomic Counting},
atomic counting with acquisition memory barrier (``AM'') is described in
\cref{sec:together:Atomic Counting With Release Memory Barrier},
and
atomic counting with check and release memory barrier (``CAM'') is described in
\cref{sec:together:Atomic Counting With Check and Release Memory Barrier}.
Use of hazard pointers is described in
\cref{sec:defer:Hazard Pointers}
on page~\ref{sec:defer:Hazard Pointers}
and in
\cref{sec:together:Hazard-Pointer Helpers}.

\fi

\subsubsection{Simple Counting}
\label{sec:together:Simple Counting}

어토믹 오퍼레이션도 메모리 배리어도 사용하지 않는 간단한 카운팅은 레퍼런스
카운터 획득과 해제가 둘 다 같은 락으로 보호될 때 사용됩니다.
이 경우, 레퍼런스 카운트 자체는 어토믹하지 않게 조정될 수 있다는게 분명한데,
락이 모든 필요한 배제, 메모리 배리어, 어토믹 오퍼레이션, 그리고 컴파일러 최적화
해제를 제공하기 때문입니다.
이는 레퍼런스 카운터 외에도 다른 오퍼레이션들을 보호하기 위해 락이 필요할 때
선택 가능한 방법입니다만, 락이 해제된 후에 이 객체로의 참조가 있어야 하는
경우는 아닙니다.
\Cref{lst:together:Simple Reference-Count API} 는 간단한 어토믹하지 않은
레퍼런스 카운팅을 구현하는데 사용될 수도 있는 간단한 API 를 보입니다---간단한
레퍼런스 카운팅은 거의 항상 open-code 되지만요.

\iffalse

Simple counting, with neither atomic operations nor memory barriers,
can be used when the reference-counter acquisition and release are
both protected by the same lock.
In this case, it should be clear that the reference count itself
may be manipulated non-atomically, because the lock provides any
necessary exclusion, memory barriers, atomic instructions, and disabling
of compiler optimizations.
This is the method of choice when the lock is required to protect
other operations in addition to the reference count, but where
a reference to the object must be held after the lock is released.
\Cref{lst:together:Simple Reference-Count API} shows a simple
API that might be used to implement simple non-atomic reference
counting---although simple reference counting is almost always
open-coded instead.

\fi

\begin{listing}[tbp]
\begin{fcvlabel}[ln:together:Simple Reference-Count API]
\begin{VerbatimL}[commandchars=\\\[\]]
struct sref {
	int refcount;
};

void sref_init(struct sref *sref)
{
	sref->refcount = 1;
}

void sref_get(struct sref *sref)
{
	sref->refcount++;
}

int sref_put(struct sref *sref,
             void (*release)(struct sref *sref))
{
	WARN_ON(release == NULL);
	WARN_ON(release == (void (*)(struct sref *))kfree);

	if (--sref->refcount == 0) {
		release(sref);
		return 1;
	}
	return 0;
}
\end{VerbatimL}
\end{fcvlabel}
\caption{Simple Reference-Count API}
\label{lst:together:Simple Reference-Count API}
\end{listing}

\subsubsection{Atomic Counting}
\label{sec:together:Atomic Counting}

간단한 원자적 카운팅은 참조를 얻는 모든 CPU 가 이미 참조를 쥐고 있을 때 사용될
수 있습니다.
이 스타일은 하나의 CPU 가 스스로의 사적 사용을 위해 객체를 생성하지만 다른 CPU,
태스크, 타이머 핸들러 등에게 액세스를 허용해야만 할 때 사용됩니다.
이 객체를 넘기는 모든 CPU 는 먼저 그걸 건네받을 쪽을 대신해 새로운 참조를
획득하거나 건네기를 한 후에 액세스를 하기를 삼가야 합니다.
리눅스 커널의 경우, \co{kref} 기능이 이 형태의 레퍼런스 카운팅을 구현하는데
사용되는데,
\cref{lst:together:Linux Kernel kref API} 에 보여진 바와 같습니다.\footnote{
	리눅스 v4.10 기준입니다.
	리눅스 v4.11 은 완화된 순서 규칙 플랫폼에서의 효율성을 개선한
	\co{refcount_t} API 를 도입했는데, 기능적으로는 대체된 \co{atomic_t} 와
	동일합니다.}

\iffalse

Simple atomic counting may be used in cases where any CPU acquiring
a reference must already hold a reference.
This style is used when a single CPU creates an object for its own private
use, but must allow for accesses from other CPUs, tasks, timer handlers,
and so on.
Any CPU that hands the object off must first acquire a new reference on
behalf of the recipient on the one hand, or refrain from further accesses
after the handoff on the other.
In the Linux kernel, the \co{kref} primitives are used to implement
this style of reference counting, as shown in
\cref{lst:together:Linux Kernel kref API}.\footnote{
	As of Linux v4.10.
	Linux v4.11 introduced a \co{refcount_t} API that improves
	efficiency weakly ordered platforms, but which is functionally
	equivalent to the \co{atomic_t} that it replaced.}

\fi

이 경우에는 락킹이 모든 레퍼런스 카운트 오퍼레이션을 보호하지 않으며, 이말은
두개의 다른 CPU 가 동시에 레퍼런스 카운트를 조정할 수도 있음을 의미하기 때문에
어토믹 카운팅이 필요합니다.
평범한 값 증가/감소가 사용되었다면, 한쌍의 CPU 는 둘 다 동시에 레퍼런스
카운트를 읽어와서 둘 다 값 ``3'' 을 얻을 수 있습니다.
둘 모두 각자의 값을 증가시키면 둘 다 ``4'' 를 얻게 되고, 둘 다 이 값을 카운터에
다시 저장하게 됩니다.
이 카운터의 새로운 값은 ``5'' 여야 하므로, 값 증가 하나가 사라졌습니다.
따라서, 어토믹 오퍼레이션은 카운터 증가와 감소 둘 다에 사용되어야만 합니다.

해제가 락킹, 해저드 포인터, 또는 RCU 로 보호된다면, 메모리 배리어는 필요하지
\emph{않지만}, 다른 이유 때문입니다.
락킹의 경우, 락은 모든 필요한 메모리 배리어를 제공하고 (그리고 컴파일러
최적화를 해제합니다), 또한 한쌍의 해제가 동시에 수행되는 것을 방지합니다.
해저드 포인터와 RCU 의 경우, 정리는 뒤로 미뤄지며, 모든 필요한 메모리 배리어나
컴파일러 최적화 해제는 해저드 퐁니터나 RCU 에 의해 제고오딥니다.
따라서, 두개의 CPU 가 마지막 두개의 참조를 동시에 해제하려 하면, 실제 정리는 두
CPU 모두 각자의 해저드 포인터를 해제하거나 RCU read-side 크리티컬 섹션을 끝낼
때까지 미뤄질 겁니다.

\iffalse

Atomic counting is required in this case because locking does not protect
all reference-count operations, which means that two different CPUs
might concurrently manipulate the reference count.
If normal increment and decrement were used, a pair of CPUs might both
fetch the reference count concurrently, perhaps both obtaining
the value ``3''.
If both of them increment their value, they will both obtain ``4'',
and both will store this value back into the counter.
Since the new value of the counter should instead be ``5'', one
of the increments has been lost.
Therefore, atomic operations must be used both for counter increments
and for counter decrements.

If releases are guarded by locking, hazard pointers, or RCU,
memory barriers are \emph{not} required, but for different reasons.
In the case of locking, the locks provide any needed memory barriers
(and disabling of compiler optimizations), and the locks also
prevent a pair of releases from running concurrently.
In the case of hazard pointers and RCU, cleanup will be deferred,
and any needed memory barriers or disabling of compiler optimizations
will be provided by the hazard-pointers or RCU infrastructure.
Therefore, if two CPUs release the final two references concurrently, the
actual cleanup will be deferred until both CPUs have released their hazard
pointers or exited their RCU read-side critical sections, respectively.

\fi

\QuickQuiz{
	한 CPU 가 마지막 참조를 해제한 직후에 다른 CPU 가 참조를 얻는 경우를
	위한 보호는 왜 필요하지 않죠?

	\iffalse

	Why isn't it necessary to guard against cases where one CPU
	acquires a reference just after another CPU releases the last
	reference?

	\fi

}\QuickQuizAnswer{
	합법적으로 또다른 참조를 얻기 위해서는 그 CPU 는 이미 참조를 가지고
	있어야 하니까요.
	따라서, 만약 어느 CPU 가 마지막 참조를 해제했다면, 새 참조를 얻을 수
	있는 CPU 는 존재하지 않습니다!

	\iffalse

	Because a CPU must already hold a reference in order to legally
	acquire another reference.
	Therefore, if one CPU releases the last reference, there had
	better not be any CPU acquiring a new reference!

	\fi

}\QuickQuizEnd

\begin{listing}[tbp]
\begin{fcvlabel}[ln:together:Linux Kernel kref API]
\begin{VerbatimL}[commandchars=\\\[\]]
struct kref {						\lnlbl[kref:b]
	atomic_t refcount;
};							\lnlbl[kref:e]

void kref_init(struct kref *kref)			\lnlbl[init:b]
{
	atomic_set(&kref->refcount, 1);
}							\lnlbl[init:e]

void kref_get(struct kref *kref)			\lnlbl[get:b]
{
	WARN_ON(!atomic_read(&kref->refcount));
	atomic_inc(&kref->refcount);
}							\lnlbl[get:e]

static inline int					\lnlbl[sub:b]
kref_sub(struct kref *kref, unsigned int count,
         void (*release)(struct kref *kref))
{
	WARN_ON(release == NULL);

	if (atomic_sub_and_test((int) count,		\lnlbl[check]
	                        &kref->refcount)) {
		release(kref);				\lnlbl[rel]
		return 1;				\lnlbl[ret:1]
	}
	return 0;
}							\lnlbl[sub:e]
\end{VerbatimL}
\end{fcvlabel}
\caption{Linux Kernel \tco{kref} API}
\label{lst:together:Linux Kernel kref API}
\end{listing}

\begin{fcvref}[ln:together:Linux Kernel kref API]
하나의 어토믹 데이터 항목으로 구성되는 \co{kref} 구조체가
\cref{lst:together:Linux Kernel kref API} 의
\clnrefrange{kref:b}{kref:e} 에 보여져 있습니다.
\Clnrefrange{init:b}{init:e} 의 \co{kref_init()} 함수는 카운터를 값 ``1'' 로
초기화 합니다.
\co{atomic_set()} 기능은 단순한 값 할당으로, 그 이름은 \co{atomic_t} 데이터
타입에서 기인했지 그 동작에서 기인한게 아님을 알아두시기 바랍니다.
\co{kref_init()} 함수는 객체 생성 중에 그 객체가 다른 CPU 에게 사용가능해지기
전에 호출되어야만 합니다.

\Clnrefrange{get:b}{get:e} 의 \co{kref_get()} 함수는 무조건적으로 이 카운터를
원자적으로 값 증가시킵니다.
\co{atomic_inc()} 기능은 명시적으로 컴파일러 최적화를 모든 플랫폼에서 해제할
필요는 없지만 \co{kref} 기능이 다른 모듈에 있고 리눅스 커널 빌드 프로세스는
모듈간 최적화를 하지 않는다는 사실 때문에 사실상 같은 효과가 있습니다.

\iffalse

\begin{fcvref}[ln:together:Linux Kernel kref API]
The \co{kref} structure itself, consisting of a single atomic
data item, is shown in \clnrefrange{kref:b}{kref:e} of
\cref{lst:together:Linux Kernel kref API}.
The \co{kref_init()} function on \clnrefrange{init:b}{init:e}
initializes the counter to the value ``1''.
Note that the \co{atomic_set()} primitive is a simple
assignment, the name stems from the data type of \co{atomic_t}
rather than from the operation.
The \co{kref_init()} function must be invoked during object creation,
before the object has been made available to any other CPU\@.

The \co{kref_get()} function on \clnrefrange{get:b}{get:e}
unconditionally atomically increments the counter.
The \co{atomic_inc()} primitive does not necessarily explicitly
disable compiler
optimizations on all platforms, but the fact that the \co{kref}
primitives are in a separate module and that the Linux kernel build
process does no cross-module optimizations has the same effect.

\fi

\Clnrefrange{sub:b}{sub:e} 의 \co{kref_sub()} 함수는 원자적으로 카운터의 값을
감소시키며, 그 결과가 0이면 \clnref{rel} 은 명시된 \co{release()} 함수를
호출하고 \clnref{ret:1} 에서 리턴하여, 호출자에게 \co{release()} 가
호출되었음을 알립니다.
그렇지 않으면 \co{kref_sub()} 은 0을 리턴하여 호출자에게 \co{release()} 가
호출되지 않았음을 알립니다.
\end{fcvref}

\iffalse

The \co{kref_sub()} function on \clnrefrange{sub:b}{sub:e}
atomically decrements the
counter, and if the result is zero, \clnref{rel} invokes the specified
\co{release()} function and \clnref{ret:1} returns, informing the caller
that \co{release()} was invoked.
Otherwise, \co{kref_sub()} returns zero, informing the caller that
\co{release()} was not called.
\end{fcvref}

\fi

\QuickQuizSeries{%
\QuickQuizB{
	\begin{fcvref}[ln:together:Linux Kernel kref API]
	\Cref{lst:together:Linux Kernel kref API}
	의 \clnref{check} 에서 \co{atomic_sub_and_test()} 가 수행된 직후에,
	어떤 다른 CPU 가 \co{kref_get()} 을 호출했다고 해봅시다.
	다른 CPU 는 이제 해제된 객체로의 불법적인 참조를 갖게 되지 않나요?
	\end{fcvref}

	\iffalse

	\begin{fcvref}[ln:together:Linux Kernel kref API]
	Suppose that just after the \co{atomic_sub_and_test()}
	on \clnref{check} of
	\cref{lst:together:Linux Kernel kref API} is invoked,
	that some other CPU invokes \co{kref_get()}.
	Doesn't this result in that other CPU now having an illegal
	reference to a released object?
        \end{fcvref}

	\fi

}\QuickQuizAnswerB{
	이 함수들이 올바르게 사용되면 그런 일은 일어날 수 없습니다.
	여러분이 이미 참조를 가지고 있지 않다면 \co{kref_get()} 을 호출하는건
	불법인데, \co{kref_sub()} 이 카운터를 0으로 감소시키지 못할 겁니다.

	\iffalse

	This cannot happen if these functions are used correctly.
	It is illegal to invoke \co{kref_get()} unless you already
	hold a reference, in which case the \co{kref_sub()} could
	not possibly have decremented the counter to zero.

	\fi

}\QuickQuizEndB
%
\QuickQuizM{
	\co{kref_sub()} 이 0을 리턴해서 \co{release()} 함수가 호출되지 않았음을
	알렸다고 해봅시다.
	호출자는 어떤 조건에서 객체의 지속되는 존재를 믿을 수 있나요?

	\iffalse

	Suppose that \co{kref_sub()} returns zero, indicating that
	the \co{release()} function was not invoked.
	Under what conditions can the caller rely on the continued
	existence of the enclosing object?

	\fi

}\QuickQuizAnswerM{
	호출자는 최소 하나의 참조가 계속 존재할 걸 아는게 아니라면 지속되는
	존재를 믿을 수 없습니다.
	일반적으로, 호출자는 이를 알 방법이 없으며, 따라서 \co{kref_sub()} 호출
	후에 객체로의 참조를 하는 것을 주의깊게 막아야만 합니다.

	관심 있는 독자 여러분은 이 제한을 RCU, 특히 \co{call_rcu()} 를 사용해
	해결해 보시기 바랍니다.

	\iffalse

	The caller cannot rely on the continued existence of the
	object unless it knows that at least one reference will
	continue to exist.
	Normally, the caller will have no way of knowing this, and
	must therefore carefully avoid referencing the object after
	the call to \co{kref_sub()}.

	Interested readers are encouraged to work around this limitation
	using RCU, in particular, \co{call_rcu()}.

	\fi

}\QuickQuizEndM
%
\QuickQuizE{
	간단하게 해제 함수로 \co{kfree()} 를 넘기지 않는 이유가 뭐죠?

	\iffalse

	Why not just pass \co{kfree()} as the release function?

	\fi

}\QuickQuizAnswerE{
	일반적으로 \co{kref} 구조체는 더 큰 구조체에 내재되므로, 그리고
	\co{kref} 필드만이 아니라 전체 구조체를 메모리 해제해야 하기
	때문입니다.
	이는 일반적으로 \co{container_of()} 에 이어 \co{kfree()} 를 호출하는
	wrapper 함수로 정의됩니다.

	\iffalse

	Because the \co{kref} structure normally is embedded in
	a larger structure, and it is necessary to free the entire
	structure, not just the \co{kref} field.
	This is normally accomplished by defining a wrapper function
	that does a \co{container_of()} and then a \co{kfree()}.

	\fi

}\QuickQuizEndE
}

\subsubsection{Atomic Counting With Release Memory Barrier}
\label{sec:together:Atomic Counting With Release Memory Barrier}

Atomic reference counting with release memory barriers is used by the
Linux kernel's networking layer to track the destination caches that
are used in packet routing.
The actual implementation is quite a bit more involved; this section
focuses on the aspects of \co{struct dst_entry} reference-count
handling that matches this use case,
shown in \cref{lst:together:Linux Kernel dst-clone API}.\footnote{
	As of Linux v4.13.
	Linux v4.14 added a level of indirection to permit more
	comprehensive debugging checks, but the overall effect in the
	absence of bugs is identical.}

\begin{listing}[tbp]
\begin{fcvlabel}[ln:together:Linux Kernel dst-clone API]
\begin{VerbatimL}[commandchars=\\\[\]]
static inline
struct dst_entry * dst_clone(struct dst_entry * dst)
{
	if (dst)
		atomic_inc(&dst->__refcnt);
	return dst;
}

static inline
void dst_release(struct dst_entry * dst)
{
	if (dst) {
		WARN_ON(atomic_read(&dst->__refcnt) < 1);
		smp_mb__before_atomic_dec();		\lnlbl[mb]
		atomic_dec(&dst->__refcnt);
	}
}
\end{VerbatimL}
\end{fcvlabel}
\caption{Linux Kernel \tco{dst_clone} API}
\label{lst:together:Linux Kernel dst-clone API}
\end{listing}

The \co{dst_clone()} primitive may be used if the caller
already has a reference to the specified \co{dst_entry},
in which case it obtains another reference that may be handed off
to some other entity within the kernel.
Because a reference is already held by the caller, \co{dst_clone()}
need not execute any memory barriers.
The act of handing the \co{dst_entry} to some other entity might
or might not require a memory barrier, but if such a memory barrier
is required, it will be embedded in the mechanism used to hand the
\co{dst_entry} off.

\begin{fcvref}[ln:together:Linux Kernel dst-clone API]
The \co{dst_release()} primitive may be invoked from any environment,
and the caller might well reference elements of the \co{dst_entry}
structure immediately prior to the call to \co{dst_release()}.
The \co{dst_release()} primitive therefore contains a memory
barrier on \clnref{mb} preventing both the compiler and the CPU
from misordering accesses.
\end{fcvref}

Please note that the programmer making use of \co{dst_clone()} and
\co{dst_release()} need not be aware of the memory barriers, only
of the rules for using these two primitives.

\subsubsection{Atomic Counting With Check and Release Memory Barrier}
\label{sec:together:Atomic Counting With Check and Release Memory Barrier}

Consider a situation where the caller must be able to acquire a new
reference to an object to which it does not already hold a reference,
but where that object's existence is guaranteed.
The fact that initial reference-count acquisition can now run concurrently
with reference-count release adds further complications.
Suppose that a reference-count release finds that the new
value of the reference count is zero, signaling that it is
now safe to clean up the reference-counted object.
We clearly cannot allow a reference-count acquisition to
start after such clean-up has commenced, so the acquisition
must include a check for a zero reference count.
This check must be part of the atomic increment operation,
as shown below.

\QuickQuiz{
	Why can't the check for a zero reference count be
	made in a simple ``if'' statement with an atomic
	increment in its ``then'' clause?
}\QuickQuizAnswer{
	Suppose that the ``if'' condition completed, finding
	the reference counter value equal to one.
	Suppose that a release operation executes, decrementing
	the reference counter to zero and therefore starting
	cleanup operations.
	But now the ``then'' clause can increment the counter
	back to a value of one, allowing the object to be
	used after it has been cleaned up.

	This use-after-cleanup bug is every bit as bad as a
	full-fledged use-after-free bug.
}\QuickQuizEnd

The Linux kernel's \co{fget()} and \co{fput()} primitives
use this style of reference counting.
Simplified versions of these functions are shown in
\cref{lst:together:Linux Kernel fget/fput API}.\footnote{
	As of Linux v2.6.38.
	Additional \co{O_PATH} functionality was added in v2.6.39,
	refactoring was applied in v3.14, and \co{mmap_sem} contention
	was reduced in v4.1.}

\begin{listing}[tbp]
\begin{fcvlabel}[ln:together:Linux Kernel fget/fput API]
\begin{VerbatimL}[commandchars=\\\@\$]
struct file *fget(unsigned int fd)
{
	struct file *file;
	struct files_struct *files = current->files;	\lnlbl@fetch$

	rcu_read_lock();				\lnlbl@rrl$
	file = fcheck_files(files, fd);			\lnlbl@lookup$
	if (file) {
		if (!atomic_inc_not_zero(&file->f_count)) { \lnlbl@acq$
			rcu_read_unlock();		\lnlbl@rru1$
			return NULL;			\lnlbl@ret:null$
		}
	}
	rcu_read_unlock();				\lnlbl@rru2$
	return file;					\lnlbl@ret:file$
}

struct file *
fcheck_files(struct files_struct *files, unsigned int fd)
{
	struct file * file = NULL;
	struct fdtable *fdt = rcu_dereference((files)->fdt);  \lnlbl@deref:fdt$

	if (fd < fdt->max_fds)				\lnlbl@range$
		file = rcu_dereference(fdt->fd[fd]);	\lnlbl@deref:fd$
	return file;					\lnlbl@ret:file2$
}

void fput(struct file *file)
{
	if (atomic_dec_and_test(&file->f_count))	\lnlbl@dec$
		call_rcu(&file->f_u.fu_rcuhead, file_free_rcu);  \lnlbl@call$
}

static void file_free_rcu(struct rcu_head *head)
{
	struct file *f;

	f = container_of(head, struct file, f_u.fu_rcuhead); \lnlbl@obtain$
	kmem_cache_free(filp_cachep, f);		\lnlbl@free$
}
\end{VerbatimL}
\end{fcvlabel}
\caption{Linux Kernel \tco{fget}/\tco{fput} API}
\label{lst:together:Linux Kernel fget/fput API}
\end{listing}

\begin{fcvref}[ln:together:Linux Kernel fget/fput API]
\Clnref{fetch} of \co{fget()} fetches the pointer to the current
process's file-descriptor table, which might well be shared
with other processes.
\Clnref{rrl} invokes \co{rcu_read_lock()}, which
enters an RCU read-side critical section.
The callback function from any subsequent \co{call_rcu()} primitive
will be deferred until a matching \co{rcu_read_unlock()} is reached
(\clnref{rru1} or~\lnref{rru2} in this example).
\Clnref{lookup} looks up the file structure corresponding to the file
descriptor specified by the \co{fd} argument, as will be
described later.
If there is an open file corresponding to the specified file descriptor,
then \clnref{acq} attempts to atomically acquire a reference count.
If it fails to do so, \clnrefrange{rru1}{ret:null} exit the RCU read-side critical
section and report failure.
Otherwise, if the attempt is successful, \clnrefrange{rru2}{ret:file}
exit the read-side
critical section and return a pointer to the file structure.

The \co{fcheck_files()} primitive is a helper function for
\co{fget()}.
\Clnref{deref:fdt} uses \co{rcu_dereference()} to safely fetch an
RCU-protected pointer to this task's current file-descriptor table,
and \clnref{range} checks to see if the specified file descriptor is in range.
If so, \clnref{deref:fd} fetches the pointer to the file structure, again using
the \co{rcu_dereference()} primitive.
\Clnref{ret:file2} then returns a pointer to the file structure or \co{NULL}
in case of failure.

The \co{fput()} primitive releases a reference to a file structure.
\Clnref{dec} atomically decrements the reference count, and, if the
result was zero, \clnref{call} invokes the \co{call_rcu()} primitives
in order to free up the file structure (via the \co{file_free_rcu()}
function specified in \co{call_rcu()}'s second argument), but only after
all currently-executing RCU read-side critical sections complete, that
is, after an RCU grace period has elapsed.

Once the grace period completes, the \co{file_free_rcu()} function
obtains a pointer to the file structure on \clnref{obtain}, and frees it
on \clnref{free}.
\end{fcvref}

This code fragment thus demonstrates how RCU can be used to guarantee
existence while an in-object reference count is being incremented.

\subsection{Counter Optimizations}
\label{sec:together:Counter Optimizations}

In some cases where increments and decrements are common, but checks
for zero are rare, it makes sense to maintain per-CPU or per-task
counters, as was discussed in \cref{chp:Counting}.
For example, see the paper on sleepable read-copy update (SRCU), which
applies this technique to RCU~\cite{PaulEMcKenney2006c}.
This approach eliminates the need for atomic instructions or memory
barriers on the increment and decrement primitives, but still requires
that code-motion compiler optimizations be disabled.
In addition, the primitives such as \co{synchronize_srcu()}
that check for the aggregate reference
count reaching zero can be quite slow.
This underscores the fact that these techniques are designed
for situations where the references are frequently acquired and
released, but where it is rarely necessary to check for a zero
reference count.

% @@@ Difficulty of managing reference counts: leaks, premature freeing.

However, it is usually the case that use of reference counts requires
writing (often atomically) to a data structure that is otherwise
read only.
In this case, reference counts are imposing expensive cache misses
on readers.

It is therefore worthwhile to look into synchronization mechanisms
that do not require readers to write to the data structure being
traversed.
One possibility is the hazard pointers covered in
\cref{sec:defer:Hazard Pointers}
and another is RCU, which is covered in
\cref{sec:defer:Read-Copy Update (RCU)}.
