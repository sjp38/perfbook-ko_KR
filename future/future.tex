% future/future.tex
% SPDX-License-Identifier: CC-BY-SA-3.0

\QuickQuizChapter{chp:Conflicting Visions of the Future}{Conflicting Visions of the Future}
%
\Epigraph{Prediction is very difficult, especially about the future.}
	 {\emph{Niels Bohr}}

이 챕터는 병렬 프로그래밍의 미래에 대한 일부 서로 다른 전망들을 소개합니다.
이 전망들 가운데 어느것이 현실이 될지, 사실 이것들 중 하나라도 현실이 될지는
명확치가 않습니다.
그러나 이것들은 각 전망을 좋아하는 지지자들이 있기 때문에 중요하며, 충분히 많은
사람들이 무언가를 열렬하게 믿는다면, 여러분은 그 지지자들의 생각, 말, 그리고
행위에서의 영향의 형태를 통한 그 존재함의 그림자만이라도 다루게 될 겁니다.
그와는 별개로, 이 전망들 중 두개 이상의 것들이 실제 현실이 될 수도 있습니다.
하지만 대부분은 가짜입니다.
뭐가 진짜고 뭐가 가짜일지 말할 수 있다면 여러분은 부자가
될겁니다~\cite{KeithRSpitz1977}!

따라서, 뒤의 섹션들은 트랜잭셔널 메모리, 하드웨어 트랜잭셔널 메모리, 그리고
병렬 함수형 프로그래밍에 대한 개괄을 제공합니다.
하지만 먼저, 예언에 대한 경고적인 이야기를 2000년대 초로부터 가져와 봅니다.
\iffalse

This chapter presents some conflicting visions of the future of parallel
programming.
It is not clear which of these will come to pass, in fact, it is not
clear that any of them will.
They are nevertheless important because each vision has its devoted
adherents, and if enough people believe in something fervently enough,
you will need to deal with at least the shadow of that thing's existence
in the form of its
influence on the thoughts, words, and deeds of its adherents.
Besides which, it is entirely possible that one or more of these visions
will actually come to pass.
But most are bogus.
Tell which is which and you'll be rich~\cite{KeithRSpitz1977}!

Therefore, the following sections give an overview of transactional
memory, hardware transactional memory,
formal verification in regression testing, and
parallel functional programming.
But first, a cautionary tale on prognostication taken from the early 2000s.
\fi

% future/cpu.tex
% SPDX-License-Identifier: CC-BY-SA-3.0

\section{The Future of CPU Technology Ain't What it Used to Be}
\label{sec:future:The Future of CPU Technology Ain't What it Used to Be}

과거란 많은 기간의 경험을 거친 렌즈를 통해 보기에는 항상 매우 간단하고 순수해
보입니다.
그리고 2000년대 초는 Moore's Law 가 그땐 전통적이었던 CPU 클락 주파수의 증가를
가져오던 현상이 깨지기 시작하는 시점에 임박했던, 순수했던 시대였습니다.
아, 그때도 기술의 한계에 대한 가끔의 경고는 있었습니다만 그런 경고들은 수십년째
이어져 오고 있었습니다.
그걸 마음에 둔 채로, 다음의 시나리오들을 고려해 봅시다:
\iffalse

Years past always seem so simple and innocent when viewed through the
lens of many years of experience.
And the early 2000s were for the most part innocent of the impending
failure of Moore's Law to continue delivering the then-traditional
increases in CPU clock frequency.
Oh, there were the occasional warnings about the limits of technology,
but such warnings had been sounded for decades.
With that in mind, consider the following scenarios:
\fi

\begin{figure}[tb]
\centering
\resizebox{3in}{!}{\includegraphics{cartoons/r-2014-CPU-future-uniprocessor-uber-alles}}
\caption{Uniprocessor \"Uber Alles}
\ContributedBy{Figure}{fig:future:Uniprocessor Uber Alles}{Melissa Broussard}
\end{figure}

\begin{figure}[tb]
\centering
\resizebox{3in}{!}{\includegraphics{cartoons/r-2014-CPU-Future-Multithreaded-Mania}}
\caption{Multithreaded Mania}
\ContributedBy{Figure}{fig:future:Multithreaded Mania}{Melissa Broussard}
\end{figure}

\begin{figure}[tb]
\centering
\resizebox{3in}{!}{\includegraphics{cartoons/r-2014-CPU-Future-More-of-the-Same}}
\caption{More of the Same}
\ContributedBy{Figure}{fig:future:More of the Same}{Melissa Broussard}
\end{figure}

\begin{figure}[tb]
\centering
\resizebox{3in}{!}{\includegraphics{cartoons/r-2014-CPU-Future-Crash-dummies}}
\caption{Crash Dummies Slamming into the Memory Wall}
\ContributedBy{Figure}{fig:future:Crash Dummies Slamming into the Memory Wall}{Melissa Broussard}
\end{figure}

\begin{enumerate}
\item	유니프로세서 \"Uber Alles
	(Figure~\ref{fig:future:Uniprocessor Uber Alles}),
\item	멀티쓰레드 매니아
	(Figure~\ref{fig:future:Multithreaded Mania}),
\item	더 많은 같은것들
	(Figure~\ref{fig:future:More of the Same}), 그리고
\item	메모리 장벽에 부닺치는 것들
	(Figure~\ref{fig:future:Crash Dummies Slamming into the Memory Wall}).
\iffalse

\item	Uniprocessor \"Uber Alles
	(Figure~\ref{fig:future:Uniprocessor Uber Alles}),
\item	Multithreaded Mania
	(Figure~\ref{fig:future:Multithreaded Mania}),
\item	More of the Same
	(Figure~\ref{fig:future:More of the Same}), and
\item	Crash Dummies Slamming into the Memory Wall
	(Figure~\ref{fig:future:Crash Dummies Slamming into the Memory Wall}).
\fi
\end{enumerate}

다음의 섹션들은 이 시나리오들 각각을 다룹니다.
\iffalse

Each of these scenarios are covered in the following sections.
\fi

\subsection{Uniprocessor \"Uber Alles}
\label{sec:future:Uniprocessor Uber Alles}

2004년에 이야기한 것~\cite{PaulEdwardMcKenneyPhD} 처럼:
\iffalse

As was said in 2004~\cite{PaulEdwardMcKenneyPhD}:
\fi

\begin{quote}
	이 시나리오에서, Moore's-Law 를 통한 CPU 클락 속도의 증가와 수평적으로
	확장되는 컴퓨팅의 계속된 발전의 조합은 SMP 시스템들을 별것 아니게
	만듭니다.
	따라서 이 시나리오는 ``Uniprocessor \"Uber Alles'', 말 그대로 다른
	모든것보다 나은 유니프로세서라고 불립니다.

	이런 유니프로세서 시스템들은 인스트럭션 오버헤드만이 문제가 될텐데,
	메모리 배리어, cache thrashing, 그리고 cache contention 은 단일 CPU
	시스템에서는 문제가 없기 때문입니다.
	이 시나리오 상에서, RCU 는 NMI 들과의 상호작용과 같은 간단한 부분에서만
	유용할 것입니다.
	이미 RCU 를 구현한 운영체제는 그대로 RCU 를 가지고 있어도 되겠지만, RCU
	가 존재하지 않는 운영 체제가 RCU 를 적용해야 할지는 분명치 않습니다.

	하지만, 최근의 멀티쓰레드 사용 CPU 의 발전은 이 시나리오가 이뤄질
	가능성은 적다고 이야기 합니다.
	\iffalse

	In this scenario, the combination of Moore's-Law increases in CPU
	clock rate and continued progress in horizontally scaled computing
	render SMP systems irrelevant.
	This scenario is therefore dubbed ``Uniprocessor \"Uber
	Alles'', literally, uniprocessors above all else.

	These uniprocessor systems would be subject only to instruction
	overhead, since memory barriers, cache thrashing, and contention
	do not affect single-CPU systems.
	In this scenario, RCU is useful only for niche applications, such
	as interacting with NMIs.
	It is not clear that an operating system lacking RCU would see
	the need to adopt it, although operating
	systems that already implement RCU might continue to do so.

	However, recent progress with multithreaded CPUs seems to indicate
	that this scenario is quite unlikely.
	\fi
\end{quote}

실제로 그렇게 되진 않을 겁니다!
하지만 더 커다란 소프트웨어 커뮤니티는 그들이 병렬성을 포용해야 한다는 사실을
받아들이기를 주저했으며, 따라서 이는 이 커뮤니티가 Moore's-Law 로 인한 CPU 코어
클락 주파수 상승의 ``공짜 점심'' 이 정말로 끝났다는 결론을 내리기 전이었습니다.
잊지 마세요: 믿음은 감정이지, 이성적이고 기술적인 생각 과정의 결과가 아닐 수
있습니다!
\iffalse

Unlikely indeed!
But the larger software community was reluctant to accept the fact that
they would need to embrace parallelism, and so it was some time before
this community concluded that the ``free lunch'' of Moore's-Law-induced
CPU core-clock frequency increases was well and truly finished.
Never forget: belief is an emotion, not necessarily the result of a
rational technical thought process!
\fi

\subsection{Multithreaded Mania}
\label{sec:future:Multithreaded Mania}

Also from 2004~\cite{PaulEdwardMcKenneyPhD}:

\begin{quote}
	A less-extreme variant of Uniprocessor \"Uber Alles features
	uniprocessors with hardware multithreading, and in fact
	multithreaded CPUs are now standard for many desktop and laptop
	computer systems.  The most aggressively multithreaded CPUs share
	all levels of cache hierarchy, thereby eliminating CPU-to-CPU
	memory latency, in turn greatly reducing the performance
	penalty for traditional synchronization mechanisms.  However,
	a multithreaded CPU would still incur overhead due to contention
	and to pipeline stalls caused by memory barriers.  Furthermore,
	because all hardware threads share all levels of cache, the
	cache available to a given hardware thread is a fraction of
	what it would be on an equivalent single-threaded CPU, which can
	degrade performance for applications with large cache footprints.
	There is also some possibility that the restricted amount of cache
	available will cause RCU-based algorithms to incur performance
	penalties due to their grace-period-induced additional memory
	consumption.  Investigating this possibility is future work.

	However, in order to avoid such performance degradation, a number
	of multithreaded CPUs and multi-CPU chips partition at least
	some of the levels of cache on a per-hardware-thread basis.
	This increases the amount of cache available to each hardware
	thread, but re-introduces memory latency for cachelines that
	are passed from one hardware thread to another.
\end{quote}

And we all know how this story has played out, with multiple multi-threaded
cores on a single die plugged into a single socket.
The question then becomes whether or not future shared-memory systems will
always fit into a single socket.

\subsection{More of the Same}
\label{sec:meas:More of the Same}

Again from 2004~\cite{PaulEdwardMcKenneyPhD}:

\begin{quote}
	The More-of-the-Same scenario assumes that the memory-latency
	ratios will remain roughly where they are today.

	This scenario actually represents a change, since to have more
	of the same, interconnect performance must begin keeping up
	with the Moore's-Law increases in core CPU performance.  In this
	scenario, overhead due to pipeline stalls, memory latency, and
	contention remains significant, and RCU retains the high level
	of applicability that it enjoys today.
\end{quote}

And the change has been the ever-increasing levels of integration
that Moore's Law is still providing.
But longer term, which will it be?
More CPUs per die?
Or more I/O, cache, and memory?

Servers seem to be choosing the former, while embedded systems on a chip
(SoCs) continue choosing the latter.

\subsection{Crash Dummies Slamming into the Memory Wall}
\label{sec:future:Crash Dummies Slamming into the Memory Wall}

\begin{figure}[tbp]
\centering
\epsfxsize=3in
\epsfbox{future/latencytrend}
% from Ph.D. thesis: related/latencytrend.eps
\caption{Instructions per Local Memory Reference for Sequent Computers}
\label{fig:future:Instructions per Local Memory Reference for Sequent Computers}
\end{figure}

\begin{figure}[htbp]
\centering
\epsfxsize=3in
\epsfbox{future/be-lb-n4-rf-all}
% from Ph.D. thesis: an/plots/be-lb-n4-rf-all.eps
\caption{Breakevens vs. $r$, $\lambda$ Large, Four CPUs}
\label{fig:future:Breakevens vs. r, lambda Large, Four CPUs}
\end{figure}

\begin{figure}[htbp]
\centering
\epsfxsize=3in
\epsfbox{future/be-lw-n4-rf-all}
% from Ph.D. thesis: an/plots/be-lw-n4-rf-all.eps
\caption{Breakevens vs. $r$, $\lambda$ Small, Four CPUs}
\label{fig:future:Breakevens vs. r, Worst-Case lambda, Four CPUs}
\end{figure}

And one more quote from 2004~\cite{PaulEdwardMcKenneyPhD}:

\begin{quote}
	If the memory-latency trends shown in
	Figure~\ref{fig:future:Instructions per Local Memory Reference for Sequent Computers}
	continue, then memory latency will continue to grow relative
	to instruction-execution overhead.
	Systems such as Linux that have significant use of RCU will find
	additional use of RCU to be profitable, as shown in
	Figure~\ref{fig:future:Breakevens vs. r, lambda Large, Four CPUs}
	As can be seen in this figure, if RCU is heavily used, increasing
	memory-latency ratios give RCU an increasing advantage over other
	synchronization mechanisms.
	In contrast, systems with minor
	use of RCU will require increasingly high degrees of read intensity
	for use of RCU to pay off, as shown in
	Figure~\ref{fig:future:Breakevens vs. r, Worst-Case lambda, Four CPUs}.
	As can be seen in this figure, if RCU is lightly used,
	increasing memory-latency ratios
	put RCU at an increasing disadvantage compared to other synchronization
	mechanisms.
	Since Linux has been observed with over 1,600 callbacks per grace
	period under heavy load~\cite{Sarma04c},
	it seems safe to say that Linux falls into the former category.
\end{quote}

On the one hand, this passage failed to anticipate the cache-warmth
issues that RCU can suffer from in workloads with significant update
intensity, in part because it seemed unlikely that RCU would really
be used for such workloads.
In the event, the \co{SLAB_DESTROY_BY_RCU} has been pressed into 
service in a number of instances where these cache-warmth issues would
otherwise be problematic, as has sequence locking.
On the other hand, this passage also failed to anticipate that
RCU would be used to reduce scheduling latency or for security.

In short, beware of prognostications, including those in the remainder
of this chapter.

% future/tm.tex

\section{Transactional Memory}
\label{sec:future:Transactional Memory}

데이터베이스 바깥의 영역에서 트랜잭션을 사용하자는 아이디어는 수십년 전으로
거슬러 오르는데~\cite{DBLomet1977SIGSOFT}, 이 아이디어는 데이터베이스에서와
데이터베이스 외에서의 트랜잭션에 대해 데이터베이스 외에서의 트랜잭션은
데이터베이스 트랜잭션을 정의하는 성질인 ``ACID'' 에서 ``D'' 를 빼냅니다.
하드웨어에서 메모리 기반의 트랜잭션, 또는 ``트랜잭셔널 메모리'' (TM) 을
지원하려는 아이디어는 최근의 것입니다만~\cite{Herlihy93a}, 안타깝게도 실제 많이
사용되는 하드웨어에서의 그런 트랜잭션의 지원은 그와 비슷한 것들에 대한 제안은
계속 있어왔지만~\cite{JMStone93}, 곧바로 이루어지지는 않았습니다.
그로부터 그렇게 오래지 않아, Shavit 과 Touitou 는 일반적으로 사용되는
하드웨어에서 동작할 수 있는, 메모리 접근 순서를 조정해서 소프트웨어만으로
이루어진 트랜잭셔널 메모리 구현 (STM) 을 제안했습니다.
이 제안은 여러 해동안 질질 끌려졌는데, 어쩌면 연구자 커뮤니티들의 관심은
non-blocking 동기화 (Section~\ref{sec:advsync:Non-Blocking Synchronization} 을
참고하세요) 에 열중되었기 때문일 수도 있습니다.
\iffalse

The idea of using transactions outside of databases goes back many
decades~\cite{DBLomet1977SIGSOFT}, with the key difference between
database and non-database transactions being that non-database transactions
drop the ``D'' in the ``ACID'' properties defining database transactions.
The idea of supporting memory-based transactions, or ``transactional memory''
(TM), in hardware
is more recent~\cite{Herlihy93a}, but unfortunately, support for such
transactions in commodity hardware was not immediately forthcoming,
despite other somewhat similar proposals being put forward~\cite{JMStone93}.
Not long after, Shavit and Touitou proposed a software-only implementation
of transactional memory (STM) that was capable of running on commodity
hardware, give or take memory-ordering issues~\cite{Shavit95}.
This proposal languished for many years, perhaps due to the fact that
the research community's attention was absorbed by non-blocking
synchronization (see Section~\ref{sec:advsync:Non-Blocking Synchronization}).
\fi

하지만 세기가 변하면서, TM 은 일부 주의의 목소리를 받기
시작했고~\cite{Blundell2005DebunkTM,McKenney2007PLOSTM}, 2000년부터 2009년
사이의 중반 쯤에는 몇가지 경고의 목소리에도
불구하고~\cite{MauriceHerlihy2005-TM-manifesto.pldi,DanGrossman2007TMGCAnalogy},
그 관심의 수준이 ``열광적'' 이라 할 수 있게 되었습니다.
\iffalse

But by the turn of the century, TM started receiving
more attention~\cite{Martinez01a,Rajwar01a}, and by the middle of the
decade, the level of interest can only be termed
``incandescent''~\cite{MauriceHerlihy2005-TM-manifesto.pldi,
DanGrossman2007TMGCAnalogy}, despite a few voices of
caution~\cite{Blundell2005DebunkTM,McKenney2007PLOSTM}.
\fi

TM 에 대한 기본 아이디어는 한 섹션의 코드를 어토믹하게 수행해서 다른 쓰레드가
그 중간의 상태를 볼 수 없게 하자는 것입니다.
그렇게, TM 의 의미는 간단히 각각의 트랜잭션을 반복적으로 획득할 수 있는 글로벌
락의 획득과 해제로 감싸는 것으로, 성능과 확장성이 처참해지긴 하겠지만, 구현될
수 있습니다.
하드웨어에서든 소프트웨어에서든 TM 구현에서는 피할 수 없는 복잡성은 동시적인
트랜잭션들이 안전하게 병렬적으로 수행될 수 있는지에 대한 파악입니다.
이 파악은 동적으로 이루어지기 때문에, 충돌하는 트랜잭션들은 취소되거나
``롤백''될 수 있고, 일부 구현에서는, 이 실패한 모드가 프로그래머에게 보여질 수
있습니다.
\iffalse

The basic idea behind TM is to execute a section of
code atomically, so that other threads see no intermediate state.
As such, the semantics of TM could be implemented
by simply replacing each transaction with a recursively acquirable
global lock acquisition and release, albeit with abysmal performance
and scalability.
Much of the complexity inherent in TM implementations, whether hardware
or software, is efficiently detecting when concurrent transactions can safely
run in parallel.
Because this detection is done dynamically, conflicting transactions
can be aborted or ``rolled back'', and in some implementations, this
failure mode is visible to the programmer.
\fi

트랜잭션 롤백은 트랜잭션 크기가 줄어듦에 따라 발생률이 줄어들 것이기 때문에, TM
은 스택, 큐, 해시 테이블, 탐색 트리들과 같은 데에 사용되는 링크드 리스트 제어와
같이 작은 크기의 메모리 기반 오퍼레이션들에 매력적일 수 있습니다.
하지만, I/O 와 프로세스 생성과 같은 메모리 기반이 아닌 오퍼레이션들을 포함하는
것들과 같은, 커다란 트랜잭션들에 TM 을 적용하기는 어렵습니다.
다음 섹션들은 ``Transactional Memory
Everywhere''~\cite{PaulEMcKenney2009TMeverywhere} 의 커다란 비전을 위해
존재하는 현재의 극복해야할 사항들을 알아봅니다.
Section~\ref{sec:future:Outside World} 은 바깥의 것들과 상호동작하는 데에
나타나는 문제점들을 알아보고,
Section~\ref{sec:future:Process Modification} 는 프로세스 수정 기능들과의
상호동작을 보며,
Section~\ref{sec:future:Synchronization} 는 다른 동기화 기능들과의 상호
동작들을 살펴보며, 마지막으로
Section~\ref{sec:future:Discussion} 에서 일부 토의와 함께 마무리 합니다.
\iffalse

Because transaction roll-back is increasingly unlikely as transaction
size decreases, TM might become quite attractive for small memory-based
operations,
such as linked-list manipulations used for stacks, queues, hash tables,
and search trees.
However, it is currently much more difficult to make the case for large
transactions, particularly those containing non-memory operations such
as I/O and process creation.
The following sections look at current challenges to the grand vision of
``Transactional Memory Everywhere''~\cite{PaulEMcKenney2009TMeverywhere}.
Section~\ref{sec:future:Outside World} examines the challenges faced
interacting with the outside world,
Section~\ref{sec:future:Process Modification} looks at interactions
with process modification primitives,
Section~\ref{sec:future:Synchronization} explores interactions with
other synchronization primitives, and finally
Section~\ref{sec:future:Discussion} closes with some discussion.
\fi

\subsection{Outside World}
\label{sec:future:Outside World}

Donal Knuth 의 말을 인용하면:
\iffalse

In the words of Donald Knuth:
\fi

\begin{quote}
	많은 컴퓨터 사용자들이 입력과 출력은 ``진짜 프로그래밍'' 의 실제 부분은
	아니고, 그것들은 그저 기계로 정보를 넣고 그로부터 정보를 뽑아내기 위해
	(불행히도) 반드시 해야만 하는것들이라고 느낍니다.
	\iffalse

	Many computer users feel that input and output are not actually part
	of ``real programming,'' they are merely things that (unfortunately)
	must be done in order to get information in and out of the machine.
	\fi
\end{quote}

우리가 입력과 출력이 ``진짜 프로그래밍'' 이라 믿든 아니라 믿든, 대부분의 컴퓨터
시스템들에 있어서, 바깥 세상과의 상호작용이 첫번째 중요도의 요구사항임은
사실입니다.
따라서 이 섹션은 트랜잭셔널의, I/O 오퍼레이션을 통해서든, 시간 딜레이를
통해서든, 또는 영구적 저장장치를 이용해서든 이루어지는, 상호작용을 위한
기능들에 대해 비평해 봅니다.
\iffalse

Whether we believe that input and output are ``real programming'' or
not, the fact is that for most computer systems, interaction with the
outside world is a first-class requirement.
This section therefore critiques transactional memory's ability to
so interact, whether via I/O operations, time delays, or persistent
storage.
\fi

\subsubsection{I/O Operations}
\label{sec:future:I/O Operations}

누군가는 I/O 오퍼레이션을 락 기반의 크리티컬 섹션 안에서 할수도, 적어도
원칙적으로는, userspace-RCU read-side 크리티컬 섹션 안에서 할수도 있습니다.
트랜잭션 안에서 I/O 오퍼레이션을 수행하려 하면 어떻게 될까요?
\iffalse

One can execute I/O operations within a lock-based critical section,
and, at least in principle, from within a userspace-RCU read-side
critical section.
What happens when you attempt to execute an I/O operation from within
a transaction?
\fi

여기 내포된 문제는 트랜잭션은 롤백될 수 있는데, 예를 들어 conflict 이 나거나
하는 경우입니다.
대략적으로 말해서, 이는 특정 트랜잭션 내에서 일어날 수 있는 모든 오퍼레이션들이
복구될 수 있어서 해당 오퍼레이션을 두번 행하는 것은 한번만 행한 것과 똑같은
효과를 내야 할 것을 필요로 합니다.
안타깝게도, I/O 는 일반적으로는 근본적으로 돌이킬 수 없는 오퍼레이션이어서,
일반적인 I/O 오퍼레이션을 트랜잭션 내에 넣기는 어렵게 합니다.
사실, 일반적인 I/O 는 돌이킬 수 없습니다:
일단 여러분이 핵탄두를 발사시키는 버튼을 눌렀다면, 되돌릴 수는 없습니다.

여기 트랜잭션 안에서 I/O 를 다루기 위한 몇가지 선택사항들이 있습니다:
\iffalse

The underlying problem is that transactions may be rolled back, for
example, due to conflicts.
Roughly speaking, this requires that all operations within any given
transaction be revocable, so that executing the operation twice has
the same effect as executing it once.
Unfortunately, I/O is in general the prototypical irrevocable
operation, making it difficult to include general I/O operations in
transactions.
In fact, general I/O is irrevocable:
Once you have pushed the button launching the nuclear warheads, there
is no turning back.

Here are some options for handling of I/O within transactions:
\fi

\begin{enumerate}
\item	트랜잭션 내에서의 I/O 를 메모리 상의 버퍼에 버퍼링 되는 I/O 로
	제약시킵니다.
	이렇게 되면 이 버퍼들은 다른 메모리 위치들이 포함될 수 있는 것과 같은
	방식으로 트랜잭션 내에 포함되어질 수 있을 겁니다.
	이는 선택될 수 있는 메커니즘으로 보이며, 이는 실제로 stream I/O 나
	대용량 I/O 와 같은, 많은 흔한 상황들에서 잘 동작합니다.
	하지만, 복수의 레코드에 기반한 출력 스트림들이 복수의 프로세스들로부터
	하나의 파일에 합쳐지는, 예컨대 ``a+'' 옵션과 함께 사용된 \co{fopen()}
	이나 \co{O_APPEND} 플래그와 함게 사용된 \co{open()} 과 같은 경우에는
	특수한 처리가 필요합니다.
	또한, 다음 섹션에서 보게 되겠지만, 일반적인 네트워킹 오퍼레이션들은
	버퍼링을 통해 처리될 수 없습니다.
\iffalse

\item	Restrict I/O within transactions to buffered I/O with in-memory
	buffers.
	These buffers may then be included in the transaction in the
	same way that any other memory location might be included.
	This seems to be the mechanism of choice, and it does work
	well in many common cases of situations such as stream I/O and
	mass-storage I/O.
	However, special handling is required in cases where multiple
	record-oriented output streams are merged onto a single file
	from multiple processes, as might be done using the ``a+''
	option to \co{fopen()} or the \co{O_APPEND}  flag to \co{open()}.
	In addition, as will be seen in the next section, common
	networking operations cannot be handled via buffering.
\fi
\item	트랜잭션 안에서의 I/O 를 금지시켜서 I/O 오퍼레이션을 행하려는 모든
	시도는 그를 둘러싼 트랜잭션을 abort 시키게 만듭니다 (그리고 복수의
	중첩된 트랜잭션들까지도요).
	이 방법은 버퍼링 되지 않은 I/O 를 위한 관습적인 TM 전략처럼 보이지만,
	TM 이 I/O 를 허용할 수 있는 다른 동기화 기능들을 포함할 것을 필요로
	합니다.
\item	트랜잭션 안에서의 I/O 를 금지시키지만, 이 금지를 강제시키는데에
	컴파일러의 협조를 받습니다.
\iffalse

\item	Prohibit I/O within transactions, so that any attempt to execute
	an I/O operation aborts the enclosing transaction (and perhaps
	multiple nested transactions).
	This approach seems to be the conventional TM approach for
	unbuffered I/O, but requires that TM interoperate with other
	synchronization primitives that do tolerate I/O.
\item	Prohibit I/O within transactions, but enlist the compiler's aid
	in enforcing this prohibition.
\fi
\item	한번에 단 하나의 \emph{되돌려질 수 있는}
	트랜잭션~\cite{SpearMichaelScott2008InevitableSTM} 만이 수행될 수 있게
	허가해서, 되돌려질 수 있는 트랜잭션들은 I/O 오퍼레이션을 포함할 수
	있도록 허가합니다.\footnote{
		이전의 문헌에서, 되돌려질 수 있는 트랜잭션들은
		\emph{inevitable} 트랜잭션이라 칭해졌습니다.}
	이는 일반적으로 동작합니다만, I/O 오퍼레이션들의 성능과 확장성을 상당히
	제한합니다.
	확장성과 성능이 병렬화의 첫번째 목표라는 점을 놓고 보면, 이 방법의
	일반성은 약간 스스로를 제한시키는 것으로 보입니다.
	더 나쁜 것은, 되돌릴 수 있는 성질을 I/O 오퍼레이션을 막는데에 사용하는
	것은 직접적인 트랜잭션 abort 오퍼레이션의 사용을 금지하는 것으로
	보입니다.\footnote{
		이 문제는 Michael Factor 에 의해 지적되었습니다.}
	마지막으로, 특정 데이터 아이템을 조정하는 도돌려질 수 있는 트랜잭션이
	존재한다면, 똑같은 데이터 아이템을 조정하는 다른 모든 트랜잭션들은
	non-blocking semantic 을 가질 수가 없습니다.
\iffalse

\item	Permit only one special
	\emph{irrevocable} transaction~\cite{SpearMichaelScott2008InevitableSTM}
	to proceed
	at any given time, thus allowing irrevocable transactions to
	contain I/O operations.\footnote{
		In earlier literature, irrevocable transactions are
		termed \emph{inevitable} transactions.}
	This works in general, but severely limits the scalability and
	performance of I/O operations.
	Given that scalability and performance is a first-class goal of
	parallelism, this approach's generality seems a bit self-limiting.
	Worse yet, use of irrevocability to tolerate I/O operations
	seems to prohibit use of manual transaction-abort operations.\footnote{
		This difficulty was pointed out by Michael Factor.}
	Finally, if there is an irrevocable transaction manipulating
	a given data item, any other transaction manipulating that
	same data item cannot have non-blocking semantics.
\fi
\item	I/O 오퍼레이션들이 트랜잭션의 하층에 들어가질 수 있는 새로운 하드웨어와
	프로토콜을 만듭니다.
	인풋 오퍼레이션의 경우, 하드웨어는 그 오퍼레이션의 결과를 올바르게
	예측할 수 있어야 할 것이고, 그 예측이 틀렸을 경우에는 그 트랜잭션을
	abort 시킬 수 있어야 할 겁니다.
\iffalse

\item	Create new hardware and protocols such that I/O operations can
	be pulled into the transactional substrate.
	In the case of input operations, the hardware would need to
	correctly predict the result of the operation, and to abort the
	transaction if the prediction failed.
\fi
\end{enumerate}

I/O 오퍼레이션은 TM 의 잘 알려진 약한 부분들이고, 트랜잭션에서 I/O 를 지원하기
위한 이 문제가 합리적이고 일반적인 해결책을 가지고 있는지는 확실치 않은데,
적어도 ``합리적'' 이란 말이 사용 가능한 성능과 확장성을 포함한다면 그렇습니다.
더도 아니고 덜도 아니고, 이 문제에 대한 계속된 시간과 관심이 이 문제에 대한
추가적인 진보를 만들어낼 겁니다.
\iffalse

I/O operations are a well-known weakness of TM, and it is not clear
that the problem of supporting I/O in transactions has a reasonable
general solution, at least if ``reasonable'' is to include usable
performance and scalability.
Nevertheless, continued time and attention to this problem will likely
produce additional progress.
\fi

\subsubsection{RPC Operations}
\label{sec:future:RPC Operations}

누군가는 RPC 들을 락 기반의 크리티컬 섹션에서도, userspace-RCU read-side
크리티컬 섹션 내에서도 수행할 수 있습니다.
여러분이 RPC 를 트랜잭션 내에서 수행하려 하면 어떻게 될까요?

RPC 요청과 그에 대한 응답이 해당 트랜잭션 내에 포함된다면, 그리고 트랜잭션의
일부 부분이 그 응답으로 리턴되는 결과에 의존적이라면, 버퍼링을 사용하는 I/O 의
경우에 사용되었던 메모리 버퍼를 사용한 트릭은 사용할 수 없습니다.
이런 버퍼링 전략을 사용하려는 모든 시도는 트랜잭션을 deadlock 에 바지게 할
것인데, 요청은 그 트랜잭션이 성공할 것이라 보장되기 전가지는 보내질 수가
없지만, 트랜잭션의 성공 여부는 응답이 도착하기 전까지는 알 수 없을 것이기
때문으로, 다음과 같은 경우가 그 예가 됩니다:
\iffalse

One can execute RPCs within a lock-based critical section, as well as
from within a userspace-RCU read-side critical section. What happens when you
attempt to execute an RPC from within a transaction?

If both the RPC request and its response are to be contained within the
transaction, and if some part of the transaction depends on the result
returned by the response, then it is not possible to use the memory-buffer
tricks that can be used in the case of buffered I/O.
Any attempt to
take this buffering approach would deadlock the transaction, as the
request could not be transmitted until the transaction was guaranteed
to succeed, but the transaction's success might not be knowable until
after the response is received, as is the case in the following example:
\fi

\vspace{5pt}
\begin{minipage}[t]{\columnwidth}
\small
\begin{verbatim}
  1 begin_trans();
  2 rpc_request();
  3 i = rpc_response();
  4 a[i]++;
  5 end_trans();
\end{verbatim}
\end{minipage}
\vspace{5pt}

이 트랜잭션의 메모리 사용량은 RPC 응답이 도착하기 전가지는 결정될 수 없고, 이
트랜잭션의 메모리 사용량이 결정되기 전까지는, 이 트랜잭션이 커밋되어도 되는지
여부를 결정할 수가 없습니다.
따라서 트랜잭션의 의미론에 있어 일관적인 유일한 동작은 무조건적으로 이
트랜잭션을 abort 시키는 것으로, 이 말은, 곧 도움이 되지 않는다는 말입니다.

여기 TM 에서 사용할 수 있는 몇가지 선택들이 있습니다:
\iffalse

The transaction's memory footprint cannot be determined until after the
RPC response is received, and until the transaction's memory footprint
can be determined, it is impossible to determine whether the transaction
can be allowed to commit.
The only action consistent with transactional semantics is therefore to
unconditionally abort the transaction, which is, to say the least,
unhelpful.

Here are some options available to TM:
\fi

\begin{enumerate}
\item	트랜잭션 내에서의 RPC 를 금지시켜서, RPC 오퍼레이션을 수행하려 하는
	모든 시도는 그를 둘러싼 트랜잭션을 (그리고 아마도 복수개의 중첩된
	트랜잭션들도) abort 시키도록 합니다.
	대안적으로, 컴파일러가 RPC 없는 트랜잭션들을 강제하도록 도움을 줄 수
	있게 합니다.
	이 방법은 동작합니다만, TM 이 다른 동기화 도구들과 상호작용할 것을
	필요로 합니다.
\item	한번에 하나의 되돌려질 수 있는 특수한
	트랜잭션~\cite{SpearMichaelScott2008InevitableSTM} 만을 허용해서, 이
	되돌려질 수 있는 트랜잭션은 RPC 오퍼레이션을 포함할 수 있도록 합니다.
	이 방법은 일반적으로 동작합니다만, RPC 오퍼레이션들의 확장성과 성능을
	상당히 제한하게 됩니다.
	확장성과 성능이 병렬화의 첫번째 목표임을 상기해보면, 이 방법의 일반성은
	약간 제한적인 것으로 보입니다.
	더 나아가서, RPC 오퍼레이션을 받아들일 수 있는, 되돌려질 수 있는
	트랜잭션들의 사용은 일단 RPC 오퍼레이션이 시작되면 손으로 작성된
	트랜잭션 abort 오퍼레이션들을 배제시킵니다.
	마지막으로, 특정 데이터 아이템을 조정하는 되돌려질 수 있는 트랜잭션이
	존재한다면, 같은 데이터 아이템을 조정하는 모든 다른 트랜잭션들은
	non-blocking semantic 을 가질 수 없습니다.
\iffalse

\item	Prohibit RPC within transactions, so that any attempt to execute
	an RPC operation aborts the enclosing transaction (and perhaps
	multiple nested transactions).
	Alternatively, enlist the compiler to enforce RPC-free
	transactions.
	This approach does work, but will require TM to
	interact with other synchronization primitives.
\item	Permit only one special
	irrevocable transaction~\cite{SpearMichaelScott2008InevitableSTM}
	to proceed at any given time, thus allowing irrevocable
	transactions to contain RPC operations.
	This works in general, but severely limits the scalability and
	performance of RPC operations.
	Given that scalability and performance is a first-class goal of
	parallelism, this approach's generality seems a bit self-limiting.
	Furthermore, use of irrevocable transactions to permit RPC
	operations rules out manual transaction-abort operations
	once the RPC operation has started.
	Finally, if there is an irrevocable transaction manipulating
	a given data item, any other transaction manipulating that
	same data item cannot have non-blocking semantics.
\fi
\item	트랜잭션의 성공이 RPC 응답이 도착하기 전에 결정될 수 있는 특수한 경우를
	정의하고, RPC 요청이 보내지기 직전에 이것들을 자동적으로 되돌려질 수
	있는 트랜잭션들로 변환시킵니다.
	물론, 만약 여러 동시적 트랜잭션들이 이 방식으로 RPC 호출을 시도한다면,
	이것들 중 하나만을 제외하고는 모두 롤백시켜야 할 것으로, 결과적으로
	성능과 확장성이 떨어질 겁니다.
	이 방법은 더도 아니고 덜도 아니고, RPC 로 마무리되는, 오랫동안 수행되는
	트랜잭션들이 있을 때에는 가치가 있을 겁니다.
	이 방법은 여전히 손으로 직접 하는 트랜잭션 abort 오퍼레이션들과의
	문제가 존재합니다.
\iffalse

\item	Identify special cases where the success of the transaction may
	be determined before the RPC response is received, and
	automatically convert these to irrevocable transactions immediately
	before sending the RPC request.
	Of course, if several concurrent transactions attempt RPC calls
	in this manner, it might be necessary to roll all but one of them
	back, with consequent degradation of performance and scalability.
	This approach nevertheless might be valuable given long-running
	transactions ending with an RPC.
	This approach still has problems with manual transaction-abort
	operations.
\fi
\item	RPC 응답이 트랜잭션의 바깥으로 옮겨질 수 있는 특수한 경우들을 정의하고,
	버퍼링을 사용한 I/O 에서 사용된 것과 비슷한 기법을 사용합니다.
\item	트랜잭션적인 구조가 RPC 클라이언트만이 아니라 서버도 포함하도록
	확장합니다.
	이건 이론적으로는 가능하며, 분산 데이터베이스들을 통해 보여졌습니다.
	하지만, 분산 데이터베이스 기법을 통해 요구되는 성능과 확장성
	요구사항들이 맞춰질 수 있을지는 불명확한데, 메모리 기반의 TM 은 느린
	디스크 드라이브에서 나오는 느린 응답시간을 감출 수 없을 것이기
	때문입니다.
	물론, 솔리드 스테이트 디스크 (SSD) 의 발전과 함께, 데이터 베이스들이 그
	자신의 응답시간들을 디스크 드라이브의 응답시간들에 숨겨둘 수 있을지도
	불명확해지고 있습니다.
\iffalse

\item	Identify special cases where the RPC response may be moved out
	of the transaction, and then proceed using techniques similar
	to those used for buffered I/O.
\item	Extend the transactional substrate to include the RPC server as
	well as its client.
	This is in theory possible, as has been demonstrated by
	distributed databases.
	However, it is unclear whether the requisite performance and
	scalability requirements can be met by distributed-database
	techniques, given that memory-based TM cannot hide such latencies
	behind those of slow disk drives.
	Of course, given the advent of solid-state disks, it is also unclear
	how much longer databases will be permitted to hide their latencies
	behind those of disks drives.
\fi
\end{enumerate}

앞의 섹션에서 이야기 된 것처럼, I/O 는 TM 의 알려진 약점이고, RPC 는 그저 I/O
의 특별히 문제가 되는 경우일 뿐입니다.
\iffalse

As noted in the prior section, I/O is a known weakness of TM, and RPC
is simply an especially problematic case of I/O.
\fi

\subsubsection{Time Delays}
\label{sec:future:Time Delays}

트랜잭션 틱한 접근과의 상호작용에 관련된 중요한 특수 케이스 하나는 트랜잭션
내에서의 명시적인 시간 딜레이와 관련됩니다.
물론, 트랜잭션 내에서의 시간 딜레이라는 이 생각은 TM 의 원자성에 위배됩니다만,
어떤 사람들은 이런 종류의 일은 완화된 원자성이 모두 그러한 것이라고 주장할 수
있습니다.
나아가서, memory-mapped I/O 와의 올바른 상호작용은 가끔 주의깊게 제어되는
타이밍을 필요로 하며, 어플리케이션들은 다양한 목적을 위해 시간 딜레이를 자주
사용합니다.

그러니, TM 은 트랜잭션 내에서의 시간 딜레이에 대해 뭘 해야 할까요?
\iffalse

An important special case of interaction with extra-transactional accesses
involves explicit time delays within a transaction.
Of course, the idea of a time delay within a transaction flies in the
face of TM's atomicity property, but one can argue that this sort of
thing is what weak atomicity is all about.
Furthermore, correct interaction with memory-mapped I/O sometimes requires
carefully controlled timing, and applications often use time delays
for varied purposes.

So, what can TM do about time delays within transactions?
\fi

\begin{enumerate}
\item	트랜잭션 내에서의 시간 딜레이를 무시합니다.
	이 방법은 우아한 모습으로 보이지만, 다른 너무 많은 ``우아한''
	해결책들처럼, 기존 코드에 사용되는순간 실패하게 됩니다.
	크리티컬 섹션 내에서 중요한 시간 딜레이를 가지고 있을 수도 있는 그런
	코드는 트랜잭션화 되는 과정에서 실패할 겁니다.
\item	시간 딜레이 오퍼레이션을 마주하는 순간 트랜잭션을 abort 시킵니다.
	이 방법은 매력적이지만, 안타깝게도 시간 딜레이 오퍼레이션을 자동적으로
	탐지하는게 항상 가능하지는 않습니다.
	어떤 짧은 루프는 정말 중요한 무언가를 계산하는 루프일가요, 아니면 그저
	시간이 지나가길 기다리는 걸까요?
\item	트랜잭션 내에서의 시간 딜레이를 금지시키기 위해 컴파일러의 도움을
	받습니다.
\item	시간 딜레이가 평범하게 수행되도록 합니다.
	안타깝게도, 일부 TM 구현은 커밋 시점에서야 수정사항을 외부에
	노출시켜서, 많은 경우에는 시간 딜레이의 목적을 달성 불가능하게
	할겁니다.
\iffalse

\item	Ignore time delays within transactions.
	This has an appearance of elegance, but like too many other
	``elegant'' solutions, fails to survive first contact with
	legacy code.
	Such code, which might well have important time delays in critical
	sections, would fail upon being transactionalized.
\item	Abort transactions upon encountering a time-delay operation.
	This is attractive, but it is unfortunately not always possible
	to automatically detect a time-delay operation.
	Is that tight loop computing something important, or is it
	instead waiting for time to elapse?
\item	Enlist the compiler to prohibit time delays within transactions.
\item	Let the time delays execute normally.
	Unfortunately, some TM implementations publish modifications only
	at commit time, which would in many cases defeat the purpose of
	the time delay.
\fi
\end{enumerate}

단 하나의 올바른 답이 있는지 여부는 분명치 않습니다.
완화된 원자성을 갖춰서 변경 사항을 트랜잭션 내에서 곧바로 외부에 노출시키는
(abort 시에는 이 변경들을 되돌리는) TM 구현은 마지막 대안에 의해 잘 처리될 수
있을 겁니다.
이런 경우라 하더라도, 트랜잭션의 다른 끝의 (또는, 심지어 하드웨어의) 코드는
abort 된 트랜잭션을 처리하기 위해 상당한 재설계가 필요할 겁니다.
이런 재설계의 필요성은 트랜잭셔널 메모리를 기존 코드에 적용하기를 더 어렵게
할겁니다.
\iffalse

It is not clear that there is a single correct answer.
TM implementations featuring weak atomicity that publish changes
immediately within the transaction (rolling these changes back upon abort)
might be reasonably well served by the last alternative.
Even in this case, the code (or possibly even hardware) at the other
end of the transaction may require a substantial redesign to tolerate
aborted transactions.
This need for redesign would make it more difficult to apply transactional
memory to legacy code.
\fi

\subsubsection{Persistence}
\label{sec:future:Persistence}

락킹 기능들에는 많은 다른 타입들이 존재합니다.
한가지 흥미로운 차이는 지속성으로, 달리 말하자면, 락이 그 락을 사용하는
프로세스의 주소 공간에 대한 의존성 없이 존재할 수 있는가 여부입니다.

지속성 없는 락은 \co{pthread_mutex_lock()}, \co{pthread_rwlock_rdlock()},
그리고 대부분의 커널 레벨 락킹 기능들을 포함합니다.
만약 지속성 없는 락의 데이터 구조를 담고 있는 메모리 위치가 사라지면, 락 역시
그렇게 됩니다.
\co{pthread_mutex_lock()} 의 일반적인 사용에 있어, 이는 프로세스가 종료될 때,
그것의 모든 락들이 사라짐을 의미합니다.
이 속성은 프로그램 종료 시점에 락 정리를 대수롭지 않게 하기 위해 사용될 수도
있습니다만, 관계없는 어플리케이션들 사이에서 락을 공유하기는, 그런 공유를
위해선 어플리케이션 사이에 메모리를 공유할 것을 필요로 하기에, 어렵게 합니다.
\iffalse

There are many different types of locking primitives.
One interesting distinction is persistence, in other words, whether the
lock can exist independently of the address space of the process using
the lock.

Non-persistent locks include \co{pthread_mutex_lock()},
\co{pthread_rwlock_rdlock()}, and most kernel-level locking primitives.
If the memory locations instantiating a non-persistent lock's data
structures disappear, so does the lock.
For typical use of \co{pthread_mutex_lock()}, this means that when the
process exits, all of its locks vanish.
This property can be exploited in order to trivialize lock cleanup
at program shutdown time, but makes it more difficult for unrelated
applications to share locks, as such sharing requires the applications
to share memory.
\fi

지속성 있는 락들은 관련없는 어플리케이션들 사이에 메모리를 공유해야할 필요를
없앱니다.
지속성 있는 락킹 API 들은 flock 계열인 \co{lockf()}, System V 세마포어,
\co{open()} 에 사용되는 \co{O_CREAT} 플래그 등을 포함합니다.
이 지속성 있는 API 들은 다양한 어플리케이션들을 구축하는 커다란 규모의
오퍼레이션들을 보호하는데에 사용될 수 있고, \co{O_CREAT} 의 경우에는 심지어
운영체제 리부팅 뒤에도 살아남습니다.
필요하다면, 락들은 분산된 락 매니저와 분산 파일 시스템들을 통해서 여러 컴퓨터
시스템들에까지 미칠 수 있습니다---그리고 이 컴퓨터 시스템들의 어떤 모든 것들의
리부팅 전후에도 지속됩니다.

지속성 있는 락들은 어떤 어플리케이션을 통해서도 사용될 수 있는데, 복수의 언어와
소프트웨어 환경을 사용해 작성된 어플리케이션도 포함됩니다.
사실, 하나의 지속성 있는 락이 C 언어로 쓰여진 어플리케이션에 의해 획득되고는
Python 으로 쓰여진 어플리케이션에 의해 해제될 수도 있는 것입니다.

TM 에는 이와 비슷한 지속성 있는 기능들이 어떻게 주어질 수 있을까요?
\iffalse

Persistent locks help avoid the need to share memory among unrelated
applications.
Persistent locking APIs include the flock family, \co{lockf()}, System
V semaphores, or the \co{O_CREAT} flag to \co{open()}.
These persistent APIs can be used to protect large-scale operations
spanning runs of multiple applications, and, in the case of \co{O_CREAT}
even surviving operating-system reboot.
If need be, locks can even span multiple computer systems via distributed
lock managers and distributed filesystems---and persist across reboots
of any or all of these computer systems.

Persistent locks can be used by any application, including applications
written using multiple languages and software environments.
In fact, a persistent lock might well be acquired by an application written
in C and released by an application written in Python.

How could a similar persistent functionality be provided for TM?
\fi

\begin{enumerate}
\item	지속성 있는 트랜잭션들을 그것들을 지원하기 위해 설계된, SQL 과 같은
	특수 목적 환경으로 제한시킵니다.
	이는 데이터베이스 시스템의 수십년의 역사로 보건대 분명히 잘
	동작합니다만, 지속성 있는 락들에 의해 제공되는 것만큼의 유연성을
	제공하지는 않습니다.
\item	일부 저장 장치나 파일시스템들에서 제공되는 스냅샷 기능들을 사용합니다.
	안타깝게도, 이는 네트워크 통신을 처리하지도, 스냅샷 기능을 제공하지
	않는, 예를 들어 메모리 스틱과 같은 장치로의 I/O 는 처리하지도 못합니다.
\item	타임머신을 만듭니다.
\iffalse

\item	Restrict persistent transactions to special-purpose environments
	designed to support them, for example, SQL.
	This clearly works, given the decades-long history of database
	systems, but does not provide the same degree of flexibility
	provided by persistent locks.
\item	Use snapshot facilities provided by some storage devices and/or
	filesystems.
	Unfortunately, this does not handle network communication,
	nor does it handle I/O to devices that do not provide snapshot
	capabilities, for example, memory sticks.
\item	Build a time machine.
\fi
\end{enumerate}

물론, 이것이 트랜잭셔널 \emph{메모리} 라 불린다는 사실이 한숨을 돌리게 할텐데,
이 이름 자체가 지속성 있는 트랜잭션의 컨셉과 들어맞지 않기 때문입니다.
이는 이 가능성을 트랜잭셔널 메모리의 피할 수 없는 한계점을 보이는 중요한 테스트
케이스로 고려할 만큼, 딱 그만큼의 가치만 있습니다.
\iffalse

Of course, the fact that it is called transactional \emph{memory}
should give us pause, as the name itself conflicts with the concept of
a persistent transaction.
It is nevertheless worthwhile to consider this possibility as an important
test case probing the inherent limitations of transactional memory.
\fi

\subsection{Process Modification}
\label{sec:future:Process Modification}

프로세스들은 영원하지 않습니다:
프로세스들은 생성되고 소멸되며, 메모리 매핑은 수정되고, 동적 라이브러리들과
링크되고, 디버깅 됩니다.
이 섹션들은 어떻게 트랜잭셔널 메모리가 항상 변화되는 실행 환경을 처리할 수
있는지 알아봅니다.
\iffalse

Processes are not eternal:
They are created and destroyed, their memory mappings are modified,
they are linked to dynamic libraries, and they are debugged.
These sections look at how transactional memory can handle an
ever-changing execution environment.
\fi

\subsubsection{Multithreaded Transactions}
\label{sec:future:Multithreaded Transactions}

락을 쥔채, 또는, 필요하다면 userspace-RCU read-side 크리티컬 섹션 내에서
프로세스나 쓰레드를 생성하는 것은 완전히 합법적인 일입니다.
이건 합법적일 뿐만 아니라, 다음 코드 조각에서 볼 수 있듯이 상당히 간단하기도
합니다.
\iffalse

It is perfectly legal to create processes and threads while holding
a lock or, for that matter, from within a userspace-RCU read-side critical
section.
Not only is it legal, but it is quite simple, as can be seen from the
following code fragment:
\fi

\vspace{5pt}
\begin{minipage}[t]{\columnwidth}
\small
\begin{verbatim}
  1 pthread_mutex_lock(...);
  2 for (i = 0; i < ncpus; i++)
  3   pthread_create(&tid[i], ...);
  4 for (i = 0; i < ncpus; i++)
  5   pthread_join(tid[i], ...);
  6 pthread_mutex_unlock(...);
\end{verbatim}
\end{minipage}
\vspace{5pt}

이 슈도코드는 CPU 당 하나의 쓰레드를 생성하기 위해 \co{pthread_create()} 를
사용하고, 각각의 쓰레드가 완료되기를 기다리기 위해 \co{pthread_join()} 을
사용하는데, 모두 \co{pthread_mutex_lock()} 의 보호 아래에서 행해집니다.
원하는 결과는 락 기반의 크리티컬 섹션을 병렬적으로 수행하는 것이고, \co{fork()}
와 \co{wait()} 를 사용해서도 비슷한 효과를 얻을 수 있을 겁니다.
물론, 이 크리티컬 섹션은 쓰레드 생성 오버헤드를 정당화 하기 충분할 만큼 커야 할
것입니다만, 제품 소프트웨어에서 커다란 크리티컬 섹션들의 예는 많이 존재합니다.

TM 은 트랜잭션 내에서의 쓰레드 생성에 대해서 무엇을 해줄까요?
\iffalse

This pseudo-code fragment uses \co{pthread_create()} to spawn one thread
per CPU, then uses \co{pthread_join()} to wait for each to complete,
all under the protection of \co{pthread_mutex_lock()}.
The effect is to execute a lock-based critical section in parallel,
and one could obtain a similar effect using \co{fork()} and \co{wait()}.
Of course, the critical section would need to be quite large to justify
the thread-spawning overhead, but there are many examples of large
critical sections in production software.

What might TM do about thread spawning within a transaction?
\fi

\begin{enumerate}
\item	트랜잭션 내에서의 \co{pthread_create()} 를 불법적인 것으로 규정해서,
	\co{pthread_create()} 사용의 경우 트랜잭션이 abort 되거나 (이게
	선호됩니다) 정해지지 않은 동작을 하도록 만듭니다.
	대안적으로는, 컴파일러가 \co{pthread_create()} 가 없는 트랜잭션만을
	강제하도록 도움을 받습니다.
\item	트랜잭션 내에서 \co{pthread_create()} 가 수행될 수 있도록 하되, 그 부모
	쓰레드만이 트랜잭션의 일부로 여겨지도록 합니다.
	이 방법은 이미 존재하는 TM 구현들과 합리적으로 호환될 수 있을 것처럼
	보입니다만, 부주의한 사람에게는 함정이 될 것으로 보입니다.
	이 방법은 몇가지 더 질문을 떠오르게 하는데, 자식 쓰레드로의 접근의
	충돌을 어떻게 처리할 것인가와 같은 것들입니다.
\iffalse

\item	Declare \co{pthread_create()} to be illegal within transactions,
	resulting in transaction abort (preferred) or undefined
	behavior. Alternatively, enlist the compiler to enforce
	\co{pthread_create()}-free transactions.
\item	Permit \co{pthread_create()} to be executed within a
	transaction, but only the parent thread will be considered to
	be part of the transaction.
	This approach seems to be reasonably compatible with existing and
	posited TM implementations, but seems to be a trap for the unwary.
	This approach raises further questions, such as how to handle
	conflicting child-thread accesses.
\fi
\item	\co{pthread_create()} 를 함수 호출로 변환시킵니다.
	이 방법 역시 매력적인 골칫거리가 되는데, 자식 쓰레드들이 서로 통신하는
	드물지 않은 경우들을 처리하지 않기 때문입니다.
	이 방법은 또한 부모 쓰레드가 트랜잭션을 커밋하기 전에 자식 쓰레드들을
	기다리지 않는다면 어떤 일이 일어나게 될것인지와 같은 흥미로운 질문 역시
	물러일으킵니다.
	더 흥미롭게도, 부모가 트랜잭션 내에 포함되어 있는 변수의 값에 기초해서
	조건적으로 \co{pthread_join()} 을 수행한다면 무슨 일이 일어날까요?
	이 질문들에 대한 답은 락킹의 경우에는 꽤 간단합니다.
	TM 에서의 이 질문들에 대한 답은 독자 여러분의 몫으로 남겨두겠습니다.
\iffalse

\item	Convert the \co{pthread_create()}s to function calls.
	This approach is also an attractive nuisance, as it does not
	handle the not-uncommon cases where the child threads communicate
	with one another.
	In addition, it does not permit parallel execution of the body
	of the transaction.
\item	Extend the transaction to cover the parent and all child threads.
	This approach raises interesting questions about the nature of
	conflicting accesses, given that the parent and children are
	presumably permitted to conflict with each other, but not with
	other threads.
	It also raises interesting questions as to what should happen
	if the parent thread does not wait for its children before
	committing the transaction.
	Even more interesting, what happens if the parent conditionally
	executes \co{pthread_join()} based on the values of variables
	participating in the transaction?
	The answers to these questions are reasonably straightforward
	in the case of locking.
	The answers for TM are left as an exercise for the reader.
\fi
\end{enumerate}

데이터베이스 쪽에서 트랜잭션들의 병렬적 수행은 일반적이기 때문에, 현재의 TM
제안들은 그것들을 위해 제공된게 아니란 점은 놀라울 수도 있습니다.
한편으로는, 앞의 예제들은 간단한 교재에서의 예제에서는 일반적으로 찾을 수 없는,
락킹의 복잡한 사용 예여서, 그것들의 부작위함이 예상될 수도 있습니다.
그렇다곤 하나, 일부 TM 연구자들이 트랜잭션 내에서의 fork/join 병렬성을 위해
노력하고 있다는 소문이 돌리고 있으므로, 이 주제는 조만간 더 완벽하게 다루어질
수도 있습니다.
\iffalse

Given that parallel execution of transactions is commonplace in the
database world, it is perhaps surprising that current TM proposals do
not provide for it.
On the other hand, the example above is a fairly sophisticated use
of locking that is not normally found in simple textbook examples,
so perhaps its omission is to be expected.
That said, there are rumors that some TM researchers are investigating
fork/join parallelism within transactions, so perhaps this topic will
soon be addressed more thoroughly.
\fi

\subsubsection{The \tco{exec()} System Call}
\label{sec:future:The exec System Call}

락을 잡은채로, 또는 RCU read-side 크리티컬 섹션 내에서도 \co{exec()} 시스템
콜을 호출할 수 있습니다.
그 호출의 정확한 의미는 기능의 타입에 따라 정해집니다.

지속성 없는 기능들 (\co{pthread_mutex_lock()}, \co{pthread_rwlock_rdlock()},
그리고 userspace-RCU 등) 의 경우에는, \co{exec()} 가 성공한다면, 모든 잡혀 있던
락들을 포함해서 전체 주소 공간이 사라집니다.
물론, \co{exec()} 가 실패한다면, 이 주소 공간은 여전히 살아있게 되어서, 모든
연관된 락들 역시 여전히 살아 있게 됩니다.
약간 이상할 수 있겠지만, 합리적으로 잘 정의된 의미입니다.

다른 한편, 지속성이 있는 기능들 (flock 부류들, \co{lockf()}, System V 세마포어,
\co{O_CREAT} 플래그와 함께 수행되는 \co{open()} 등) 은 \co{exec()} 의 성공이나
실패 여부와 관계 없이 살아남아서, \co{exec()} 된 프로그램은 그것들을 놓아줘야
할 겁니다.
\iffalse

One can execute an \co{exec()} system call while holding a lock, and
also from within an userspace-RCU read-side critical section.
The exact semantics depends on the type of primitive.

In the case of non-persistent primitives (including
\co{pthread_mutex_lock()}, \co{pthread_rwlock_rdlock()}, and userspace RCU),
if the \co{exec()} succeeds, the whole address space vanishes, along
with any locks being held.
Of course, if the \co{exec()} fails, the address space still lives,
so any associated locks would also still live.
A bit strange perhaps, but reasonably well defined.

On the other hand, persistent primitives (including the flock family,
\co{lockf()}, System V semaphores, and the \co{O_CREAT} flag to
\co{open()}) would survive regardless of whether the \co{exec()}
succeeded or failed, so that the \co{exec()}ed program might well
release them.
\fi

\QuickQuiz{}
	메모리의 \co{mmap()} 리전 내의 데이터 구조체로 표현되는 지속성 없는
	기능들은 어떨까요?
	그런 기능들로 만들어진 크리팈러 섹션 내에서의 \co{exec()} 가 존재한다면
	어떤 일이 벌어질까요?
	\iffalse

	What about non-persistent primitives represented by data
	structures in \co{mmap()} regions of memory?
	What happens when there is an \co{exec()} within a critical
	section of such a primitive?
	\fi
\QuickQuizAnswer{
	\co{exec()} 된 프로그램이 그 똑같은 메모리 영역을 매핑한다면, 이
	프로그램은 원칙적으로 그 락을 놓아주어야 합니다.
	이 방법이 소프트웨어 엔지니어링 관점에서도 말이 되는 소리인지에 대한
	판단은 독자 여러분의 몫으로 남겨두도록 하겠습니다.
	\iffalse

	If the \co{exec()}ed program maps those same regions of
	memory, then this program could in principle simply release
	the lock.
	The question as to whether this approach is sound from a
	software-engineering viewpoint is left as an exercise for
	the reader.
	\fi
} \QuickQuizEnd

트랜잭션 안에서 \co{exec()} 시스템 콜을 수행하려 한다면 어떤 일이 벌어질까요?
\iffalse

What happens when you attempt to execute an \co{exec()} system call
from within a transaction?
\fi

\begin{enumerate}
\item	트랜잭션 내에서의 \co{exec()} 를 금지시켜서, \co{exec()} 가 호출되면
	그를 둘렀나 트랜잭션들이 abort 되게 합니다.
	이는 괜찮은 정의입니다만, TM 외의 동기화 기능들이 \co{exec()} 와의 관계
	안에서도 잘 동작할 수 있을 것을 필요로 합니다.
\item	트랜잭션 내에서의 \co{exec()} 를 금지시키며, 컴파일러가 이 금지를
	강제하도록 합니다.
	이 방법을 취해서, 함수들이 \co{transaction_safe} 와
	\co{transaction_unsafe} attribute 로 장식되로고 하는, C++ 에서의 TM 에
	대한 작성중인 명세가 있습니다.\footnote{
		이 명세서를 짚어준 Mark Moir 와, 그보다 앞의 버전을 알려준
		Michael Wong 에게 감사를  드립니다.}
	이 방법은 실행시간 중에 트랜잭션을 abort 하는 것에 비해 몇가지 장점이
	있습니다만, 역시 \co{exec()} 와 관련해서 사용되어야 하는 TM 이 아닌
	동기화 기능들을 필요로 합니다.
\iffalse

\item	Disallow \co{exec()} within transactions, so that the enclosing
	transactions abort upon encountering the \co{exec()}.
	This is well defined, but clearly requires non-TM synchronization
	primitives for use in conjunction with \co{exec()}.
\item	Disallow \co{exec()} within transactions, with the compiler
	enforcing this prohibition.
	There is a draft specification for TM in C++ that takes
	this approach, allowing functions to be decorated with
	the \co{transaction_safe} and \co{transaction_unsafe}
	attributes.\footnote{
		Thanks to Mark Moir for pointing me at this spec, and
		to Michael Wong for having pointed me at an earlier
		revision some time back.}
	This approach has some advantages over aborting the transaction
	at runtime, but again requires non-TM synchronization primitives
	for use in conjunction with \co{exec()}.
\fi
\item	트랜잭션을 지속성 없는 락킹 기능들과 비슷한 방식을 취급해서,
	\co{exec()} 가 실패한다면 트랜잭션이 성공하고, \co{exec()} 가 성공하면
	말없이 트랜잭션을 커밋해 버립니다.
	일부 변수들이 \co{mmap()} 으로 매핑된 메모리에 존재하는 경우에 대해서는
	(따라서 성공적인 \co{exec()} 시스템 콜 뒤에는 살아남들 변수들) 독자
	여러분의 몫으로 남겨두겠습니다.
\item	\co{exec()} 시스템 콜이 성공할 것 같다면 트랜잭션을 (그리고 \co{exec()}
	시스템 콜을) abort 시키고, \co{exec()} 시스템 콜이 실패할 것 같다면
	트랜잭션을 지속시킵니다.
	이는 어떤 의미에서는 ``올바른'' 방법입니다만, 성이 차지 않는 결과를
	위해서는 상당한 작업을 필요로 할 겁니다.
\iffalse

\item	Treat the transaction in a manner similar to non-persistent
	Locking primitives, so that the transaction survives if \co{exec()}
	fails, and silently commits if the \co{exec()} succeeds.
	The case where some of the variables affected by the transaction
	reside in \co{mmap()}ed memory (and thus could survive a successful
	\co{exec()} system call) is left as an exercise for the reader.
\item	Abort the transaction (and the \co{exec()} system call) if the
	\co{exec()} system call would have succeeded, but allow the
	transaction to continue if the \co{exec()} system call would
	fail.
	This is in some sense the ``correct'' approach, but it would
	require considerable work for a rather unsatisfying result.
\fi
\end{enumerate}

아마도 \co{exec()} 시스템콜은 보편적인 TM 적용성에 대한 가장 이상한 반론의 예
중 하나일 텐데, 어떤 방법이 말이 되는지가 전혀 분명치 않고, 누구가는 이건 그저
실제 삶에서 중역들과 일하는 위험의 반영일 뿐이라고 주장할 수도 있을 겁니다.
그렇다곤 하나, 트랜잭션 내에서의 \co{exec()} 를 금지시키는 두가지 방법은 그나마
가장 논리적일 겁니다.

비슷한 문제들이 \co{exit()} 과 \co{kill()} 시스템 콜에 대해서도 존재합니다.
\iffalse

The \co{exec()} system call is perhaps the strangest example of an
obstacle to universal TM applicability, as it is not completely clear
what approach makes sense, and some might argue that this is merely a
reflection of the perils of interacting with execs in real life.
That said, the two options prohibiting \co{exec()} within transactions
are perhaps the most logical of the group.

Similar issues surround the \co{exit()} and \co{kill()} system calls.
\fi

\subsubsection{Dynamic Linking and Loading}
\label{sec:future:Dynamic Linking and Loading}

락 기반의 크리티컬 섹션과 userspace-RCU read-side 크리티컬 섹션 모두 C/C++ 공유
라이브러리나 Java 클래스 라이브러리와 같은, 동적으로 링크되고 로드되는 함수들을
수행시키는 코드를 합법적으로 담고 있을 수 있습니다.
물론, 이 라이브러리에 들어있는 코드에 대해서는 그 정의에 따라, 컴파일 타임에는
알 수 없습니다.
그러니, 동적으로 로드된 함수가 트랜잭션 안에서 수행된다면 어떤 일이 벌어질까요?

이 질문은 두개의 부분으로 구성됩니다: (1)~트랜잭션 안에서 어떻게 함수를
동적으로 링크하고 로드하는지와 (b)~이 함수 안의 알 수 없는 코드에 대해서 무엇을
해야 할까요?
공정해지기 위해, 항목 (b) 는 적어도 이론 상으로는 락킹과 userspace-RCU 모두에게
일부 도전적 과제를 갖게 합니다.
예를 들어서, 동적으로 링크된 함수는 락킹에 있어 데드락을 가져올 수도 있고
(에러에 의해) userspace-RCU read-side 크리티컬 섹션 안에 quiescent state 를
가져올 수도 있습니다.
차이점은, 락킹과 RCU 크리티컬 섹션 안에서 허용되는 오퍼레이션들의 클래스는
여전히 잘 이해되어 있지만, TM 의 경우에 대해서는 상당한 불확실성이 존재한다는
점입니다.
사실, 서로 다른 구현의 TM 은 서로 다른 제한을 갖게 될 것으로 보입니다.

그러니 동적으로 링크되고 로드되는 라이브러리 함수에 대해서 TM 은 무엇을 해야
할까요?
실제 코드를 로드하는 (a) 부분의 선택사항은 다음과 같은 것들을 포함합니다:
\iffalse

Both lock-based critical sections and userspace-RCU read-side critical sections
can legitimately contain code that invokes dynamically linked and loaded
functions, including C/C++ shared libraries and Java class libraries.
Of course, the code contained in these libraries is by definition
unknowable at compile time.
So, what happens if a dynamically loaded function is invoked within
a transaction?

This question has two parts: (a)~how do you dynamically link and load a
function within a transaction and (b)~what do you do about the unknowable
nature of the code within this function?
To be fair, item (b) poses some challenges for locking and userspace-RCU
as well, at least in theory.
For example, the dynamically linked function might introduce a deadlock
for locking or might (erroneously) introduce a quiescent state into a
userspace-RCU read-side critical section.
The difference is that while the class of operations permitted in locking
and userspace-RCU critical sections is well-understood, there appears
to still be considerable uncertainty in the case of TM.
In fact, different implementations of TM seem to have different restrictions.

So what can TM do about dynamically linked and loaded library functions?
Options for part (a), the actual loading of the code, include the following:
\fi

\begin{enumerate}
\item	동적 링킹과 로딩을 페이지 폴트와 비슷한 형태로 취급해서, 함수가
	로드되고 링크되면 해당 프로세스의 트랜잭션을 abort 시킵니다.
	만약 그 트랜잭션이 abort 되면, 재시도는 그 함수가 이미 존재함을 보게
	될거고, 해당 트랜잭션은 이제 정상적으로 진행될 것으로 예상될 수
	있습니다.
\item	트랜잭션 안에서의 함수의 동적 링킹과 로딩을 금지시킵니다.
\iffalse

\item	Treat the dynamic linking and loading in a manner similar to a
	page fault, so that the function is loaded and linked, possibly
	aborting the transaction in the process.
	If the transaction is aborted, the retry will find the function
	already present, and the transaction can thus be expected to
	proceed normally.
\item	Disallow dynamic linking and loading of functions from within
	transactions.
\fi
\end{enumerate}

아직 로드되지 않은 함수 안에서의 TM 에 친화적이지 않은 오퍼레이션들을 파악해낼
수 없는 특성에 대한 (b) 부분에 대한 선택 사항은 다음과 같은 것들이 있을 수
있습니다:
\iffalse

Options for part (b), the inability to detect TM-unfriendly operations
in a not-yet-loaded function, possibilities include the following:
\fi

\begin{enumerate}
\item	그냥 코드를 수행합니다: 해당 함수 안에 TM 에 친화적이지 않은
	오퍼레이션들이 존재한다면, 그냥 트랜잭션을 abort 시킵니다.
	불행히도, 이 방법은 컴파일러가 특정 트랜잭션들이 안전하게 구성되어
	있는지 여부를 알 수 없게 합니다.
	그에 상관없이 조합이 가능하게 할 수 있는 방법 한가지는 되돌이킬 수 있는
	트랜잭션의 사용입니다만, 현재의 구현들은 한번에 하나의 되돌이킬 수 있는
	트랜잭션의 수행만을 허용해서, 성능과 확장성을 상당히 떨어뜨릴 수
	있습니다.
	되돌이킬 수 있는 트랜잭션들은 직접적인 트랜잭션 abort 오퍼레이션의
	사용을 제외시킬 것으로 보여집니다.
	마지막으로, 특정 데이터 아이템을 조정하는 되돌이킬 수 있는 트랜잭션이
	하나 존재한다면, 똑같은 데이터 아이템을 조정하는 모든 다른 트랜잭션은
	non-blocking 시맨틱을 가질 수 없습니다.
\iffalse

\item	Just execute the code: if there are any TM-unfriendly operations
	in the function, simply abort the transaction.
	Unfortunately, this approach makes it impossible for the compiler
	to determine whether a given group of transactions may be safely
	composed.
	One way to permit composability regardless is irrevocable
	transactions, however, current implementations permit only a
	single irrevocable transaction to proceed at any given time,
	which can severely limit performance and scalability.
	Irrevocable transactions also seem to rule out use of manual
	transaction-abort operations.
	Finally, if there is an irrevocable transaction manipulating
	a given data item, any other transaction manipulating that
	same data item cannot have non-blocking semantics.
\fi
\item	함수 선언 부분을 어떤 함수들이 TM 친화적인지 알리도록 장식합니다.
	이렇게 되면, 이 장식들은 컴파일러의 타입 시스템에 의해 강제될 수
	있습니다.
	물론, 많은 언어들에 있어서, 이는 연관된 시간 딜레이와 함께 언어의
	확장이 제시되고, 표준화 되고, 구현될 것을 필요로 합니다.
	그렇다고는 하나, 그런 표준화를 위한 노력이 이미 진행
	중입니다~\cite{Ali-Reza-Adl-Tabatabai2009CppTM}.
\item	앞에서와 같이, 트랜잭션 안에서의 함수의 동적 링킹과 로딩을 불허합니다.
\iffalse
	
\item	Decorate the function declarations indicating which functions
	are TM-friendly.
	These decorations can then be enforced by the compiler's type system.
	Of course, for many languages, this requires language extensions
	to be proposed, standardized, and implemented, with the
	corresponding time delays.
	That said, the standardization effort is already in
	progress~\cite{Ali-Reza-Adl-Tabatabai2009CppTM}.
\item	As above, disallow dynamic linking and loading of functions from
	within transactions.
\fi
\end{enumerate}

물론 I/O 오퍼레이션은 TM 의 알려진 약점이고, 동적 링킹과 로딩은 I/O 의 또다른
특수 케이스 중 하나로 여겨질 수 있습니다.
더도 아니고 덜도 아니고, TM 의 제안자들은 이 문제를 해결하거나, TM 이 병렬
프로그래머의 도구상자의 여러 도구들 가운데 하나일 뿐이라고 스스로를 인정하게
해야 합니다.
(공평을 위해 말하자면, 많은 수의 TM 제안자들이 그들 스스로는 TM 외의 것들도
갖춘 세계에 존재하는 존재 가운데 하나일 뿐임을 인정해 왔습니다.)
\iffalse

I/O operations are of course a known weakness of TM, and dynamic linking
and loading can be thought of as yet another special case of I/O.
Nevertheless, the proponents of TM must either solve this problem, or
resign themselves to a world where TM is but one tool of several in the
parallel programmer's toolbox.
(To be fair, a number of TM proponents have long since resigned themselves
to a world containing more than just TM.)
\fi

\subsubsection{Memory-Mapping Operations}
\label{sec:future:Memory-Mapping Operations}

락 기반의 크리티컬 섹션 내에서 (\co{mmap()}, \co{shmat()}, 그리고
\co{munmap()}~\cite{TheOpenGroup1997SUS} 과 같은) 메모리 매핑 오퍼레이션들을
사용하는건 완전히 합법적이고, 적어도 원칙적으로는, userspace-RCU read-side
크리티컬 섹션 내에서의 사용도 드렇습니다.
그런 오퍼레이션을 트랜잭션 내에서 수행하려 시도하면 무슨 일이 벌어질까요?
더 나아가서, 다시 매핑된 메모리 영역이 현재 쓰레드의 트랜잭션에 사용되는 변수를
담고 있다면 어떻게 될까요?
그리고 이 메모리 영역이 다른 스레드의 트랜잭션에 사용되는 변수를 담고 있다면
어떻게 될까요?

대부분의 락킹 기능들이 락 변수들을 다시 매핑하는 행위의 결과를 정의하지
않는다는 점을 놓고 보면 TM 시스템의 메타데이터가 다시 매핑되는 케이스의 경우는
고려하지 않아도 될것입니다.

TM 에서 사용 가능한 메모리 매핑에서의 선택지들이 여기 있습니다:
\iffalse

It is perfectly legal to execute memory-mapping operations (including
\co{mmap()}, \co{shmat()}, and \co{munmap()}~\cite{TheOpenGroup1997SUS})
within a lock-based critical section, and, at least in principle, from
within a userspace-RCU read-side critical section.
What happens when you attempt to execute such an operation from within
a transaction?
More to the point, what happens if the memory region being remapped
contains some variables participating in the current thread's transaction?
And what if this memory region contains variables participating in some
other thread's transaction?

It should not be necessary to consider cases where the TM system's
metadata is remapped, given that most locking primitives do not define
the outcome of remapping their lock variables.

Here are some memory-mapping options available to TM:
\fi

\begin{enumerate}
\item	트랜잭션 안에서의 메모리 재 매핑은 불법으로 간주되어서, 모든 메모리 재
	매핑을 둘러싼 트랜잭션이 abort 됩니다.
	이는 일을 단순화 시킵니다만, TM 이 크리티컬 섹션 내에서 재 매핑을
	처리할 수 있는 동기화 기능들과 상호작용할 수 있을 것을 필요로 합니다.
\item	트랜잭션 안에서의 메모리 재 매핑은 불법으로 간주되고, 컴파일러가 이
	금지사항을 강제할 수 있도록 돕습니다.
\item	트랜잭션 내에서의 메모리 매핑은 합법입니다만, 매핑된 영역 안에 변수를
	가지고 있는 모든 다른 트랜잭션들은 abort 됩니다.
\item	트랜잭션 내에서의 메모리 매핑은 합법입니다만, 매핑되는 영역이 현재
	트랜잭션의 메모리 사용영역과 겹친다면 그 매핑 오퍼레이션은 실패합니다.
\item	트랜잭션 안에서든 밖에서든 행해지는 모든 메모리 매핑 오퍼레이션은
	시스템의 모든 트랜잭션의 메모리 사용 범위와 겹쳐지는지를 체크합니다.
	만약 겹친다면, 해당 메모리 매핑 오퍼레이션은 실패합니다.
\item	시스템의 어떤 트랜잭션의 메모리 사용 영역과 겹치는 메모리 매핑
	오퍼레이션의 영향은 TM conflict manager 에 의해 결정되는데, 메모리 매핑
	오퍼레이션을 실패시킬지 모든 충돌하는 트랜잭션들을 abort 시킬지를
	동적으로 결정합니다.
\iffalse

\item	Memory remapping is illegal within a transaction, and will result
	in all enclosing transactions being aborted.
	This does simplify things somewhat, but also requires that TM
	interoperate with synchronization primitives that do tolerate
	remapping from within their critical sections.
\item	Memory remapping is illegal within a transaction, and the
	compiler is enlisted to enforce this prohibition.
\item	Memory mapping is legal within a transaction, but aborts all
	other transactions having variables in the region mapped over.
\item	Memory mapping is legal within a transaction, but the mapping
	operation will fail if the region being mapped overlaps with
	the current transaction's footprint.
\item	All memory-mapping operations, whether within or outside a
	transaction, check the region being mapped against the memory
	footprint of all transactions in the system.
	If there is overlap, then the memory-mapping operation fails.
\item	The effect of memory-mapping operations that overlap the memory
	footprint of any transaction in the system is determined by the
	TM conflict manager, which might dynamically determine whether
	to fail the memory-mapping operation or abort any conflicting
	transactions.
\fi
\end{enumerate}

\co{munmap()} 은 메모리의 관련된 영역을 매핑되지 않은 채로 놔두어서, 추가적인
재미있는 영향을 가질 수 있음은 알아둘만 합니다.\footnote{
	매핑과 매핑 해제 사이의 이 차이점은 Josh Triplett 에 의해 이야기
	되었습니다.}
\iffalse

It is interesting to note that \co{munmap()} leaves the relevant region
of memory unmapped, which could have additional interesting
implications.\footnote{
	This difference between mapping and unmapping was noted by
	Josh Triplett.}
\fi

\subsubsection{Debugging}
\label{sec:future:Debugging}

브레이크포인트와 같은 일반적인 디버깅 오퍼레이션들은 락 기반의 크리티컬 섹션과
userspace-RCU read-side 크리티컬 섹션 안에서는 평범하게 동작합니다.
하지만, 초기의 트랜잭셔널 메모리 하드웨어 구현~\cite{DaveDice2009ASPLOSRockHTM}
에서는 트랜잭션 안에서의 exception 은 그 트랜잭션을 abort 시켰는데, 이는
브레이크포인트는 그를 둘러싼 트랜잭션들을 abort 시켰음을 의미합니다.

트랜잭션은 어떻게 디버깅 될 수 있을까요?
\iffalse

The usual debugging operations such as breakpoints work normally within
lock-based critical sections and from usespace-RCU read-side critical sections.
However, in initial transactional-memory hardware
implementations~\cite{DaveDice2009ASPLOSRockHTM} an exception within
a transaction will abort that transaction, which in turn means that
breakpoints abort all enclosing transactions.

So how can transactions be debugged?
\fi

\begin{enumerate}
\item	브레이크포인트를 담고 있는 트랜잭션에 대해서 소프트웨어 에뮬레이션
	테크닉을 사용합니다.
	물론, 모든 트랜잭션의 범위 내에서 브레이크포인트가 설정될 때마다 모든
	트랜잭션을 에뮬레이션 해야 할 필요가 있을 겁니다.
	만약 런타임 시스템이 특정 브레이크포인트가 트랜잭션의 범위 안에 있는지
	아닌지 여부를 결정할 수가 없다면, 안전한 쪽으로 있기 위해 모든
	트랜잭션을 에뮬레이션 해야 할 필요가 있을 겁니다.
	하지만, 이 방법은 상당한 오버헤드를 가져오게 될 것이어서, 쫓고 있는
	버그를 찾기 어렵게 만들 겁니다.
\iffalse

\item	Use software emulation techniques within transactions containing
	breakpoints.
	Of course, it might be necessary to emulate all transactions
	any time a breakpoint is set within the scope of any transaction.
	If the runtime system is unable to determine whether or not a
	given breakpoint is within the scope of a transaction, then it
	might be necessary to emulate all transactions just to be on
	the safe side.
	However, this approach might impose significant overhead, which
	might in turn obscure the bug being pursued.
\fi
\item	브레이크포인트 exception 을 처리할 수 있는 하드웨어 TM 구현만을
	사용합니다.
	불행히도, 지금 (2008년 9월) 글을 쓰는 시점에서는, 모든 그런 구현들은
	연구용 프로토타입들 뿐입니다.
\item	하드웨어 TM 구현들의 더 간단한 것들보다는 (매우 간략히 말해서) 더
	exception 을 잘 처리해주는 소프트웨어 TM 구현만을 사용합니다.
	물론, 소프트웨어 TM 은 하드웨어 TM 보다 높은 오버헤드를 갖는 경향이
	있으므로, 이 방법이 모든 상황에서 적합하지는 않을 겁니다.
\item	더 주의깊게 프로그램을 짜서, 트랜잭션 안에서는 버그가 존재하지 않는
	것을 첫번째 목적으로 합니다.
	이걸 어떻게 할 수 있는지 알아낸다면, 부디 모두에게 그 비밀을 공유해
	주세요!
\iffalse

\item	Use only hardware TM implementations that are capable of
	handling breakpoint exceptions.
	Unfortunately, as of this writing (September 2008), all such
	implementations are strictly research prototypes.
\item	Use only software TM implementations, which are
	(very roughly speaking) more tolerant of exceptions than are
	the simpler of the hardware TM implementations.
	Of course, software TM tends to have higher overhead than hardware
	TM, so this approach may not be acceptable in all situations.
\item	Program more carefully, so as to avoid having bugs in the
	transactions in the first place.
	As soon as you figure out how to do this, please do let everyone
	know the secret!
\fi
\end{enumerate}

트랜잭셔널 메모리가 다른 동기화 메커니즘들에 비해서 더 나은 생산성을 가져다 줄
것임을 믿을 만한 몇가지 이유가 있습니다만, 전통적인 디버깅 테크닉들이
트랜잭션에 적용될 수가 없다면 이런 개선점들은 사라질 수도 있을 듯 합니다.
특히나 트랜잭셔널 메모리에 익숙지 않은 사람이 커다란 트랜잭션을 트랜잭셔널
메모리로 사용하려 한다면 특히나 그렇게 될 가능성이 큽니다.
대조적으로, 거친 ``탑 클래스의'' 프로그래머들은, 특히나 작은 트랜잭션에
대해서는 그런 디버깅의 필요 자체가 없게 할 수도 있을 겁니다.

따라서, 트랜잭셔널 메모리가 익숙지 않은 프로그래머들에게도 생산성을 약속해 줄
수 있으려면, 디버깅 문제는 풀릴 필요가 있습니다.
\iffalse

There is some reason to believe that transactional memory will deliver
productivity improvements compared to other synchronization mechanisms,
but it does seem quite possible that these improvements could easily
be lost if traditional debugging techniques cannot be applied to
transactions.
This seems especially true if transactional memory is to be used by
novices on large transactions.
In contrast, macho ``top-gun'' programmers might be able to dispense with
such debugging aids, especially for small transactions.

Therefore, if transactional memory is to deliver on its productivity
promises to novice programmers, the debugging problem does need to
be solved.
\fi

\subsection{Synchronization}
\label{sec:future:Synchronization}

어느날 트랜잭셔널 메모리가 모두에게의 모든 것이 된다면, 다른 동기화 메커니즘을
사용할 필요가 없어질 겁니다.
그렇게 되기 전까지는, 트랜잭셔널 메모리가 할 수 없는 것을 할 수 있거나 주어진
상황에서 보다 자유롭게 동작할 수 있는 동기화 메커니즘과 함께 사용되어야
할겁니다.
다음 섹션들은 이 영역에 존재하는 현재의 challenge 들을 정리해 봅니다.
\iffalse

If transactional memory someday proves that it can be everything to everyone,
it will not need to interact with any other synchronization mechanism.
Until then, it will need to work with synchronization mechanisms that
can do what it cannot, or that work more naturally in a given situation.
The following sections outline the current challenges in this area.
\fi

\subsubsection{Locking}
\label{sec:future:Locking}

락을 잡은채로 다른 락을 잡는 것은 일반적인 일인데, 최소한, 데드락을 막기 위해
일반적으로 잘 알려진 소프트웨어 엔지니어링 테크닉이 사용된다면 이는 잘
동작합니다.
RCU read-side 크리티컬 섹션 안에서 락을 잡는 것은 흔하지 않은데, RCU read-side
기능들은 락 기반의 데드락 사이클에 포함될 수가 없기 때문에 이는 데드락에 대한
고려를 줄여줍니다.
하지만 트랜잭션 안에서 락을 잡으려 한다면 무슨 일이 일어날까요?
\iffalse

It is commonplace to acquire locks while holding other locks, which works
quite well, at least as long as the usual well-known software-engineering
techniques are employed to avoid deadlock.
It is not unusual to acquire locks from within RCU read-side critical
sections, which eases deadlock concerns because RCU read-side primitives
cannot participate in lock-based deadlock cycles.
But what happens when you attempt to acquire a lock from within a transaction?
\fi

이론적으로, 그에 대한 답은 간단합니다: 해당 락을 나타내는 데이터 구조를
트랜잭션의 한 부분으로 잡으면, 모든 것이 잘 동작합니다.
실제로는, TM 시스템 의 상세한 구현에 따라서 여러개의 명확치 않은 복잡한
문제~\cite{Volos2008TRANSACT} 가 존재할 수 있습니다.
이런 복잡한 문제들은 해결될 수 있긴 하지만, 트랜잭션의 바깥에서 잡는 락들에
대해서는 45\,\% 의 오버헤드 증가, 트랜잭션 안에서 잡는 락들에 대해서는 300\,\% 의
오버헤드 증가를 그 비용으로 갖게 됩니다.
적은 양의 락킹을 포함하고 있는 트랜잭션을 사용하는 프로그램에서 이런 오버헤드는
받아들일 만도 하게씾만, 때때로 트랜잭션을 사용하고자 하는 락 기반의 제품 수준
프로그램에서는 받아들여질 수 없을 겁니다.
\iffalse

In theory, the answer is trivial: simply manipulate the data structure
representing the lock as part of the transaction, and everything works
out perfectly.
In practice, a number of non-obvious complications~\cite{Volos2008TRANSACT}
can arise, depending on implementation details of the TM system.
These complications can be resolved, but at the cost of a 45\,\% increase in
overhead for locks acquired outside of transactions and a 300\,\% increase
in overhead for locks acquired within transactions.
Although these overheads might be acceptable for transactional
programs containing small amounts of locking, they are often completely
unacceptable for production-quality lock-based programs wishing to use
the occasional transaction.
\fi

\begin{enumerate}
\item	락킹에 친화적인 TM 구현만을 사용합니다.
	불행히도, 락킹에 친화적이지 않은 구현들은 성공적인 트랜잭션을 위한 낮은
	오버헤드와 극단적으로 큰 트랜잭션을 수용할 수 있는 능력과 같은 장점을
	일부 갖습니다.
\item	TM 을 락 기반의 프로그램에 사용할 때 ``조금만'' TM 을 사용해서 락킹에
	친화적인 TM 구현의 한계점들을 받아들입니다.
\item	락킹 기반의 기존 시스템을 전부 무시해서, 모든 것을 트랜잭션 기준으로
	다시 구현합니다.
	이 방법은 지지하는 사람이 부족하지 않습니다만, 이는 이 시리즈에서
	이야기된 모든 문제들이 해결되기를 필요로 합니다.
	이 문제들을 해결하는데 걸리는 시간 동안, 경쟁상대인 동기화 메커니즘들
	역시 개선될 수 있을 겁니다.
\item	TxLinux~\cite{ChistopherJRossbach2007a} 그룹에 의해서 행해진 것처럼, TM
	을 락킹 기반의 시스템에서의 최적화로만 사용합니다.
	이 방법은 잘 동작할 듯 들리지만, (데드락을 막기 위한 필요와 같은) 락킹
	설계 문제들을 그대로 남겨둡니다.
\item	락킹 기능들로 추가되는 오버헤드를 줄이려 노력합니다.
\iffalse

\item	Use only locking-friendly TM implementations.
	Unfortunately, the locking-unfriendly implementations have some
	attractive properties, including low overhead for successful
	transactions and the ability to accommodate extremely large
	transactions.
\item	Use TM only ``in the small'' when introducing TM to lock-based
	programs, thereby accommodating the limitations of
	locking-friendly TM implementations.
\item	Set aside locking-based legacy systems entirely, re-implementing
	everything in terms of transactions.
	This approach has no shortage of advocates, but this requires
	that all the issues described in this series be resolved.
	During the time it takes to resolve these issues, competing
	synchronization mechanisms will of course also have the
	opportunity to improve.
\item	Use TM strictly as an optimization in lock-based systems, as was
	done by the TxLinux~\cite{ChistopherJRossbach2007a} group.
	This approach seems sound, but leaves the locking design
	constraints (such as the need to avoid deadlock) firmly in place.
\item	Strive to reduce the overhead imposed on locking primitives.
\fi
\end{enumerate}

많은 사람에게 놀랑루 수 있는 TM 과 락킹 사이를 잇는 문제가 존재할 수 있다는
사실은 실제 제품 소프트웨어에서 새로운 메커니즘과 기능들을 사용해야 함을
강조합니다.
다행히도, 오픈 소스의 진보는 연구자들을 포함해서 모두에게 많은 그런
소프트웨어가 사용 가능함을 의미합니다.
\iffalse

The fact that there could possibly be a problem interfacing TM and locking
came as a surprise to many, which underscores the need to try out new
mechanisms and primitives in real-world production software.
Fortunately, the advent of open source means that a huge quantity of
such software is now freely available to everyone, including researchers.
\fi

\subsubsection{Reader-Writer Locking}
\label{sec:future:Reader-Writer Locking}

락을 잡은 채로 reader-writer 락의 읽기 권한 획득을 하는 것은 잘 동작하며,
적어도 데드락을 막기 위해 잘 알려진 소프트웨어 엔지니어링 테크닉이 사용되는
곳에서는 흔한 일입니다.
RCU read-side 크리티컬 섹션 아넹서 reader-writer 락의 읽기 권한 획득을 하는 것
역시 동작하며, RCU read-side 기능들은 락 기반의 데드락 사이클에 포함될 수가
없기 때문에 데드락에 대한 주의를 완화시켜줍니다.
하지만 트랜잭션 안에서 reader-writer 락의 읽기 권한을 획득하려 하면 무슨 일이
벌어질까요?
\iffalse

It is commonplace to read-acquire reader-writer locks while holding
other locks, which just works, at least as long as the usual well-known
software-engineering techniques are employed to avoid deadlock.
Read-acquiring reader-writer locks from within RCU read-side critical
sections also works, and doing so eases deadlock concerns because RCU
read-side primitives cannot participate in lock-based deadlock cycles.
But what happens when you attempt to read-acquire a reader-writer lock
from within a transaction?
\fi

안타깝게도, 트랜잭션 안에서 전통적인 카운터 기반의 reader-writer 락의 읽기
권한을 획득하려는 시도는 reader-writer 락의 목적을 와해시킬 수 있습니다.
이를 보기 위해, 같은 reader-writer 락의 읽기 권한을 동시에 획득하려 하는 한쌍의
트랜잭션을 생각해 봅시다.
읽기 권한 획득은 reader-writer 락의 데이터 구조의 수정에 연관되기 때문에,
충돌이 날 것이고, 이는 두 트랜잭션 가운데 하나를 roll back 시킬 겁니다.
이 동작은 reader-writer 락의 목적인 동시적인 읽기 쓰레드들을 허용하는 것과 전혀
일관적이지 못합니다.

TM 에서 해볼 수 있는 선택사항 몇가지가 다음과 같습니다:
\iffalse

Unfortunately, the straightforward approach to read-acquiring the
traditional counter-based reader-writer lock within a transaction defeats
the purpose of the reader-writer lock.
To see this, consider a pair of transactions concurrently attempting to
read-acquire the same reader-writer lock.
Because read-acquisition involves modifying the reader-writer lock's
data structures, a conflict will result, which will roll back one of
the two transactions.
This behavior is completely inconsistent with the reader-writer lock's
goal of allowing concurrent readers.

Here are some options available to TM:
\fi

\begin{enumerate}
\item	Per-CPU 또는 per-thread reader-writer 락킹~\cite{WilsonCHsieh92a} 을
	사용해서, 특정 CPU (또는, 쓰레드) 가 락의 읽기 권한 획득 시에 로컬
	데이터만을 조정하도록 합니다.
	이는 동시에 락의 읽기 권한을 획득하려는 두개의 트랜잭션이 충돌하는
	현상을 막아서 원래 의도대로 동시에 둘 다 진행되도록 합니다.
	안타깝게도, (1)~per-CPU/thread 락킹에서의 쓰기 권한 획득 오버헤드는
	상당히 크며, (2)~per-CPU/thread 락킹의 메모리 오버헤드가 금지될 수
	있으며, (3)~이 변환은 여러분이 문제가 되는 소스 코드에 접근 권한이 있을
	때에만 사용할 수 있는 방법입니다.
	더 최근의, 다른 확장성 있는 reader-writer
	락~\cite{YossiLev2009SNZIrwlock} 은 이런 문제들을 일부 또는 전부 없앨
	수도 있습니다.
\iffalse

\item	Use per-CPU or per-thread reader-writer
	locking~\cite{WilsonCHsieh92a}, which allows a
	given CPU (or thread, respectively) to manipulate only local
	data when read-acquiring the lock.
	This would avoid the conflict between the two transactions
	concurrently read-acquiring the lock, permitting both to proceed,
	as intended.
	Unfortunately, (1)~the write-acquisition overhead of
	per-CPU/thread locking can be extremely high, (2)~the memory
	overhead of per-CPU/thread locking can be prohibitive, and
	(3)~this transformation is available only when you have access to
	the source code in question.
	Other more-recent scalable
	reader-writer locks~\cite{YossiLev2009SNZIrwlock}
	might avoid some or all of these problems.
\fi
\item	TM 을 락 기반의 프로그램에 접목할 때 ``작은 부분'' 에서만 TM 을
	사용해서 트랜잭션 안에서 reader-writer 락의 일기 권한 획득을 막습니다.
\item	락킹 기반의 기존 시스템을 완전히 분리시켜서, 모든 것을 트랜잭션으로
	다시 구현합니다.
	이 방법은 지지자가 부족하지는 않지만, 이는 이 글에서 설명된 \emph{모든}
	문제들이 해결될 것을 필요로 합니다.
	이 문제들을 해결하는 시간이 지나는 동안, 다른 경쟁상대 동기화
	메커니즘들이 더 나아질 가능성도 물론 존재합니다.
\item	TxLinux~\cite{ChistopherJRossbach2007a} 그룹에 의해서 그랬던 것처럼 TM
	을 완전히 락 기반의 시스템의 최적화에만 사용합니다.
	이 방법은 잘 동작할 듯 들리지만, 락킹 설계의 (데드락 방지 등과 같은)
	제약점 들을 그대로 둡니다.
	더 나아가서, 이 방법은 여러 트랜잭션들이 같은 락에 대해 읽기 권한을
	획득하려 핧 때에 불필요한 트랜잭션 롤백을 초래할 수 있습니다.
\iffalse

\item	Use TM only ``in the small'' when introducing TM to lock-based
	programs, thereby avoiding read-acquiring reader-writer locks
	from within transactions.
\item	Set aside locking-based legacy systems entirely, re-implementing
	everything in terms of transactions.
	This approach has no shortage of advocates, but this requires
	that \emph{all} the issues described in this series be resolved.
	During the time it takes to resolve these issues, competing
	synchronization mechanisms will of course also have the
	opportunity to improve.
\item	Use TM strictly as an optimization in lock-based systems, as was
	done by the TxLinux~\cite{ChistopherJRossbach2007a} group.
	This approach seems sound, but leaves the locking design
	constraints (such as the need to avoid deadlock) firmly in place.
	Furthermore, this approach can result in unnecessary transaction
	rollbacks when multiple transactions attempt to read-acquire
	the same lock.
\fi
\end{enumerate}

물론, 배타적 락킹에 대해서도 그랬던 것처럼, TM 을 reader-writer 락킹에 조합하는
것을 둘러싼 또다른 명확치 않은 이슈들이 있을 수 있습니다.
\iffalse

Of course, there might well be other non-obvious issues surrounding
combining TM with reader-writer locking, as there in fact were with
exclusive locking.
\fi

\subsubsection{RCU}
\label{sec:future:RCU}

Read-copy update (RCU) 는 리눅스 커널에서 주로 사용되기 대문에, RCU 와 TM 을
조합하려는 학계에서의 시도는 없었을 거라고 생각할 수 있을 겁니다.\footnote{
	하지만, user-space
	RCU~\cite{MathieuDesnoyers2009URCU,MathieuDesnoyers2012URCU} 의
	발전으로 커널 안에서만 사용될 것이라는 변명은 힘을 잃습니다.}
하지만, 오스틴 텍사스 대학교의 TxLinux 그룹은 다른 선택이
없었습니다~\cite{ChistopherJRossbach2007a}.
그들은 RCU 를 사용하는 리눅스 2.6 커널에 TM 을 적용했기 때문에 TM 이 RCU
업데이트를 위한 락킹의 자리를 대신하는 방식으로 TM 과 RCU 를 조합해야만
했습니다.
해당 논문은 RCU 구현의 락들 (예: \co{rcu_ctrlblk.lock}) 이 트랜잭션으로
변환되었다고 이야기 하긴 했지만, 안타깝게도 RCU 기반의 업데이트데 사용된 락들
(예: \co{dcache_lock}) 에는 무슨 일이 있었는지는 이야기 되지 않았습니다.
\iffalse

Because read-copy update (RCU) finds its main use in the Linux kernel,
one might be forgiven for assuming that there had been no academic work
on combining RCU and TM.\footnote{
	However, the in-kernel excuse is wearing thin with the advent
	of user-space RCU~\cite{MathieuDesnoyers2009URCU,MathieuDesnoyers2012URCU}.}
However, the TxLinux group from the University of Texas at Austin had
no choice~\cite{ChistopherJRossbach2007a}.
The fact that they applied TM to the Linux 2.6 kernel, which uses RCU,
forced them to integrate TM and RCU, with TM taking the place of locking
for RCU updates.
Unfortunately, although the paper does state that the RCU implementation's
locks (e.g., \co{rcu_ctrlblk.lock}) were converted to transactions,
it is silent about what happened to locks used in RCU-based updates
(e.g., \co{dcache_lock}).
\fi

RCU 는 읽기 쓰레드들과 업데이트 쓰레드들이 동시적으로 수행될 수 있게 하며, 더
나아가서 RCU 읽기 쓰레드들은 업데이트가 진행되고 있는 데이터에도 접근할 수
있도록 한다는 점을 알아두는게 중요합니다.
물론, 성능, 확장성, real-time 응답성 에서의 이득이 될 수 있는, 이런 RCU 의
속성은 TM 의 원자성 속성을 무시하고 동작합니다.

그러니 TM 기반의 업데이트가 동시의 RCU 읽기 쓰레드들과 상호작용할 때에는 어떻게
해야 할까요?
몇가지 가능한 것들은 다음과 같습니다:
\iffalse

It is important to note that RCU permits readers and updaters to run
concurrently, further permitting RCU readers to access data that is in
the act of being updated.
Of course, this property of RCU, whatever its performance, scalability,
and real-time-response benefits might be, flies in the face of the
underlying atomicity properties of TM.

So how should TM-based updates interact with concurrent RCU readers?
Some possibilities are as follows:
\fi

\begin{enumerate}
\item	RCU 읽기 쓰레드들이 충돌하는 동시의 TM 업데이트들을 abort 시킵니다.
	이는 실제로 TxLinux 프로젝트에서 취해진 방법입니다.
	이 방법은 RCU semantic 을 유지하고, 또한 RCU 의 read-side 성능, 확장성,
	그리고 real-time 응답 속성을 유지합니다만, 불행히도, 불필요하게
	충돌되는 업데이트들을 abort 시키는 부작용을 갖게 됩니다.
	최악의 경우에는 길게 유지되는 RCU 읽기 쓰레드들은 모든 업데이트
	쓰레드들을 진행하지 못하게 만들 잠재성을 가지고 있어서 이는 이론적으로
	시스템이 멎게 되는 결과를 초래할 수 있습니다.
	또한, 모든 TM 구현이 이런 방법에 필요한 강한 원자성을 제공하는 것은
	아닙니다.
\iffalse

\item	RCU readers abort concurrent conflicting TM updates.
	This is in fact the approach taken by the TxLinux project.
	This approach does preserve RCU semantics, and also preserves
	RCU's read-side performance, scalability, and real-time-response
	properties, but it does have the unfortunate side-effect of
	unnecessarily aborting conflicting updates.
	In the worst case, a long sequence of RCU readers could
	potentially starve all updaters, which could in theory result
	in system hangs.
	In addition, not all TM implementations offer the strong atomicity
	required to implement this approach.
\fi
\item	충돌하는 TM 업데이트들과 동시에 수행되는 RCU 읽기 쓰레드들은 충돌하는
	어떤 RCU load 로부터든 과거의 (이전 트랜잭션의) 값을 가져오도록 합니다.
	이는 RCU semantic 과 성능을 유지하며, RCU 업데이트 쪽의 starvation 을
	방지합니다.
	하지만, 모든 TM 구현이 수행중인 트랜잭션에 의해 임시적으로 업데이트된
	변수들의 기존 값들에 대한 빠른 접근을 제공할 수 있는 건 아닙니다.
	구체적으로, 기존 값들을 로그에 유지하는 (그렇게 함으로써 훌륭한 TM 커밋
	성능을 제공하는) 로그 기반의 TM 구현은 이런 방법에 대해서는 즐거울 수
	없을 겁니다.
	RCU 가 커다란 길이의 트랜잭션 구현 안에서 기존 값을 접근하는 것을
	가능하게 하기 위해 아마도 \co{rcu_dereference()} 기능이 사용될 수 있을
	겁니다만, 성능은 여전히 문제가 될 수 있을 겁니다.
	더도 아니고 덜도 아니고, 이런 방법으로 쉽고 효율적으로 RCU 와 조합될 수
	있는 유명한 TM 구현들이
	존재합니다~\cite{DonaldEPorter2007TRANSACT,PhilHoward2011RCUTMRBTree,PhilipWHoward2013RCUrbtree}.
\iffalse

\item	RCU readers that run concurrently with conflicting TM updates
	get old (pre-transaction) values from any conflicting RCU loads.
	This preserves RCU semantics and performance, and also prevents
	RCU-update starvation.
	However, not all TM implementations can provide timely access
	to old values of variables that have been tentatively updated
	by an in-flight transaction.
	In particular, log-based TM implementations that maintain
	old values in the log (thus making for excellent TM commit
	performance) are not likely to be happy with this approach.
	Perhaps the \co{rcu_dereference()} primitive can be leveraged
	to permit RCU to access the old values within a greater range
	of TM implementations, though performance might still be an issue.
	Nevertheless, there are popular TM implementations that can
	be easily and efficiently integrated with RCU in this
	manner~\cite{DonaldEPorter2007TRANSACT,PhilHoward2011RCUTMRBTree,
	PhilipWHoward2013RCUrbtree}.
\fi
\item	RCU 읽기 쓰레드가 현재 수행중인 트랜잭션과 충돌하는 액세스를
	수행한다면, 그 RCU 액세스는 충돌하는 트랜잭션이 커밋하거나 abort 되기
	전까지 딜레이 되도록 합니다.
	이 방법은 RCU semantic 을 유지합니다만, RCU 의 성능이나 real-time
	응답은 유지하지 않으며, 특히나 긴시간 수행되는 트랜잭션의 존재 시에는
	더욱 그렇습니다.
	또한, 모든 TM 구현에서 충돌되는 액세스들을 딜레이 시키는게 가능하지는
	않습니다.
	그렇다고는 하나, 이 방법은 작은 트랜잭션들만을 지원하는 하드웨어 TM
	구현들에는 확실히 합리적일 것으로 보입니다.
\iffalse

\item	If an RCU reader executes an access that conflicts with an
	in-flight transaction, then that RCU access is delayed until
	the conflicting transaction either commits or aborts.
	This approach preserves RCU semantics, but not RCU's performance
	or real-time response, particularly in presence of long-running
	transactions.
	In addition, not all TM implementations are capable of delaying
	conflicting accesses.
	That said, this approach seems eminently reasonable for hardware
	TM implementations that support only small transactions.
\fi
\item	RCU 읽기 쓰레드들을 트랜잭션으로 변환시킵니다.
	이 방법은 RCU 가 모든 TM 구현과 완전히 호환될 것을 보장합니다만, RCU
	read-side 크리티컬 섹션에서의 TM 의 롤백이 가능해서 RCU 의 real-time
	응답 보장을 지켜지지 못하게 하고, 따라서 RCU 의 read-side 성능을
	떨어뜨립니다.
	더 나아가서, 이 방법은 RCU read-side 크리티컬 섹션들 가운데 하나라도
	사용되는 TM 구현이 처리할 수 없는 오퍼레이션들을 포함하고 있는 경우에
	대해서는 적절치 못합니다.
\iffalse

\item	RCU readers are converted to transactions.
	This approach pretty much guarantees that RCU is compatible with
	any TM implementation, but it also imposes TM's rollbacks on RCU
	read-side critical sections, destroying RCU's real-time response
	guarantees, and also degrading RCU's read-side performance.
	Furthermore, this approach is infeasible in cases where any of
	the RCU read-side critical sections contains operations that
	the TM implementation in question is incapable of handling.
\fi
\item	RCU 의 많은 update-side 사용들이 새로운 데이터 구조를 공개하기 위해
	하나의 포인터를 수정합니다.
	이런 일부 경우들에 있어서, RCU 는 이어서 롤백될 트랜잭션에 의핸 포니터
	업데이트를 안전하게 볼 수 있는데, 해당 트랜잭션이 메모리 순서 규칙을
	지켜주는 한, 그리고 이 롤백 프로세스가 연관된 구조체를 메모리에서
	해제시키는데에 \co{call_rcu()} 를 사용하는 한은 그렇습니다.
	안타깝게도, 모든 TM 구현들이 트랜잭션 안에서의 메모리 배리어를 따르지는
	않습니다.
	명백하게, 이 생각은 트랜잭션들은 원자적일 것으로 여겨지기 때문에,
	트랜잭션 안에서의 액세스들의 순서는 문제가 되지 않을 것으로 여겨진다는
	것입니다.
\item	RCU 업데이트 에서의 TM 의 사용을 금지시킵니다.
	이는 잘 동작할 것으로 보장됩니다만, 약간 제한적일 것으로 보입니다.
\iffalse

\item	Many update-side uses of RCU modify a single pointer to publish
	a new data structure.
	In some of these cases, RCU can safely be permitted to see a
	transactional pointer update that is subsequently rolled back,
	as long as the transaction respects memory ordering and as long
	as the roll-back process uses \co{call_rcu()} to free up the
	corresponding structure.
	Unfortunately, not all TM implementations respect memory barriers
	within a transaction.
	Apparently, the thought is that because transactions are supposed
	to be atomic, the ordering of the accesses within the transaction
	is not supposed to matter.
\item	Prohibit use of TM in RCU updates.
	This is guaranteed to work, but seems a bit restrictive.
\fi
\end{enumerate}

추가적인 방법들 역시 모습을 드러내게 될 것으로 보이는데, 특히나 user-level RCU
구현의 발전에 따라서 특히 그렇습니다.\footnote{
	여러개의 앞의 대안들을 가져와준 TxLinux 그룹, Maged Michael, 그리고
	Josh Triplett 에게 감사를.}
\iffalse

It seems likely that additional approaches will be uncovered, especially
given the advent of user-level RCU implementations.\footnote{
	Kudos to the TxLinux group, Maged Michael, and Josh Triplett
	for coming up with a number of the above alternatives.}
\fi

\subsubsection{Extra-Transactional Accesses}
\label{sec:future:Extra-Transactional Accesses}

락 기반의 크리티컬 섹션 내에서, 동시에 액세스 되는 변수를 조정하는 것은 완전히
합법적이며, 심지어 그 락의 크리팈러 밖에서 수정되고 있는 변수에 액세스 하는
것도 합법적인데, 그런 흔한 예 중 하나는 통계적 카운터일 것입니다.
같은 일이 RCU read-side 크리티컬 섹션 안에서도 비슷하며, 이는 흔한 일입니다.

제품화된 데이터베이스 시스템에서도 흔한 ``dirty reads'' 라 불리는 이런
메커니즘을 놓고 보면, TM 의 제안자로부터 extra-transactional access 들이 상당한
관심을 받은 것은 놀랄만한 일이 아닌데, 완화된, 그리고 강화된
원자성~\cite{Blundell2006TMdeadlock} 이 이런 의미에서의 하나의 케이스입니다.

여기 extra-transactional 선택 사항들이 몇가지 있습니다:
\iffalse

Within a lock-based critical section, it is perfectly legal to manipulate
variables that are concurrently accessed or even modified outside that
lock's critical section, with one common example being statistical
counters.
The same thing is possible within RCU read-side critical
sections, and is in fact the common case.

Given mechanisms such as the so-called ``dirty reads'' that are
prevalent in production database systems, it is not surprising
that extra-transactional accesses have received serious attention
from the proponents of TM, with the concepts of weak and strong
atomicity~\cite{Blundell2006TMdeadlock} being but one case in point.

Here are some extra-transactional options:
\fi

\begin{enumerate}
\item	Extra-transactional 액세스로 인한 충돌은 항상 트랜잭션을 abort
	시킵니다.
	이는 강한 원자성입니다.
\item	Extra-transactional 액세스로 인한 충돌은 무시되어서, 트랜잭션 끼리의
	충돌만이 트랜잭션들을 abort 시킵니다.
	이는 완화된 원자성입니다.
\item	트랜잭션은 메모리를 할당할 때나 락 기반의 크리티컬 섹션과 상호작용할
	때와 같이 특수한 경우에 트랜잭션에서 할 수 없는 동작도 할 수 있도록
	허용됩니다.
\item	복수개의 트랜잭션에 의해 하나의 변수에 동시에 수행될 수 있는 일부
	오퍼레이션 (예를 들면, 더하기) 이 가능하게 하는 하드웨어 확장장치를
	만듭니다.
\item	트랜잭셔널 메모리에 완화된 semantic 을 가져옵니다.
	한가지 방법은 Section~\ref{sec:future:RCU} 에서 설명된 RCU 와의
	조합으로, Gramoli 와 Guerraoui 가 예를 들면 커다란 ``신축성 있는''
	트랜잭션들을 작은 트랜잭션들로의 제한적 분할로 (맥빠지는 성능과
	확장성에도 불구하고) 충돌 가능성을 줄이는 식의 많은 수의 완화된
	트랜잭션 방법을 연구했습니다~\cite{Gramoli:2014:DTP:2541883.2541900}.
\iffalse

\item	Conflicts due to extra-transactional accesses always abort
	transactions.
	This is strong atomicity.
\item	Conflicts due to extra-transactional accesses are ignored,
	so only conflicts among transactions can abort transactions.
	This is weak atomicity.
\item	Transactions are permitted to carry out non-transactional
	operations in special cases, such as when allocating memory or
	interacting with lock-based critical sections.
\item	Produce hardware extensions that permit some operations
	(for example, addition) to be carried out concurrently on a
	single variable by multiple transactions.
\item	Introduce weak semantics to transactional memory.
	One approach is the combination with RCU described in
	Section~\ref{sec:future:RCU}, while Gramoli and Guerraoui
	survey a number of other weak-transaction
	approaches~\cite{Gramoli:2014:DTP:2541883.2541900}, for example,
	restricted partitioning of large
	``elastic'' transactions into smaller transactions, thus
	reducing conflict probabilities (albeit with tepid performance
	and scalability).
	Perhaps further experience will show that some uses of
	extra-transactional accesses can be replaced by weak
	transactions.
\fi
\end{enumerate}

트랜잭션들은 어떤 다른 동기화 메커니즘과의 상호작용의 필요 없이 혼자 존재할 수
있도록 만들어진 것으로 보입니다.
만약 그렇다면, 트랜잭션들을 트랜잭션으로 인한 것이 아닌 액세스와 조합할 때에
많은 혼란과 복잡성이 등장하는 것은 놀라운 일이 아닙니다.
하지만 트랜잭션들이 격리된 데이터 구조로의 적은 업데이트로만 국한되어 있는 것이
아니라면, 또는 기존에 존재하던 커다란 병렬 코드와 상호작용하지 않는 새로운
프로그램에만 국한되어 있는 게 아니라면, 트랜잭션은 커다란 수준의 실용적인
효과를 내기 위해서는 트랜잭션이 아닌 액세스와 조합되어야만 합니다.
\iffalse

It appears that transactions were conceived as standing alone, with no
interaction required with any other synchronization mechanism.
If so, it is no surprise that much confusion and complexity arises when
combining transactions with non-transactional accesses.
But unless transactions are to be confined to small updates to isolated
data structures, or alternatively to be confined to new programs
that do not interact with the huge body of existing parallel code,
then transactions absolutely must be so combined if they are to have
large-scale practical impact in the near term.
\fi

% @@@ Huge transactions.  Or perhaps conflict handling.

\subsection{Discussion}
\label{sec:future:Discussion}

보편적인 TM 적용에 대한 문제들은 다음과 같은 결론을 이끌어냅니다:
\iffalse

The obstacles to universal TM adoption lead to the following
conclusions:
\fi

\begin{enumerate}
\item	TM 의 흥미로운 속성 가운데 하나는 트랜잭션들은 롤백과 재시도에 걸리기
	쉽다는 사실입니다.
	이 속성은 버퍼링 되지 않는 I/O, RPC, 메모리 매핑 오퍼레이션, 시간
	딜레이, 그리고 \co{exec()} 시스템 콜과 같이 되돌이킬 수 없는
	오퍼레이션들에 대한 TM 의 어려움을 암시합니다.
	이 속성은 또한 안타깝게도, 개발자에게 보여질 수 있는 형태로 존재하는,
	실패의 가능성으로 인해 존재하는 동기화 도구들에 내재된 복잡성을
	가져오는 결론을 이끌어 냅니다.
\item	Shpeisman 등~\cite{TatianaShpeisman2009CppTM} 에 의해 이야기된 TM 의
	또다른 흥미로운 속성은, TM 은 자신이 보호하는 데이터와 동기화가
	뒤엉켜진다는 것입니다.
	이 속성은 I/O, 메모리 매핑 오퍼레이션들, 트랜잭션 이외의 접근들, 그리고
	디버깅 브레이크포인트들과 연관되는 TM 의 문제들을 암시합니다.
	대조적으로, 락킹과 RCU 를 포함하는 전통적인 동기화 기능들은 동기화
	기능들과 그들이 보호하는 데이터 사이의 분명한 구분을 유지합니다.
\item	TM 분야의 많은 사용자들의 이야기된 목표들 가운데 하나는 커다란 순차적
	프로그램의 병렬화를 쉽게 하는 것입니다.
	그렇게 되면, 개별적인 트랜잭션들은 흔히 순차적으로 수행될 것으로
	기대되어서, TM 의 멀티쓰레드로 수행되는 트랜잭션들에 대한 문제들을
	상당수 설명할 수도 있을 겁니다.
\iffalse

\item	One interesting property of TM is the fact that transactions are
	subject to rollback and retry.
	This property underlies TM's difficulties with irreversible
	operations, including unbuffered I/O, RPCs, memory-mapping
	operations, time delays, and the \co{exec()} system call.
	This property also has the unfortunate consequence of introducing
	all the complexities inherent in the possibility of
	failure into synchronization primitives, often in a developer-visible
	manner.
\item	Another interesting property of TM, noted by
	Shpeisman et al.~\cite{TatianaShpeisman2009CppTM}, is that TM
	intertwines the synchronization with the data it protects.
	This property underlies TM's issues with I/O, memory-mapping
	operations, extra-transactional accesses, and debugging
	breakpoints.
	In contrast, conventional synchronization primitives, including
	locking and RCU, maintain a clear separation between the
	synchronization primitives and the data that they protect.
\item	One of the stated goals of many workers in the TM area is to
	ease parallelization of large sequential programs.
	As such, individual transactions are commonly expected to
	execute serially, which might do much to explain TM's issues
	with multithreaded transactions.
\fi
\end{enumerate}

TM 연구자들과 개발자들은 이런 모든 것에 대해 무엇을 해야 할까요?

한가지 방법은 TM 에 대한 집중은 작게 하고, 다른 동기화 기능들에 대비해 상당한
장점을 잠재적으로 하드웨어가 줄 수 있는 상황에 집중하는 것입니다.
이는 실제로 Sun 이 Rock 연구용 CPU 에서 취한
방법입니다~\cite{DaveDice2009ASPLOSRockHTM}.
일부 TM 연구자들은 이 방법에 동의하는 것으로 보입니다만, 다른 사람들은 TM 에
그보다 훨씬 많은 기대를 갖고 있습니다.

물론, TM 이 커다란 문제들에서 취해질 수 있게 되는 것도 상당히 가능할 것이고, 이
섹션은 TM 이 이 높은 목표를 달성하기 위해서는 해결해야 할 문제들을 일부
나열했습니다.

물론, 관련된 모든 사람들은 이를 배우는 경험으로 다뤄야 합니다.
TM 연구자들은 커다란 소프트웨어 시스템을 전통적인 동기화 기능들로 성공적으로
만들어낸 실무자들로부터 배우기 위해 많은 거래를 하게 될 것으로 보입니다.

그리고 반대도 마찬가지죠.
\iffalse

What should TM researchers and developers do about all of this?

One approach is to focus on TM in the small, focusing on situations
where hardware assist potentially provides substantial advantages over
other synchronization primitives.
This is in fact the approach Sun took with its Rock research
CPU~\cite{DaveDice2009ASPLOSRockHTM}.
Some TM researchers seem to agree with this approach, while others have
much higher hopes for TM.

Of course, it is quite possible that TM will be able to take on larger
problems, and this section lists a few of the issues that
must be resolved if TM is to achieve this lofty goal.

Of course, everyone involved should treat this as a learning experience.
It would seem that TM researchers have great deal to learn from
practitioners who have successfully built large software systems using
traditional synchronization primitives.

And vice versa.
\fi

\begin{figure}[tb]
\centering
\resizebox{3in}{!}{\includegraphics{cartoons/TM-the-vision}}
\caption{The STM Vision}
\ContributedBy{Figure}{fig:future:The STM Vision}{Melissa Broussard}
\end{figure}

\begin{figure}[tb]
\centering
\resizebox{2.7in}{!}{\includegraphics{cartoons/TM-the-reality-conflict}}
\caption{The STM Reality: Conflicts}
\ContributedBy{Figure}{fig:future:The STM Reality: Conflicts}{Melissa Broussard}
\end{figure}

\begin{figure}[tb]
\centering
\resizebox{3in}{!}{\includegraphics{cartoons/TM-the-reality-nonidempotent}}
\caption{The STM Reality: Irrevocable Operations}
\ContributedBy{Figure}{fig:future:The STM Reality: Irrevocable Operations}{Melissa Broussard}
\end{figure}

\begin{figure}[tb]
\centering
\resizebox{2.7in}{!}{\includegraphics{cartoons/TM-the-reality-realtime}}
\caption{The STM Reality: Realtime Response}
\ContributedBy{Figure}{fig:future:The STM Reality: Realtime Response}{Melissa Broussard}
\end{figure}

하지만, STM 의 현재 상황은 일련의 만화로 요약될 수 있을 것 같습니다.
먼저,
Figure~\ref{fig:future:The STM Vision}
는 STM 의 전망을 보이고 있습니다.
항상 그렇듯, 실제는
Figures~\ref{fig:future:The STM Reality: Conflicts},
\ref{fig:future:The STM Reality: Irrevocable Operations},
그리고 ~\ref{fig:future:The STM Reality: Realtime Response}
에 그려진 것처럼 좀 더 미묘한 차이가 있습니다.

최근의 구할 수 있는 하드웨어에서의 발전은 뒤의 섹션에서 설명될 HTM 의
변종으로의 문을 열었습니다.
\iffalse

But for the moment, the current state of STM
can best be summarized with a series of cartoons.
First,
Figure~\ref{fig:future:The STM Vision}
shows the STM vision.
As always, the reality is a bit more nuanced, as fancifully depicted by
Figures~\ref{fig:future:The STM Reality: Conflicts},
\ref{fig:future:The STM Reality: Irrevocable Operations},
and~\ref{fig:future:The STM Reality: Realtime Response}.
Less fanciful STM retrospectives are also
available~\cite{JoeDuffy2010RetroTM,JoeDuffy2010RetroTM2}.

Recent advances in commercially available hardware have opened the door
for variants of HTM, which are addressed in the following section.
\fi

% @@@ Don Porter's user-level TM work for Linux.

% future/htm.tex

\section{Hardware Transactional Memory}
\label{sec:future:Hardware Transactional Memory}

2017 년에 이르러, 하드웨어 트랜잭셔널 메모리 (HTM) 이 상품으로 구입할 수
있는 흔한 컴퓨터 시스템들 일부에도 도입되기 시작하고 있습니다.
이 섹션은 병렬 프로그래머의 도구상자에 이 하드웨어 트랜잭셔널 메모리가 위치할
자리를 알아보기 위한 시도를 해 봅니다.

개념적 관점에서, HTM 은 명시된 명령문의 그룹 (하나의 ``트랜잭션'') 을 다른
프로세서에서 수행되는 모든 다른 트랜잭션들의 시야에 원자적으로 보이도록 그
효과를 만들기 위해 프로세서 캐시와 예측적 수행을 사용합니다.
이 트랜잭션은 begin-transaction 기계 인스트럭션을 통해 초기화 되고
commit-transaction 기계 인스트럭션을 통해 완료됩니다.
일반적으로 abort-transaction 기계 인스트럭션도 존재하는데, 이는
(begin-transaction 인스트럭션과 그를 뒤따른 인스트럭션들이 수행되지 않은
것처럼) 예측적 수행 내용을 짓이기고 failure handler 의 수행을 시작합니다.
이 failure handler 의 위치는 begin-transaction 인스트럭션에 의해서 명시적인
failure-handler 의 주소를 통해서든 또는 해당 인스트럭션 자체의 조건적 코드에
의해서든 주어집니다.
각각의 트랜잭션은 모든 다른 트랜잭션에 대해 어토믹하게 수행됩니다.
\iffalse

As of 2017, hardware transactional memory (HTM) is available on several
types of commercially available commodity computer systems.
This section makes a first attempt to find its place in the parallel
programmer's toolbox.

From a conceptual viewpoint, HTM uses processor caches and speculative
execution to make a designated group of statements (a ``transaction'')
take effect atomically
from the viewpoint of any other transactions running on other processors.
This transaction is initiated by a
begin-transaction machine instruction and completed by a commit-transaction
machine instruction.
There is typically also an abort-transaction machine instruction, which
squashes the speculation (as if the begin-transaction instruction and
all following instructions had not executed) and commences execution
at a failure handler.
The location of the failure handler is typically specified by the
begin-transaction instruction, either as an explicit failure-handler
address or via a condition code set by the instruction itself.
Each transaction executes atomically with respect to all other transactions.
\fi

HTM 은 여러가지 중요한 장점을 갖는데, 데이터 구조의 자동적인 동적 분할, 동기화
기능의 캐시 미스의 감소, 그리고 상당수의 실용적 어플리케이션의 지원 등이 이런
장점에 포함됩니다.

하지만, 항상 작은 문자로 인쇄된 부분을 읽어야 하고, HTM 역시 예외가 아닙니다.
이 섹션의 중요한 요점은 어떤 조건 하에서 HTM 의 장점들이 그것의 작은 문자로
인쇄된 내용에 감추어져 있는 복잡성을 앞서는가를 결정하는 것입니다.
이런 결론 아래,
Section~\ref{sec:future:HTM Benefits WRT to Locking}
은 HTM 의 장점을 설명하고
Section~\ref{sec:future:HTM Weaknesses WRT Locking}
은 그 단점들을 설명합니다.
이는 이전의 논문들~\cite{McKenney2007PLOSTM,PaulEMcKenney2010OSRGrassGreener}
에서 취해진 것과 같은 방법입니다만, 전체적으로 TM 보다는 HTM 에 초점을
맞춥니다.\footnote{
	그리고 저는 다른 저자들인 Maged Michael, Josh Triplett, Jonathan
	Walpole, 그리고 Andi Kleen  과의 자극적인 토론들에 기꺼이 감사를
	드립니다.}
\iffalse

HTM has a number of important benefits, including automatic
dynamic partitioning of data structures, reducing synchronization-primitive
cache misses, and supporting a fair number of practical applications.

However, it always pays to read the fine print, and HTM is no exception.
A major point of this section is determining under what conditions HTM's
benefits outweigh the complications hidden in its fine print.
To this end, Section~\ref{sec:future:HTM Benefits WRT to Locking}
describes HTM's benefits and
Section~\ref{sec:future:HTM Weaknesses WRT Locking} describes its weaknesses.
This is the same approach used in earlier
papers~\cite{McKenney2007PLOSTM,PaulEMcKenney2010OSRGrassGreener},
but focused on HTM rather than TM as a whole.\footnote{
	And I gratefully acknowledge many stimulating
	discussions with the other authors, Maged Michael, Josh Triplett,
	and Jonathan Walpole, as well as with Andi Kleen.}
\fi

이어서
Section~\ref{sec:future:HTM Weaknesses WRT to Locking When Augmented}
은 리눅스 커널에서 (그리고 일부 user-space 어플리케이션에서) 사용되는 동기화
도구들의 조합과 관련해서 HTM 의 단점을 설명합니다.
Section~\ref{sec:future:Where Does HTM Best Fit In?}
은 병렬 프로그래머의 도구상자에서 어다에 HTM 이 가장 잘 어울릴지를 알아보고,
Section~\ref{sec:future:Potential Game Changers}
은 HTM 의 범위와 매력을 상당히 증가시킬 수 있을 몇가지 사건들을 나열합니다.
마지막으로,
Section~\ref{sec:future:Conclusions}
에서는 결론을 내립니다.
\iffalse

Section~\ref{sec:future:HTM Weaknesses WRT to Locking When Augmented} then describes
HTM's weaknesses with respect to the combination of synchronization
primitives used in the Linux kernel (and in some user-space applications).
Section~\ref{sec:future:Where Does HTM Best Fit In?} looks at where HTM
might best fit into the parallel programmer's toolbox, and
Section~\ref{sec:future:Potential Game Changers} lists some events that might
greatly increase HTM's scope and appeal.
Finally, Section~\ref{sec:future:Conclusions}
presents concluding remarks.
\fi

\subsection{HTM Benefits WRT to Locking}
\label{sec:future:HTM Benefits WRT to Locking}

HTM 의 주된 장점들은 (1)~다른 동기화 기능들에 의해 종종 일어나는 캐시 미스의
제거, (2)~동적으로 데이터 구조를 파티셔닝 하는 능력, (3)~상당한 수의 실용적
어플리케이션이 존재한다는 사실입니다.
저는 두가지 이유로 TM 의 전통과 달리 사용의 편의성을 별도로 열거하지 않습니다.
첫째로, 사용의 편의성은 이 섹션이 초점을 맞추고 있는, HTM 의 주요 장점으로부터
기인합니다.
둘째, 날 프로그래밍 재능을 위한 테스트를 하려는 시도를
둘러싼~\cite{RichardBornat2006SheepGoats,SaeedDehnadi2009SheepGoats}, 그리고
심지어 취직을 위한 면접에서의 작은 프로그래밍 연습문제의 사용을
둘러싼~\cite{RegBraithwaite2007FizzBuzz} 상당한 논쟁이 존재했습니다.
이는 우리가 무엇이 프로그래밍을 쉽게 하고 어렵게 하는지에 대한 진정한 이해를
가지고 있지 못함을 의미합니다.
따라서, 이 섹션은 앞서 나열한 세개의 장점에 대해서만, 각각 뒤의 섹션들에서
초점을 맞추도록 하겠습니다.
\iffalse

The primary benefits of HTM are
(1)~its avoidance of the cache misses that are often incurred by
other synchronization primitives,
(2)~its ability to dynamically partition
data structures,
and (3)~the fact that it has
a fair number of practical applications.
I break from TM tradition by not listing ease of use separately
for two reasons.
First, ease of use should stem from HTM's primary benefits,
which this section focuses on.
Second, there has been considerable controversy surrounding attempts to
test for raw programming
talent~\cite{RichardBornat2006SheepGoats,SaeedDehnadi2009SheepGoats}
and even around the use of small programming exercises in job
interviews~\cite{RegBraithwaite2007FizzBuzz}.
This indicates that we really do not have a grasp on what makes
programming easy or hard.
Therefore, this section focuses on the three benefits listed above,
each in one of the following sections.
\fi

\subsubsection{Avoiding Synchronization Cache Misses}
\label{sec:future:Avoiding Synchronization Cache Misses}

대부분의 동기화 메커니즘들은 어토믹 인스트럭션으로 조정되는 데이터 구조에
기반합니다.
일반적으로 이 어토믹 인스트럭션들은 먼저 연관된 캐시 라인이 수행되고 있는 CPU
에 의해 소유되도록 하기 때문에, 다른 CPU 에서 같은 동기화 기능 인스턴스의
수행을 뒤따라 하게 되면, 캐시 미스를 초래하게 됩니다.
이러한 통신으로 인한 캐시 미스 이벤트들은 전통적인 동기화 메커니즘들의 성능과
확장성을 상당히 ㄸ러어뜨립니다~\cite[Section 4.2.3]{Anderson97}.
\iffalse

Most synchronization mechanisms are based on data structures that are
operated on by atomic instructions.
Because these atomic instructions normally operate by first causing
the relevant cache line to be owned by the CPU that they are running on,
a subsequent execution
of the same instance of that synchronization primitive on some other
CPU will result in a cache miss.
These communications cache misses severely degrade both the performance and
scalability of conventional synchronization
mechanisms~\cite[Section 4.2.3]{Anderson97}.
\fi

반면에, HTM 은 CPU 의 캐시를 사용해서 동기화를 하므로 동기화 데이터 구조의
필요와 그로 말미암은 캐시 미스가 없습니다.
HTM 의 장점은 락 데이터 구조가 별개의 캐시 라인에 위치해 있을 때에 극대화
되는데, 크리티컬 섹션을 HTM 트랜잭션으로 변환함으로써 전체 캐시 미스로 인한
크리티컬 섹션의 오버헤드를 줄여주는 경우가 그런 경우입니다.
이런 이득은 짧은 크리티컬 섹션을 갖는 흔한 경우에 특히 클 수 있으며, 최소한
생략된 락이 그 락에 의해 보호되는, 자주 쓰여지는 변수와 캐시 라인을 공유하지
않을 때의 상황에서는 그렇습니다.
\iffalse

In contrast, HTM synchronizes by using the CPU's cache, avoiding the need
for a synchronization data structure and resultant cache misses.
HTM's advantage is greatest in cases where a lock data structure is
placed in a separate cache line, in which case, converting a given
critical section to an HTM transaction can reduce that critical section's
overhead by a full cache miss.
These savings can be quite significant for the common case of short
critical sections, at least for those situations where the elided lock
does not share a cache line with an oft-written variable protected by
that lock.
\fi

\QuickQuiz{}
	해당 락 변수와 캐시 라인을 공유하는, 자주 쓰여지는 변수가 왜
	문제가 될까요?
	\iffalse

	Why would it matter that oft-written variables shared the cache
	line with the lock variable?
	\fi
\QuickQuizAnswer{
	락이 그것이 보호하는 변수와 같은 캐시라인에 있다면, 하나의 CPU 에 의한
	그 변수들로의 쓰기는 모든 다른 CPU 들에 있는 그 캐시 라인을 무효화
	시킵니다.
	이런 무효화는 많은 충돌과 재시도를 만들어내고, 심지어 락킹에 비해
	성능과 확장성을 떨어뜨릴 수도 있을 겁니다.
	\iffalse

	If the lock is in the same cacheline as some of the variables
	that it is protecting, then writes to those variables by one CPU
	will invalidate that cache line for all the other CPUs.
	These invalidations will
	generate large numbers of conflicts and retries, perhaps even
	degrading performance and scalability compared to locking.
	\fi
} \QuickQuizEnd

\subsubsection{Dynamic Partitioning of Data Structures}
\label{sec:future:Dynamic Partitioning of Data Structures}

일부 전통적 동기화 메커니즘의 사용에 대한 주요한 방해는 정적으로 데이터 구조를
분할해야 하는 필요성입니다.
간단하게 분할될 수 있는 데이터 구조들이 여럿 존재하는데, 유명한 예로 해시
테이블이 존재하는데, 여기서는 각각의 해시 체인이 하나의 파티션을 구성하게
됩니다.
각각의 해시 체인을 위한 락을 할당하는 것으로 해시 테이블을 해당 체인에 국한된
오퍼레이션들로 병렬화 시키게 됩니다.\footnote{
	그리고 이 방법을 여러 해시 체인에 접근하는 오퍼레이션들로 그런
	오퍼레이션들이 연관된 체인들을 위한 락들을 해시 순서대로 모두 잡도록
	하는 것으로 쉽게 확장할 수 있습니다.}
배열, radix tree, 그리고 일부 다른 데이터 구조들에 대해서도 파티셔닝은 비슷하게
간단합니다.
\iffalse

A major obstacle to the use of some conventional synchronization mechanisms
is the need to statically partition data structures.
There are a number of data structures that are trivially
partitionable, with the most prominent example being hash tables,
where each hash chain constitutes a partition.
Allocating a lock for each hash chain then trivially parallelizes
the hash table for operations confined to a given chain.\footnote{
	And it is also easy to extend this scheme to operations accessing
	multiple hash chains by having such operations acquire the
	locks for all relevant chains in hash order.}
Partitioning is similarly trivial for arrays, radix trees, and a few
other data structures.
\fi

하지만, 많은 종류의 tree 와 graph 에 있어 파티셔닝은 상당히 어렵고, 그 결과는
종종 복잡합니다~\cite{Ellis80}.
일반적인 데이터 구조를 파티셔닝 하는데에 two-phased 락킹과 해시된 락 배열들을
사용하는 것도 가능하지만, 다른 방법들이 더
선호되었는데~\cite{DavidSMiller2006HashedLocking}, 이것들은
Section~\ref{sec:future:HTM Weaknesses WRT to Locking When Augmented}
에서 논의될 겁니다.
동기화 캐시 미스의 제거를 놓고 볼 때, HTM 은 최소한 상대적으로 적은 업데이트를
가정하면 커다란 파티셔닝 불가능한 데이터 구조를 위한 그럴싸한 방법입니다.
\iffalse

However, partitioning for many types of trees and graphs is quite
difficult, and the results are often quite complex~\cite{Ellis80}.
Although it is possible to use two-phased locking and hashed arrays
of locks to partition general data structures, other techniques
have proven preferable~\cite{DavidSMiller2006HashedLocking},
as will be discussed in
Section~\ref{sec:future:HTM Weaknesses WRT to Locking When Augmented}.
Given its avoidance of synchronization cache misses,
HTM is therefore a very real possibility for large non-partitionable
data structures, at least assuming relatively small updates.
\fi

\QuickQuiz{}
	HTM 성능과 확장성에 상대적으로 적은 업데이트가 중요한 이유가 뭐죠?
	\iffalse

	Why are relatively small updates important to HTM performance
	and scalability?
	\fi
\QuickQuizAnswer{
	업데이트가 많을수록, 충돌의 가능성이 커지고, 따라서 재시도의 가능성이
	커져서 성능이 하락됩니다.
	\iffalse

	The larger the updates, the greater the probability of conflict,
	and thus the greater probability of retries, which degrade
	performance.
	\fi
} \QuickQuizEnd

\subsubsection{Practical Value}
\label{sec:future:Practical Value}

HTM 의 실질적 가치에 대한 몇몇 증거들이 Sun
Rock~\cite{DaveDice2009ASPLOSRockHTM} 와 Azul Vega~\cite{CliffClick2009AzulHTM}
을 포함한 여러 하드웨어 플랫폼들에서 보여졌습니다.
실질적 이점들이 더 최신의 IBM Blue Gene/Q, Intel Haswell TSX, 그리고 AMD ASF
시스템들에서도 나올 것이라 가정하는 것은 합리적입니다.

예상되는 실질적 이점들은 다음과 같습니다:
\iffalse

Some evidence of HTM's practical value has been demonstrated in a number
of hardware platforms, including
Sun Rock~\cite{DaveDice2009ASPLOSRockHTM} and
Azul Vega~\cite{CliffClick2009AzulHTM}.
It is reasonable to assume that practical benefits will flow from the
more recent IBM Blue Gene/Q, Intel Haswell TSX, and AMD ASF systems.

Expected practical benefits include:
\fi

\begin{enumerate}
\item	In-memory 데이터 액세스와 업데이트를 위한 락
	생략~\cite{Martinez01a,Rajwar02a}.
\item	커다란 파티셔닝 불가능한 데이터 구조로의 동시적인 액세스와 약간의
	무작위적 업데이트들.
\iffalse

\item	Lock elision for in-memory data access and
	update~\cite{Martinez01a,Rajwar02a}.
\item	Concurrent access and small random updates to large non-partitionable
	data structures.
\fi
\end{enumerate}

하지만, HTM 은 또한 실질적인 한계점들도 가지고 있는데, 이에 대해서는 다음
섹션에서 이야기 하겠습니다.
\iffalse

However, HTM also has some very real shortcomings, which will be discussed
in the next section.
\fi

\subsection{HTM Weaknesses WRT Locking}
\label{sec:future:HTM Weaknesses WRT Locking}

HTM 의 컨셉은 상당히 간단합니다: 메모리로의 액세스와 업데이트가 그룹 단위로
어토믹하게 일어난다는 것입니다.
하지만, 많은 간단한 아이디어들이 그러하듯이, 이를 실제 세계의 실제 시스템에
적용할 때에서야 복잡성이 나타납니다.
이 복잡성들은 다음과 같은 것들입니다:
\iffalse

The concept of HTM is quite simple: A group of accesses and updates to
memory occurs atomically.
However, as is the case with many simple ideas, complications arise
when you apply it to real systems in the real world.
These complications are as follows:
\fi

\begin{enumerate}
\item	트랜잭션 크기 한계.
\item	Conflict 처리.
\item	Abort 와 롤백.
\item	진행 보장의 부재.
\item	되돌이킬 수 없는 오퍼레이션들.
\item	Semantic 상의 차이점들.
\iffalse

\item	Transaction-size limitations.
\item	Conflict handling.
\item	Aborts and rollbacks.
\item	Lack of forward-progress guarantees.
\item	Irrevocable operations.
\item	Semantic differences.
\fi
\end{enumerate}

이 각각의 복잡성들은 다음의 섹션들에서 다루어지고, 그 뒤를 이어 요약을 합니다.
\iffalse

Each of these complications is covered in the following sections,
followed by a summary.
\fi

\subsubsection{Transaction-Size Limitations}
\label{sec:future:Transaction-Size Limitations}

현재 HTM 구현들의 트랜잭션 크기 한계점은 그 트랜잭션에 영향받는 데이터를 쥐고
있기 위해 프로세서의 캐시를 사용한다는 데서 나옵니다.
이는 특정 CPU 가 트랜잭션을 자신의 캐시 안에 국한된 채로 수행시켜서 해당
트랜잭션이 다른 CPU 들에게 어토믹 하게 보이도록 하는 것을 가능하게 하지만, 이는
또한 캐시에 들어가지 않는 모든 트랜잭션은 abort 될것을 의미하기도 합니다.
더 나아가서, 인터럽트, 시스템 콜, exception, trap, 그리고 컨텍스트 스위치와
같이 수행 문맥을 바꾸는 이벤트들은 해당 CPU 에서 수행중인 트랜잭션을 모두 abort
시키거나 다른 수행 컨텍스트에 의한 캐시 사용량으로 인해 트랜잭션의 크기를
제한해야만 합니다.

물론, 최신 CPU 들은 커다란 캐시를 갖는 경향이 있고, 많은 트랜잭션들에 필요한
데이터는 1 메가바이트 캐시 안에도 잘 들어갈 겁니다.
불행히도, 캐시에 있어서, 크기만이 모든 문제는 아닙니다.
문제는, 대부분의 캐시들은 하드웨어로 구현된 해시 테이블로 생각될 수 있다는
것입니다.
하지만, 하드웨어 캐시는 (일반적으로 \emph{set} 이라 불리는) 버킷들을 연결시키지
않고, set 당 고정된 수의 캐시라인들을 제공합니다.
특정 캐시에서 각각의 set 에 제공되는 원소들의 수는 해당 캐시의
\emph{associativity} 라고 불립니다.
\iffalse

The transaction-size limitations of current HTM implementations
stem from the use of the processor caches to hold the data
affected by the transaction.
Although this allows a given CPU to make the transaction appear atomic to
other CPUs by executing the transaction within the confines of its cache,
it also means that any transaction that does not fit must be aborted.
Furthermore, events that change execution context, such as interrupts,
system calls, exceptions, traps, and context switches either must
abort any ongoing transaction on the CPU in question or must further
restrict transaction size due to the cache footprint of the other
execution context.

Of course, modern CPUs tend to have large caches, and the data required
for many transactions would fit easily in a one-megabyte cache.
Unfortunately, with caches, sheer size is not all that matters.
The problem is that most caches
can be thought of hash tables implemented in hardware.
However, hardware caches do not chain their buckets (which are normally
called \emph{sets}), but rather
provide a fixed number of cachelines per set.
The number of elements provided for each set in a given cache
is termed that cache's \emph{associativity}.
\fi

Cache associativity 는 다양하지만, 제가 지금 타자를 치고 있는 랩탑의 8-way
associativity level-0 캐시는 흔하지 않습니다.
이게 의미하는바는 특정 트랜잭션이 9개의 캐시 라인을 건드리게 되고, 그 9개의
캐시 라인들이 모두 같은 set 으로 매핑된다면, 그 트랜잭션은 해당 캐시에 얼마나
많은 메가바이트의 용량들이 남아있는가에 관계없이 성공할 수 없다는 것입니다.
그렇습니다, 특정 데이터 구조에서 무작위적으로 골라진 데이터 원소들에 있어서, 그
트랜잭션이 커밋에 성공할 확률은 매우 높긴 합니다만, 어떤 보장사항도 없습니다.

이 한계점을 경감시키기 위한 일부 연구들이 있었습니다.
Fully associative \emph{victim cache} 는 associativity 한계를 경감시킬 수
있습니다만, victim cache 의 성능과 에너지 효율성에 대한 많은 한계가 현재
존재합니다.
그렇다고는 하나, 수정되지 않은 캐시 라인들을 위한 HTM victim cache 는 주소만을
가지고 있을 수도 있으므로, 상당히 작을 수 있습니다:
데이터 주소 자체는 충돌되는 쓰기를 파악하기에 충분하며~\cite{RaviRajwar2012TSX}
데이터 자체는 메모리에 쓰여지거나 다른 캐시에 shadow 될수 있습니다.
\iffalse

Although cache associativity varies, the eight-way associativity of
the level-0 cache on the laptop I am typing this on is not unusual.
What this means is that if a given transaction needed to touch
nine cache lines, and if all nine cache lines mapped to the same
set, then that transaction cannot possibly complete, never mind how
many megabytes of additional space might be available in that cache.
Yes, given randomly selected data elements in a given data structure,
the probability of that transaction being able to commit is quite
high, but there can be no guarantee.

There has been some research work to alleviate this limitation.
Fully associative \emph{victim caches} would alleviate the associativity
constraints, but there are currently stringent performance and
energy-efficiency constraints on the sizes of victim caches.
That said, HTM victim caches for unmodified cache lines can be quite
small, as they need to retain only the address:
The data itself can be written to memory or shadowed by other caches,
while the address itself is sufficient to detect a conflicting
write~\cite{RaviRajwar2012TSX}.
\fi

\emph{Unbounded transactional memory} (UTM)
방법~\cite{CScottAnanian2006,KevinEMoore2006} 은 DRAM 을 극단적으로 커다란
victim cache 로 사용합니다만, 그런 방법을 제품 품질의 캐시 일관성 메커니즘과
결합하는 것은 여전히 해결되지 않은 문제입니다.
또한, DRAM 을 victim cache 로 사용하는 것은 성능과 에너지 효율성의 저하를
가져올 수 있는데, victim cache 가 fully associative 하다면 특히 그렇습니다.
마지막으로, UTM 의 ``unbounded'' 라는 측면은 DRAM 이 모두 victim cache 로
사용될 수 있다는 가정을 합니다만, 실제로는 커다랗긴 하지만 고정된 양의, 해당
CPU 에 주어진 DRAM 의 용량만으로 해당 CPU 의 트랜잭션의 크기가 제한될 겁니다.
다른 방법들은 하드웨어 트랜잭셔널 메모리와 소프트웨어 트랜잭셔널 메모리의
조합을 사용하고~\cite{SanjeevKumar2006}, HTM 의 fallback 메커니즘으로 STM 을
사용하는 방법을 생각해 볼 수도 있을 겁니다.

하지만, 제가 알기로는, 현재로써 사용 가능한 시스템들은 이런 연구 아이디어들을
구현한 바가 없습니다.
\iffalse

\emph{Unbounded transactional memory} (UTM)
schemes~\cite{CScottAnanian2006,KevinEMoore2006}
use DRAM as an extremely large victim cache, but integrating such schemes
into a production-quality cache-coherence mechanism is still an unsolved
problem.
In addition, use of DRAM as a victim cache may have unfortunate
performance and energy-efficiency consequences, particularly
if the victim cache is to be fully associative.
Finally, the ``unbounded'' aspect of UTM assumes that all of DRAM
could be used as a victim cache, while in reality
the large but still fixed amount of DRAM assigned to a given CPU
would limit the size of that CPU's transactions.
Other schemes use a combination of hardware and software transactional
memory~\cite{SanjeevKumar2006} and one could imagine using STM as a
fallback mechanism for HTM.

However, to the best of my knowledge, currently available systems do
not implement any of these research ideas, and perhaps for good reason.
\fi

\subsubsection{Conflict Handling}
\label{sec:future:Conflict Handling}

첫번째 문제는 \emph{conflict} 의 가능성입니다.
예를 들어, transaction~A 와~B 가 다음과 같이 정의되었다고 생각해 봅시다:
\iffalse

The first complication is the possibility of \emph{conflicts}.
For example, suppose that transactions~A and~B are defined as follows:
\fi

\vspace{5pt}
\begin{minipage}[t]{\columnwidth}
\begin{verbatim}
Transaction A       Transaction B

x = 1;              y = 2;
y = 3;              x = 4;
\end{verbatim}
\end{minipage}
\vspace{5pt}

각각의 트랜잭션이 각자의 프로세서 위에서 동시에 수행된다고 생각해 봅시다.
만약 transaction~A 가 \co{x} 에 쓰기를 하는 동시에 transaction~B 가 \co{y} 에
쓰기를 한다면, 두 트랜잭션 모두 진행될 수 없습니다.
이를 보기 위해, transaction~A 가 \co{y} 로의 쓰기를 수행한다고 생각해 봅시다.
그럼 transaction~A 는 transaction~B 와 섞여들어가게 되는데, 이는 트랜잭션이
상대방의 시점에 어토믹하게 수행되어야 한다는 트랜잭션의 요구사항을 위반하는
것입니다.
Transaction~B 가 \co{x} 로의 저장을 수행하게 허락한다면, 이는 비슷하게 어토믹
수행 요구사항을 위반하는 것입니다.
이 상황은 \emph{conflict} 라 명명되는데, 두개의 동시에 수행되는 트랜잭션들이
똑같은 변수에 접근하게 되며 그 접근들 가운데 최소한 하나는 쓰기인 경우에
발생합니다.
따라서 시스템은 수행이 진행될 수 있도록 하기 위해 이 트랜잭션들 가운데 하나나
두개 모두를 abort 시킬 의무를 갖습니다.
정확히 어떤 트랜잭션을 abort 시킬 것인가에 대한 선택은 Ph.D. 학위논문을 만들
능력이 있을 만큼 흥미로운 주제인데, 그런 예~\cite{EgeAkpinar2011HTM2TLE}도
있으니, 보시기 바랍니다.\footnote{
	``Toxic Transactions'' 라는 제목의 Liu 와 Spear 의
	논문~\cite{YujieLiu2011ToxicTransactions}  은 이런 면에서 특히
	유명합니다.}
이 섹션의 목적을 위해서, 우린 시스템이 무작위적 선택을 한다고 가정하겠습니다.
\iffalse

Suppose that each transaction executes concurrently on its own processor.
If transaction~A stores to \co{x} at the same time that transaction~B
stores to \co{y}, neither transaction can progress.
To see this, suppose that transaction~A executes its store to \co{y}.
Then transaction~A will be interleaved within transaction~B, in violation
of the requirement that transactions execute atomically with respect to
each other.
Allowing transaction~B to execute its store to \co{x} similarly violates
the atomic-execution requirement.
This situation is termed a \emph{conflict}, which happens whenever two
concurrent transactions access the same variable where at least one of
the accesses is a store.
The system is therefore obligated to abort one or both of the transactions
in order to allow execution to progress.
The choice of exactly which transaction to abort is an interesting topic
that will very likely retain the ability to generate Ph.D. dissertations for
some time to come, see for
example~\cite{EgeAkpinar2011HTM2TLE}.\footnote{
	Liu's and Spear's paper entitled ``Toxic
	Transactions''~\cite{YujieLiu2011ToxicTransactions} is
	particularly instructive in this regard.}
For the purposes of this section, we can assume that the system makes
a random choice.
\fi

또하나의 문제는 conflict 파악으로, 적어도 가장 간단한 경우에 있어서는 비교적
간단한 편입니다.
프로세서가 트랜잭션을 수행할 때, 프로세서는 그 트랜잭션에 의해 접근되는 모든
캐시라인을 표시해 둡니다.
만약 이 프로세서의 캐시가 현재 트랜잭션에 의해 접근된 것으로 표시된 캐시라인에
대한 요청을 받게 되면, 잠재적 conflcit 이 일어난 것입니다.
더 세련된 시스템들은 현재 프로세서의 트랜잭션이 그 요청을 보낸 프로세서의
트랜잭션을 앞서도록 순서잡을 것이고, 이 프로세스의 최적화는 또한 Ph.D.
학위논문을 쓰기 위한 능력을 얻을 수 있게 해줄 겁니다.
하지만 이 섹션은 매우 간단한 conflict 파악 전략을 가정합니다.
\iffalse

Another complication is conflict detection, which is comparatively
straightforward, at least in the simplest case.
When a processor is executing a transaction, it marks every cache line
touched by that transaction.
If the processor's cache receives a request involving a cache line that
has been marked as touched by the current transaction, a potential
conflict has occurred.
More sophisticated systems might try to order the current processors'
transaction to precede that of the processor sending the request,
and optimization of this process will likely also retain the ability
to generate Ph.D. dissertations for quite some time.
However this section assumes a very simple conflict-detection strategy.
\fi

하지만, HTM 이 효율적으로 동작하려면 conflict 의 가능성이 충분히 낮아야 하는데,
이는 데이터 구조가 충분히 낮은 conflict 가능성을 유지하도록 구성되어야 할것을
필요로 합니다.
예를 들어, 간단한 삽입, 삭제, 그리고 탐색 오퍼레이션을 제공하는 red-black
트리는 이런 경우에 적합합니다만, 트리의 모든 원소의 정확한 수를 유지해야 하는
red-black 트리는 그렇지 않습니다.\footnote{
	이 수를 업데이트 해야할 필요성이 트리로의 삽입과 트리로부터의 삭제가
	서로 conflict 를 발생시키게 해서 strong non-commutativity 를 초래할
	겁니다~\cite{HagitAttiya2011LawsOfOrder,Attiya:2011:LOE:1925844.1926442,PaulEMcKenney2011SNC}.}
또다른 예로, 트리의 모든 원소를 하나의 트랜잭션에서 열거하는 red-black 트리는
높은 conflict 확률을 가질 것이고, 성능과 확장성을 떨어뜨릴 겁니다.
결과적으로, 많은 순차적 프로그램들은 HTM 이 효과적으로 동작하도록 하기 위해
일부 재구성을 필요로 할 겁니다.
몇몇 경우에 있어서, 실무자들은 그런 추가적 단계를 취하거나 (red-black 트리의
경우에 있어서, radix 트리나 해시 테이블과 같은 파티셔닝 가능한 데이터 구조로의
전환 같은) 그냥 락킹을 사용하는 것을 선호할 수 있는데, HTM 이 모든 관련된
구조에서 사용 가능하기 충분한 시간이 오기 전까지는 특히 그럴 수
있습니다~\cite{CliffClick2009AzulHTM}.
\iffalse

However, for HTM to work effectively, the probability of conflict must
be suitably low, which in turn requires that the data structures
be organized so as to maintain a sufficiently low probability of conflict.
For example, a red-black tree with simple insertion, deletion, and search
operations fits this description, but a red-black
tree that maintains an accurate count of the number of elements in
the tree does not.\footnote{
	The need to update the count would result in additions to and
	deletions from the tree conflicting with each other, resulting
	in strong non-commutativity~\cite{HagitAttiya2011LawsOfOrder,Attiya:2011:LOE:1925844.1926442,PaulEMcKenney2011SNC}.}
For another example, a red-black tree that enumerates all elements in
the tree in a single transaction will have high conflict probabilities,
degrading performance and scalability.
As a result, many serial programs will require some restructuring before
HTM can work effectively.
In some cases, practitioners will prefer to take the extra steps
(in the red-black-tree case, perhaps switching to a partitionable
data structure such as a radix tree or a hash table), and just
use locking, particularly during the time before HTM is readily available
on all relevant
architectures~\cite{CliffClick2009AzulHTM}.
\fi

\QuickQuiz{}
	동기화 메커니즘의 선택에 관계 없이 어떻게 red-black 트리가 트리 내의
	모든 원소의 열거를 효율적으로 할 수 있을까요???
	\iffalse

	How could a red-black tree possibly efficiently enumerate all
	elements of the tree regardless of choice of synchronization
	mechanism???
	\fi
\QuickQuizAnswer{
	많은 경우에, 이 열거는 정확하지 않아도 좋습니다.
	이런 경우들에 있어서, hazard pointer 나 RCU 가 특정한 삽입이나 삭제와의
	낮은 conflict 확률을 유지하면서 읽기 쓰레드들을 보호하는데 사용될 수
	있습니다.
	\iffalse

	In many cases, the enumeration need not be exact.
	In these cases, hazard pointers or RCU may be used to protect
	readers with low probability of conflict with any given insertion
	or deletion.
	\fi
} \QuickQuizEnd

더 나아가서, conflict 이 일어날 수 있다는 사실은 다음 섹션에 이야기되는 것처럼
failure 처리를 어떻게 할 것인지 그림을 가져올 수 있게 해줍니다.
\iffalse

Furthermore, the fact that conflicts can occur brings failure handling
into the picture, as discussed in the next section.
\fi

\subsubsection{Aborts and Rollbacks}
\label{sec:future:Aborts and Rollbacks}

모든 트랜잭션이 언제든 abort 될 수 있으므로, 트랜잭션은 롤백될 수 없는 명령을
포함해선 안된다는 점이 중요합니다.
이 말은 트랜잭션은 I/O, 시스템콜, 또는 디버깅 브레이크포인트 (HTM
트랜잭션에서의 디버거에서의 single step 수행이 안됩니다!!!) 를 가질 수 없음을
의미합니다.
대신, 트랜잭션은 스스로를 평범한 캐시된 메모리에만 접근하도록 국한시켜야만
합니다.
더 나아가서, 일부 시스템에서는, 인터럽트, exception, trap, TLB 미스, 그리고
다른 이벤트들 역시 트랜잭션을 abort 시킵니다.
잘못된 에러 조건의 처리로 초래된 많은 수의 버그들을 생각해보면, 사용성을 위해
abort 와 rollback 이 어떤 효과를 갖는지 알아보는게 좋을 겁니다.
\iffalse

Because any transaction might be aborted at any time, it is important
that transactions contain no statements that cannot be rolled back.
This means that transactions cannot do I/O, system calls, or debugging
breakpoints (no single stepping in the debugger for HTM transactions!!!).
Instead, transactions must confine themselves to accessing normal
cached memory.
Furthermore, on some systems, interrupts, exceptions, traps,
TLB misses, and other events will also abort transactions.
Given the number of bugs that have resulted from improper handling
of error conditions, it is fair to ask what impact aborts and rollbacks
have on ease of use.
\fi

\QuickQuiz{}
	하지만 왜 디버거는 트랜잭션의 앞의 인스턴스의 스텝들을 다시 추적하기
	위해 재시도에 의조하면서 브레이크포인트를 트랜잭션의 성공되는 명령문
	줄에 설정해 두는 것으로 single stepping 을 흉내낼 수 없나요?
	\iffalse

	But why can't a debugger emulate single stepping by setting
	breakpoints at successive lines of the transaction, relying
	on the retry to retrace the steps of the earlier instances
	of the transaction?
	\fi
\QuickQuizAnswer{
	이 방법은 높은 확률로 동작할 수 있을 겁니다만, 대부분의 사용자들에게는
	상당히 놀라운 형태로 실패할 수 있습니다.
	이를 보기 위해, 다음 트랜잭션을 생각해 봅시다:
	\iffalse

	This scheme might work with reasonably high probability, but it
	can fail in ways that would be quite surprising to most users.
	To see this, consider the following transaction:
	\fi

	\vspace{5pt}
	\begin{minipage}[t]{\columnwidth}
	\small
\begin{verbatim}
  1 begin_trans();
  2 if (a) {
  3   do_one_thing();
  4   do_another_thing();
  5 } else {
  6   do_a_third_thing();
  7   do_a_fourth_thing();
  8 }
  9 end_trans();
\end{verbatim}
	\end{minipage}
	\vspace{5pt}

	사용자가 line~3 에 트랜잭션을 abort 시키고 디버거에 들어가게 될
	브레이크포인트를 설정했다고 생각해 봅시다.
	브레이크포인트가 시작되고 디버거가 모든 쓰레드를 정지시키는 사이에,
	어떤 다른 쓰레드가 \co{a} 의 값을 0으로 설정했다고 생각해 봅시다.
	사용자가 이 프로그램을 single-step 하면, 짜잔!
	프로그램은 이제 then-절 대신 else-절에 들어와 있습니다.

	이는 제가 사용성이 좋은 디버거라 부르는 것이 \emph{아닙니다}.
	\iffalse

	Suppose that the user sets a breakpoint at line~3, which triggers,
	aborting the transaction and entering the debugger.
	Suppose that between the time that the breakpoint triggers
	and the debugger gets around to stopping all the threads, some
	other thread sets the value of \co{a} to zero.
	When the poor user attempts to single-step the program, surprise!
	The program is now in the else-clause instead of the then-clause.

	This is \emph{not} what I call an easy-to-use debugger.
	\fi
} \QuickQuizEnd

물론, abort 와 롤백은 HTM 이 real-time 시스템에 유용할 수 있는 것인지라는
질문을 떠올리게 합니다.
HTM 의 성능적 이득이 abort 와 롤백의 비용을 넘을까요, 그리고 그렇다면 어떤 조건
하에서 그럴까요?
트랜잭션은 우선순위 향상 기능을 사용할 수 있을까요?
아니면 높은 우선순위 쓰레드를 위한 트랜잭션은 낮은 우선순위 쓰레드들을
우선적으로 abort 시켜야 할까요?
만약 그렇다면, 하드웨어는 어떻게 효율적으로 우선순위를 알 수 있을까요?
실제 세계에서의 HTM 의 사용에 대한 환경은 협소한데, 연구자들이 트랜잭션들이 비
real-time 환경에서 잘 동작하기에는 충분한 것들보다도 더 많은 문제들을 찾고 있기
때문일 수도 있습니다.
\iffalse

Of course, aborts and rollbacks raise the question of whether HTM can
be useful for hard real-time systems.
Do the performance benefits of HTM outweigh the costs of the aborts
and rollbacks, and if so under what conditions?
Can transactions use priority boosting?
Or should transactions for high-priority threads instead preferentially
abort those of low-priority threads?
If so, how is the hardware efficiently informed of priorities?
The literature on real-time use of HTM is quite sparse, perhaps because
researchers are finding more than enough problems in getting
transactions to work well in non-real-time environments.
\fi

현재의 HTM 구현들은 결정론적으로 특정 트랜잭션을 abort 시킬 수도 있기 때문에,
소프트웨어는 fallback 코드를 제공해야만 합니다.
이 fallback 코드는 예를 들면 락킹과 같은, 어떤 다른 형태의 동기화를 사용해야만
합니다.
만약 이 fallback 이 빈번하게 사용된다면, 데드락의 가능성을 포함한 모든 락킹의
제한점들이 다시 나타납니다.
물론, 이 fallback 이 자주 사용되지 않아서 더 간단하고 디드락이 나타나기 쉽지
않은 설계가 사용될 수 있게 되기를 바랄 수 있습니다.
하지만 이는 시스템은 락 기반의 fallback 에서 트랜잭션으로 어떻게 전환할
것인지에 대한 질문을 떠올리게 합니다.\footnote{
	어플리케이션이 fallback 모드에서 멈춰서게 되는 가능성은 Dave Dice 가
	만드는데 기여한, ``lemming effect'' 라고 명명되었습니다.}
한가지 방법은 test-and-test-and-set 방법~\cite{Martinez02a} 의 사용으로, 모두가
락이 해제될 때까지 기다림으로써 시스템이 트랜잭션적으로 깨끗한 백지 상태에서
시작할 수 있도록 하는 것입니다.
하지만, 이는 상당한 spinning 을 초래할 수 있는데, 이는 락을 쥔 쓰레드가
블락되어 있거나 preemption 당했다면 현명하지 못한 행위일 수 있습니다.
또다른 방법은 트랜잭션이 락을 쥔 쓰레드와 병렬적으로 수행될 수 있도록 하는
것~\cite{Martinez02a} 입니다만, 이 방법은 원자성을 유지하는데에 어려움을
낳으며, 특히 그 쓰레드가 락을 잡고 있는 이유가 연관된 트랜잭션이 캐시에
들어가지 않기 때문이라면 더 그렇습니다.
\iffalse

Because current HTM implementations might deterministically abort a
given transaction, software must provide fallback code.
This fallback code must use some other form of synchronization, for
example, locking.
If the fallback is used frequently, then all the limitations of locking,
including the possibility of deadlock, reappear.
One can of course hope that the fallback isn't used often, which might
allow simpler and less deadlock-prone locking designs to be used.
But this raises the question of how the system transitions from using
the lock-based fallbacks back to transactions.\footnote{
	The possibility of an application getting stuck in fallback
	mode has been termed the ``lemming effect'', a term that
	Dave Dice has been credited with coining.}
One approach is to use a test-and-test-and-set discipline~\cite{Martinez02a},
so that everyone holds off until the lock is released, allowing the
system to start from a clean slate in transactional mode at that point.
However, this could result in quite a bit of spinning, which might not
be wise if the lock holder has blocked or been preempted.
Another approach is to allow transactions to proceed in parallel with
a thread holding a lock~\cite{Martinez02a}, but this raises difficulties
in maintaining atomicity, especially if the reason that the thread is
holding the lock is because the corresponding transaction would not fit
into cache.
\fi

마지막으로, abort 와 롤백 가능성을 처리하는 것은 개발자에게 모든 가능한 에러
조건들의 조합을 올바르게 처리해야 한다는 추가적인 부담을 지우는 것으로 보일 수
있습니다.

HTM 의 사용자들이 fallback 코드 수행경로와 fallback 에서 트랜잭션 코드로의 전환
모두에 상당한 검증을 위한 노력을 기울여야 함은 분명합니다.
\iffalse

Finally, dealing with the possibility of aborts and rollbacks seems to
put an additional burden on the developer, who must correctly handle
all combinations of possible error conditions.

It is clear that users of HTM must put considerable validation effort
into testing both the fallback code paths and transition from fallback
code back to transactional code.
\fi

\subsubsection{Lack of Forward-Progress Guarantees}
\label{sec:future:Lack of Forward-Progress Guarantees}

트랜잭션 크기, conflict, 그리고 abort/rollback 이 모두 트랜잭션을 abort 되게
만들 수 있다고는 하지만, 충분히 작고 짧은 기간동안 수행되는 트랜잭션은 결국은
성공할 것이라 보장된다고 희망할 수 있을 것입니다.
이는 compare-and-swap (CAS) 와 load-linked/store-conditional (LL/SC)
오퍼레이션들이 이 인스트럭션들을 어토믹 오퍼레이션을 구현하는데에 사용하는
코드에서 무조건적으로 재시도 되는 것과 똑같이 트랜잭션이 무조건적으로 재시도
되도록 허용하도록 할 수 있을 것입니다.

불행히도, 대부분의 현재 사용 가능한 HTM 구현들은 어떤 종류의 진행 보장도
제공하지 않는데, 이는 HTM 은 시스템의 데드락을 막는데에 사용될 수 없음을
의미합니다.\footnote{
	HTM 은 데드락의 가능성을 줄이는데 사용될 수는 있습니다만, fallback
	코드가 수행될 가능성이 존재하는한, 데드락의 가능성은 존재합니다.}
미래의 HTM 구현은 어떤 진행 보장을 제공할 수도 있을 겁니다.
그러기 전까지는, HTM 은 real-time 어플리케이션은 상당한 주의 아래 사용되어야만
합니다.\footnote{
	2012년 중반까지는, 트랜잭셔널 메모리의 real-time 특성에 대해서는
	놀라만큼 적은 작업만이 있었습니다.}
\iffalse

Even though transaction size, conflicts, and aborts/rollbacks can all
cause transactions to abort, one might hope that sufficiently small and
short-duration transactions could be guaranteed to eventually succeed.
This would permit a transaction to be unconditionally retried, in the
same way that compare-and-swap (CAS) and load-linked/store-conditional
(LL/SC) operations are unconditionally retried in code that uses these
instructions to implement atomic operations.

Unfortunately, most currently available HTM implementation refuse to
make any
sort of forward-progress guarantee, which means that HTM cannot be
used to avoid deadlock on those systems.\footnote{
	HTM might well be used to reduce the probability of deadlock,
	but as long as there is some possibility of the fallback
	code being executed, there is some possibility of deadlock.}
Hopefully future implementations of HTM will provide some sort of
forward-progress guarantees.
Until that time, HTM must be used with extreme caution in real-time
applications.\footnote{
	As of mid-2012, there has been surprisingly little work on
	transactional memory's real-time characteristics.}
\fi

2013년에 있어 이 우울한 그림에 대한 한가지 예외는 특수한 \emph{제약된
트랜잭션}~\cite{ChristianJacobi2012MainframeTM} 을 시작하는데 사용되는 별도의
인스트럭션을 제공하는 IBM 메인프레임의 차기 버전들입니다.
이름에서 추측할 수 있듯이, 그런 트랜잭션은 다음과 같은 제약 아래에 살아남을 수
있어야만 합니다:
\iffalse

The one exception to this gloomy picture as of 2013 is upcoming versions
of the IBM mainframe, which provides a separate instruction that may be
used to start a special
\emph{constrained transaction}~\cite{ChristianJacobi2012MainframeTM}.
As you might guess from the name, such transactions must live within
the following constraints:
\fi

\begin{enumerate}
\item	각각의 트랜잭션의 데이터 사용량은 4개의 32-byte 메모리 블락 안에 들어갈
	수 있어야만 합니다.
\item	각각의 트랜잭션은 최대 32 개의 어셈블러 인스트럭션을 수행하도록
	허용됩니다.
\item	트랜잭션은 뒤로 돌아가는 브랜치 (e.x: 루프를 가질 수 없음) 를 가질 수
	없습니다.
\item	각각의 트랜잭션의 코드는 256 바이트의 메모리로 제한됩니다.
\item	특정 트랜잭션의 데이터 사용량의 한 부분이 4K 페이지 안에 위치한다면, 그
	4K 페이지는 트랜잭션의 인스트럭션을 담을 수 없습니다.
\iffalse

\item	Each transaction's data footprint must be contained within
	four 32-byte blocks of memory.
\item	Each transaction is permitted to execute at most 32 assembler
	instructions.
\item	Transactions are not permitted to have backwards branches
	(e.g., no loops).
\item	Each transaction's code is limited to 256 bytes of memory.
\item	If a portion of a given transaction's data footprint resides
	within a given 4K page, then that 4K page is prohibited from
	containing any of that transaction's instructions.
\fi
\end{enumerate}

이런 제약은 가혹합니다만, 이는 더도 아니고 덜도 아니고 스택, 큐, 해시 테이블,
등등을 포함해서 다양한 데이터 구조의 업데이트가 구현될 수 있도록 합니다.
이런 오퍼레이션들은 결국은 완료될 것이 보장되고, 따라서 데드락과 라이브락
조건으로부터 자유롭습니다.

시간의 흐름에 따라 하드웨어의 진행 보장이 얼마나 지우너될 것인지 보는 것은 꽤
흥미로울 것입니다.
\iffalse

These constraints are severe, but they nevertheless permit a wide variety
of data-structure updates to be implemented, including stacks, queues,
hash tables, and so on.
These operations are guaranteed to eventually complete, and are free of
deadlock and livelock conditions.

It will be interesting to see how hardware support of forward-progress
guarantees evolves over time.
\fi

\subsubsection{Irrevocable Operations}
\label{sec:future:Irrevocable Operations}

Abort 와 롤백의 또다른 결론은 HTM 트랜잭션은 되돌이켜질 수 없는 오퍼레이션들을
담을 수 없다는 것입니다.
현재의 HTM 구현들은 일반적으로 트랜잭션 안에서의 모든 액세스가 캐시될 수 있는
메모리 안으로만 국한되도록 하고 (따라서 MMIO 액세스를 금지하고) 인터럽트, trap,
그리고 exception 시에는 트랜잭션을 어보팅 시키는 것으로 (따라서 시스템콜을
금지시키는 것으로) 이 제한을 강제합니다.

HTM 트랜잭션은 buffered I/O 역시 버퍼의 fill/flush 오퍼레이션이 트랜잭션 외에서
일어나는 한은 담겨질 수 있음을 알아두시기 바랍니다.
이게 동작하는 이유는 버퍼에 데이터를 넣고 빼는 것은 되돌이켜질 수 있기
때문입니다: 실제 버퍼 fill/flush 오퍼레이션들만이 되돌이켜질 수 없습니다.
물론, 이 buffered-I/O 방법은 I/O 를 트랜잭션의 흔적에 포함시키는 효과를 내서,
트랜잭션의 크기를 키우고 그로 인해 트랜잭션 실패 확률을 증가시킵니다.
\iffalse

Another consequence of aborts and rollbacks is that HTM transactions
cannot accommodate irrevocable operations.
Current HTM implementations typically enforce this limitation by
requiring that all of the accesses in the transaction be to cacheable
memory (thus prohibiting MMIO accesses) and aborting transactions on
interrupts, traps, and exceptions (thus prohibiting system calls).

Note that buffered I/O can be accommodated by HTM transactions as
long as the buffer fill/flush operations occur extra-transactionally.
The reason that this works is that adding data to and removing data
from the buffer is revocable: Only the actual buffer fill/flush
operations are irrevocable.
Of course, this buffered-I/O approach has the effect of including the I/O
in the transaction's footprint, increasing the size of the transaction
and thus increasing the probability of failure.
\fi

\subsubsection{Semantic Differences}
\label{sec:future:Semantic Differences}

HTM 이 많은 경우에 락킹의 대체제로 사용될 수 있기는 하지만 (그래서
transactional lock elision~\cite{DaveDice2008TransactLockElision} 이란 명칭이
있습니다), semantic 상에 약간의 차이가 있습니다.
트랜잭션으로 수행되면 deadlock 이나 livelock 을 초래할 수 있는, 락 기반의
크리티컬 섹션과 연관된 특히 골치아픈 예가 Blundell 에 의해
알려졌습니다만~\cite{Blundell2006TMdeadlock}, 훨씬 간단한 예는 텅 빈 크리티컬
섹션입니다.

락 기반의 프로그램에서, 텅 빈 크리티컬 섹션은 기존에 그 락을 쥐고 있던 모든
프로세스들이 지금은 그것을 해제한 상태일 것을 보장합니다.
이 idiom 은 2.4 리눇 크너러의 네트워킹 스택에서 설정 변경을 조정하기 위해
사용되었습니다.
하지만 이 텅 빈 크리티컬 섹션이 트랜잭션으로 변환된다면, 그 결과는 no-op
입니다.
모든 앞의 크리티컬 섹션들이 종료되었을 것이라는 보장은 사라집니다.
달리 말해서, transactional lock elision 은 락킹의 데이터 보호 semantic 을
유지하지만 락킹의 시간에 기반한 메세지 semantic 은 잃어버립니다.
\iffalse

Although HTM can in many cases be used as a drop-in replacement for locking
(hence the name transactional lock
elision~\cite{DaveDice2008TransactLockElision}),
there are subtle differences in semantics.
A particularly nasty example involving coordinated lock-based critical
sections that results in deadlock or livelock when executed transactionally
was given by Blundell~\cite{Blundell2006TMdeadlock}, but a much simpler
example is the empty critical section.

In a lock-based program, an empty critical section will guarantee
that all processes that had previously been holding that lock have
now released it.
This idiom was used by the 2.4 Linux kernel's networking stack to
coordinate changes in configuration.
But if this empty critical section is translated to a transaction,
the result is a no-op.
The guarantee that all prior critical sections have terminated is
lost.
In other words, transactional lock elision preserves the data-protection
semantics of locking, but loses locking's time-based messaging semantics.
\fi

\QuickQuiz{}
	하지만 \emph{누가} 텅 빈 락 기반의 크리티컬 섹션을 필요로 하나요???
	\iffalse

	But why would \emph{anyone} need an empty lock-based critical
	section???
	\fi
\QuickQuizAnswer{
	Section~\ref{sec:locking:Exclusive Locks}
	의 \QuickQuizARef{\QlockingQemptycriticalsection} 의 답을 보시기 바랍니다.

	하지만, 진행 보장이 없는 강력한 어토믹 HTM 구현들에 대해서, 텅 빈
	크리티컬 섹션에 기반한 메모리 기반의 락킹 설계는 transactional lock
	elision 의 존재에도 올바르게 동작할 거라는 주장이 있습니다.
	비록 저는 이 주장에 대한 증명을 보지는 못했습니다만, 이 주장에는
	직선적인 합리성이 존재합니다.
	주요 아이디어는, 강력한 어토믹 HTM 구현에 있어서, 특정 트랜잭션의
	결과는 그 트랜잭션이 성공적으로 완료되기 전까지는 보여지지 않는다는
	것입니다.
	따라서, 트랜잭션이 시작된 것을 볼 수 있다면, 그것이 이미 완료되었음이
	보장되는데, 이 말은 뒤따르는 텅 빈 락 기반의 크리티컬 섹션은 성공적으로
	그것을 ``기다릴'' 것을 의미합니다---무엇보다도, 기다림이 요구되지
	않습니다.
	\iffalse

	See the answer to \QuickQuizARef{\QlockingQemptycriticalsection} in
	Section~\ref{sec:locking:Exclusive Locks}.

	However, it is claimed that given a strongly atomic HTM
	implementation without forward-progress guarantees, any
	memory-based locking design based on empty critical sections
	will operate correctly in the presence of transactional
	lock elision.
	Although I have not seen a proof of this statement, there
	is a straightforward rationale for this claim.
	The main idea is that in a strongly atomic HTM implementation,
	the results of a given transaction are not visible until
	after the transaction completes successfully.
	Therefore, if you can see that a transaction has started,
	it is guaranteed to have already completed, which means
	that a subsequent empty lock-based critical section will
	successfully ``wait'' on it---after all, there is no waiting
	required.
	\fi

	이 이야기는 (많은 STM 구현을 포함하는) weakly atomic 시스템에는
	적용되지 않고, 통신에 메모리 이외의 수단을 사용하는 락 기반의
	프로그램들에는 적용되지 않습니다.
	그런 수단으로는 시간의 흐름 (예를 들어, hard real-time 시스템)
	이나 우선순위의 흐름 (예를 들어, soft real-time 시스템) 이 있습니다.

	Priority boosting 에 의존하는 락킹 설계는 특히 흥미로운 경우입니다.
	\iffalse

	This line of reasoning does not apply to weakly atomic
	systems (including many STM implementation), and it also
	does not apply to lock-based programs that use means other
	than memory to communicate.
	One such means is the passage of time (for example, in
	hard real-time systems) or flow of priority (for example,
	in soft real-time systems).

	Locking designs that rely on priority boosting are of particular
	interest.
	\fi
} \QuickQuizEnd

\QuickQuiz{}
	락 기반의 텅 빈 크리티컬 섹션들을 생략하지 않는 방법으로 간단하게
	락킹의 시간 기반 메세징 semantic 을 transactional lock elision 에서
	처리할 수는 없을까요?
	\iffalse
	
	Can't transactional lock elision trivially handle locking's
	time-based messaging semantics
	by simply choosing not to elide empty lock-based critical sections?
	\fi
\QuickQuizAnswer{
	그럴수도 있습니다만, 이는 불필요하고 불충분할 겁니다.

	텅 빈 크리티컬 섹션이 조건적 컴파일에 의한 것이라면 이 방법은 필요가
	없습니다.
	여기서는, 해당 락의 유일한 목적은 데이터를 보호하는 것이므로, 이를
	생략시키는 것이 해야할 옳은 일일 것입니다.
	실제로, 락 기반의 텅 빈 크리티컬 섹션을 남겨두는 것은 성능과 확장성을
	떨어뜨릴 수 있습니다.

	또다른 한편, 락 기반의 비어있지 않은 크리티컬 섹션이 락킹의 데이터
	보호화 시간 기반의 메세징 semantic 에 의존하고 있을 수도 있습니다.
	그런 경우에 transactional lock elision 을 사용하는 것은 올바르지 않고,
	버그를 초래할 수 있습니다.
	\iffalse

	It could do so, but this would be both unnecessary and
	insufficient.

	It would be unnecessary in cases where the empty critical section
	was due to conditional compilation.
	Here, it might well be that the only purpose of the lock was to
	protect data, so eliding it completely would be the right thing
	to do.
	In fact, leaving the empty lock-based critical section would
	degrade performance and scalability.

	On the other hand, it is possible for a non-empty lock-based
	critical section to be relying on both the data-protection
	and time-based and messaging semantics of locking.
	Using transactional lock elision in such a case would be
	incorrect, and would result in bugs.
	\fi
} \QuickQuizEnd

\QuickQuiz{}
	최신 하드웨어~\cite{PeterOkech2009InherentRandomness} 에서, 누가 병렬
	소프트웨어가 타이밍에 의존해서 동작할 거라고 기대할 수 있겠습니까?
	\iffalse

	Given modern hardware~\cite{PeterOkech2009InherentRandomness},
	how can anyone possibly expect parallel software relying
	on timing to work?
	\fi
\QuickQuizAnswer{
	짧게 답하자면 일반적으로 구할수 있는 하드웨어에서, 짧은 시간 단위의
	타이밍에 기반한 동기화 설계들은 무모하고 모든 조건 하에서 올바르게
	동작할 거라고 예상될 수 없습니다.

	그렇다고는 하나, 훨씬 더 결정론적인, hard real-time 에서의 사용을 위해
	설계된 시스템들이 있습니다.
	여러분이 그런 시스템을 사용하게 되는 (있을 수 없을법한) 상황이라면 여기
	시간 기반의 동기화가 동작할 수 있는지 보이는 예가 있습니다.
	다시 말하지만, 일반적인 마이크로프로세서는 매우 비결정론적인 성능
	특성들을 가지고 있으므로, 이를 일반적인 마이크로프로세서 위에서는
	시도하지 \emph{마세요}.
	\iffalse

	The short answer is that on commonplace commodity hardware,
	synchronization designs based on any sort of fine-grained
	timing are foolhardy and cannot be expected to operate correctly
	under all conditions.

	That said, there are systems designed for hard real-time use
	that are much more deterministic.
	In the (very unlikely) event that you are using such a system,
	here is a toy example showing how time-based synchronization can
	work.
	Again, do \emph{not} try this on commodity microprocessors,
	as they have highly nondeterministic performance characteristics.
	\fi

	이 예는 하나의 제어 쓰레드와 함께 복수개의 worker 쓰레드를 사용합니다.
	각각의 worker 쓰레드는 외부로 나가는 데이터에 연관되고, 각 단위의 일을
	수행한 후에 (예를 들어, \co{clock_gettime()} 시스템콜로 얻어오는) 현재
	시간을 per-thread \co{my_timestamp} 변수에 저장합니다.
	이 예의 real-time 특성은 다음과 같은 제한을 갖습니다:
	\iffalse

	This example uses multiple worker threads along with a control
	thread.
	Each worker thread corresponds to an outbound data feed, and
	records the current time (for example, from the
	\co{clock_gettime()} system call) in a per-thread
	\co{my_timestamp} variable after executing each unit
	of work.
	The real-time nature of this example results in the following
	set of constraints:
	\fi

	\begin{enumerate}
	\item	특정 worker 쓰레드가 자신의 타임스탬프를 \co{MAX_LOOP_TIME}
		기간이 넘도록 업데이트 하지 못하는 것은 치명적인 에러입니다.
	\item	락들은 글로벌 상태에 접근하고 업데이트하는데 절약적으로
		사용되어야 합니다.
	\item	락들은 각각의 쓰레드 우선순위 내에서는 엄격한 FIFO 순서로
		얻어집니다.
	\iffalse

	\item	It is a fatal error for a given worker thread to fail
		to update its timestamp for a time period of more than
		\co{MAX_LOOP_TIME}.
	\item	Locks are used sparingly to access and update global
		state.
	\item	Locks are granted in strict FIFO order within
		a given thread priority.
	\fi
	\end{enumerate}

	Worker 쓰레드들이 일을 받는걸 완료하면, 이들은 어플리케이션의 다른
	부분들로부터 그들을 풀어내고 $-1$ 로 초기화 한 per-thread
	\co{my_status} 변수의 상태값을 설정해야 합니다.
	쓰레드들은 종료되지 않습니다; 그대신 뒤이어 처리해야할 것들을
	처리해주기 위해 쓰레드 풀에 들어갑니다.
	제어 쓰레드는 필요한 만큼 worker 쓰레드를 할당 (그리고 재할당) 하고,
	또한 쓰레드 상태들의 히스토그램을 관리합니다.
	제어 쓰레드는 worker 쓰레드보다 높지는 않은 real-time 우선순위로
	동작합니다.

	Worker 쓰레드의 코드는 다음과 같습니다:
	\iffalse

	When worker threads complete their feed, they must disentangle
	themselves from the rest of the application and place a status
	value in a per-thread \co{my_status} variable that is initialized
	to $-1$.
	Threads do not exit; they instead are placed on a thread pool
	to accommodate later processing requirements.
	The control thread assigns (and re-assigns) worker threads as
	needed, and also maintains a histogram of thread statuses.
	The control thread runs at a real-time priority no higher than
	that of the worker threads.

	Worker threads' code is as follows:
	\fi

	\vspace{5pt}
	\begin{minipage}[t]{\columnwidth}
	\scriptsize
\begin{verbatim}
  1   int my_status = -1;  /* Thread local. */
  2 
  3   while (continue_working()) {
  4     enqueue_any_new_work();
  5     wp = dequeue_work();
  6     do_work(wp);
  7     my_timestamp = clock_gettime(...);
  8   }
  9 
 10   acquire_lock(&departing_thread_lock);
 11 
 12   /*
 13    * Disentangle from application, might
 14    * acquire other locks, can take much longer
 15    * than MAX_LOOP_TIME, especially if many
 16    * threads exit concurrently.
 17    */
 18   my_status = get_return_status();
 19   release_lock(&departing_thread_lock);
 20 
 21   /* thread awaits repurposing. */
\end{verbatim}
	\end{minipage}
	\vspace{5pt}

	제어 쓰레드의 코드는 다음과 같습니다:
	\iffalse

	The control thread's code is as follows:
	\fi

	\vspace{5pt}
	\begin{minipage}[t]{\columnwidth}
	\scriptsize
\begin{verbatim}
  1   for (;;) {
  2     for_each_thread(t) {
  3       ct = clock_gettime(...);
  4       d = ct - per_thread(my_timestamp, t);
  5       if (d >= MAX_LOOP_TIME) {
  6         /* thread departing. */
  7         acquire_lock(&departing_thread_lock);
  8         release_lock(&departing_thread_lock);
  9         i = per_thread(my_status, t);
 10         status_hist[i]++; /* Bug if TLE! */
 11       }
 12     }
 13     /* Repurpose threads as needed. */
 14   }
\end{verbatim}
	\end{minipage}
	\vspace{5pt}

	Line~5 는 해당 쓰레드가 종료되었는지를 추론하는데에 시간의 흐름을
	사용하고, 만약 그렇다면 line~6-10 을 수행합니다.
	Line~7 과~8 의 락 기반의 텅 빈 크리티컬 섹션은 종료되는 프로세스의 모든
	쓰레드는 완료되었음을 보장합니다 (락은 FIFO 순서로 얻어짐을
	기억하세요!).

	다시 말하건대, 이것들을 일반적인 마이크로프로세서 위에서 수행하려
	시도하지 마세요.
	무엇보다도, hard real-time 사용을 위해 설계된 시스템을 사용할 권한을
	얻기도 충분히 어렵습니다!
	\iffalse

	Line~5 uses the passage of time to deduce that the thread
	has exited, executing lines~6-10 if so.
	The empty lock-based critical section on lines~7 and~8
	guarantees that any thread in the process of exiting
	completes (remember that locks are granted in FIFO order!).

	Once again, do not try this sort of thing on commodity
	microprocessors.
	After all, it is difficult enough to get right on systems
	specifically designed for hard real-time use!
	\fi
} \QuickQuizEnd

락킹과 트랜잭션 사이의 중요한 semantic 상의 차이 하나는 락 기반의 real-time
프로그램에서 우선순위 역전을 막기 위해 상요되는 priority boosting 입니다.
우선순위 역전이 일어날 수 있는 한가지 경우는 락을 쥐고 있는 낮은 우선순위의
쓰레드가 중간 우선순위의 CPU 를 많이 사용하는 쓰레드에 의해 preemption 당하는
경우입니다.
만약 CPU 마다 그런 중간 우선순위 쓰레드가 최소 하나씩은 있다면, 낮은 우선순위
쓰레드는 수행될 기회를 결코 얻지 못할 겁니다.
만약 높은 우선순위 쓰레드가 이제 그 락을 얻으려 시도하면, 이 쓰레드는 블록될
겁니다.
이 쓰레드는 낮은 우선순위 쓰레드가 락을 놓기 전까지는 그 락을 얻지 못할 것이고,
이 낮은 우선순위 쓰레드는 수행될 기회를 얻기 전까지는 그 락을 해제하지를 못할
것이며, 낮은 우선순위 쓰레드는 중간 우선순위 쓰레드가 CPU 를 놓기 전까지는
수행될 기회를 얻지를 못할 겁니다.
따라서, 중간 우선순위 쓰레드는 실질적으로 높은 우선순위 프로세스를 블록하고
있는 셈인데, 이게 바로 ``우선순위 역전'' 이라 불리는 이유입니다.
\iffalse

One important semantic difference between locking and transactions
is the priority boosting that is used to avoid priority inversion
in lock-based real-time programs.
One way in which priority inversion can occur is when a
low-priority thread holding a lock
is preempted by a medium-priority CPU-bound thread.
If there is at least one such medium-priority thread per CPU, the
low-priority thread will never get a chance to run.
If a high-priority thread now attempts to acquire the lock,
it will block.
It cannot acquire the lock until the low-priority thread releases it,
the low-priority thread cannot release the lock until it gets a chance
to run, and it cannot get a chance to run until one of the medium-priority
threads gives up its CPU.
Therefore, the medium-priority threads are in effect blocking the
high-priority process, which is the rationale for the name ``priority
inversion.''
\fi

\begin{figure}[tbp]
{ \scriptsize
\begin{verbbox}
  1 void boostee(void)
  2 {
  3   int i = 0;
  4 
  5   acquire_lock(&boost_lock[i]);
  6   for (;;) {
  7     acquire_lock(&boost_lock[!i]);
  8     release_lock(&boost_lock[i]);
  9     i = i ^ 1;
 10     do_something();
 11   }
 12 }
 13 
 14 void booster(void)
 15 {
 16   int i = 0;
 17 
 18   for (;;) {
 19     usleep(1000); /* sleep 1 ms. */
 20     acquire_lock(&boost_lock[i]);
 21     release_lock(&boost_lock[i]);
 22     i = i ^ 1;
 23   }
 24 }
\end{verbbox}
}
\centering
\theverbbox
\caption{Exploiting Priority Boosting}
\label{fig:future:Exploiting Priority Boosting}
\end{figure}

우선순위 역전을 막기 위한 한가지 방법은 \emph{priority inheritance} 로, 락에
의해 블록된 높은 우선순위 쓰레드가 임시적으로 자신의 우선순위를 락을 쥔
쓰레드에게 넘겨주는 것인데, \emph{priority boosting} 이라고도 불립니다.
하지만, priority boosting 은 우선순위 역전을 막는것 외에도
Figure~\ref{fig:future:Exploiting Priority Boosting} 에 보여진 것처럼도 사용될
수 있습니다.
이 그림의 line~1-12 는 매 밀리세컨드마다 수행되어야 하는 낮은 우선순위
프로세스를 보이고 있고, line~14-24 는 \co{boostee()} 가 주기적으로 필요한 만큼
수행되는 것을 보장하기 위해 priority boosting 을 사용하는 높은 우선순위
프로세스를 보입니다.

\co{boostee()} 함수는 두개의 \co{boost_lock[]} 락들 가운데 하나를 항상 쥐고
있어서 \co{booster()} 의 line~20-21 이 우선순위를 필요한만큼 증폭시킬 수 있게
함으로써 이를 가능하게 합니다.
\iffalse

One way to avoid priority inversion is \emph{priority inheritance},
in which a high-priority thread blocked on a lock temporarily donates
its priority to the lock's holder, which is also called \emph{priority
boosting}.
However, priority boosting can be used for things other than avoiding
priority inversion, as shown in
Figure~\ref{fig:future:Exploiting Priority Boosting}.
Lines~1-12 of this figure show a low-priority process that must
nevertheless run every millisecond or so, while lines~14-24 of
this same figure show a high-priority process that uses priority
boosting to ensure that \co{boostee()} runs periodically as needed.

The \co{boostee()} function arranges this by always holding one of
the two \co{boost_lock[]} locks, so that lines~20-21 of
\co{booster()} can boost priority as needed.
\fi

\QuickQuiz{}
	하지만
	Figure~\ref{fig:future:Exploiting Priority Boosting}
	의 \co{boostee()} 함수는 그 락을 반대 순서로 잡고 있어요!
	이는 deadlock 을 초래할 수 있지 않을까요?
	\iffalse

	But the \co{boostee()} function in
	Figure~\ref{fig:future:Exploiting Priority Boosting}
	alternatively acquires its locks in reverse order!
	Won't this result in deadlock?
	\fi
\QuickQuizAnswer{
	Deadlock 은 일어나지 않을 겁니다.
	Deadlock 이 일어나려면, 두개의 다른 쓰레드가 각각 두개의 락을 반대
	순서로 잡아야 하는데, 이 예에서는 그런 일은 없습니다.
	하지만, lockdep~\cite{JonathanCorbet2006lockdep} 과 같은 deadlock
	detector 들은 이에 거짓 양성 반응을 보일 수 있습니다.
	\iffalse

	No deadlock will result.
	To arrive at deadlock, two different threads must each
	acquire the two locks in oppposite orders, which does not
	happen in this example.
	However, deadlock detectors such as
	lockdep~\cite{JonathanCorbet2006lockdep}
	will flag this as a false positive.
	\fi
} \QuickQuizEnd

이 구조는 \co{boostee()} 가 시스템이 바빠지기 전에 line~5 에서 첫번째 락을 잡을
것을 필요로 합니다만, 이는 최신 하드웨어에서조차도 쉽게 가능합니다.

안타깝게도, 이 구조는 transactional lock elision 의 존재 하에서는 깨질 수
있습니다.
\co{boostee()} 함수의 겹쳐지는 크리티컬 섹션들은 잠시후든 나중이든 abort 될
하나의 무한한 트랜잭션이 되어버리는데,  예를 들면 \co{boostee()} 함수를
수행하는 쓰레드가 preemption 당하는 첫번째 시점이 되겠습니다.
이 시점에서, \co{boostee()} 는 락킹으로 물러나게 됩니다만, 그 낮은 우선순위와
초기화 단계가 완료되었다는 사실 때문에 (이게 바로 \co{boostee()} 가 preemption
을 당한 이유입니다), 이 쓰레드는 다시 수행될 기회를 얻지 못하게 됩니다.

그리고 만약 \co{boostee()} 쓰레드가 락을 잡고 있지 않다면, \co{booster()}
쓰레드의
Figure~\ref{fig:future:Exploiting Priority Boosting}
line~20 과~21 에서의 텅빈 크리티컬 섹션은 아무런 효과를 갖지 못하는 텅빈
트랜잭션이 되어서, \co{boostee()} 는 결코 수행되지 않을 겁니다.
이 예는 트랜잭셔널 메모리의 rollback-and-retry 시맨틱의 미묘한 결론을 그리고
있습니다.
\iffalse

This arrangement requires that \co{boostee()} acquire its first
lock on line~5 before the system becomes busy, but this is easily
arranged, even on modern hardware.

Unfortunately, this arrangement can break down in presence of transactional
lock elision.
The \co{boostee()} function's overlapping critical sections become
one infinite transaction, which will sooner or later abort,
for example, on the first time that the thread running
the \co{boostee()} function is preempted.
At this point, \co{boostee()} will fall back to locking, but given
its low priority and that the quiet initialization period is now
complete (which after all is why \co{boostee()} was preempted),
this thread might never again get a chance to run.

And if the \co{boostee()} thread is not holding the lock, then
the \co{booster()} thread's empty critical section on lines~20 and~21 of
Figure~\ref{fig:future:Exploiting Priority Boosting}
will become an empty transaction that has no effect, so that
\co{boostee()} never runs.
This example illustrates some of the subtle consequences of
transactional memory's rollback-and-retry semantics.
\fi

경험이 추가적인 묘한 semantic 상의 차이를 더 드러낼 것인데, HTM 기반의 lock
elision 의 커다란 프로그램으로의 적용은 주의 하에 이루어져야 합니다.
그렇다곤 하나, 적용이 되는 곳이라면, HTM 기반 lock elision 은 해당 락 변수에
연관된 캐시 미스들을 제거할 수 있어서 2015년 초에 있어서 실제 세계의 커다란
소프트웨어 시스템들에서는 수십 퍼센트의 성능 향상을 가져옵니다.
따라서 우리는 이 기술을 지원하는 하드웨어에서의 이 테크닉의 상당한 사용을
기대합니다.
\iffalse

Given that experience will likely uncover additional subtle semantic
differences, application of HTM-based lock elision to large programs
should be undertaken with caution.
That said, where it does apply, HTM-based lock elision can eliminate
the cache misses associated with the lock variable, which has resulted
in tens of percent performance increases in large real-world software
systems as of early 2015.
We can therefore expect to see substantial use of this technique on
hardware supporting it.
\fi

\QuickQuiz{}
	그래서 많은 사람들이 락킹을 대신하는 작업을 시작하고는 대부분은 락킹을
	최적화 하는 것으로 결론을 내리나요???
	\iffalse

	So a bunch of people set out to supplant locking, and they
	mostly end up just optimizing locking???
	\fi
\QuickQuizAnswer{
	그들은 최소한 어떤 유용한 것을 얻습니다!
	그리고 시간이 흐름에 따라 HTM 에 추가적인 진보가 있을 수 있습니다.
	\iffalse

	At least they accomplished something useful!
	And perhaps there will be additional HTM progress over time.
	\fi
} \QuickQuizEnd

\subsubsection{Summary}
\label{sec:future:HTM Weaknesses WRT Locking: Summary}

\input{future/HTMtable}

HTM 이 강력한 사용 예들을 갖는 것처럼 보이긴 하지만, 현재의 구현들은 주의깊은
처리를 필요로 하는, 트랜잭션 크기, conflict 처리의 복잡성, abort-and-rollback
이슈, 그리고 semantic 상의 차이와 같은 심각한 한계들을 가지고 있습니다.  HTM 의
현재 상황이 락킹과 비교해서
Table~\ref{tab:future:Comparison of Locking and HTM} 에 요약되어 있습니다.
보여지듯이, HTM 의 현재 상황이 락킹의 일부 심각한
단점들을 경감시키긴 하지만,\footnote{
	공정성을 위해 말해두자면, 락킹의 단점들은 잘 알려져 있고 널리 사용되고
	있는, deadlock detector~\cite{JonathanCorbet2006lockdep}, 락킹에 적용된
	데이터 구조의 풍부함, 그리고
	Section~\ref{sec:future:HTM Weaknesses WRT to Locking When Augmented}
	에서 이야기한 것과 같이 결합되어 사용되어온 긴 역사를 포함한,
	엔지니어링 단계에서의 해결책들이 있음을 강조해둘 필요가 있습니다.
	한가지 더 말하자면, 락킹이 정말로 많은 학계의 논문들을 살짝만 보아도
	믿어질 만큼 끔찍한 것이었다면, 그 수많은 락 기반의 (FOSS 와 독점의)
	병렬 프로그램들은 대체 어디서 나왔을까요?}
HTM 역시 그 자체의 상당히 많은 단점들을 포함하고 있습니다.  이러한 단점들은 TM
커뮤니티의 리더들에 의해서도 인정된 바
있습니다~\cite{AlexanderMatveev2012PessimisticTM}.\footnote{
	또한, 2011년 초에, 저는 트랜잭셔널 메모리를 둘러싼 일부 가정에 대한
	비평을 하도록 초대된 바 있습니다~\cite{PaulEMcKenney2011Verico}.
	제가 발표를 하기 위해 시차에 매우 시달렸기 때문에 저를 편하게 해주기
	위해서였을지는 몰라도, 청중은 놀라우리만큼 적대적이지 않았습니다.}
\iffalse

Although it seems likely that HTM will have compelling use cases,
current implementations have serious transaction-size limitations,
conflict-handling complications, abort-and-rollback issues, and
semantic differences that will require careful handling.
HTM's current situation relative to locking is summarized in
Table~\ref{tab:future:Comparison of Locking and HTM}.
As can be seen, although the current state of HTM alleviates some
serious shortcomings of locking,\footnote{
	In fairness, it is important to emphasize that locking's shortcomings
	do have well-known and heavily used engineering solutions, including
	deadlock detectors~\cite{JonathanCorbet2006lockdep}, a wealth
	of data structures that have been adapted to locking, and
	a long history of augmentation, as discussed in
	Section~\ref{sec:future:HTM Weaknesses WRT to Locking When Augmented}.
	In addition, if locking really were as horrible as a quick skim
	of many academic papers might reasonably lead one to believe,
	where did all the large lock-based parallel programs (both
	FOSS and proprietary) come from, anyway?}
it does so by introducing a significant
number of shortcomings of its own.
These shortcomings are acknowledged by leaders in the TM
community~\cite{AlexanderMatveev2012PessimisticTM}.\footnote{
	In addition, in early 2011, I was invited to deliver a critique of
	some of the assumptions underlying transactional
	memory~\cite{PaulEMcKenney2011Verico}.
	The audience was surprisingly non-hostile, though perhaps they
	were taking it easy on me due to the fact that I was heavily
	jet-lagged while giving the presentation.}
\fi

또한, 이게 전부가 아닙니다.
락킹은 일반적으로 그 자체만으로 사용되지 않고, 보통 레퍼런스 카운팅, 어토믹
오퍼레이션, non-blocking 데이터 구조, 해저드
포인터~\cite{MagedMichael04a,HerlihyLM02}, 그리고 read-copy update
(RCU)~\cite{McKenney98,McKenney01a,ThomasEHart2007a,PaulEMcKenney2012ELCbattery}
등과 같은 다른 동기화 메커니즘들과 결합되어 사용됩니다.
다음 섹션은 그러한 결합이 이 수식을 어떻게 바꾸어놓는지 봅니다.
\iffalse

In addition, this is not the whole story.
Locking is not normally used by itself, but is instead typically
augmented by other synchronization mechanisms,
including reference counting, atomic operations, non-blocking data structures,
hazard pointers~\cite{MagedMichael04a,HerlihyLM02},
and RCU~\cite{McKenney98,McKenney01a,ThomasEHart2007a,PaulEMcKenney2012ELCbattery}.
The next section looks at how such augmentation changes the equation.
\fi

\subsection{HTM Weaknesses WRT to Locking When Augmented}
\label{sec:future:HTM Weaknesses WRT to Locking When Augmented}

\input{future/HTMtableRCU}

실무자들은 락킹의 일부 단점들을 막기 위해 오랫동안 레퍼런스 카운팅, 어토믹
오퍼레이션, non-blocking 데이터 구조, 해저드 포인터, 그리고 RCU 를 사용해
왔습니다.
예를 들어, deadlock 은 많은 경우에 레퍼런스 카운트, 해저드 포인터, 또는 RCU 를
데이터 구조를 보호하는데에 사용함으로써 막아질 수 있고, 특히나 read-only
크리티컬 섹션에서는
그렇습니다~\cite{MagedMichael04a,HerlihyLM02,MathieuDesnoyers2012URCU,DinakarGuniguntala2008IBMSysJ,ThomasEHart2007a}.
이 방법은 또한
Chapter~\ref{chp:Data Structures} 에서 봤던 것처럼 데이터 구조를 분할할 필요를
줄여줍니다.
RCU 는 더 나아가서 contention 에서 자유로운 wait-free read-side 기능들을
제공합니다~\cite{MathieuDesnoyers2012URCU}.
이런 점을
Table~\ref{tab:future:Comparison of Locking and HTM} 에 추가하면
Table~\ref{tab:future:Comparison of Locking (Augmented by RCU or Hazard Pointers) and HTM}
에 보인, augmented locking 과 HTM 사이의 업데이트된 비교가 나옵니다.
두개의 표간의 차이점을 요약해보면 다음과 같습니다:
\iffalse

Practitioners have long used reference counting, atomic operations,
non-blocking data structures, hazard pointers, and RCU to avoid some
of the shortcomings of locking.
For example, deadlock can be avoided in many cases by using reference
counts, hazard pointers, or RCU to protect data structures,
particularly for read-only critical
sections~\cite{MagedMichael04a,HerlihyLM02,MathieuDesnoyers2012URCU,DinakarGuniguntala2008IBMSysJ,ThomasEHart2007a}.
These approaches also reduce the need to partition data
structures, as was see in Chapter~\ref{chp:Data Structures}.
RCU further provides contention-free wait-free read-side
primitives~\cite{MathieuDesnoyers2012URCU}.
Adding these considerations to
Table~\ref{tab:future:Comparison of Locking and HTM}
results in the updated comparison between augmented locking and HTM
shown in
Table~\ref{tab:future:Comparison of Locking (Augmented by RCU or Hazard Pointers) and HTM}.
A summary of the differences between the two tables is as follows:
\fi

\begin{enumerate}
\item	Non-blocking read-side 메커니즘의 사용은 deadlock 문제를 줄여줍니다.
\item	해저드 포인터와 RCU 와 같은 Read-side 메커니즘들은 분할이 불가능한
	데이터에서 효과적으로 동작할 수 있습니다.
\item	해저드 포인터와 RCU 는 서로간에 또는 업데이트 쓰레드와 충돌하지 않아서,
	읽기가 대부분인 워크로드에서는 훌륭한 성능과 확장성을 제공합니다.
\item	해저드 포인터와 RCU 는 진행 보장을 제공합니다 (각각 lock freedom 과
	wait-freedom 을 제공합니다).
\item	해저드 포인터와 RCU 에서의 privatization 오퍼레이션들은 간단합니다.
\iffalse

\item	Use of non-blocking read-side mechanisms alleviates deadlock issues.
\item	Read-side mechanisms such as hazard pointers and RCU can operate
	efficiently on non-partitionable data.
\item	Hazard pointers and RCU do not contend with each other or with
	updaters, allowing excellent performance and scalability for
	read-mostly workloads.
\item	Hazard pointers and RCU provide forward-progress guarantees
	(lock freedom and wait-freedom, respectively).
\item	Privatization operations for hazard pointers and RCU are
	straightforward.
\fi
\end{enumerate}

물론, 다음 섹션에서 논의하듯이 HTM 을 다른 기능들과 결합하는 것도 가능합니다.
\iffalse

Of course, it is also possible to augment HTM,
as discussed in the next section.
\fi

\subsection{Where Does HTM Best Fit In?}
\label{sec:future:Where Does HTM Best Fit In?}

HTM 의 적용 영역이
page~\pageref{fig:defer:RCU Areas of Applicability} 의
Figure~\ref{fig:defer:RCU Areas of Applicability} 에 보여진 RCU 처럼 그려지기
전인 것 같긴 하지만, 그런 방향으로 움직이기 시작하지 않을 이유는 없습니다.

HTM 은 커다란 멀티프로세서에서 돌아가는 상대적으로 커다란 in-memory 데이터
구조의 겹치지 않는 영역에 대한 상대적으로 작은 변경에 연관된 업데이트 위주의
워크로드에 들어맞을 텐데, 이런 워크로드는 현재 HTM 구현의 크기 제약을 맞출 수
있고 conflict 과 그로인한 abort 와 롤백의 확률을 최소화 시켜줄 것이기
때문입니다.
이 시나리오는 현재의 동기화 도구들을 가지고는 처리하기가 상대적으로 어려운
시나리오이기도 합니다.
\iffalse

Although it will likely be some time before HTM's area of applicability
can be as crisply delineated as that shown for RCU in
Figure~\ref{fig:defer:RCU Areas of Applicability} on
page~\pageref{fig:defer:RCU Areas of Applicability}, that is no reason not to
start moving in that direction.

HTM seems best suited to update-heavy workloads involving relatively
small changes to disparate portions of relatively large in-memory
data structures running on large multiprocessors,
as this meets the size restrictions of current HTM implementations while
minimizing the probability of conflicts and attendant aborts and
rollbacks.
This scenario is also one that is relatively difficult to handle given
current synchronization primitives.
\fi

HTM 과 함께 락킹을 사용하는 것은 돌이킬 수 없는 오퍼레이션들에 대한 HTM 의
어려움을 극복해줄 수 있을 것으로 보이며, RCU 나 해저드 포인터의 사용은 데이터
구조의 커다란 부분을 횡단하는, read-only 오퍼레이션들에 대해서 HTM 의 트랜잭션
크기 제한을 해결해 줄 수 있을 것입니다.
현재의 HTM 구현들은 RCU 나 해저드 포인터 읽기 쓰레드와 conflict 나는 업데이트
트랜잭션을 무조건적으로 abort 시키고 있지만, 미래의 HTM 구현들은 이런 동기화
메커니즘들과 더 부드럽게 작용할 것입니다.
그전까지는, 커다란 RCU 나 해저드 포인터 read-side 크리티컬 섹션과 conflict 나는
업데이트의 확률은 동일한 read-only 트랜잭션과 conflict 날 확률에 비해 훨씬
작아야 합니다.\footnote{
	NoSQL 데이터베이스들이 데이터베이스 어플리케이션의 엄격한 트랜잭션에의
	의존도를 완화시키고 있는 상황에서 shared-memory 시스템에서는 엄격한
	트랜잭션 메커니즘이 떠오르는 것은 상당히 아이러닉합니다.}
더도 아니고 덜도 아니고, RCU 나 해저드 포인터 읽기 쓰레드들의 느린 흐름이
연관된 conflict 의 느린 흐름으로 인해 업데이트 쓰레드를 starve 시킬 수도
있습니다.
이런 취약점은 앞의 트랜잭션의 로드된 메모리 로케이션의 복사본에 대한 트랜잭션
외적인 읽기 방법을 제공하는 것으로 (상당한 하드웨어 비용과 복잡성을 요하겠지만)
제거될 수도 있습니다.
\iffalse

Use of locking in conjunction with HTM seems likely to overcome HTM's
difficulties with irrevocable operations, while use of RCU or
hazard pointers might alleviate HTM's transaction-size limitations
for read-only operations that traverse large fractions of the data
structure.
Current HTM implementations unconditionally abort an update transaction
that conflicts with an RCU or hazard-pointer reader, but perhaps future
HTM implementations will interoperate more smoothly with these
synchronization mechanisms.
In the meantime, the probability of an update conflicting with a
large RCU or hazard-pointer read-side critical section should be
much smaller than the probability of conflicting with the equivalent
read-only transaction.\footnote{
	It is quite ironic that strictly transactional mechanisms are
	appearing in shared-memory systems at just about the time
	that NoSQL databases are relaxing the traditional
	database-application reliance on strict transactions.}
Nevertheless, it is quite possible that a steady stream of RCU or
hazard-pointer readers might starve updaters due to a corresponding
steady stream of conflicts.
This vulnerability could be eliminated (perhaps at significant
hardware cost and complexity) by giving extra-transactional
reads the pre-transaction copy of the memory location being loaded.
\fi

HTM 트랜잭션들이 fallback 을 가져야만 한다는 사실로 인해 어떤 경우에는 데이터
구조들의 정적 분할 가능성을 강제해야만 할수도 있습니다.
이 제약점은 미래의 HTM 구현들이 진행 보장을 제공한다면 어떤 경우에는 fallback
코드의 필요성을 제거할 것이므로 없어질 수 있을 것인데, 이는 HTM 이 높은
conflict 확률 아래의 환경에서도 효과적으로 사용될 수 있을 것입니다.

요약하자면, HTM 이 중요한 사용과 응용 분야를 가질 수 있을 것이지만, 병렬
프로그래머의 도구상자의 또다른 도구일 뿐이지, 도구상자 전체의 대체제는
아닙니다.
\iffalse

The fact that HTM transactions must have fallbacks might in some cases
force static partitionability of data structures back onto HTM.
This limitation might be alleviated if future HTM implementations
provide forward-progress guarantees, which might eliminate the need
for fallback code in some cases, which in turn might allow HTM to
be used efficiently in situations with higher conflict probabilities.

In short, although HTM is likely to have important uses and applications,
it is another tool in the parallel programmer's toolbox, not a replacement
for the toolbox in its entirety.
\fi

\subsection{Potential Game Changers}
\label{sec:future:Potential Game Changers}

HTM 의 필요성을 상당히 증가시킬 Game changer 들은 다음과 같은 것들이 있습니다:
\iffalse

Game changers that could greatly increase the need for HTM include
the following:
\fi

\begin{enumerate}
\item	진행 보장.
\item	트랜잭션 크기 증가.
\item	개선된 디버깅 지원.
\item	완화된 원자성.
\iffalse

\item	Forward-progress guarantees.
\item	Transaction-size increases.
\item	Improved debugging support.
\item	Weak atomicity.
\fi
\end{enumerate}

이것들은 다음 섹션들에서 확장되어 설명됩니다.
\iffalse

These are expanded upon in the following sections.
\fi

\subsubsection{Forward-Progress Guarantees}
\label{sec:future:Forward-Progress Guarantees}

Section~\ref{sec:future:Lack of Forward-Progress Guarantees} 에서 논의한 것과
같이, 현재의 HTM 구현들은 진행 보장을 갖지 않아서 HTM 의 실패된 트랜잭션을
처리하기 위한 fallback 소프트웨어를 필요로 합니다.
물론, 보장을 추가하는 건 쉽습니다만, 그걸 제공하는건 항상 쉽지는 않습니다.
HTM 의 경우에, 보장을 제공하는데 걸리는 문제는 캐시 크기와 캐시 associativity,
TLB 크기와 TLB asociativity, 트랜잭션의 시간적 길이와 인터럽트 빈도, 그리고
스케쥴러 구현 등이 포함됩니다.
\iffalse

As was discussed in
Section~\ref{sec:future:Lack of Forward-Progress Guarantees},
current HTM implementations lack forward-progress guarantees, which requires
that fallback software be available to handle HTM failures.
Of course, it is easy to demand guarantees, but not always easy
to provide them.
In the case of HTM, obstacles to guarantees can include cache size and
associativity, TLB size and associativity, transaction duration and
interrupt frequency, and scheduler implementation.
\fi

캐시 크기와 캐시 associativity 는
Section~\ref{sec:future:Transaction-Size Limitations} 에서 현재의 한계점들을
우회하기 위한 일부 연구 작업들과 함께 논의된 바 있습니다.
하지만, HTM 의 진행 보장은 크기에 대하 한계와 함께 제공될 겁니다.
그런데 현재의 HTM 구현들은, 예를 들자면 캐시의 associativity 의 한계에 맞춰진
작은 트랜잭션들에 대해서는 진행 보장을 제공하지 않는 걸까요?
한가지 잠재적인 이유는 하드웨어 failure를 처리해야 하는 필요성일 수 있습니다.
예를 들어, 문제가 생긴 캐시의 SRAM 셀은 해당 문제가 생긴 셀을 비활성화 시키는
것으로 처리될 수 있는데, 이는 캐시의 associativity 를 줄이게 되고 따라서 진행
보장이 제공될 수 있ㄴ느 트랜잭션의 최대 크기 역시 줄이게 됩니다.
이는 단순히 보장된 트랜잭션 크기를 줄일 뿐이란 점을 생각해 보면, 실제로는 다른
이유들도 있을 것으로 보입니다.
제품 수준의 하드웨어에서 진행 보장을 제공하는 것은 아마도 소프트웨어에서 진행
보장을 제공하는데 걸리는 어려움보다도 더 어려울 수 있습니다.
따라서 문제를 더 쉽게 풀기 위해, 문제를 소프트웨어에서 하드웨어로 옮기는 게
필요합니다.
\iffalse

Cache size and associativity was discussed in
Section~\ref{sec:future:Transaction-Size Limitations},
along with some research intended to work around current limitations.
However, HTM forward-progress guarantees would
come with size limits, large though these limits might one day be.
So why don't current HTM implementations provide forward-progress
guarantees for small transactions, for example, limited to the
associativity of the cache?
One potential reason might be the need to deal with hardware failure.
For example, a failing cache SRAM cell might be handled by deactivating
the failing cell, thus reducing the associativity of the cache and
therefore also the maximum size of transactions that can be guaranteed
forward progress.
Given that this would simply decrease the guaranteed transaction size,
it seems likely that other reasons are at work.
Perhaps providing forward progress guarantees on production-quality
hardware is more difficult than one might think, an entirely plausible
explanation given the difficulty of making forward-progress guarantees
in software.
Moving a problem from software to hardware does not necessarily make
it easier to solve.
\fi

물리적으로 tag 되고 index 되는 캐시에 있어서는 트랜잭션이 캐시 안에 들어간다는
것만으로는 충분치 않습니다.
여기서는 주소변환이 TLB 에 들어 맞기도 해야 합니다.
따라서 모든 진행 보장은 TLB 크기와 TLB associativity 도 신경을 써야 합니다.

현재의 HTM 구현들에서는 인터럽트, trap, 그리고 exception 이 트랜잭션을 abort
시킨다는 점을 생각해 보면, 특정 트랜잭션의 수행 시간 길이가 인터럽트 간의
예상되는 시간 간격보다 짧아야 할 필요가 있습니다.
특정 트랜잭션이 얼마나 적은 데이터만 건드리는가와는 관계 없이, 너무 오랫동안
수행된다면, 해당 트랜잭션은 abort 될겁니다.
따라서, 모든 진행 보장은 트랜잭션 크기만이 아니라 트랜잭션 수행시간에 대해서도
조정되어야만 합니다.
\iffalse

Given a physically tagged and indexed cache, it is not enough for the
transaction to fit in the cache.
Its address translations must also fit in the TLB.
Any forward-progress guarantees must therefore also take TLB size
and associativity into account.

Given that interrupts, traps, and exceptions abort transactions in current
HTM implementations, it is necessary that the execution duration of
a given transaction be shorter than the expected interval between
interrupts.
No matter how little data a given transaction touches, if it runs too
long, it will be aborted.
Therefore, any forward-progress guarantees must be conditioned not only
on transaction size, but also on transaction duration.
\fi

진행 보장은 conflict 되는 여러개의 트랜잭션들 가운데 어떤 트랜잭션이 aobrt
되어야 하는지 결정하는 능력에도 특히 의존적입니다.
각각 앞의 트랜잭션을 aobrt 시키고 자기 자신도 뒤의 트랜잭션에 의해서 aobrt
되어서 어떤 트랜잭션도 실질적으로 커밋하지 못하는 무한한 트랜잭션의 연속을 쉽게
생각해 볼 수 있습니다.
Conflict 처리의 복잡성은 제안된 바 있는 많은 수의 HTM conflict 해결
정책들~\cite{EgeAkpinar2011HTM2TLE,YujieLiu2011ToxicTransactions} 로 알 수
있습니다.
트랜잭션 외적인 액세스들로 인해 추가적인 복잡성들이 나타나는데, Blundell에 의해
알려졌습니다~\cite{Blundell2006TMdeadlock}.
이런 모든 문제들에 대해서 트랜잭션 외적인 액세스들을 탓하기는 쉽습니다만, 이런
생각의 어리석음은 각각의 트랜잭션 외적인 액세슫르을 그 자신만의 하나의 액세스로
구성된 트랜잭션으로 교체함으로써 쉽게 드러납니다.
이는 그 자체로 문제인 액세스 패턴이지, 그것들이 트랜잭션 안에서 일어나느냐
아니냐의 문제가 아닙니다.
\iffalse

Forward-progress guarantees depend critically on the ability to determine
which of several conflicting transactions should be aborted.
It is all too easy to imagine an endless series of transactions, each
aborting an earlier transaction only to itself be aborted by a later
transactions, so that none of the transactions actually commit.
The complexity of conflict handling is
evidenced by the large number of HTM conflict-resolution policies
that have been proposed~\cite{EgeAkpinar2011HTM2TLE,YujieLiu2011ToxicTransactions}.
Additional complications are introduced by extra-transactional accesses,
as noted by Blundell~\cite{Blundell2006TMdeadlock}.
It is easy to blame the extra-transactional accesses for all of these
problems, but the folly of this line of thinking is easily demonstrated
by placing each of the extra-transactional accesses into its own
single-access transaction.
It is the pattern of accesses that is the issue, not whether or not they
happen to be enclosed in a transaction.
\fi

마지막으로, 트랜잭션을 위한 모든 진행 보장은 해당 트랜잭션을 수행하는 쓰레드가
커밋을 성공하기에 충분할 만큼 길게 수행될 수 있도록 해줄 수 있는 스케쥴러에
의존적입니다.

따라서 진행 보장을 제공하는 HTM 제작사들에게는 상당한 문제가 존재합니다.
하지만, 그것들을 해결하면 얻을 수 있는 효과는 대단합니다.
그렇게 되면 HTM 트랜잭션은 소프트웨어 fallback 이 더이상 필요없어지는데, 이는
HTM 은 마침내 TM 의 deadlock 제거의 약속을 선사하게 됨을 의미합니다.

그리고 2012년 말, IBM Mainframe 은 일반적인 최선의 HTM 구현에 더해서
\emph{constrained transaction} 을 포함하는 HTM 구현을
발표했습니다~\cite{ChristianJacobi2012MainframeTM}.
constrained transaction 은 best-effort 트랜잭션을 시작하는데 사용되는
\co{tbegin} 명령과 달리 \co{tbeginc} 명령으로 시작됩니다.
Constrained transaction 은 항상 (결국은) 성공할 것이 보장되어 있으며, 따라서
만약 어떤 트랜잭션이 abort 된다면, (best-effort 트랜잭션이 그렇듯이) fallback
path 로 분기되기 보다는 하드웨어가 그 트랜잭션을 \co{tbeginc} 명령으로부터
재시작 시킵니다.
\iffalse

Finally, any forward-progress guarantees for transactions also
depend on the scheduler, which must let the thread executing the
transaction run long enough to successfully commit.

So there are significant obstacles to HTM vendors offering forward-progress
guarantees.
However, the impact of any of them doing so would be enormous.
It would mean that HTM transactions would no longer need software
fallbacks, which would mean that HTM could finally deliver on the
TM promise of deadlock elimination.

And as of late 2012, the IBM Mainframe announced an HTM
implementation that includes \emph{constrained transactions}
in addition to the usual best-effort HTM
implementation~\cite{ChristianJacobi2012MainframeTM}.
A constrained transaction starts with the \co{tbeginc} instruction
instead of the \co{tbegin} instruction that is used for best-effort
transactions.
Constrained transactions are guaranteed to always complete (eventually),
so if a transaction aborts, rather than branching to a fallback path
(as is done for best-effort transactions), the hardware instead restarts
the transaction at the \co{tbeginc} instruction.
\fi

이 Mainframe 구조는 이 진행 보장을 위해서 상당한 측정을 해야 했습니다.
특정 constrained transaction 이 반복적으로 실패한다면, 이 CPU 는 branch
prediction 을 비활성화 하고, in-order execution 을 강제하고, 심지어 pipelining
을 비활성화 시킬 수도 있습니다.
만약 반복된 failure 가 높은 contention 때문이라면, CPU 는 speculative fetche 를
비활성화 하고, 무작위적 delay 를 집어넣고, 심지어 충돌하는 CPU 들의 수행을
직렬화 시킬 수조차 있습니다.
``흥미로운'' 진행 보장 시나리오는 두개의 CPU 만이 관여될수도, 백개의 CPU 들이
관여될 수도 있습니다.
아마도 이런 상당한 측정은 왜 다른 CPU 들이 constrained transaction 을 제공하는
것을 그렇게도 자제했는지에 대한 이해를 일부 제공합니다.

그 이름이 이야기 하듯이, constrained transaction 은 실제로 상당히 제약되어
있습니다:
\iffalse

The Mainframe architects needed to take extreme measures to deliver on
this forward-progress guarantee.
If a given constrained transaction repeatedly fails, the CPU
might disable branch prediction, force in-order execution, and even
disable pipelining.
If the repeated failures are due to high contention, the CPU might
disable speculative fetches, introduce random delays, and even
serialize execution of the conflicting CPUs.
``Interesting'' forward-progress scenarios involve as few as two CPUs
or as many as one hundred CPUs.
Perhaps these extreme measures provide some insight as to why other CPUs
have thus far refrained from offering constrained transactions.

As the name implies, constrained transactions are in fact severely constrained:
\fi

\begin{enumerate}
\item	최대 데이터 사용량은 메모리의 4개 블럭으로 제한되는데, 각각의 블럭은 32
	바이트 이상이 될 수 없습니다.
\item	최대 코드 크기는 256 바이트 입니다.
\item	만약 특정 4K 페이지가 constrained transaction 의 코드를 담고 있다면,
	해당 페이지는 그 트랜잭션의 데이터를 담고 있을 수 없습니다.
\item	실행될 수 있는 assembly 인스트럭션의 최대 갯수는 32 입니다.
\item	뒤 방향으로의 분기는 금지됩니다.
\iffalse

\item	The maximum data footprint is four blocks of memory,
	where each block can be no larger than 32 bytes.
\item	The maximum code footprint is 256 bytes.
\item	If a given 4K page contains a constrained transaction's code,
	then that page may not contain that transaction's data.
\item	The maximum number of assembly instructions that may be executed
	is 32.
\item	Backwards branches are forbidden.
\fi
\end{enumerate}

더도 아니고 덜도 아니고, 이러한 제약들은 링크드 리스트, 스택, 큐, 그리고 배열과
같은 여러개의 중요한 데이터 구조들을 지원합니다.
따라서 constrained HTM 은 병렬 프로그래머의 도구상자에서 중요한 도구가 될 수
있을 것으로 보입니다.

이런 진행 보장성은 절대적일 필요는 없다는 것을 알아두시기 바랍니다.
예를 들어, HTM 의 사용이 fallback 으로 global lock 을 사용한다고 생각해 봅시다.
이 fallback 메카니즘이
Section~\ref{sec:future:Aborts and Rollbacks} 에서 이야기된 ``lemming effect''
를 피할 수 있도록 잘 설계되었다고 하면, HTM 롤백들이 충분히 드물게 일어난다면,
이 global lock 은 병목이 되지 않을 겁니다.
그렇다곤 하나, 시스템이 커질수록, 크리티컬 섹션이 길어질수록, ``lemming
effect'' 로부터 회복되는데 필요한 시간이 길어질수록, 더욱 드문 ``충분한
드물음'' 이 필요해질 겁니다.
\iffalse

Nevertheless, these constraints support a number of important data structures,
including linked lists, stacks, queues, and arrays.
Constrained HTM therefore seems likely to become an important tool in
the parallel programmer's toolbox.

Note that these forward-progress guarantees need not be absolute.
For example, suppose that a use of HTM uses a global lock as fallback.
Assuming that the fallback mechanism has been carefully designed to
avoid the ``lemming effect'' discussed in
Section~\ref{sec:future:Aborts and Rollbacks},
then if HTM rollbacks are sufficiently infrequent, the global lock
will not be a bottleneck.
That said, the larger the system, the longer the critical sections,
and the longer the time required to recover from the ``lemming effect'',
the more rare ``sufficiently infrequent'' needs to be.
\fi

\subsubsection{Transaction-Size Increases}
\label{sec:future:Transaction-Size Increases}

진행 보장은 중요하지만, 우리가 봤듯이, 트랜잭션 크기와 시간에 기반한 조건적
보장이 될 겁니다.
작은 크기의 트랜잭션에 대한 보장도 상당히 유용함을 알아둘 것이 필요합니다.
예를 들어, 두개의 캐시 라인 크기에 대한 보장은 스택, 큐, dequeue 에 충분합니다.
하지만, 더 큰 데이터 구조는 더 큰 트랜잭션에 대한 보장을 필요로 하는데, 예를
들어 tree 를 순서대로 횡단하는데에는 tree 의 노드의 수만큼의 크기에 대한 보장이
필요합니다.

따라서, 보장의 크기를 늘리는 것은 HTM 의 유용성 역시 늘려주고, 따라서 CPU 들이
HTM 을 제공하거나 훌륭하고 충분한 우회적 해결방법을 제공할 필요를 증가시킵니다.
\iffalse

Forward-progress guarantees are important, but as we saw, they will
be conditional guarantees based on transaction size and duration.
It is important to note that even small-sized guarantees will be
quite useful.
For example,
a guarantee of two cache lines is sufficient for a stack, queue, or dequeue.
However, larger data structures require larger guarantees, for example,
traversing a tree in order requires a guarantee equal to the number
of nodes in the tree.

Therefore, increasing the size of the guarantee also increases the
usefulness of HTM, thereby increasing the need for CPUs to either
provide it or provide good-and-sufficient workarounds.
\fi

\subsubsection{Improved Debugging Support}
\label{sec:future:Improved Debugging Support}

트랜잭션 크기에 대한 또다른 억제 요소는 트랜잭션을 디버깅 해야할 필요입니다.
현재 메커니즘에서의 문제는 single-step exception 이 자신을 둘러싼 트랜잭션을
abort 시킨다는 것입니다.
이 문제를 위한 우회적 해결 방법으로 프로세서 에뮬레이션 (느려요!), HTM 을 STM
으로의 대체 (느리고 시맨틱이 약간 다릅니다!), 진행을 에뮬레이션 하기 위한
반복적 재시도를 사용한 playback 테크닉 (이상한 failure 모드가 존재합니다!),
그리고 HTM 트랜잭션에서의 디버깅 지원 (복잡해요!) 등의 여러가지 방버들이
있습니다.

HTM 제조사들 가운데 누군가는 브레이크포인트, single stepping, 그리고 print
명령을 포함한 고전적인 디버깅 방법을 트랜잭션 안에서 사용할 수 있는 간단한
방법을 가능하게 하는 HTM 시스템을 제공해야 하며, 이는 HTM 을 더욱 강력하게
만들어줄 겁니다.
2013년에 이르러, 일부 트랜잭셔널 메모리 연구자들은 이 문제를 인식하고 있으며
하드웨어가 돕는 디버깅 기능들에 관한 제안도
있습니다~\cite{JustinGottschlich2013TMdebug}.
물론, 이 제안은 그런 기능들을 실제로 가지고 있고 사용할 수 있는 하드웨어에
의존적입니다.
\iffalse

Another inhibitor to transaction size is the need to debug the transactions.
The problem with current mechanisms is that a single-step exception
aborts the enclosing transaction.
There are a number of workarounds for this issue, including emulating
the processor (slow!), substituting STM for HTM (slow and slightly
different semantics!),
playback techniques using repeated retries to emulate forward
progress (strange failure modes!), and
full support of debugging HTM transactions (complex!).

Should one of the HTM vendors produce an HTM system that allows
straightforward use of classical debugging techniques within
transactions, including breakpoints, single stepping, and
print statements, this will make HTM much more compelling.
Some transactional-memory researchers are starting to recognize this
problem as of 2013, with at least one proposal involving hardware-assisted
debugging facilities~\cite{JustinGottschlich2013TMdebug}.
Of course, this proposal depends on readily available hardware gaining such
facilities.
\fi

\subsubsection{Weak Atomicity}
\label{sec:future:Weak Atomicity}

HTM 이 가까운 미래에 어떤 종류의 크기 제한을 갖게 될 것이라는 점을 놓고 보면,
HTM 은 다른 메커니즘들과 부드럽게 연동될 수 있어야 할겁니다.
해저드 포인터와 RCU 같은 읽기가 대부분인 경우를 위한 메커니즘들과 HTM 의 연동
능력은 트랜잭션 외적인 읽기가 같은 목적지에 쓰기를 하는 트랜잭션을 무조건적으로
abort 시키지 않는다면 개선될 수 있을 겁니다---대신, 해당 읽기는 간단히 해당
트랜잭션 전의 값을 얻어올 수 있을 겁니다.
이런 방식으로, 해저드 포인터와 RCU 는 HTM 이 커다란 데이터 구조를 처리하고
conflict 확률을 줄이는데 사용될 수 있을 겁니다.

하지만, 이는 간단하지가 않습니다.
이 방법을 구현하는 가장 간단한 방법은 각각의 캐시라인과 bus 에 추가적인 상태를
가질 것을 필요로 하는데, 이는 추가적 비용을 필요로 합니다.
이 비용과 함께 생기는 장점은 커다란 영역을 접근하는 읽기 쓰레드들이 연속된
conflict 로 인해 업데이트 쓰레드들이 starve 하게 되는 문제 없이 수행될 수 있게
해준다는 것입니다.
Binary search tree 에 큰 효과를 가져다준 Siakvaras 등의 대안적인
방법~\cite{Siakavaras2017CombiningHA} 은 read-only traversal 에만 RCU 를
사용하고 실제 업데이트에는 HTM 만 사용합니다.
이 조합은 다른 transactional-memory 기법들을 220\% 가량 상회하는 성능을
보이는데, Howard 와 Walpole~\cite{PhilHoward2011RCUTMRBTree} 이 RCU 와 STM 을
조합해서 보인 것과 비슷한 성능 향상입니다.
두 경우 모두, 이 완화된 atomicity 는 하드웨어가 아니라 소프트웨어로
구현되었습니다.
하지만 완화된 aotmicity 를 하드웨어와 소프트웨어 모두로 구현해서 추가적인
성능향상을 보는 것도 흥미로울 겁니다.
\iffalse

Given that HTM is likely to face some sort of size limitations for the
foreseeable future, it will be necessary for HTM to interoperate
smoothly with other mechanisms.
HTM's interoperability with read-mostly mechanisms such as hazard pointers
and RCU would be improved if extra-transactional reads did not
unconditionally abort transactions with conflicting writes---instead,
the read could simply be provided with the pre-transaction value.
In this way, hazard pointers and RCU could be used to allow HTM to handle
larger data structures and to reduce conflict probabilities.

This is not necessarily simple, however.
The most straightforward way of implementing this requires an additional
state in each cache line and on the bus, which is a non-trivial added
expense.
The benefit that goes along with this expense is permitting
large-footprint readers without the risk of starving updaters due
to continual conflicts.
An alternative approach, applied to great effect to binary search trees
by Siakavaras et al.~\cite{Siakavaras2017CombiningHA},
is to use RCU for read-only traversals and HTM
only for the actual updates themselves.
This combination outperformed other transactional-memory techniques by
up to 220\,\%, a speedup similar to that observed by
Howard and Walpole~\cite{PhilHoward2011RCUTMRBTree}
when they combined RCU with STM.
In both cases, the weak atomicity is implemented in software rather than
in hardware.
It would nevertheless be interesting to see what additional speedups
could be obtained by implementing weak atomicity in both hardware and
software.
\fi

\subsection{Conclusions}
\label{sec:future:Conclusions}

현재의 HTM 구현들이 일부 경우에 대해서는 실질적인 이득을 가져다 주긴 했지만,
또한 상당한 단점들을 가지고 있기도 합니다.
가장 심각한 단점은 제한적인 트랜잭션 사이즈, conflict 처리, abort 와 롤백의
필요, 진행 보장의 부재, 되돌이켜질 수 없는 오퍼레이션들의 처리 불가능성, 그리고
락킹과의 미묘한 semantic 상의 차이입니다.
\iffalse

Although current HTM implementations have delivered real performance
benefits in some situations, they also have significant shortcomings.
The most significant shortcomings appear to be
limited transaction sizes,
the need for conflict handling, the need for aborts and rollbacks,
the lack of forward-progress guarantees,
the inability to handle irrevocable operations,
and subtle semantic differences
from locking.
\fi

이러한 단점들 가운데 일부는 미래의 구현들에서는 줄어들 수 있겠습니다만, 기존에
언급된 바~\cite{McKenney2007PLOSTM,PaulEMcKenney2010OSRGrassGreener} 와 같이
많은 다른 종류의 동기화 메커니즘들과 함께 동작할 수 있어야 할 필요성은 계속될
것으로 보입니다.

요약해서, 현재의 HTM 구현들은 병렬 프로그래머의 도구박스에 들어오면 좋은,
그리고 유용한 추가적 도구가 되겠고, 그것들을 사용하기 위해서는 많은 흥미롭고
도전적인 작업들을 필요로 합니다.
하지만, 그것들은 모든 병렬 프로그래밍에서의 문제들을 모두 한번에 처리해줄
마법봉으로 여겨질 수는 없습니다.
\iffalse

Some of these shortcomings might be alleviated in future implementations,
but it appears that there will continue to be a strong need to make
HTM work well with the many other types of synchronization mechanisms,
as noted earlier~\cite{McKenney2007PLOSTM,PaulEMcKenney2010OSRGrassGreener}.

In short, current HTM implementations appear to be welcome and useful
additions to the parallel programmer's toolbox, and much interesting
and challenging work is required to make use of them.
However, they cannot be
considered to be a magic wand with which to wave away all parallel-programming
problems.
\fi

% future/formalregress.tex
% mainfile: ../perfbook.tex
% SPDX-License-Identifier: CC-BY-SA-3.0

\section{Formal Regression Testing?}
\label{sec:future:Formal Regression Testing?}
%
\epigraph{Theory without experiments: Have we gone too far?}
	 {\emph{Michael Mitzenmacher}}

정형적 검증은 여러 제품 환경에서 유용한 것으로
증명되었습니다~\cite{JamesRLarus2004RightingSoftware,AlBessey2010BillionLoCLater,ByronCook2018FormalAmazon,CaitlinSadowski2018staticAnalysisGoogle,DinoDistefano2019FBstaticAnalysis}.
그러나, 리눅스 커널과 같은 복잡한 동시성 코드베이스에 결합된 지속적 통합에
사용되는 자동화된 회귀 테스트 집합에 하드코어 정형 검증이 포함될 수 있기는
할지는 의문입니다.
리눅스 커널 SRCU 를 위한 컨셉 증명은 존재하지만~\cite{LanceRoy2017CBMC-SRCU},
이 테스트는 가장 간단한 RCU 구현 중 하나의 작은 부분을 위한 것이고, 계속
변화하는 리눅스 커널과 함께 유지하기는 어려운 것으로 증명되었습니다.
따라서 리눅스 커널의 회귀 테스트에 정형 검증을 첫번째 멤버로 포함시키기 위해선
무엇이 필요할지 물어볼 가치가 있습니다.

다음 리스트는 좋은 시작이 될 수 있을
겁니다~\cite[slide 34]{PaulEMcKenney2015DagstuhlVerification}:

\iffalse

Formal verification has long proven useful in a number of production
environments~\cite{JamesRLarus2004RightingSoftware,AlBessey2010BillionLoCLater,ByronCook2018FormalAmazon,CaitlinSadowski2018staticAnalysisGoogle,DinoDistefano2019FBstaticAnalysis}.
However, it is an question as to whether hard-core formal verification
will ever be included in the automated regression-test suites used for
continuous integration within complex concurrent codebases, such as the
Linux kernel.
Although there is already a proof of concept for Linux-kernel
SRCU~\cite{LanceRoy2017CBMC-SRCU}, this test is for a small portion
of one of the simplest RCU implementations, and has proven difficult
to keep it current with the ever-changing Linux kernel.
It is therefore worth asking what would be required to incorporate
formal verification as first-class members of the Linux kernel's
regression tests.

The following list is a good
start~\cite[slide 34]{PaulEMcKenney2015DagstuhlVerification}:

\fi

\begin{enumerate}
\item	모든 필요한 변환은 자동화 되어야 합니다.
\item	환경은 (메모리 순서 규칙 포함) 올바르게 처리되어야만 합니다.
\item	메모리와 CPU 오버헤드는 받아들여질 수 있을만큼 적어야 합니다.
\item	버그의 위치를 향하는 구체적 정보가 주어져야 합니다.
\item	소스코드와 입력 이상의 정보의 규모는 작아야만 합니다.
\item	발견된 버그는 코드의 사용자에게 적합해야만 합니다.

\iffalse

\item	Any required translation must be automated.
\item	The environment (including memory ordering) must be correctly
	handled.
\item	The memory and CPU overhead must be acceptably modest.
\item	Specific information leading to the location of the bug
	must be provided.
\item	Information beyond the source code and inputs must be
	modest in scope.
\item	The bugs located must be relevant to the code's users.

\fi

\end{enumerate}

이 리스트는 Richard Bornat 의 명언에 기반하지만 그보다 더 수수합니다: ``정형적
검증 연구자들은 개발자들이 작성하는 코드를 그들이 작성하는 언어로 그것이
수행되는 환경에서 그들이 작성하는 방법으로 검증해야 한다.;;
다음 섹션들은 앞의 요구사항 각자를 논하며, 이 필요성에 몇가지 도구들은 얼마나
잘 화답하고 있는지 보이는 섹션이 그를 뒤따릅니다.

\iffalse

This list builds on, but is somewhat more modest than, Richard Bornat's
dictum: ``Formal-verification researchers should verify the code that
developers write, in the language they write it in, running in the
environment that it runs in, as they write it.''
The following sections discuss each of the above requirements, followed
by a section presenting a scorecard of how well a few tools stack up
against these requirements.

\fi

\subsection{Automatic Translation}
\label{sec:future:Automatic Translation}

Promela 와 \co{spin} 은 귀중한 설계상의 도움이 되지만, 여러분이 C 언어
프로그램을 정형적으로 회귀 테스트 하려 한다면 여러분은 여러분이 코드를 다시
검증하고 싶을 때마다 그 코드를 Promela 로 직접 변환해야 합니다.
여러분의 코드가 매 60-90 일마다 릴리즈 되는 리눅스 커널이라면, 여러분은 매년
네번에서 여섯번 가량 직접 변환을 해야할 겁니다.
시간이 흐름에 따라, 사람의 오류가 발생할 것인데, 이는 이 검증이 소스코드와 맞지
않음을 의미해서 이 검증을 쓸모없게 만들 겁니다.
반복된 검증은 이 정형적 검증 도구이 여러분의 코드를 직접 입력받게 하거나
버그로부터 자유로운 여러분의 코드를 검증에 필요한 형태로 자동으로 변환하는
도구를 가지게 해야 할 겁니다.

\iffalse

Although Promela and \co{spin}
are invaluable design aids, if you need to formally regression-test
your C-language program, you must hand-translate to Promela each time
you would like to re-verify your code.
If your code happens to be in the Linux kernel, which releases every
60--90 days, you will need to hand-translate from four to six times
each year.
Over time, human error will creep in, which means that the verification
won't match the source code, rendering the verification useless.
Repeated verification clearly requires either that the formal-verification
tooling input your code directly, or that there be bug-free automatic
translation of your code to the form required for verification.

\fi

PPCMEM 과 \co{herd} 는 이론상 입력 어셈블리어와 C++ 코드를 입력받을 수 있으나,
이 도구들은 매우 작은 리트머스 테스트에만 동작해서, 일반적으로는 여러분이
여러분의 메커니즘의 핵심 부분을 직접 노출시켜야만 합니다.
Promela 와 \co{spin} 에서처럼, PPCMEM 과 \co{herd} 는 매우 유용하나, 회귀
테스트에 잘 맞지는 않습니다.

\iffalse

PPCMEM and \co{herd} can in theory directly input assembly language
and C++ code, but these tools work only on very small litmus tests,
which normally means that you must extract the core of your
mechanism---by hand.
As with Promela and \co{spin}, both PPCMEM and \co{herd} are
extremely useful, but they are not well-suited for regression suites.

\fi

대조적으로, \co{cbmc} 와 Nidhugg 는 합리적 (여전히 상당히 제한되어 있지만)
크기의 C 프로그램을 입력받으며, 그 능력이 계속 성장한다면, 회귀 테스트 도구에
훌륭한 추가품이 될 수 있을 겁니다.
Coverity 정적 분석 도구 또한 리눅스 커널을 포함한 상당히 큰 크기의 C 프로그램을
입력으로 받습니다.
물론, Coverity 의 정적 분석은 \co{cbmc} 와 Nidhugg 의 것에 비하면 상당히
단순합니다.
다른 한편, Coverity 는 특수한 도전을~\cite{AlBessey2010BillionLoCLater} 갖는
총괄적인 ``C 프로그램'' 에 대한 정의를 가졌습니다.
Amazon Web Services 는 \co{cbmc} 를 포함한 다양한 정형적 검증 도구를 사용하며,
이 도구들 중 일부를 회귀 테스팅에 사용합니다~\cite{ByronCook2018FormalAmazon}.
Google 은 상대적으로 간단한 정적 분석 도구를 논란의 여기가 있지만 C
코드베이스보다 덜 다양한 거대한 Java 코드베이스에
사용합니다~\cite{CaitlinSadowski2018staticAnalysisGoogle}.
Facebook 은 동시성 분석을 포함해 그들의 코드베이스에 보다 공격적인 형태의 정형
검증을
사용합니다만~\cite{DinoDistefano2019FBstaticAnalysis,PeterWOHearn2019incorrectnessLogic}
아직 리눅스 커널에까지 적용하진 않았습니다.
마지막으로, 마이크로소프트는 그들의 코드베이스에 정적 분석을 오랜 기간 사용해
왔습니다~\cite{JamesRLarus2004RightingSoftware}.

\iffalse

In contrast, \co{cbmc} and Nidhugg can input C programs of reasonable
(though still quite limited) size, and if their capabilities continue
to grow, could well become excellent additions to regression suites.
The Coverity static-analysis tool also inputs C programs, and of very
large size, including the Linux kernel.
Of course, Coverity's static analysis is quite simple compared to that
of \co{cbmc} and Nidhugg.
On the other hand, Coverity had an all-encompassing definition of
``C program'' that posed special challenges~\cite{AlBessey2010BillionLoCLater}.
Amazon Web Services uses a variety of formal-verification tools,
including \co{cbmc}, and applies some of these tools to regression
testing~\cite{ByronCook2018FormalAmazon}.
Google uses a number of relatively simple static analysis tools directly
on large Java code bases, which are arguably less diverse than C code
bases~\cite{CaitlinSadowski2018staticAnalysisGoogle}.
Facebook uses more aggressive forms of formal verification against its
code bases, including analysis of concurrency~\cite{DinoDistefano2019FBstaticAnalysis,PeterWOHearn2019incorrectnessLogic},
though not yet on the Linux kernel.
Finally, Microsoft has long used static analysis on its code
bases~\cite{JamesRLarus2004RightingSoftware}.

\fi

이 리스트를 놓고 보면, 제품 수준 소스 코드를 직접 사용할 수 있는 잘 만들어진
정형적 검증 도구를 만들 수 있을 것이 분명합니다.

그러나, C 코드를 입력으로 받는 것의 한가지 단점은 컴파일러가 올바를 거라
가정한다는 겁니다.
한가지 대안은 C 컴파일러가 생성한 바이너리를 입력으로 받아서 모든 연관된
컴파일러 버그를 처리하게 하는 겁니다.
이 방법은 여러 검증 노력에서 사용되었는데, 가장 두곽을 드러내는 것은 SEL4
프로젝트~\cite{ThomasSewell2013L4binaryVerification} 일 겁니다.

\iffalse

Given this list, it is clearly possible to create sophisticated
formal-verification tools that directly consume production-quality
source code.

However, one shortcoming of taking C code as input is that it assumes
that the compiler is correct.
An alternative approach is to take the binary produced by the C compiler
as input, thereby accounting for any relevant compiler bugs.
This approach has been used in a number of verification efforts,
perhaps most notably by the SEL4
project~\cite{ThomasSewell2013L4binaryVerification}.

\fi

\QuickQuiz{
	SEL4 프로젝트에서 사용된 여러 검증 도구의 놀라운 본성을 놓고 보면, 왜
	이 챕터는 그걸 더 다루지 않는지 궁금하군요?

	\iffalse

	Given the groundbreaking nature of the various verifiers used
	in the SEL4 project, why doesn't this chapter cover them in
	more depth?

	\fi

}\QuickQuizAnswer{
	SEL4 프로젝트에서 사용된 검증 도구들이 상당히 쓸모있을 거라는 데에는
	의심의 여지가 없습니다.
	그러나, SEL4 는 단일 CPU 프로젝트로 시작되었습니다.
	그리고 SEL4 가 멀티 프로세서 기능을 얻었지만, 현재로써는 리눅스 커널의
	과거의 Big Kernel Lock (BKL) 과 유사한 큰 규모의 락킹을 사용하고
	있습니다.
	SEL4 의 검증 도구를 병렬 프로그래밍에 대한 책에 추가하는 게 말이 되는
	날이 오길 기대합니다만, 아직은 그 날이 아닙니다.

	\iffalse

	There can be no doubt that the verifiers used by the SEL4
	project are quite capable.
	However, SEL4 started as a single-CPU project.
	And although SEL4 has gained multi-processor
	capabilities, it is currently using very coarse-grained
	locking that is similar to the Linux kernel's old
	Big Kernel Lock (BKL).
	There will hopefully come a day when it makes sense to add
	SEL4's verifiers to a book on parallel programming, but
	this is not yet that day.

	\fi

}\QuickQuizEnd

그러나, 소스나 바이너리에서 직접 검증을 하는 것은 사람에 의한 변환 과정에서의
오류를 제거하는 장점을 갖는데, 안정적인 회귀 테스팅에 치명적으로 중요합니다.

이는 특수 목적 언어를 사용하는 도구가 쓸모없다는 말이 아닙니다.
그와 반대로, 그것들은
\cref{chp:Formal Verification} 에서 논의 했듯 설계 시점 검증에 매우 도움이 될
수 있습니다.
그러나, 그런 도구는 이 섹션의 주제인 자동화된 회귀 테스트에서 특별히 도움되진
않습니다.

\iffalse

However, verifying directly from either the source or binary both have the
advantage of eliminating human translation errors, which is critically
important for reliable regression testing.

This is not to say that tools with special-purpose languages are useless.
On the contrary, they can be quite helpful for design-time verification,
as was discussed in
\cref{chp:Formal Verification}.
However, such tools are not particularly helpful for automated regression
testing, which is in fact the topic of this section.

\fi

\subsection{Environment}
\label{sec:future:Environment}

정형적 검증 도구들이 올바르게 그들의 환경을 모델링하는 것은 치명적으로
중요합니다.
한가지 너무 흔한 누락은 메모리 모델로, Promela/\co{spin} 을 포함한 수많은
정형적 검증 도구들이 sequential consistency 로의 제약을 갖습니다.
\Cref{sec:formal:Is QRCU Really Correct?} 와 연관된 QRCU 경험은 중요한 주의를
기울여야 할 이야기 입니다.

Promela 와 \co{spin} 은
\cref{chp:Advanced Synchronization: Memory Ordering} 에서 알아봤듯 현대의
컴퓨터 시스템과는 잘 들어맞지 않는 sequential consistency 를 가정합니다.
대조적으로, PPCMEM 과 \co{herd} 의 강력한 장점 중 하나는 x86, \ARM, Power,
그리고 \co{herd} 의 경우 리눅스 커널 버전 v4.17 에서 받아들여진 리눅스 커널
메모리 모델~\cite{Alglave:2018:FSC:3173162.3177156} 을 포함한 다양한 CPU
제품군들의 메모리 모델에 대한 자세한 모델링입니다.

\iffalse

It is critically important that formal-verification tools correctly
model their environment.
One all-too-common omission is the memory model, where a great
many formal-verification tools, including Promela/\co{spin}, are
restricted to sequential consistency.
The QRCU experience related in
\cref{sec:formal:Is QRCU Really Correct?}
is an important cautionary tale.

Promela and \co{spin} assume sequential consistency, which is not a
good match for modern computer systems, as was seen in
\cref{chp:Advanced Synchronization: Memory Ordering}.
In contrast, one of the great strengths of PPCMEM and \co{herd}
is their detailed modeling of various CPU families memory models,
including x86, \ARM, Power, and, in the case of \co{herd},
a Linux-kernel memory model~\cite{Alglave:2018:FSC:3173162.3177156},
which was accepted into Linux-kernel version v4.17.

\fi

\co{cmbc} 와 Nihugg 도구들은 메모리 모델을 선택할 수 있는 어떤 능력을 제공하나,
PPCMEM 과 \co{herd} 만큼의 다양성은 아닙니다.
그러나, 시간이 갈수록 거대 규모 도구들이 상당히 다양한 메모리 모델을 수용할
겁니다.

장기적으로는, 정형적 검증 도구가 I/O 를 포함하는 것이 도움이
될테지만~\cite{PaulEMcKenney2016LinuxKernelMMIO} 그러기까진 시간이 좀 걸릴
겁니다.

그렇다고 하나, 환경을 맞추는데 실패하는 도구들도 여전히 유용할 수 있습니다.
예를 들어, 상당히 많은 동시성 버그가 가상의 sequential consistency 제공
시스템에서도 버그일 것이며, 이 버그들은 이 시스템의 메모리 모델을 sequential
consistency 로 지나치게 간략화 하는 도구에 의해서도 발견될 수 있습니다.
그러나, 이런 도구들은 누락된 메모리 순서 지시어에 연관된 버그들은 찾지 못할
텐데, 이 점은
\cref{sec:formal:Is QRCU Really Correct?} 의 주의를 요하는 이야기에서
언급되었습니다.

\iffalse

The \co{cbmc} and Nidhugg tools provide some ability to select
memory models, but do not provide the variety that PPCMEM and
\co{herd} do.
However, it is likely that the larger-scale tools will adopt
a greater variety of memory models as time goes on.

In the longer term, it would be helpful for formal-verification
tools to include I/O~\cite{PaulEMcKenney2016LinuxKernelMMIO},
but it may be some time before this comes to pass.

Nevertheless, tools that fail to match the environment can still
be useful.
For example, a great many concurrency bugs would still be bugs on
a mythical sequentially consistent system, and these bugs could
be located by a tool that over-approximates the system's memory model
with sequential consistency.
Nevertheless, these tools will fail to find bugs involving missing
memory-ordering directives, as noted in the aforementioned
cautionary tale of
\cref{sec:formal:Is QRCU Really Correct?}.

\fi

\subsection{Overhead}
\label{sec:future:Overhead}

모든 하드코어 정형적 검증 도구들은 본성적으로 폭발적인데, 흥미로운 소프트웨어
질문 중 대부분이 실제로 비결정적이라 생각하기 전까지는 실망스러워 보일 겁니다.
그러나, 폭발적인 성질에도 정도의 차이가 있습니다.

PPCMEM 은 설계상 최적화 되어 있지 않은데, 문제의 메모리 모델이 올바르게
표현되었음을 확실히 보장하기 위함입니다.
\co{herd} 는
\cref{sec:formal:Axiomatic Approaches} 에서 언급되었듯 더 적극적인 최적화를
하는데, 따라서 PPCMEM 보다 수십 수백배 빠릅니다.
그렇다고 하나, PPCMEM 과 \co{herd} 둘 다 커다란 코드의 몸통보다는 작은 리트머스
테스트를 목표로 합니다.

대조적으로, Promela/\co{spin}, \co{cbmc} 그리고 Nidhugg 는 (어떤) 더 큰 분량의
코드를 위해 설계되었습니다.
Promela/\co{spin} 은 Curiosity rover 의 파일시스템을 검증하는데
사용되었고~\cite{DBLP:journals/amai/GroceHHJX14}, 앞서 언급되었듯 \co{cbmc} 와
Nidhugg 는 리눅스 커널 RCU 에 적용되었습니다.

\iffalse

Almost all hard-core formal-verification tools are exponential
in nature, which might seem discouraging until you consider that
many of the most interesting software questions are in fact undecidable.
However, there are differences in degree, even among exponentials.

PPCMEM by design is unoptimized, in order to provide greater assurance
that the memory models of interest are accurately represented.
The \co{herd} tool optimizes more aggressively, as described in
\cref{sec:formal:Axiomatic Approaches}, and is thus orders of magnitude
faster than PPCMEM\@.
Nevertheless, both PPCMEM and \co{herd} target very small litmus tests
rather than larger bodies of code.

In contrast, Promela/\co{spin}, \co{cbmc}, and Nidhugg are designed for
(somewhat) larger bodies of code.
Promela/\co{spin} was used to verify the Curiosity rover's
filesystem~\cite{DBLP:journals/amai/GroceHHJX14} and, as noted earlier,
both \co{cbmc} and Nidhugg were appled to Linux-kernel RCU\@.

\fi

만약 휴리스틱 상의 발전이 지난 30년간의 속도와 비슷하게 계속된다면 우린 정형적
검증의 오버헤드가 크게 감소할 것을 기대할 수 있습니다.
그러나, 조합상의 폭발 (combinatorial explosion) 은 여전히 조합상의 폭발이므로,
휴리스틱의 계속된 개선과 관계없이 검증될 수 있는 프로그램의 크기를 분명하게
제한할 것입니다.

그러나, 조합상의 폭발의 이면은 마케도니아의 Philip II 의 조언입니다: ``분할하고
정복하라.''
만약 큰 프로그램이 분할될 수 있고 그 조각들이 검증된다면, 그 결과는 조합상의
\emph{내파 (implosion)}~\cite{PaulEMcKenney2011Verico} 입니다.
분할을 할 자연스러운 장소는 API 경계인데, 예를 들ㅇ면 락킹 기능의 그것들입니다.
그러면 하나의 검증 경로는 이 락킹 구현이 올바른지를 검증하고, 추가적인 검증
경로는 락킹 API 의 올바른 사용을 검증할 수 있습니다.

\iffalse

If advances in heuristics continue at the rate of the past three
decades, we can look forward to large reductions in overhead for
formal verification.
That said, combinatorial explosion is still combinatorial explosion,
which would be expected to sharply limit the size of programs that
could be verified, with or without continued improvements in
heuristics.

However, the flip side of combinatorial explosion is Philip II of
Macedon's timeless advice: ``Divide and rule.''
If a large program can be divided and the pieces verified, the result
can be combinatorial \emph{implosion}~\cite{PaulEMcKenney2011Verico}.
One natural place to divide is on API boundaries, for example, those
of locking primitives.
One verification pass can then verify that the locking implementation
is correct, and additional verification passes can verify correct
use of the locking APIs.

\fi

\begin{listing}[tbp]
\input{CodeSamples/formal/herd/C-SB+l-o-o-u+l-o-o-u-C@whole.fcv}
\caption{Emulating Locking with \tco{cmpxchg_acquire()}}
\label{lst:future:Emulating Locking with cmpxchg}
\end{listing}

\begin{table}[tbh]
\rowcolors{1}{}{lightgray}
\renewcommand*{\arraystretch}{1.1}
\small
\centering
\begin{tabular}{S[table-format=1.0]S[table-format=1.3]S[table-format=2.3]}
	\toprule
	\multicolumn{1}{c}{\# Threads} & \multicolumn{1}{c}{Locking} &
			\multicolumn{1}{c}{\tco{cmpxchg_acquire}} \\
	\midrule
	2 & 0.004 &  0.022 \\
	3 & 0.041 &  0.743 \\
	4 & 0.374 & 59.565 \\
	5 & 4.905 &        \\
	\bottomrule
\end{tabular}
\caption{Emulating Locking: Performance (s)}
\label{tab:future:Emulating Locking: Performance (s)}
\end{table}

이 방법의 성능상 이득은 리눅스 커널 메모리
모델~\cite{Alglave:2018:FSC:3173162.3177156} 을 사용해 선보일 수 있습니다.
이 모델은 \co{spin_lock()} 과 \co{spin_unlock()} 기능을 제공합니다만 이
기능들은 또한
\cref{lst:future:Emulating Locking with cmpxchg}
(\path{C-SB+l-o-o-u+l-o-o-*u.litmus} 와 \path{C-SB+l-o-o-u+l-o-o-u*-C.litmus})
에서 보이듯 \co{cmpxchg_acquire()} 와 \co{smp_store_release()} 를 사용해
에뮬레이션 될 수 있습니다.
\Cref{tab:future:Emulating Locking: Performance (s)}
는 이 모델의 \co{spin_lock()} 과 \co{spin_unlock()} 을 사용하는 것의 성능과
확장성을 이 기능들을 에뮬레이션 하는 것과 비교합니다.
차이는 사소하지 않습니다: 네개의 프로세스에서, 이 모델은 에뮬레이션보다 수백배
이상 빠릅니다!

\iffalse

The performance benefits of this approach can be demonstrated using
the Linux-kernel memory
model~\cite{Alglave:2018:FSC:3173162.3177156}.
This model provides \co{spin_lock()} and \co{spin_unlock()}
primitives, but these primitives can also be emulated using
\co{cmpxchg_acquire()} and \co{smp_store_release()}, as shown in
\cref{lst:future:Emulating Locking with cmpxchg}
(\path{C-SB+l-o-o-u+l-o-o-*u.litmus} and \path{C-SB+l-o-o-u+l-o-o-u*-C.litmus}).
\Cref{tab:future:Emulating Locking: Performance (s)}
compares the performance and scalability of using the model's
\co{spin_lock()} and \co{spin_unlock()} against emulating these
primitives as shown in the listing.
The difference is not insignificant: At four processes, the model
is more than two orders of magnitude faster than emulation!

\fi

\QuickQuiz{
\begin{fcvref}[ln:future:formalregress:C-SB+l-o-o-u+l-o-o-u-C:whole]
	\Cref{lst:future:Emulating Locking with cmpxchg}
	의 라인~\lnref{filter_} 에서는 왜 단순하게 그 조건을 \co{exists} 절에
	추가하는 대신 별도의 \co{filter} 커맨드를 사용하나요?
	그리고 \co{cmpxchg_acquire()} 대신 \co{xchg_acquire()} 를 사용하는게 더
	간단하지 않을까요?

	\iffalse

	Why bother with a separate \co{filter} command on line~\lnref{filter_} of
	\cref{lst:future:Emulating Locking with cmpxchg}
	instead of just adding the condition to the \co{exists} clause?
	And wouldn't it be simpler to use \co{xchg_acquire()} instead
	of \co{cmpxchg_acquire()}?

	\fi

\end{fcvref}
}\QuickQuizAnswer{
	이 \co{filter} 절은 \co{herd} 도구가 \co{exists} 절이 그러는 것보다
	이른 처리 단계에서 수행을 폐기하게 해줘서 상당한 속도향상을 가져옵니다.

	\iffalse

	The \co{filter} clause causes the \co{herd} tool to discard
	executions at an earlier stage of processing than does
	the \co{exists} clause, which provides significant speedups.

	\fi

\begin{table}[tbh]
\rowcolors{7}{lightgray}{}
\renewcommand*{\arraystretch}{1.1}
\small
\centering
\begin{tabular}{S[table-format=1.0]S[table-format=1.3]S[table-format=2.3]
		S[table-format=3.3]S[table-format=2.3]S[table-format=3.3]}
	\toprule
	& & \multicolumn{2}{c}{\tco{cmpxchg_acquire()}}
		& \multicolumn{2}{c}{\tco{xchg_acquire()}} \\
	\cmidrule(l){3-4} \cmidrule(l){5-6}
	\multicolumn{1}{c}{\#} & \multicolumn{1}{c}{Lock}
		& \multicolumn{1}{c}{\tco{filter}}
			& \multicolumn{1}{c}{\tco{exists}}
				& \multicolumn{1}{c}{\tco{filter}}
					& \multicolumn{1}{c}{\tco{exists}} \\
	\cmidrule{1-1} \cmidrule(l){2-2} \cmidrule(l){3-4} \cmidrule(l){5-6}
	2 & 0.004 &  0.022 &   0.039 &  0.027 &  0.058 \\
	3 & 0.041 &  0.743 &   1.653 &  0.968 &  3.203 \\
	4 & 0.374 & 59.565 & 151.962 & 74.818 & 500.96 \\
	5 & 4.905 &        &         &        &        \\
	\bottomrule
\end{tabular}
\caption{Emulating Locking: Performance Comparison (s)}
\label{tab:future:Emulating Locking: Performance Comparison (s)}
\end{table}

	\co{xchg_acquire()} 의 경우, 이 어토믹 오퍼레이션은 락 획득이
	성공했는지와 관계없이 쓰기를 할건데, 이는 \co{xchg_acquire()} 를
	사용하는 모델은 락 획득 실패의 경우 쓰기를 하지 않을
	\co{cmpxchg_acquire()} 를 사용하는 것보다 더 맣은 오퍼레이션을 갖게
	됨을 의미합니다.
	더 많은 쓰기는 폭발할 수 있는 더 많은 조합을 의미하는데,
	\cref{tab:future:Emulating Locking: Performance Comparison (s)} 에 보인
	것과 같습니다
	(\path{C-SB+l-o-o-u+l-o-o-*u.litmus},
	\path{C-SB+l-o-o-u+l-o-o-u*-C.litmus},
	\path{C-SB+l-o-o-u+l-o-o-u*-CE.litmus},
	\path{C-SB+l-o-o-u+l-o-o-u*-X.litmus}, 그리고
	\path{C-SB+l-o-o-u+l-o-o-u*-XE.litmus}).
	이 표는 \co{cmpxchg_acquire()} 가 \co{xchg_acquire()} 보다, 그리고
	\co{filter} 의 사용이 \co{exists} 절의 사용보다 성능이 나음을 분명하게
	보입니다.

	\iffalse

	As for \co{xchg_acquire()}, this atomic operation will do a
	write whether or not lock acquisition succeeds, which means
	that a model using \co{xchg_acquire()} will have more operations
	than one using \co{cmpxchg_acquire()}, which won't do a write
	in the failed-acquisition case.
	More writes means more combinatorial to explode, as shown in
	\cref{tab:future:Emulating Locking: Performance Comparison (s)}
	(\path{C-SB+l-o-o-u+l-o-o-*u.litmus},
	\path{C-SB+l-o-o-u+l-o-o-u*-C.litmus},
	\path{C-SB+l-o-o-u+l-o-o-u*-CE.litmus},
	\path{C-SB+l-o-o-u+l-o-o-u*-X.litmus}, and
	\path{C-SB+l-o-o-u+l-o-o-u*-XE.litmus}).
	This table clearly shows that \co{cmpxchg_acquire()}
	outperforms \co{xchg_acquire()} and that use of the
	\co{filter} clause outperforms use of the \co{exists} clause.

	\fi

}\QuickQuizEnd

도구들이 자동으로 커다란 프로그램을 분할하고 그 조각들을 검증하고 그 조각들의
조합들을 검증할 수 있다면 매우 유용할 겁니다.
그 전까지는, 커다란 프로그램의 검증은 상당한 직접적 개입을 필요로 할 겁니다.
이 개입은 스크립트를 사용하는 것으로 중개되는게 선호될 것이고, 각 릴리즈마다
반복된 검증을 안정적으로 이끌어 갈 수 있을수록 좋을 것이고, 종국적으로는 잘
짜여진 지속적 통합의 형태면 좋을 겁니다.
그리고 Facebook 의 Infer 도구는 compositionality 와 abstraction 을
통해~\cite{SamBlackshear2018RacerD,DinoDistefano2019FBstaticAnalysis} 그걸 위한
중요한 단계를 밟고 있습니다.

어떤 경우든, 우린 정형적 검증 기능이 시간에 따라 증가하길, 그리고 그런 증가가
결국 정형적 검증의 회귀 테스트에서의 적합도를 증가시킬 것을 예상할 수 있습니다.

\iffalse

It would of course be quite useful for tools to automatically divide
up large programs, verify the pieces, and then verify the combinations
of pieces.
In the meantime, verification of large programs will require significant
manual intervention.
This intervention will preferably mediated by scripting, the better to
reliably carry out repeated verifications on each release, and
preferably eventually in a manner well-suited for continuous integration.
And Facebook's Infer tool has taken important steps towards doing just
that, via compositionality and
abstraction~\cite{SamBlackshear2018RacerD,DinoDistefano2019FBstaticAnalysis}.

In any case, we can expect formal-verification capabilities to continue
to increase over time, and any such increases will in turn increase
the applicability of formal verification to regression testing.

\fi

\subsection{Locate Bugs}
\label{sec:future:Locate Bugs}

모든 크기의 모든 소프트웨어 작품은 버그를 갖습니다.
따라서, 버그의 존재나 부재만을 알리는 정형적 검증 도구는 특별히 유용하지
않습니다.
필요한 건 그 버그가 어디에 있는지와 그 버그의 본성에 대한 최소한의 \emph{어떤}
정보를 제공하는 도구입니다.

\co{cbmc} 의 출력은 소스코드로 매핑되는 추적 기록을 포함하는데
Promela/\co{spin} 과 Nidhugg 의 것과 유사합니다.
물론, 이 기록은 무척 길 수 있으며, 그걸 분석하는 건 매우 귀찮을 수 있습니다.
그러나, 그러는게 과거 방법으로 버그를 찾는 것보다는 보통 훨씬 빠르고
즐겁습니다.

또한, 정형적 검증 도구의 가장 간단한 테스트 중 하나는 버그 주입입니다.
어쨌건, 아무나 \co{printf("VERIFIED\\n")} 을 작성할 수 있는 건 아니지만, 정형적
검증 도구의 개발자들 역시 나머지 사람들 만큼이나 버그에 취약하다는 게
사실입니다.
따라서, 버그가 존재한다고 주장하기만 하는 정형적 검증 도구는 그걸 실제 세계의
코드에서 검증하기가 더 어려우므로 기본적으로 덜 믿음직합니다.

\iffalse

Any software artifact of any size contains bugs.
Therefore, a formal-verification tool that reports only the
presence or absence of bugs is not particularly useful.
What is needed is a tool that gives at least \emph{some} information
as to where the bug is located and the nature of that bug.

The \co{cbmc} output includes a traceback mapping back to the source
code, similar to Promela/\co{spin}'s, as does Nidhugg.
Of course, these tracebacks can be quite long, and analyzing them
can be quite tedious.
However, doing so is usually quite a bit faster
and more pleasant than locating bugs the old-fashioned way.

In addition, one of the simplest tests of formal-verification tools is
bug injection.
After all, not only could any of us write
\co{printf("VERIFIED\\n")}, but the plain fact is that
developers of formal-verification tools are just as bug-prone as
are the rest of us.
Therefore, formal-verification tools that just proclaim that a
bug exists are fundamentally less trustworthy because it is
more difficult to verify them on real-world code.

\fi

이 모든 것을 차치하고, 정형적 검증 도구를 작성하는 사람들은 존재하는 도구들을
사용할 수 있습니다.
예를 들어, 심각하고 드문 버그의 존재와 부재를 탐지하도록 설계된 도구는
bisection 을 도울 수도 있습니다.
만약 이 프로그램의 어떤 과거 버전이 버그를 포함하지 않았지만 새 버전은
포함한다면, bisection 은 이 버그를 주입한 커밋을 빠르게 찾는데 사용될 수
있으며, 그 정보는 버그를 찾고 고치는데 충분할 수도 있습니다.
물론, 이런 종류의 전략은 흔한 버그에서는 잘 동작하지 않을텐데, 이 경우
bisection 은 모든 커밋이 최소 하나의 흔한 버그는 가질 것이기 때문입니다.

따라서, 많은 정형적 검증 도구들에 의해 제공되는 수행 기록은 계속해서 가치있을
것이며, 특히 복잡하고 이해하기 어려운 버그에서 그럴 겁니다.
또한, 최근의 작업은 full-up 정확성 증명을 위해 사용되던 전통적 Hoare 로직을
생각나게 하는 \emph{incorrectness-logic} 정형화를, 그러나 버그를 찾는
목적만으로~\cite{PeterWOHearn2019incorrectnessLogic} 적용합니다. 

\iffalse

All that aside, people writing formal-verification tools are
permitted to leverage existing tools.
For example, a tool designed to determine only the presence
or absence of a serious but rare bug might leverage bisection.
If an old version of the program under test did not contain the bug,
but a new version did, then bisection could be used to quickly
locate the commit that inserted the bug, which might be
sufficient information to find and fix the bug.
Of course, this sort of strategy would not work well for common
bugs because in this case bisection would fail due to all commits
having at least one instance of the common bug.

Therefore, the execution traces provided
by many formal-verification tools will continue to be valuable,
particularly for complex and difficult-to-understand bugs.
In addition, recent work applies \emph{incorrectness-logic}
formalism reminiscent of the traditional Hoare logic used for
full-up correctness proofs, but with the sole purpose of finding
bugs~\cite{PeterWOHearn2019incorrectnessLogic}.

\fi

\subsection{Minimal Scaffolding}
\label{sec:future:Minimal Scaffolding}

과거, 정형적 검증 연구자들은 소프트웨어의 무엇이 검증될 것인지에 대한 전체
명세서를 요구했습니다.
불행히도, 수학적으로 엄격한 명세서는 실제 코드보다 클 수도 있고, 명세의 각 행은
코드의 각 행이 그렇듯 버그를 포함하고 있을 수 있습니다.
코드가 명세를 올바르게 구현했다는 걸 증명하기 위한 정형적 검증 노력은 둘 사이의
버그 대비 버그 비교의 증명이 될텐데, 이는 그다지 도움되지 않을 수 있습니다.

더 나쁜게, 리눅스 커널 RCU 를 포함한 여러 소프트웨어 작품에 대한 그 요구사항은
본성적으로 경험에 편중되어
있습니다~\cite{PaulEMcKenney2015RCUreqts1,PaulEMcKenney2015RCUreqts2,PaulEMcKenney2015RCUreqts3}.\footnote{
	또는, 정형적 검증의 말투로 말하자면, 리눅스 커널 RCU 는 \emph{불완전한
	명세} 를 가졌습니다.}
이런 흔한 종류의 소프트웨어에서, 완전환 명세는 세련된 허구입니다.
완전한 명세란 2017년 말의 Meltdown 과 Spectre 사이드 채널 공격으로
분명해진 하드웨어에서의 허구보다 덜하지도
않습니다~\cite{JannHorn2018MeltdownSpectre}.

\iffalse

In the old days, formal-verification researchers demanded a full
specification against which the software would be verified.
Unfortunately, a mathematically rigorous specification might well
be larger than the actual code, and each line of specification
is just as likely to contain bugs as is each line of code.
A formal verification effort proving that the code faithfully implemented
the specification would be a proof of bug-for-bug compatibility between
the two, which might not be all that helpful.

Worse yet, the requirements for a number of software artifacts,
including Linux-kernel RCU, are empirical in
nature~\cite{PaulEMcKenney2015RCUreqts1,PaulEMcKenney2015RCUreqts2,PaulEMcKenney2015RCUreqts3}.\footnote{
	Or, in formal-verification parlance, Linux-kernel RCU has an
	\emph{incomplete specification}.}
For this common type of software, a complete specification is a
polite fiction.
Nor are complete specifications any less fictional for hardware,
as was made clear by the late-2017 Meltdown and Spectre side-channel
attacks~\cite{JannHorn2018MeltdownSpectre}.

\fi

이 상황은 실제 세계의 소프트웨어와 하드웨어 작품에 대한 정형적 검증의 희망을
포기하게 할 수도 있겠으나, 할 수 있는게 상당히 있다는 것이 드러났습니다.
예를 들어, 설계와 코딩 규칙은 코드에 포함된 단정문들처럼 부분적 명세로 동작할
수 있습니다.
그리고 실제로 \co{cbmc} 와 Nidhugg 같은 정형적 검증 도구는 발동될 수 있는
단정문들을 검사함으로써 암묵적으로 이 단정문들을 명세의 부분으로 취급합니다.
그러나, 단정문들도 코드의 한 부분인데, 이는 특히나 그 코드가 스트레스 테스트에
적합하다면 그게 쓸모없어질 확률을 낮춥니다.\footnote{
	그리고 여러분은 여러분의 코드를 스트레스 테스트 \emph{합니다}, 그렇지
	않습니까?}
\co{cbmc} 도구 역시 array-out-of-bound 참조를 검사하며, 따라서 암묵적으로 그걸
명세에 포함시킵니다.
앞서 언급된 비정확성 로직 또한 명세를 사용하는 걸로 생각될 수
있습니다~\cite{PeterWOHearn2019incorrectnessLogic}.

\iffalse

This situation might cause one to give up all hope of formal verification
of real-world software and hardware artifacts, but it turns out that there is
quite a bit that can be done.
For example, design and coding rules can act as a partial specification,
as can assertions contained in the code.
And in fact formal-verification tools such as \co{cbmc} and Nidhugg
both check for assertions that can be triggered, implicitly treating
these assertions as part of the specification.
However, the assertions are also part of the code, which makes it less
likely that they will become obsolete, especially if the code is
also subjected to stress tests.\footnote{
	And you \emph{do} stress-test your code, don't you?}
The \co{cbmc} tool also checks for array-out-of-bound references,
thus implicitly adding them to the specification.
The aforementioned incorrectness logic can also be thought of as using
an implicit bugs-not-present
specification~\cite{PeterWOHearn2019incorrectnessLogic}.

\fi

이 암묵적 명세 접근법은 상당히 말이 되는데, 특히 여러분이 정형적 검증을 완전한
올바름의 증명이 아니라 일반적인 경우보다 다른 강점과 약점을 갖는 검증의 대안적
형태, 즉 테스트로 생각한다면 그렇습니다.
이 관점에서, 소프트웨어는 항상 버그를 가질 것이며, 따라서 버그를 찾는 걸 돕는
모든 종류의 모든 도구는 실제로 아주 좋은 것입니다.

\iffalse

This implicit-specification approach makes quite a bit of sense, particularly
if you look at formal verification not as a full proof of correctness,
but rather an alternative form of validation with a different set of
strengths and weaknesses than the common case, that is, testing.
From this viewpoint, software will always have bugs, and therefore any
tool of any kind that helps to find those bugs is a very good thing
indeed.

\fi

\subsection{Relevant Bugs}
\label{sec:future:Relevant Bugs}

버그를 발견하는 것---그리고 고치는 것---은 물론 모든 종류의 검증 노력의 모든
요점입니다.
분명, 위양성 (false positive) 는 막아져야 합니다.
그러나 위양성의 부재 하에서 조차도, 보그는 존재합니다.

예를 들어, 어떤 소프트웨어 작품이 정확히 100개의 버그를 가지고 있으며, 이것들
각자는 평균적으로 백만년의 수행 시간 중 한번 발생한다고 해봅시다.
더 나아가서 박식한 정형 검증 도구가 개발자들은 마땅히 고쳐야 할 이 모든 100개의
버그를 찾아냈다고 해 봅시다.
이 소프트웨어 제품의 안정성에는 무슨 일이 벌어지겠습니까?

답은 안정성의 \emph{하락} 입니다.

\iffalse

Finding bugs---and fixing them---is of course the whole point of any
type of validation effort.
Clearly, false positives are to be avoided.
But even in the absence of false positives, there are bugs and there are bugs.

For example, suppose that a software artifact had exactly 100 remaining
bugs, each of which manifested on average once every million years
of runtime.
Suppose further that an omniscient formal-verification tool located
all 100 bugs, which the developers duly fixed.
What happens to the reliability of this software artifact?

The answer is that the reliability \emph{decreases}.

\fi

이를 보기 위해, 역사적 경험은 약 7\,\% 의 수정이 새로운 버그를
불러들인다고~\cite{RexBlack2012SQA} 함을 기억하십시오.
따라서, 조합된 실패까지의 중간 시간 (meat time to failure: MTBF) 약 10,000 년을
갖는 이 100개의 버그를 고치는 것은 일곱개의 버그를 더 불러들입니다.
역사적 통계는 새 버그 각각은 70,000 년보다 훨씬 적은 MTBF 를 가질 것이라
말합니다.
이는 결국 이 일곱개의 새 버그의 조합된 MTBF 는 10,000 년보다 훨씬 적을 것임을
의미하며, 이는 결국 좋은 의도로 만들어진 원래의 100개의 버그의 수정이 실제로는
전체 소프트웨어의 안정성을 실제로는 하락시켰음을 의미하게 됩니다.

\iffalse

To see this, keep in mind that historical experience indicates that
about 7\,\% of fixes introduce a new bug~\cite{RexBlack2012SQA}.
Therefore, fixing the 100 bugs, which had a combined mean time to failure
(MTBF) of about 10,000 years, will introduce seven more bugs.
Historical statistics indicate that each new bug will have an MTBF
much less than 70,000 years.
This in turn suggests that the combined MTBF of these seven new bugs
will most likely be much less than 10,000 years, which in turn means
that the well-intentioned fixing of the original 100 bugs actually
decreased the reliability of the overall software.

\fi

\QuickQuizSeries{%
\QuickQuizB{
	알려진 버그의 MTBF 들이 아직 발견되지 않은 버그의 MTBF 를 예측하기 좋은
	정보임을 어떻게 아나요?

	\iffalse

	How do we know that the MTBFs of known bugs is a good estimate
	of the MTBFs of bugs that have not yet been located?

	\fi

}\QuickQuizAnswerB{
	우린 모릅니다, 그렇지만 그건 중요치 않습니다.

	이를 보기 위해, 7\,\% 라는 숫자는 뒤따라서 발견된 버그의 주입에만
	적용됨을 기억하십시오: 이는 발견되지 못한 버그의 주입은 완전히 무시해야
	합니다.
	따라서, 이 알려진 버그에 대한 MTBF 통계는 뒤따라서 발견된 주입된 버그에
	대한 좋은 추정이 됩니다.

	이 전체 섹션의 핵심 요점은, 결코 발견되지 못한 버그보다는 사용자를
	불편하게 하는 버그를 더 주의해야 한다는 겁니다.
	이는 물론 우리가 아직 사용자를 불편하게 하지 않은 버그를 완전히
	무시해야 한다는 말이 \emph{아니라}, 우리는 가장 중요하고 시급한 버그를
	고치는데 있어 노력의 우선순위를 올바르게 잡아야 한다는 것입니다.

	\iffalse

	We don't, but it does not matter.

	To see this, note that the 7\,\% figure only applies to injected
	bugs that were subsequently located: It necessarily ignores
	any injected bugs that were never found.
	Therefore, the MTBF statistics of known bugs is likely to be
	a good approximation of that of the injected bugs that are
	subsequently located.

	A key point in this whole section is that we should be more
	concerned about bugs that inconvenience users than about
	other bugs that never actually manifest.
	This of course is \emph{not} to say that we should completely
	ignore bugs that have not yet inconvenienced users, just that
	we should properly prioritize our efforts so as to fix the
	most important and urgent bugs first.

	\fi

}\QuickQuizEndB
%
\QuickQuizE{
	하지만 정형적 검증 도구는 그 수정에 의해 만들어진 버그를 곧바로
	찾아낼텐데 왜 이게 문제인가요?

	\iffalse

	But the formal-verification tools should immediately find all the
	bugs introduced by the fixes, so why is this a problem?

	\fi

}\QuickQuizAnswerE{
	실제 세계의 정형적 검증 도구는 (더 목소리 높은 제안자의 정형 검증의
	상상에만 존재하는 것과는 달리) 현명치 못하기 때문에, 그리고 따라서 특정
	종류의 버그를 찾아내지 못하기 때문에 문제가 됩니다.
	한가지만 예를 들어보자면, 정형적 검증 도구는 누락된 단정문에 연관된,
	또는 명세의 발견되지 못한 부분에 연관된 버그를 찾지 못할 겁니다.

	\iffalse

	It is a problem because real-world formal-verification tools
	(as opposed to those that exist only in the imaginations of
	the more vociferous proponents of formal verification) are
	not omniscient, and thus are only able to locate certain types
	of bugs.
	For but one example, formal-verification tools are unlikely to
	spot a bug corresponding to an omitted assertion or, equivalently,
	a bug corresponding to an undiscovered portion of the specification.

	\fi

}\QuickQuizEndE
}

더 나쁜게, 평균적으로 매일 한번씩 나타나는 하나의 버그와 백만년마다 나타나는
99개의 버그를 갖는 소프트웨어 작품을 생각해 봅시다.
어떤 정형적 검증 도구가 이 99 백만년짜리 버그를 발견했지만, 이 하루짜리 버그를
찾지 못했다고 해봅시다.
이 99개의 발견된 버그를 고치는데에는 시간과 노력을 필요로 할 것이고 안정성을
떨어뜨리며, 매일 나타나는 버그를 고치기 위해서는 아무것도 하지 않게 될 것인데,
이는 부끄럽고 훨씬 나쁜 일입니다.

따라서, 가장 문제가 되는 버그를 찾는걸 우선시하는 검증 도구를 갖는게 최선일
겁니다.
그러나,
\cref{sec:future:Locate Bugs}
에서 이야기된 바와 같이, 추가적인 도구를 사용하는게 가능합니다.
그런 한가지 강력한 도구는 평범한 과거의 테스트 기법입니다.
버그에 대한 정보를 가졌다면, 이를 위한 구체적 테스트를 만드는게 가능할 것이고,
버그가 드러날 확률을 높이기 위해
\cref{sec:debugging:Hunting Heisenbugs}
에서 설명된 기법들을 사용할 수 있을 겁니다.
이 기법들은 버그의 평소 실패 확률을 대략적으로 추정하는 계산을 가능하게 할
것이고, 이는 결국 버그 수정 노력의 우선순위를 정하는데 사용될 수 있을 겁니다.

\iffalse

Worse yet, imagine another software artifact with one bug that fails
once every day on average and 99 more that fail every million years
each.
Suppose that a formal-verification tool located the 99 million-year
bugs, but failed to find the one-day bug.
Fixing the 99 bugs located will take time and effort, decrease
reliability, and do nothing at all about the pressing each-day failure
that is likely causing embarrassment and perhaps much worse besides.

Therefore, it would be best to have a validation tool that
preferentially located the most troublesome bugs.
However, as noted in
\cref{sec:future:Locate Bugs},
it is permissible to leverage additional tools.
One powerful tool is none other than plain old testing.
Given knowledge of the bug, it should be possible to construct
specific tests for it, possibly also using some of the techniques
described in
\cref{sec:debugging:Hunting Heisenbugs}
to increase the probability of the bug manifesting.
These techniques should allow calculation of a rough estimate of the
bug's raw failure rate, which could in turn be used to prioritize
bug-fix efforts.

\fi

\QuickQuiz{
	하지만 많은 정형적 검증 도구는 한번에 하나의 버그만 찾을 수 있으므로,
	이 도구가 다음 버그를 찾기 전에 각 버그가 고쳐져야만 합니다.
	그런 도구가 주어졌을 때 어떻게 버그 수정 노력의 우선순위를 정할 수
	있겠습니까?

	\iffalse

	But many formal-verification tools can only find one bug at
	a time, so that each bug must be fixed before the tool can
	locate the next.
	How can bug-fix efforts be prioritized given such a tool?

	\fi

}\QuickQuizAnswer{
	한가지 방법은 제품 환경에서는 적합하지 않을 수도 있으나 그 도구가 다음
	버그를 찾는 것은 허용할 수 있는 간단한 수정을 제공하는 것입니다.
	또다른 방법은 설정이나 입력을 제한해서 지금까지 발견된 버그가 발생하지
	못하게 하는 겁니다.
	여러 비슷한 방법들이 있습니다만, 공통적인 주제는 이 도구의 관점에서
	버그를 고치는 것은 제품 품질 수정을 만들고 검증하는 것보다 훨씬 쉽다는
	것이며, 핵심은 제품 품질 수정을 만들고 검증하는데 필요한 더 거대한
	노력을 우선순위화 조정 하자는 겁니다.

	\iffalse

	One approach is to provide a simple fix that might not be
	suitable for a production environment, but which allows
	the tool to locate the next bug.
	Another approach is to restrict configuration or inputs
	so that the bugs located thus far cannot occur.
	There are a number of similar approaches, but the common theme
	is that fixing the bug from the tool's viewpoint is usually much
	easier than constructing and validating a production-quality fix,
	and the key point is to prioritize the larger efforts required
	to construct and validate the production-quality fixes.

	\fi

}\QuickQuizEnd

더 적은 수의 preemption 이 더 발생 가능성 높다는 합리적 가정 하에 더 적은
preemption 을 갖는 수행의 우선순위를 높이는 정형 검증 작업이 최근에 있었습니다.

연관된 버그를 찾는 것은 너무 큰 요구로 들릴 수도 있겠으나, 우리가 소프트웨어의
안정성을 정말로 높이고자 한다면 정말로 필요한 것입니다.

\iffalse

There has been some recent formal-verification work that prioritizes
executions having fewer preemptions, under that reasonable assumption
that smaller numbers of preemptions are more likely.

Identifying relevant bugs might sound like too much to ask, but it is what
is really required if we are to actually increase software reliability.

\fi

\subsection{Formal Regression Scorecard}
\label{sec:future:Formal Regression Scorecard}

\begin{table*}[tbh]
% \rowcolors{6}{}{lightgray}
%\renewcommand*{\arraystretch}{1.1}
\small
\centering
\setlength{\tabcolsep}{2pt}
\begin{tabular}{lcccccccccc}
	\toprule
	& & Promela & & PPCMEM & & \tco{herd} & & \tco{cbmc} & & Nidhugg \\
	\midrule
	(1) Automated &
		& \cellcolor{red!50} &
			& \cellcolor{orange!50} &
				& \cellcolor{orange!50} &
					& \cellcolor{blue!50} &
						& \cellcolor{blue!50} \\
	\addlinespace[3pt]
	(2) Environment &
		& \cellcolor{red!50} (MM) &
			& \cellcolor{green!50} &
				& \cellcolor{blue!50} &
					& \cellcolor{yellow!50} (MM) &
						& \cellcolor{orange!50} (MM) \\
	\addlinespace[3pt]
	(3) Overhead &
		& \cellcolor{yellow!50} &
			& \cellcolor{red!50} &
				& \cellcolor{yellow!50} &
					& \cellcolor{yellow!50} (SAT) &
						& \cellcolor{green!50} \\
	\addlinespace[3pt]
	(4) Locate Bugs &
		& \cellcolor{yellow!50} &
			& \cellcolor{yellow!50} &
				& \cellcolor{yellow!50} &
					& \cellcolor{green!50} &
						& \cellcolor{green!50} \\
	\addlinespace[3pt]
	(5) Minimal Scaffolding &
		& \cellcolor{green!50} &
			& \cellcolor{yellow!50} &
				& \cellcolor{yellow!50} &
					& \cellcolor{blue!50} &
						& \cellcolor{blue!50} \\
	\addlinespace[3pt]
	(6) Relevant Bugs &
		& \cellcolor{yellow!50} ??? &
			& \cellcolor{yellow!50} ??? &
				& \cellcolor{yellow!50} ??? &
					& \cellcolor{yellow!50} ??? &
						& \cellcolor{yellow!50} ??? \\
	\bottomrule
\end{tabular}
\caption{Formal Regression Scorecard}
\label{tab:future:Formal Regression Scorecard}
\end{table*}

\Cref{tab:future:Formal Regression Scorecard}
shows a rough-and-ready scorecard for the formal-verification tools
covered in this chapter.
Shorter wavelengths are better than longer wavelengths.

Promela requires hand translation and supports only sequential
consistency, so its first two cells are red.
It has reasonable overhead (for formal verification, anyway)
and provides a traceback, so its next two cells are yellow.
Despite requiring hand translation, Promela handles assertions
in a natural way, so its fifth cell is green.

PPCMEM usually requires hand translation due to the small size of litmus
tests that it supports, so its first cell is orange.
It handles several memory models, so its second cell is green.
Its overhead is quite high, so its third cell is red.
It provides a graphical display of relations among operations, which
is not as helpful as a traceback, but is still quite useful, so its
fourth cell is yellow.
It requires constructing an \co{exists} clause and cannot take
intra-process assertions, so its fifth cell is also yellow.

The \co{herd} tool has size restrictions similar to those of PPCMEM,
so \co{herd}'s first cell is also orange.
It supports a wide variety of memory models, so its second cell is blue.
It has reasonable overhead, so its third cell is yellow.
Its bug-location and assertion capabilities are quite similar to those
of PPCMEM, so \co{herd} also gets yellow for the next two cells.

The \co{cbmc} tool inputs C code directly, so its first cell is blue.
It supports a few memory models, so its second cell is yellow.
It has reasonable overhead, so its third cell is also yellow, however,
perhaps SAT-solver performance will continue improving.
It provides a traceback, so its fourth cell is green.
It takes assertions directly from the C code, so its fifth cell is blue.

Nidhugg also inputs C code directly, so its first cell is also blue.
It supports only a couple of memory models, so its second cell is orange.
Its overhead is quite low (for formal-verification), so its
third cell is green.
It provides a traceback, so its fourth cell is green.
It takes assertions directly from the C code, so its fifth cell is blue.

So what about the sixth and final row?
It is too early to tell how any of the tools do at finding the right bugs,
so they are all yellow with question marks.

\QuickQuizSeries{%
\QuickQuizB{
	How would testing stack up in the scorecard shown in
	\cref{tab:future:Formal Regression Scorecard}?
}\QuickQuizAnswerB{
	It would be blue all the way down, with the possible
	exception of the third row (overhead) which might well
	be marked down for testing's difficulty finding
	improbable bugs.

	On the other hand, improbable bugs are often also
	irrelevant bugs, so your mileage may vary.

	Much depends on the size of your installed base.
	If your code is only ever going to run on (say) 10,000
	systems, Murphy can actually be a really nice guy.
	Everything that can go wrong, will.
	Eventually.
	Perhaps in geologic time.

	But if your code is running on 20~billion systems,
	like the Linux kernel was said to be by late 2017,
	Murphy can be a real jerk!
	Everything that can go wrong, will, and it can go wrong
	really quickly!!!
}\QuickQuizEndB
%
\QuickQuizE{
	But aren't there a great many more formal-verification systems
	than are shown in
	\cref{tab:future:Formal Regression Scorecard}?
}\QuickQuizAnswerE{
	Indeed there are!
	This table focuses on those that Paul has used, but others are
	proving to be useful.
	Formal verification has been heavily used in the SEL4
	project~\cite{ThomasSewell2013L4binaryVerification},
	and its tools can now handle modest levels of concurrency.
	More recently, Catalin Marinas used Lamport's
	TLA tool~\cite{Lamport:2002:SST:579617} to locate some
	forward-progress bugs in the Linux kernel's queued spinlock
	implementation.
	Will Deacon fixed these bugs~\cite{WillDeacon2018qspinlock},
	and Catalin verified Will's
	fixes~\cite{CatalinMarinas2018qspinlockTLA}.

	Lighter-weight formal verification tools have been used heavily
	in production~\cite{JamesRLarus2004RightingSoftware,AlBessey2010BillionLoCLater,ByronCook2018FormalAmazon,CaitlinSadowski2018staticAnalysisGoogle,DinoDistefano2019FBstaticAnalysis}.
}\QuickQuizEndE
}

Once again, please note that this table rates these tools for use in
regression testing.
Just because many of them are a poor fit for regression testing does
not at all mean that they are useless, in fact,
many of them have proven their worth many times over.\footnote{
	For but one example, Promela was used to verify the file system
	of none other than the Curiosity Rover.
	Was \emph{your} formal verification tool used on software that
	currently runs on Mars???}
Just not for regression testing.

However, this might well change.
After all, formal verification tools made impressive strides in the 2010s.
If that progress continues, formal verification might well become an
indispensible tool in the parallel programmer's validation toolbox.


\section{Functional Programming for Parallelism}
\label{sec:future:Functional Programming for Parallelism}

1980 년대 초에 제가 처음으로 함수형 프로그래밍 수업을 들었을 때, 교수님은
side-effect-free 한 함수형 프로그래밍 스타일은 사소한 병렬화와 분석에 잘
맞는다고 이야기했습니다.
30년이 지나서도 이 말은 여전히 남아있습니다만, 프로그램은 상태도 I/O 도 갖지
않아야만 한다고 이야기한 이 교수님의 또하나의 말과는 달리 병렬 함수형 언어를
사용하는 주류 업체는 적습니다.
Erlang 과 같은 함수형 언어들의 틈새 사용예가 존재하고, 일부 다른 함수형
언어들에 멀티쓰레드 지원이 추가되었습니다만, 주류 업체의 언어 사용은
(일반적으로 OpenMP, MPI, 또는 Fortran 의 경우, coarrays 와 연동되는) C, C++,
Java, 그리고 Fortran 과 같은 절차형 언어의 것으로 남아있습니다.

이 상황은 기본적으로 ``분석이 목표라면, 분석을 하기 전에 절차형 언어를 함수형
언어로 변환하는게 어떨까?'' 하는 질문을 이끌어냅니다.
이 접근방법에 대해서는 물론 여러가지 반대의견들이 있는데, 여기선 세개만 나열해
보자면 다음과 같습니다:
\iffalse

When I took my first-ever functional-programming class in the early 1980s,
the professor asserted that the side-effect-free functional-programming
style was well-suited to trivial parallelization and analysis.
Thirty years later, this assertion remains, but mainstream production
use of parallel functional languages is minimal, a state of affairs
that might well stem from this professor's additional assertion that
programs should neither maintain state nor do I/O.
There is niche use of functional languages such as Erlang, and
multithreaded support has been added to several other functional languages,
but mainstream production usage remains the province of procedural
languages such as C, C++, Java, and Fortran (usually augmented with
OpenMP, MPI, or, in the case of Fortran, coarrays).

This situation naturally leads to the question ``If analysis is the goal,
why not transform the procedural language into a functional language before
doing the analysis?''
There are of course a number of objections to this approach, of which
I list but three:
\fi

\begin{enumerate}
\item	절차적 언어들은 종종 글로벌 변수들을 많이 사용하곤 하는데, 이 변수들은
	다른 함수들, 또는, 더 나쁘게도, 여러 쓰레드들에서 접근될 수 있습니다.
	Haskell 의 \emph{monad} 는 단일 쓰레드의 글로벌 상태를 다루기 위해
	만들어졌고, 글로벌 상태로의 복수 쓰레드에서의 접근은 함수적
	모델에 대한 추가적 위반을 필요로 함을 알아두세요.
\item	멀티쓰레드를 지원하는 절차적 언어들은 종종 락, 어토믹 오퍼레이션,
	트랜잭션과 같은, 함수형 모델에 대한 위반을 추가하는 동기화 기능들을
	종종 사용합니다.
\item	절차적 언어들은, 예를 들면 하나의 함수에의 같은 호출에 두개의 서로 다른
	인자에 같은 구조체로의 포인터를 넘김으로써 함수 인자들을 \emph{alias}
	할 수 있습니다.
	이는 함수가 알지 못한채로 그 구조체를 두개의 서로 다른 (그리고 겹칠 수
	있는) 코드 흐름을 통해 수정하는 결과를 초래할 수 있는데, 이는 분석을
	상당히 복잡하게 만듭니다.
\iffalse

\item	Procedural languages often make heavy use of global variables,
	which can be updated independently by different
	functions, or, worse yet, by multiple threads.
	Note that Haskell's \emph{monads} were invented to deal with
	single-threaded global state, and that multi-threaded access to
	global state requires additional violence to the functional model.
\item	Multithreaded procedural languages often use synchronization
	primitives such as locks, atomic operations, and transactions,
	which inflict added violence upon the functional model.
\item	Procedural languages can \emph{alias} function arguments,
	for example, by passing a pointer to the same structure via two
	different arguments to the same invocation of a given function.
	This can result in the function unknowingly updating that
	structure via two different (and possibly overlapping) code
	sequences, which greatly complicates analysis.
\fi
\end{enumerate}

물론, 글로벌 상태, 동기화 기능들, aliasing 의 중요성으로 인해, 영리한 함수형
프로그래밍 전문가들은 함수형 프로그래밍 모델을 그것들에 조화시키기 위한
방법들을 여럿 제안했고, monads 는 그런 것들 중 하나에 불과합니다.

또다른 접근방법은 병렬 절차적 프로그램을 함수형 프로그램으로 변환하고, 그 결과
나온 프로그램을 분석하는데에 함수형 프로그래밍 도구들을 사용하는 것입니다.
하지만 모든 실제 컴퓨팅은 유한한 시간 간격동안 유한한 입력과 함께 동작하는
커다란 finite-state machine 이라는 점을 놓고 보면, 이보다 훨씬 잘 할 수 있을 법
합니다.
이는, 모든 실제 프로그램은 비록 실용적이지 못할만큼 커다란 것이라 할지라도
하나의 수식으로 변환될 수 있음을 의미합니다~\cite{VijayDSilva2012-sas}.
\iffalse

Of course, given the importance of global state, synchronization
primitives, and aliasing, clever functional-programming experts have
proposed any number of attempts to reconcile the function programming
model to them, monads being but one case in point.

Another approach is to compile the parallel procedural program into
a functional program, then to use functional-programming tools to analyze
the result.
But it is possible to do much better than this, given that any real
computation is a large finite-state machine with finite input that
runs for a finite time interval.
This means that any real program can be transformed into an expression,
possibly albeit an impractically large one~\cite{VijayDSilva2012-sas}.
\fi

하지만, 여러개의 병렬 알고리즘의 낮은 단계의 알맹이들이 현대의 컴퓨터들의
메모리에 들어가기에 알맞을 만큼 충분히 작은 크기의 수식으로 변환될 수 있습니다.
만약 그런 수식이 단정문과 결합된다면, 해당 단정문이 들어맞는지 알아보는 것은
충족 가능성 문제가 됩니다.
충족 가능성 문제는 NP-complete 하긴 하지만, 이 문제들은 전체 상태 공간을
만들어내는데 필요한 것보다는 훨씬 적은 시간 안에 풀이될 수 있는 경우가
많습니다.
또한, 이 풀이에 걸리는 시간은 그 아래 깔려있는 메모리 모델과는 조금만 의존적인
것으로 나타나서, 완화된 순서 규칙의 시스템에서 수행되는 알고리즘 역시 검사될 수
있습니다~\cite{JadeAlglave2013-cav}.

일반적인 접근방법은 프로그램을 single-static-assignment (SSA) 형태로 변환시켜서
하나의 변수로의 각각의 값 할당이 그 변수의 별개의 버전을 만들도록 하는
것입니다.
이는 모든 동작중인 쓰레드로부터의 값 할당에 적용되어서, 그로부터 말미암은
표현은 검사하고자 하는 코드의 모든 가능한 수행경로를 담고 있게 됩니다.
단정문의 추가는 입력과 초기 값들의 어떤 조합이든 단정문이 틀리게 되는 경우를
만들 수 있는지에 대한 질문, 즉 앞에서 이야기한 충족 가능성 여부에 대한 질문을
수반 합니다.
\iffalse

However, a number of the low-level kernels of parallel algorithms transform
into expressions that are small enough to fit easily into the memories
of modern computers.
If such an expression is coupled with an assertion, checking to see if
the assertion would ever fire becomes a satisfiability problem.
Even though satisfiability problems are NP-complete, they can often
be solved in much less time than would be required to generate the
full state space.
In addition, the solution time appears to be only weakly dependent on
the underlying memory model, so that algorithms running on weakly ordered
systems can also be checked~\cite{JadeAlglave2013-cav}.

The general approach is to transform the program into single-static-assignment
(SSA) form, so that each assignment to a variable creates a separate
version of that variable.
This applies to assignments from all the active threads, so that the
resulting expression embodies all possible executions of the code
in question.
The addition of an assertion entails asking whether any combination of
inputs and initial values can result in the assertion firing, which,
as noted above, is exactly the satisfiability problem.
\fi

여기서 있을법한 반대의견 중 하나는, 임의의 루프 구조에 대해서는 제대로 처리를
하지 못한다는 점입니다.
하지만, 많은 경우에 이는 루프를 유한한 횟수만큼 풀어놓는 것으로 처리될 수
있습니다.
또한, 아마도 일부 루프는 귀납법으로는 비난을 면치 못하게 될것이 증명될 겁니다.
\iffalse

One possible objection is that it does not gracefully handle arbitrary
looping constructs.
However, in many cases, this can be handled by unrolling the loop a
finite number of times.
In addition, perhaps some loops will also prove amenable to collapse
via inductive methods.
\fi

또하나의 있을법한 반대의견은 스핀락은 임의의 긴 루프에 관계되고, 유한 횟수
루프를 풀어놓는 방법은 해당 스핀락의 모든 동작을 담지는 못할 것이라는 점입니다.
이 반대의견은 쉽게 극복될 수 있음이 드러났습니다.
전체 스핀락을 모델링하는 대신에, 락을 얻으려 시도하며, 만약 즉시 락을 얻지
못한다면 abort 하도록 하는 trylock 을 모델링 하는 것입니다.
이렇게 되면 단정문은 락이 곧바로 얻을 수 없어서 abort 되는 스핀락에 대해서는
단정문 실패가 되지 않도록 수정되어야 합니다.
논리 수식은 시간과는 무관하기 때문에, 모든 가능한 동시성 동작들은 이 방법을
통해 담아질 수 있을 겁니다.
\iffalse

Another possible objection is that spinlocks involve arbitrarily long
loops, and any finite unrolling would fail to capture the full behavior
of the spinlock.
It turns out that this objection is easily overcome.
Instead of modeling a full spinlock, model a trylock that attempts to
obtain the lock, and aborts if it fails to immediately do so.
The assertion must then be crafted so as to avoid firing in cases
where a spinlock aborted due to the lock not being immediately available.
Because the logic expression is independent of time, all possible
concurrency behaviors will be captured via this approach.
\fi

마지막 반대의견은 이 테크닉은 리눅스 커널을 만드는 수백만 줄의 코드로 이루어진
것과 같은 실제 전체 크기의 소프트웨어 작품을 다루기에는 적합치 않을 것이라는
것입니다.
이건 그럴 수도 있습니다만, 리눅스 커널 내의 그보다 훨씬 작은 병렬 기능들 각각을
제대로 검증하는 것도 상당히 가치있는 것이라는 사실은 그대로 남아있습니다.
그리고 실제로 연구자들은 이 방법을 리눅스 커널의 Tree RCU 구현을 포함한 (RCU 의
덜 심오한 속성들 중 하나를 검증하는 것이기는 하지만) 단순하지 않은 실제 세계의
코드에
적용했습니다~\cite{LihaoLiang2016VerifyTreeRCU,MichalisKokologiannakis2017NidhuggRCU}.

이 테크닉이 얼마나 넓게 적용될 수 있을 것인지를 볼 필요가 있습니다만, 이는
formal verification 분야의 더 ㅎ으미로운 혁신들 중 하나입니다.
함수형 프로그래밍 대변자들이 이야기하는, 피할 수 없는 함수형 프로그래밍의 지배
시대가 올수도 있겠지만, 이 오래도록 끌려온 방법론이 formal-verification 에서의
신뢰성 있는 경쟁이 보이기 시작하는 것은 분명한 사건입니다.
따라서 피할 수 없는 함수형 프로그래밍의 지배 시대에 대해 의심을 가질 이유는
있습니다.
\iffalse

A final objection is that this technique is unlikely to be able to handle
a full-sized software artifact such as the millions of lines of code making
up the Linux kernel.
This is likely the case, but the fact remains that exhaustive validation
of each of the much smaller parallel primitives within the Linux kernel
would be quite valuable.
And in fact the researchers spearheading this approach have applied it
to non-trivial real-world code, including the Tree RCU implementation in
the Linux
kernel~\cite{LihaoLiang2016VerifyTreeRCU,MichalisKokologiannakis2017NidhuggRCU}.

It remains to be seen how widely applicable this technique is, but
it is one of the more interesting innovations in the field of
formal verification.
And it might be more well-received than the traditional advice of
writing all programs in functional form.
Although it might well be that the functional-programming advocates
are at long last correct in their assertion of the inevitable
dominance of functional programming, it is clearly the case
that this long-touted methodology is starting to see credible
competition on its formal-verification home turf.
There is therefore continued reason to doubt the inevitability of
functional-programming dominance.
\fi
